Il primo punto all'ordine del giorno riguardava la decisione del nome del gruppo. Ogni componente ha fatto dalle due alle tre proposte di nomi, che sono state successivamente messe ai voti tra tutti i componenti del gruppo e, dopo due votazioni, a maggioranza ha prevalso il nome \GRUPPO. Agostinetto Matteo ha fatto le sue proposte di nomi e affrontato le successive votazioni tramite \textit{WhatsApp}\footnote{Servizio di messaggistica istantanea.}. Successivamente si è discusso su quale logo utilizzare per il gruppo e sullo stile dell'impaginazione dei documenti da presentare alle varie revisioni.

\noindent Nel secondo punto all'ordine del giorno si è discusso riguardo la scelta del capitolato da svolgere. Dopo aver ristretto a due soli capitolati, C2 e C4, per limiti di capienza raggiunti dagli altri, il gruppo, di comune accordo, ha deciso di scegliere il capitolato C4. Le considerazioni dettagliate riguardo tale scelta sono riportate nel documento \textit{Studio di Fattibilità v1.0.0} . Successivamente a tale decisione sono stati contattati i Proponenti del capitolato per informarli della decisione ed essere sicuri che avessero la disponibilità effettiva a seguire il lavoro del gruppo.

\noindent Nel terzo punto all'ordine del giorno si è discusso e deciso quali software utilizzare per coordinare al meglio il lavoro di gruppo durante lo svolgimento del progetto. E' stata creata una mail di gruppo per le comunicazioni esterne verso i Proponenti del capitolato, mentre per le comunicazioni interne si è deciso di utilizzare il servizio di messaggistica istantanea \textit{WhatsApp} e la chat di \textit{\gls{Facebook}}. Inoltre si è deciso di utilizzare \textit{\gls{GitHub}} per avere un servizio di \gls{repository} gratuito e \textit{GanttProject} per produrre i diagrammi di \gls{Gantt}. Per generare i diagrammi \gls{UML} il gruppo ha optato per il software \gls{Astah}. Infine, per calendarizzare con cura le varie attività ed eventuali impegni dei componenti del gruppo ci si appoggerà al calendario offerto da Google, ossia \gls{Google Calendar}. I dettagli di tutti questi aspetti saranno discussi approfonditamente nel documento \textit{Norme di Progetto v1.0.0} .

\noindent Nell'ultimo punto dell'ordine del giorno si è deciso come effettuare la suddivisione dei vari ruoli richiesti dal progetto e di come gestire le successive rotazioni. I dettagli di queste decisioni sono riportati con cura nel documento \textit{Piano di Progetto v1.0.0} .

