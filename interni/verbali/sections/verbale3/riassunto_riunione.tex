In seguito al primo incontro col Proponente, dopo aver fatto passare un paio di giorni per approfondire i contenuti discussi in tale occasione, il gruppo si è riunito per decidere quali requisiti desiderabili realizzare nel progetto \PROGETTO{} e soprattutto quale sarà la feature che ci possa distinguere da altri software simili nel mercato, così come richiesto nella suddetta discussione col Proponente\footnote{Vedi Verbale Esterno del 2015-03-06.}.
In seguito a varie proposte dei componenti del gruppo, seguite da un'ampia discussione, sono stati individuati i seguenti requisiti opzionali:
\begin{itemize}
	\item \textbf{Scorciatoie da tastiera:} permettono di rompere la linearità della presentazione e potersi spostare sia in avanti sia indietro (\textit{es.} premendo TAB si potrebbe visualizzare una piccola finestra che mostri i titoli di tutte le slide della presentazione);
	\item \textbf{Infografiche:} strumenti che integrano la presentazione. Vengono utilizzate solo per la stampa. Sarebbe interessante dare la possibilità di inserire e creare vari template per personalizzarne l'aspetto grafico;
	\item \textbf{Dashboard presentatore:}  prevede anche una sezione in cui annotarsi parole chiave o brevi appunti nello stile di Keynote\footnote{Software di presentazioni sviluppato da Apple.};
	\item \textbf{Template di presentazioni:} dà la possibilità agli utenti di creare presentazioni a partire da stili grafici predefiniti;
	\item \textbf{Effetti grafici:} effetti di \textit{Zoom Out} e \textit{Zoom In} nei cambi di capitolo durante la presentazione.
\end{itemize}

\noindent Per la feature che ci possa differenziare si è scelta la possibilità di inserire del codice embedded nelle slide per poter visualizzare dati real-time (\textit{es.} presentazione economica: valori di cambio valuta in tempo reale), prevedendo anche dei fallback nel caso in cui la presentazione sia visualizzata offline e fornendo alcuni esempi già implementati a titolo di esempio.

\noindent Si è deciso infine di permettere la registrazione di un utente anche tramite il proprio account di Facebook, Twitter e Google+.