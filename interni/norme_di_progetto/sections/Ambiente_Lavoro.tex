In questa sezione verrà illustrato l'ambiente di lavoro che sarà utilizzato durante lo sviluppo del progetto \PROGETTO.\\

\subsection{Sistema operativo}

Il sistema operativo utilizzato è lasciato a discrezione di ogni membro del gruppo. Questa scelta è dovuta principalmente al fatto che il progetto dovrà supportare più piattaforme.
I membri del gruppo utilizzeranno i seguenti sistemi operativi:

\begin{itemize}
	\item Windows 7 64 bit;
	\item Windows 8.1;
	\item Ubuntu 14.10;
	\item Mac OS X 10.10.2.
\end{itemize}

\subsection{Coordinamento}

Il coordinamento del gruppo avviene tramite:
\begin{itemize}
	\item Repository Git;
	\item Google Drive;
	\item Google Calendar;
\end{itemize}

\subsubsection{Repository Git}

Nonostante siano disponibili molti repository (Git, Mercurial, SVN) è stato scelto di utilizzare Git in quanto il servizio soddisfa pienamente le necessità di hosting e versionamento necessarie per lo sviluppo di questo progetto, inoltre diversi membri del gruppo avevano già usato tale servizio. Esso permette di lavorare senza una connessione attiva a internet e da la possibilità di ignorare alcune estensioni specificate in un file chiamato .gitignore\footnote{File globale di Git che contiene una lista di regole per ignorare i file.}.\\ Per i membri del gruppo che desiderano utilizzare un client, si consigliano i seguenti:
\begin{itemize}
	\item \textbf{SourceTree};
	\item \textbf{GitHub};
\end{itemize}
Per maggiori dettagli si rimanda alla sezione \ref{repository}.

\subsubsection{Google Drive}

Abbiamo scelto di utilizzare questo servizio cloud per condividere tutti i documenti che non necessitano di versionamento e vengono utilizzati frequentemente da parte dei membri del gruppo. 
È un servizio molto semplice e accessibile direttamente da browser che permette di lavorare su documenti creati con Google Docs.

\subsubsection{Google Calendar}
Google Calendar viene utilizzato per la gestione delle risorse umane e per tenere traccia degli eventi importanti; infatti è stato creato un calendario condiviso con tutti i membri del gruppo in modo da conoscere le date in cui una persona è assente o non reperibile e le date rilevanti per il gruppo.\\
Inoltre è possibile modificare la gestione delle notifiche per un evento o più eventi in modo che venga inviata automaticamente una email a tutti i membri del gruppo. In questo caso il preavviso deve essere di almeno un giorno.

\subsection{Ambiente documentale}

\subsubsection{Stesura documenti}

Per la stesura dei documenti verrà utilizzato il linguaggio di markup \LaTeX.
Come editor è consigliato TeXstudio\footnote{\url{http://texstudio.sourceforge.net}}, il quale è disponibile per tutti i principali sistemi operativi.

\subsubsection{Script}

Per facilitare la stesura dei documenti sono stati creati alcuni script:

\begin{itemize}
	\item Generazione di tutti i documenti PDF: con il comando make all verranno generati tutti i PDF dei documenti contenuti nella directory corrente;
	%\item Controllo ortografico: con il comando make aspell verrà invocato il programma aspell su tutti i documenti della directory corrente;
	%\item Eliminazione file errati o vecchi: con il comando make clean verranno eliminatati i file generati da compilazioni vecchio o file non necessari della directory corrente;
	\item Evidenziare glossario: con il comando java glossary verrà eseguito uno script che selezionerà nei documenti, della directory corrente, le parole contenute nella versione più recente del glossario e le evidenzierà con il simbolo "\G".

\end{itemize}

\subsubsection{Pianificazione delle attività}
Per pianificare le attività e la gestione del progetto e delle risorse umane si è scelto di usare GanttProject\footnote{\url{http://www.ganttproject.biz}}.

\subsection{Ambiente di sviluppo}
Secondo il decreto legislativo n.81, ogni centoventi minuti passati in maniera continuativa davanti al computer il ogni membro deve fare almeno quindici minuti interruzione.\footnote{\url{http://www.pmi.it/impresa/normativa/news/51788/dipendenti-al-pc-obblighi-di-legge-per-i-datori-di-lavoro.html}}

\subsection{Diagrammi UML}

Per la realizzazione dei diagrammi UML è stato scelto Astah Professional Edition\footnote{\url{http://astah.net/download}}.
E' possibile ricevere gratuitamente la licenza della versione professional inviando una richiesta al sito \url{http://astah.net/student-license-request}.

\subsection{Verifica}

\subsubsection{Verifica dei documenti}

La verifica dei documenti verrà eseguita ogni volta che è stata effettuata una modifica ad un documento e debba essere approvato.
Per una corretta verifica di un documento vanno seguite le seguenti pratiche:

\begin{itemize}
	\item \textbf{Controllo tipografico: }tramite l'utilizzo di TeXstudio verranno individuati errori tipografici presenti nel documento;
	\item \textbf{Controllo lessicale: }il \textit{Verificatore} dovrà controllore che il documento non presenti errori lessicali attraverso un'attenta analisi del testo utilizzando la tecnica inspection o walkthrough;
	\item \textbf{Controllo glossario: }il \textit{Verificatore} dovrà controllore che ogni parola, nel testo, presente nel glossario sia correttamente evidenziata;
	\item \textbf{Controllo contenuto: }il \textit{Verificatore} dovrà controllare che il documento contenga tutto il necessario e che sia impaginato adeguatamente;
	\item \textbf{Rispetto delle norme del progetto: }il \textit{Verificatore} dovrà controllare che il documento segua le norme di progetto stabilite;
	\item \textbf{Lista di controllo: }si dovrà stilare una lista di errori più frequenti, per semplificare le successive verifiche dei documenti;
	\item \textbf{Rispetto indice Gulpease: }il \textit{Verificatore} dovrà calcolare e controllare, per ogni documento, che gli indici di Gulpease risiedano nel range di valori specificato nel \textit{Piano di Qualifica}, altrimenti si dovrà effettuare una ricerca walkthrough alla ricerca delle frasi troppo lunghe e complesse;
	\item \textbf{Segnalazione errori: }una volta completata la verifica di un documento, se sono stati riscontrati errori, il \textit{Verificatore} dovrà aprire dei ticket per segnalarli.
\end{itemize}

\subsection{Verifica dei diagrammi}

I diagrammi devono essere verificati manualmente dal \textit{Verificatore} e deve controllare che aderiscano correttamente allo standard UML 2.0.
In particolare deve controllare che i diagrammi di flusso siano rappresentati in maniera corretta e che i casi d'uso utilizzino correttamente le inclusioni e le estensioni.