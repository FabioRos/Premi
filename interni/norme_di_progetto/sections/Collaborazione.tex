	\subsection{Comunicazioni}
		\subsubsection{Comunicazioni interne}
Le comunicazioni interne verranno eseguite attraverso la mailing list: \url{dazzleworksgroup@gmail.com}. \\ 
Quando un membro del gruppo vuole inviare una email a tutti i componenti, deve inviare il messaggio dalla sua casella di posta elettronica personale verso l'indirizzo \url{dazzleworksgroup@gmail.com}, dalla quale un inoltro automatico provvederà a mandare l'email agli altri indirizzi di posta elettronica personali dei componenti del gruppo, tenendo traccia di tutte le comunicazioni.\\ 
I membri sono tenuti a prestare attenzione al numero di messaggi diffusi. È sempre importante non arrivare a una sovraesposizione del pubblico interno, in quanto si creerebbe solo senso di smarrimento e confusione. Questo non esclude che uno stesso messaggio non sia proposto su più mezzi di informazione, azione spesso necessaria, ma questo presuppone un intervento ponderato e non casuale.\\
Con lo scopo di facilitare la comunicazione tra i membri del gruppo vengono utilizzati strumenti di messaggistica istantanea e videochiamata come Skype e Google Hangouts.
È necessario redigere un \gls{verbale} nel caso in cui siano state prese decisioni o siano emersi dettagli inerenti allo sviluppo del progetto. 
		\subsubsection{Comunicazioni esterne}
Per le comunicazioni esterne è stato creato un indirizzo di posta elettronica:
\begin{center}
\url{dazzleworksgroup@gmail.com}
\end{center}
Tale indirizzo deve essere l'unico canale di comunicazione verso l'esterno. Sarà solo il \textit{Responsabile di progetto} ad utilizzare l'indirizzo di posta per intrattenere le corrispondenze con i proponenti e i Committenti. È compito del \textit{Responsabile di progetto} informare i membri del gruppo delle discussioni avvenute ed eventualmente inoltrare i messaggi alla mailing list.

	\subsection{Composizione email}
In questo paragrafo viene descritta la forma che deve avere una email sia per comunicazione interna che esterna.
		\subsubsection{Mittente}
\begin{itemize}
	\item \textbf{Interno:} indirizzo personale di chi scrive il messaggio;
	\item \textbf{Esterno:} l'unico indirizzo utilizzabile per comunicare verso l'esterno è \url{dazzleworksgroup@gmail.com} e deve essere usato esclusivamente dal \textit{Responsabile di progetto}.
\end{itemize}
	\subsubsection{Destinatario}
\begin{itemize}
	\item \textbf{Esterno:} l'indirizzo del destinatario può avere variazioni a seconda che si voglia comunicare con il Prof. Tullio Vardanega, il Prof. Riccardo Cardin o con i Committenti del progetto;
	\item \textbf{Interno:} l'unico indirizzo utilizzabile è \url{dazzleworksgroup@gmail.com}.
\end{itemize}
Le uniche eccezioni permesse sono:
\begin{itemize}
	\item \textbf{Proposte all'\textit{Amministratore di progetto}:} per eventuali proposte di cambiamento delle norme da parte di un membro del gruppo, quest'ultimo dovrà contattarlo al suo indirizzo di posta elettronica personale;
	\item \textbf{Comunicazione ristretta tra alcuni membri del gruppo:} in questi casi i membri del gruppo utilizzeranno i loro indirizzi personali.
\end{itemize}

		\subsubsection{Oggetto}
L'oggetto deve sintetizzare il contenuto della email. Deve essere chiaro, breve e possibilmente univoco in modo da riconoscerlo dagli altri precedenti.\\
Nel caso si debba mandare un messaggio alla mailing list è obbligatorio di aggiungere "Group:" all'inizio dell'oggetto.
		\subsubsection{Corpo}
Il corpo del messaggio deve essere chiaro, diretto e avere tutte le informazioni necessarie per permettere a tutti i destinatari di capire correttamente l'argomento trattato. Per riferirsi ad alcuni componenti del gruppo o ruoli di progetto si dovrà usare la seguente sintassi: "\textbf{Cognome Nome}" o "\textbf{Nome Ruolo}". Alla fine del corpo del messaggio, il mittente dovrà sempre firmarsi con il suo nome, cognome e ruolo all'interno del gruppo.
		\subsubsection{Allegati}
Viene consentito l'uso di allegati che deve essere limitato solo al caso in cui essi siano realmente necessari. È buona norma presentare gli allegati scrivendo alcune righe di spiegazione. Prestare attenzione al formato(preferibilmente PDF) del documento che si sta inviando. 

	\subsection{Riunioni}
Il \textit{Responsabile di progetto} ha il compito di indire le riunioni, sia interne che esterne. Per ogni riunione sarà necessario specificare data, ora, luogo, e l'ordine del giorno. Le informazioni sulla riunione dovranno essere rese disponibili con almeno quattro giorni di anticipo.
		\subsubsection{Interne}
Sarà il \textit{Responsabile di Progetto} a convocare i componenti del gruppo alle riunioni generali. Tutti i membri del team dovranno partecipare fisicamente o in video chiamata Le riunioni avranno una frequenza almeno quindicinale. Ogni componente del gruppo è tenuto a leggere la posta elettronica e rispondere ad eventuali richieste di riunioni interne. \\
Il \textit{Responsabile di progetto} può anticipare o posticipare una riunione in base alla disponibilità data dai componenti del gruppo. È possibile richiedere una riunione generale da parte di un qualsiasi componente del gruppo. La richiesta verrà inoltrata al \textit{Responsabile di Progetto} che potrà decidere se accoglierla o rifiutarla.\\
Inoltre, è possibile e auspicabile che siano necessarie riunioni tra specifici membri del gruppo, ad esempio: in fase di analisi può essere utile che solo gli \textit{Analisti} si incontrino tra di loro. I componenti che non hanno preceduto la riunione saranno comunque informati sui contenuti e le decisioni prese tramite invio di una email alla mailing list o alla pubblicazione di un \gls{verbale} sul \gls{repository} nel caso siano state prese decisioni importanti.
		\subsubsection{Esterne}
Sarà il \textit{Responsabile di progetto} a fissare le riunioni esterne con i Proponenti e/o i Committenti utilizzando la casella di posta creata appositamente. Prima di contattare le parti esterne, il \textit{Responsabile di progetto} dovrà verificare la presenza dei membri interni al gruppo. \\
In caso di risposta positiva si provvederà a contattare i proponenti e/o i Committenti per fissare la data dell'incontro.
È compito del \textit{Responsabile di progetto} a redigere il \gls{verbale} dell'incontro avvenuto.

	\subsection{Verbale}
Per verbali si intende quanto si è discusso nel corso di una riunione e le decisioni che eventualmente sono state assunte. \\
Nella stesura di un \gls{verbale} occorre essere sintetici, essenziali ed esaustivi, cercando di rappresentare gli interventi, le osservazioni o le decisioni assunte in maniera oggettiva, senza interpretazioni personali, ambigue o fuorvianti. Il tempo dei verbi che deve essere utilizzato è il presente.\\
I verbali dovranno obbligatoriamente includere le seguenti informazioni:
\begin{itemize}
	\item Dove e quando si svolge la riunione, i partecipanti, gli assenti, il nome di chi presiede e quello del segretario verbalizzante;
	\item Tipo di incontro(interno o esterno);
	\item Le questioni e/o i problemi ancora irrisolti;
	\item I (nuovi) argomenti posti in discussione all'ordine del giorno;
	\item Gli interventi più significativi rappresentati in ordine temporale; 
	\item I suggerimenti, le proposte, le conclusioni e le decisioni assunte; 
	\item I compiti assegnati ad ognuno con indicazione delle rispettive scadenze; 
	\item Gli argomenti da trattare nella prossima riunione. 
\end{itemize}

	\subsection{Repository}\label{repository}
In questa sezione sono definiti gli strumenti utilizzati per la condivisione dei file. \\
Per la gestione della documentazione e della codifica è stato creato un \gls{repository}. Il \gls{repository} è privato e quindi accessibile solo dai membri del gruppo \GRUPPO.\\
Dopo l'ultima revisione il progetto verrà reso open-source.
Per il \gls{repository} abbiamo scelto di usare il servizo \gls{GitHub}, il quale utilizza il sistema di \gls{versionamento} \gls{Git}.
		\subsubsection{Utilizzo del servizio}
È necessario eseguire le operazioni di sincronizzazione all'inizio e alla fine di ogni sessione di lavoro. \\
In particolare ad ogni sessione di lavoro andranno eseguite le seguenti operazioni:
\begin{itemize}
	\item \textbf{Pull} - scaricare da la versione più aggiornata del progetto per poterci poi lavorare offline;
	\item \textbf{Add} - aggiungere i file ad uno specifico elenco generando una proposta di modifica;
	\item \textbf{Commit} - validare le proposte fatte in precedenza;
	\item \textbf{Push} - serve a pubblicare i risultati online, sulla piattaforma di sviluppo;
	\item \textbf{Merge} - permette di fondere un \gls{branch} con il \gls{repository} padre, così da implementare le modifiche apportate ai file e alle cartelle originarie.
\end{itemize}
È vietato utilizzare "git add *" per evitare di includere file nascosti essendo consapevoli dell'esatto contenuto. Ogni commit deve essere corredato di un messaggio che descriva in modo \gls{conciso} il lavoro svolto.

		\subsubsection{Struttura}
Tutta la documentazione si può trovare in un \gls{repository} disponibile al seguente indirizzo: \url{https://github.com/FabioRos/Premi}. \\
Esso avrà questa struttura:
\begin{itemize}
	\item \textbf{Interni:}
		\begin{itemize}
				\item \textbf{Studio di fattibilità};
				\item \textbf{Norme di Progetto}.
		\end{itemize}
	\item \textbf{Esterni:}
		\begin{itemize}
				\item \textbf{Analisi dei Requisiti};
				\item \textbf{Piano di progetto};
				\item \textbf{Piano di qualifica};
				\item \textbf{Glossario};
				\item \textbf{Lettera di presentazione};
				\item \textbf{Verbali}.
		\end{itemize}
\end{itemize}


