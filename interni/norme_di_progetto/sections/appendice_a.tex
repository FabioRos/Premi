Di seguito verrà presentata la lista degli errori più comuni trovati durante le verifiche dei documenti:
\begin{itemize}
	\item \textbf{Norme stilistiche:} la conoscenza \textbf{non} approfondita delle norme per la stesura dei documenti potrebbe portare a errori:
		\begin{itemize}
			\item Nel caso degli elenchi potrebbero esserci elementi che non iniziano con la lettera maiuscola;
			\item Nel caso degli elenchi potrebbe non terminare con il "." l'ultimo elemento oppure potrebbe mancare il ";" per uno o più elementi tranne l'ultimo; 
			\item Potrebbe mancare la marcatura \G{} in pedice per i termini presenti nel \textit{Glossario v4.0.0};
			\item Il non utilizzo del grassetto per le fasi principali del progetto, il corsivo per i documenti e i ruoli di progetto;
			\item Le note a piè di pagina che non iniziano con una maiuscola e non finiscono con il ".".
		\end{itemize}
	\item \textbf{Lingua italiana:} 
		\begin{itemize}
			\item Più tempi verbali all'interno della stessa frase;
			\item Utilizzo di termini con significato ambiguo;
			\item Proponente e committente: viene confuso il loro significato;
			\item L'utilizzo di frasi troppo lunghe rendono il concetto di difficile comprensione;
			\item Utilizzo erroneo del termine "fase";
		\end{itemize}
	\item \textbf{\LaTeX:}
	\begin{itemize}
			\item Non viene considerato il carattere di spaziatura dopo l'inserimento dei  comandi \LaTeX;
			\item Macro \LaTeX non viene scritta usando l'apposito commando "\_LaTeX";
			\item Non viene considerato il carattere di spaziatura dopo l'inserimento dei  comandi \LaTeX;
	\end{itemize}
	\item \textbf{UML:}
	\begin{itemize}
		\item Direzione delle frecce non corretta;
		\item Non viene utilizzata la notazione UML 2.0 per la rappresentazione delle interfacce;
		\item Il sistema non deve mai essere un attore;
		\item Utilizzo della relazione di aggregazione invece della relazione di composizione o viceversa nei diagrammi delle classi;
		\item Viene erroneamente scambiato il verso di associazione;
	\end{itemize}
	\item \textbf{Riferimenti:}
	\begin{itemize}
		\item Non inserimento della versione dei documenti citati;
		\item I riferimento informativi verso libri e siti web devono essere dettagliati, indicando l'edizione di riferimento e i capitoli utilizzati.w
	\end{itemize}
	\item \textbf{Tracciamento requisiti:}
	\begin{itemize}
		\item Nei casi d'uso c'è il rischio di confondere inclusione con estensione;
		\item Ogni requisito deve essere atomico.
	\end{itemize}
	\item \textbf{Altro:} errori dovuti a distrazioni e/o errori di battitura.
\end{itemize}

