Di seguito verrà presentata la lista degli errori più comuni trovati durante le verifiche dei documenti:
\begin{itemize}
	\item \textbf{Norme stilistiche:} la conoscenza \textbf{non} approfondita delle norme per la stesura dei documenti potrebbe portare a errori:
		\begin{itemize}
			\item Nel caso degli elenchi potrebbero esserci elementi che non iniziano con la lettera maiuscola;
			\item Nel caso degli elenchi potrebbe non terminare con il "." l'ultimo elemento oppure potrebbe mancare il ";" per uno o più elementi tranne l'ultimo; 
			\item Potrebbe mancare la marcatura "G" in pedice per i termini presenti nel \textit{Glossario v2.0.0};
			\item Il non utilizzo del grassetto per le fasi principali del progetto, il corsivo per i documenti e i ruoli di progetto;
			\item Le note a piè di pagina che non iniziano con una maiuscola e non finiscono con il ".".
		\end{itemize}
	\item \textbf{Lingua italiana:} 
		\begin{itemize}
			\item Più tempi verbali all'interno della stessa frase;
			\item Utilizzo di termini con significato ambiguo.
		\end{itemize}
	\item \textbf{\LaTeX:} non viene considerato il carattere di spaziatura dopo l'inserimento dei  comandi \LaTeX;
	\item \textbf{Altro:} errori dovuti a distrazioni e/o errori di battitura.
\end{itemize}

