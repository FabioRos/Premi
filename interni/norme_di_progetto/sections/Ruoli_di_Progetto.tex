Lo sviluppo prevede la collaborazione di individui a cui verranno assegnati ruoli diversi. Tali ruoli rappresentano figure aziendali specializzate, indispensabili per il buon esito del progetto. Ogni membro del gruppo dovrà ricoprire ogni ruolo almeno una volta. Si deve controllare attentamente che non vi siano conflitti di interesse specialmente nelle attività di approvazione e verifica. Per garantire che la rotazione dei ruoli non provochi conflitti è necessario che le attività vengano pianificate con attenzione e che i membri interessati rispettino il loro ruolo e compito assegnato. Sarà il compito del \textit{Verificatore} a controllare che tutte le condizioni sopra indicate siano rispettate. Se il \textit{Verificatore} troverà delle incongruenze con quanto menzionato sopra avrà il compito di avvisare il \textit{Responsabile di Progetto} che avrà la responsabilità di risolvere la questione.\\
Si descrivano i diversi ruoli di progetto con le relative responsabilità e le modalità operative.
\subsection{Responsabile di Progetto}
Il \textit{Responsabile di Progetto} rappresenta il progetto, in quanto accentra su di sé la responsabilità di scelta e approvazione, ed il gruppo, poiché presenta al committente i risultati del lavoro svolto.
Detiene il potere decisionale, quindi la responsabilità in merito a:
\begin{itemize}
	\item Pianificazione, coordinamento e controllo delle attività;
	\item Gestione e controllo delle risorse;
	\item Analisi e gestione dei rischi;
	\item Approvazione dei documenti;
	\item Approvazione dell'offerta economica.
\end{itemize}
Di conseguenza, ha il compito:
\begin{itemize}
	\item Assicurarsi che le attività di verifica e validazione vengano svolte sistematicamente seguende le \textit{Norme di Progetto};
	\item Garantire che vengano rispettati i ruoli e le competenze assegnate nel \textit{Piano di Progetto};
	\item Garantire che non vi siano conflitti tra \textit{Verificatori} e \textit{Redattori};
	\item Gestire la creazione e l'assegnazione dei ticket e di assegnarli ad ogni membro del gruppo.
\end{itemize}

