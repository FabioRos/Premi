Lo sviluppo prevede la collaborazione di individui a cui verranno assegnati ruoli diversi. Tali ruoli rappresentano figure aziendali specializzate, indispensabili per il buon esito del progetto.\\
Di seguito i diversi ruoli di progetto con le relative responsabilità e modalità operative.
\subsection{Responsabile di Progetto}
Il \textit{Responsabile di Progetto} rappresenta il progetto, in quanto accentra su di sé la responsabilità di scelta e approvazione, ed il gruppo, poiché presenta al committente i risultati del lavoro svolto.
Detiene il potere decisionale, quindi la responsabilità in merito a:
\begin{itemize}
	\item Pianificazione, coordinamento e controllo delle attività;
	\item Gestione e controllo delle risorse;
	\item Analisi e gestione dei rischi;
	\item Approvazione dei documenti;
	\item Approvazione dell'offerta economica.
\end{itemize}
Di conseguenza, ha il compito:
\begin{itemize}
	\item Assicurarsi che le attività di verifica e validazione vengano svolte sistematicamente seguendo le \textit{Norme di Progetto};
	\item Garantire che vengano rispettati i ruoli e le competenze assegnate nel \textit{Piano di Progetto};
	\item Garantire che non vi siano conflitti tra \textit{Verificatori} e \textit{Redattori};
	\item Gestire la creazione e l'assegnazione dei \gls{ticket} ad ogni membro del gruppo.
\end{itemize}
\subsection{Ammministratore}
L'\textit{Amministratore di Progetto} è il responsabile dell'efficienza e del rendimento nell'ambiente di lavoro. Le mansioni che gli competono sono le seguenti:
\begin{itemize}
	\item Agevolare le attività richieste dalle \textit{Norme di Progetto v2.0.0};
	\item Ricercare e rendere operativi tutti gli strumenti necessari all'automatizzazione del maggior numero di compiti;
	\item Fornire procedure e strumenti per il controllo e la segnalazione del controllo qualità;
	\item Gestire l'archiviazione e il \gls{versionamento} della documentazione;
	\item Controllare le versioni dei prodotti e gestire le loro configurazioni.
\end{itemize}
Redige le \textit{Norme di Progetto}, dove spiega e norma l'utilizzo degli strumenti, e redige la sezione del \textit{Piano di Qualifica} dove vengono descritti strumenti e metodi di verifica.
\subsection{Analista}
L'\textit{Analista} è il responsabile delle attività di analisi. Le mansioni che gli competono sono:
\begin{itemize}
	\item Comprendere appieno la natura e la complessità del problema;
	\item Produrre una \textit{specifica di progetto} motivata in ogni suo punto e comprensibile a tutti gli interessati.
\end{itemize}
Redige lo \textit{Studio di Fattibilità}, l'\textit{Analisi dei Requisiti} e partecipa alla redazione del \textit{Piano di Qualifica} in quanto conosce l'ambito di progetto.

\subsection{Progettista}
Il \textit{Progettista} è il responsabile delle attività di progettazione. La mansioni che gli competono sono:
\begin{itemize}
	\item Produrre una soluzione comprensibile, attuabile e motivata;
	\item Effettuare scelte su aspetti progettuali che applichino al prodotto soluzioni note ed ottimizzate;
	\item Effettuare scelte su aspetti progettuali e tecnologici che rendano il prodotto facilmente manutenibile.
\end{itemize}
Redige la \textit{Specifica Tecnica}, la \textit{Definizione di Prodotto} e le sezioni relative alle metriche di verifica della programmazione del \textit{Piano di Qualifica}.

\subsection{Verificatore}
Il \textit{Verificatore} è il responsabile delle attività di verifica. Le mansioni che gli competono sono:
\begin{itemize}
	\item Garantire che l'attuazione delle attività sia conforme alle norme stabilite;
	\item Verificare che non siano stati introdotti errori lungo il percorso;
	\item Controllare la conformità di ogni stadio del ciclo di vita del prodotto.
\end{itemize}
Redige la sezione del \textit{Piano di Qualifica} che illustra l'esito e la completezza delle verifiche
e delle prove effettuate.

\subsection{Programmatore}
Il \textit{Programmatore} è responsabile delle attività di codifica e delle componenti di ausilio necessarie per l'esecuzione delle prove di verifica e validazione. Le mansioni che gli competono sono:
\begin{itemize}
	\item Implementare in maniera rigorosa le soluzioni descritte dal \textit{Progettista};
	\item Scrivere codice che sia documentato, versionato, manutenibile e che rispetti le metriche stabilite per la scrittura del codice;
	\item Implementare i test sul codice prodotto, necessari per le prove di verifica e validazione.
\end{itemize}
Redige il \textit{Manuale Utente}.

\subsection{Rotazione dei ruoli}
Ogni membro del gruppo dovrà ricoprire tutti i ruoli definiti nel \textit{Piano di Progetto v2.0.0}. Il \textit{Responsabile di Progetto} avrà il compito di pianificare l'impiego delle risorse in modo equo e in modo che ogni risorsa ricopra tutti i ruoli. 
Si deve controllare attentamente che non vi siano conflitti di interesse specialmente nelle attività di approvazione e verifica. Per garantire che la rotazione dei ruoli non provochi conflitti è necessario che le attività vengano pianificate con attenzione e che i membri interessati rispettino ruoli e compiti loro assegnati. Spetterà al \textit{Verificatore} controllare che tutte le condizioni sopra indicate vengano rispettate. Se il \textit{Verificatore} troverà delle incongruenze con quanto menzionato sopra avrà il compito di avvisare il \textit{Responsabile di Progetto} che dovrà risolvere la questione.\\
Ogni componente del gruppo potrà consultare, in qualsiasi momento, i diagrammi di Gantt che descrivono la gestione delle risorse e dei ruoli, in maniera tale che ognuno potrà sempre essere consapevole del ruolo ricoperto dagli altri componenti.

