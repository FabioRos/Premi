\subsection{Linguaggi di codifica}
Dopo un'analisi del capitolato d'appalto e dei requisiti si è deciso che per lo sviluppo del software richiesto si utilizzeranno i linguaggi HTML e Javascript.

\subsection{Convenzioni di codifica}
Di seguito è riportato l'insieme di norme e convenzioni che il gruppo dovrà seguire nella scrittura e documentazione del codice.
L'unica lingua ammessa per i nomi di variabili, classi, metodi e commenti è l'inglese.

\subsection{Nomenclatura}
Per l'assegnazione di nomi a variabili, metodi e costanti andranno seguite le seguenti regole:
\begin{itemize}
	\item Funzioni: va utilizzata la notazione mixed case, con la prima lettera minuscola;
	\item Variabili: va utilizzata la notazione mixed case, con la prima lettera minuscola;
	\item Costanti: va scritto il nome interamente in maiuscolo, separando le varie parole con il carattere "\_" (underscore).
\end{itemize}

\subsection{Intestazione di un file Javascript}

\begin{flushleft}

/*\\
\vspace{3mm}
\begin{tabular}{l}
	File\\
	Autore\\
	Data\\
	Descrizione\\
\end{tabular}\\
\vspace{5mm}
 Modifiche:\\
 \vspace{3mm}
\begin{tabular}{| c c c c c c c c c |}
	\hline
	Versione & - & Data & - & Programmatore & - & Modifica & - & Descrizione\\
	\hline
	x.y.z & - & aaaa-mm-gg & - & Nome Cognome & - & Funzione & - & Descrizione modifica\\
	\hline
\end{tabular}\\
\vspace{3mm}
*/\\

\end{flushleft}

\begin{itemize}
	\item File : nome del file;
	\item Autore : nome e cognome del creatore del file;
	\item Data : data di creazione del file nel formato aaaa-mm-gg;
	\item Descrizione : poche righe di descrizione delle funzionalità contenute nel file;
	\item Cambiamenti : tabella dello stato di avanzamento del file, contenente tutte le modifiche effettuate :
		\begin{itemize}
			\item Versione : versione una volta effettuata ala modifica;
			\item Data : data della modifica;
			\item Programmatore : nome e cognome del programmatore che ha effettuato la modifica;
			\item Modifica : segnatura della funzione a cui è stata apportata una modifica;
			\item Descrizione : breve descrizione della modifica effettuata.
		\end{itemize}
\end{itemize}

\subsection{Commenti}

Prima di ogni funzione dovrà essere presente un commento con la seguente forma:

\begin{flushleft}
/*\\
\vspace{3mm}
\begin{tabular}{l}
	Descrizione della funzione\\
	Descrizione dei parametri\\		
	Descrizione del tipo di ritorno\\
\end{tabular}\\
\vspace{3mm}
*/

\end{flushleft}

Ogni variabile di particolare importanza dovrà essere fornita di commento che ne spieghi il suo scopo e funzionamento.