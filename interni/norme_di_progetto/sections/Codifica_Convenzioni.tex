\subsubsection{Linguaggi di codifica}
Dopo un'analisi del capitolato d'appalto e dei requisiti si è deciso che per lo sviluppo del software richiesto si utilizzeranno i linguaggi HTML5 e Javascript.

\subsubsection{Convenzioni di codifica}
Di seguito è riportato l'insieme di norme e convenzioni che il gruppo dovrà seguire nella scrittura e documentazione del codice.
L'unica lingua ammessa per i nomi di variabili, metodi e commenti è l'italiano.

\subsubsection{File HTML}

Ogni file HTML deve iniziare con il tag <!DOCTYPE html> che serve ad indicare che verrà utilizzata la versione HTML5.
Ogni tag deve contenere un id e può contenere una o più classi.
Gli id e le classi dovranno essere contenute in un file .css a parte per mantenere il più possibile la separazione tra layout e contenuto.
Le pagine HTML devono rispettare gli standard del W3C.

\subsubsection{Nomenclatura}
Per l'assegnazione di nomi a variabili, metodi e costanti andranno seguite le seguenti regole:
\begin{itemize}
	\item \textbf{Funzioni:} va utilizzata la notazione mixed case, con la prima lettera minuscola;
	\item \textbf{Variabili:} va utilizzata la notazione mixed case, con la prima lettera minuscola;
	\item \textbf{Costanti:} va scritto il nome interamente in maiuscolo, separando le varie parole con il carattere "\_" (underscore).
\end{itemize}

\subsubsection{Intestazione di un file Javascript}

\begin{flushleft}

/*\\
\vspace{3mm}
\begin{tabular}{l}
	File\\
	Autore\\
	Data\\
	Descrizione\\
\end{tabular}\\
\vspace{5mm}
 Modifiche:\\
 \vspace{3mm}
\begin{tabular}{| c c c c c c c c c |}
	\hline
	Versione & - & Data & - & Programmatore & - & Modifica & - & Descrizione\\
	\hline
	x.y.z & - & aaaa-mm-gg & - & Nome Cognome & - & Funzione & - & Descrizione modifica\\
	\hline
\end{tabular}\\
\vspace{3mm}
*/\\

\end{flushleft}

\begin{itemize}
	\item \textbf{File:} nome del file;
	\item \textbf{Autore:} nome e cognome del creatore del file;
	\item \textbf{Data:} data di creazione del file nel formato aaaa-mm-gg;
	\item \textbf{Descrizione:} poche righe di descrizione delle funzionalità contenute nel file;
	\item \textbf{Cambiamenti:} tabella dello stato di avanzamento del file, contenente tutte le modifiche effettuate :
		\begin{itemize}
			\item \textbf{Versione:} versione una volta effettuata la modifica;
			\item \textbf{Data:} data della modifica;
			\item \textbf{Programmatore:} nome e cognome del programmatore che ha effettuato la modifica;
			\item \textbf{Modifica:} segnatura della funzione a cui è stata apportata una modifica;
			\item \textbf{Descrizione:} breve descrizione della modifica effettuata.
		\end{itemize}
\end{itemize}

\subsubsection{Commenti}

Prima di ogni funzione dovrà essere presente un commento con la seguente forma:

\begin{flushleft}
/*\\
\vspace{3mm}
\begin{tabular}{l}
	Descrizione della funzione\\
	Descrizione dei parametri\\		
	Descrizione del tipo di ritorno\\
\end{tabular}\\
\vspace{3mm}
*/

\end{flushleft}

Ogni variabile di particolare importanza dovrà essere fornita di commento che ne spieghi scopo e funzionamento.

