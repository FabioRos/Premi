\subsubsection{Linguaggi di codifica}
Dopo un'analisi del capitolato d'appalto e dei requisiti si è deciso che per lo sviluppo del software richiesto si utilizzeranno i linguaggi \gls{HTML5} , \gls{PHP} e \gls{Javascript}.

\subsubsection{Framework e librerie}
Per semplificare la realizzazione della nostra applicazione web si è deciso di utilizzare il \gls{framework} \gls{Angular}, per migliorare le interfacce utente, la libreria \gls{Chart.js}, utilizzata per generare grafici e \gls{Laravel}, per lo sviluppo della parte di back-end in \gls{PHP}.

\subsubsection{Software per l’integrazione continua}
Il termine \gls{integrazione continua} identifica una serie di operazioni da eseguire in modo automatico, utili per la compilazione del progetto. L'obiettivo è di verificare il codice in modo che sia sempre compilabile e utilizzabile da tutto il team. Nel caso siano presenti errori in ciò che è stato prodotto, la compilazione viene interrotta e viene notificato all'autore della parte di codice di correggere il problema.
A questo scopo, è stato scelto di utilizzare la piattaforma \textbf{Travis CI} (\url{https://travis-ci.org}), creando degli script che richiamano gli appositi tool per la verifica del codice.

\subsubsection{Convenzioni di codifica}
Di seguito è riportato l'insieme di norme e convenzioni che il gruppo dovrà seguire nella scrittura e documentazione del codice.
L'unica lingua ammessa per i nomi di variabili, metodi e commenti è l'inglese.

\subsubsection{File HTML}

Ogni file \gls{HTML} deve iniziare con il tag <!DOCTYPE html> che serve ad indicare che verrà utilizzata la versione \gls{HTML5}.
Ogni tag deve contenere un id e può contenere una o più classi.
Gli id e le classi dovranno essere contenute in un file .css a parte per mantenere il più possibile la separazione tra \gls{layout} e contenuto.
Le pagine \gls{HTML} devono rispettare gli standard del \gls{W3C}.

\subsubsection{Nomenclatura}
Per l'assegnazione di nomi a variabili, metodi e costanti andranno seguite le seguenti regole:
\begin{itemize}
	\item \textbf{Funzioni:} va utilizzata la notazione mixed case, con la prima lettera minuscola;
	\item \textbf{Variabili:} va utilizzata la notazione mixed case, con la prima lettera minuscola;
	\item \textbf{Costanti:} va scritto il nome interamente in maiuscolo, separando le varie parole con il carattere "\_" (underscore).
\end{itemize}

\newpage
\subsubsection{Intestazione di un file}

\begin{flushleft}

/*\\
\vspace{3mm}
\begin{tabular}{l}
	*file\\
	*author\\
	*date\\
	*description\\
\end{tabular}\\
\vspace{5mm}
 *Changes:\\
 \vspace{3mm}
\begin{tabular}{| c c c c c c c c c |}
	\hline
	Version & - & Date & - & Programmer & - & Change & - & Description\\
	\hline
	x.y.z & - & aaaa-mm-gg & - & Nome Cognome & - & Funzione/classe & - & Descrizione\\
	\hline
\end{tabular}\\
\vspace{3mm}
*/\\

\end{flushleft}

\begin{itemize}
	\item \textbf{File:} nome del file;
	\item \textbf{Author:} creatore del file;
	\item \textbf{Date:} data di creazione del file nel formato aaaa-mm-gg;
	\item \textbf{Description:} poche righe di descrizione delle funzionalità contenute nel file;
	\item \textbf{Changes:} tabella dello stato di avanzamento del file, contenente tutte le modifiche effettuate :
		\begin{itemize}
			\item \textbf{Version:} versione una volta effettuata la modifica;
			\item \textbf{Date:} data della modifica;
			\item \textbf{Programmer:} nome e cognome del programmatore che ha effettuato la modifica;
			\item \textbf{Change:} segnatura della classe/metodo a cui è stata apportata una modifica;
			\item \textbf{Description:} breve descrizione della modifica effettuata.
		\end{itemize}
\end{itemize}

\subsubsection{Commenti}

Prima di ogni funzione dovrà essere presente un commento con la seguente forma:

\begin{flushleft}
	/**\\
	 *@Descrizione dettagliata del metodo\\
	 *@param Descrizione del primo parametro\\
	 *@param Descrizione del n-esimo parametro\\
	 *@return Il tipo del valore di ritorno\\
	*/
\end{flushleft}

Ogni variabile di particolare importanza, o il quale utilizzo è particolarmente complesso, dovrà essere accompagnata da una breve descrizione del suo scopo.

\subsubsection{Esecuzione dei test di unità}
Questa attività consiste nell'esecuzione dei test di unità, in modo da assicurare che soddisfino i requisiti definiti. I risultati devono essere documentati.

