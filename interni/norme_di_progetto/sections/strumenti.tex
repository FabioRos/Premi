In questa sezione verrà illustrato l'ambiente di lavoro che sarà utilizzato durante lo sviluppo del progetto \PROGETTO.\\

\paragraph{Sistema operativo}

Il sistema operativo utilizzato è lasciato a discrezione di ogni membro del gruppo. Questa scelta è dovuta principalmente al fatto che il progetto dovrà supportare più piattaforme.
I membri del gruppo utilizzeranno i seguenti sistemi operativi:

\begin{itemize}
	\item \gls{Windows} 7 64 bit;
	\item \gls{Windows} 8.1;
	\item \gls{Ubuntu} 14.10;
	\item Mac OS 10.10.2.
\end{itemize}

\paragraph{Coordinamento}

Il coordinamento del gruppo avviene tramite:
\begin{itemize}
	\item \gls{Google Drive};
	\item \gls{Google Calendar};	
	\item \gls{Repository} \gls{Git}.
\end{itemize}

\subparagraph{Google Drive:}

Abbiamo scelto di utilizzare questo \gls{servizio cloud} per condividere tutti i documenti che non necessitano di \gls{versionamento} e vengono utilizzati frequentemente da parte dei membri del gruppo. 
È un servizio molto semplice e accessibile direttamente da \gls{browser} che permette di lavorare su documenti creati con \gls{Google Docs}.

\subparagraph{Google Calendar:}
\gls{Google Calendar} viene utilizzato per la gestione delle risorse umane e per tenere traccia degli eventi importanti. È stato creato un calendario condiviso con tutti i membri del gruppo in modo da conoscere le date in cui una persona è assente o non reperibile e le date rilevanti per il gruppo.\\
Inoltre è possibile modificare la gestione delle notifiche per un evento o più eventi in modo che venga inviata automaticamente una email a tutti i membri del gruppo. In questo caso il preavviso deve essere di almeno un giorno.

\subparagraph{Repository Git:}

Nonostante siano disponibili molti \gls{repository} (\gls{Git}, \gls{Mercurial}, \gls{SVN}) è stato scelto di utilizzare \gls{Git} in quanto il servizio soddisfa pienamente le necessità di \gls{hosting} e \gls{versionamento} necessarie per lo sviluppo di questo progetto, inoltre diversi membri del gruppo avevano già usato tale servizio. Esso permette di lavorare senza una connessione attiva a internet e dà la possibilità di ignorare alcune estensioni specificate in un file chiamato .gitignore\footnote{File globale di \gls{Git} che contiene una lista di regole per ignorare i file.}.\\ Per i membri del gruppo che desiderano utilizzare un client, si consigliano i seguenti:
\begin{itemize}
	\item \textbf{SourceTree};
	\item \textbf{\gls{GitHub}};
\end{itemize}
Per maggiori dettagli sulle norme che riguardano l'utilizzo si rimanda alla sezione \ref{repository}.

\newpage
\paragraph{Ambiente documentale}

\subparagraph{Stesura documenti:}

Per la stesura dei documenti verrà utilizzato il \gls{linguaggio di markup} \LaTeX.
Come editor è consigliato TeXstudio\footnote{\url{http://texstudio.sourceforge.net}}, il quale è disponibile per tutti i principali sistemi operativi.

\subparagraph{Script:}

Per facilitare la stesura dei documenti sono stati creati alcuni script:

\begin{itemize}
	\item Generazione di tutti i documenti PDF: con il comando "make all" verranno generati tutti i PDF dei documenti contenuti nella directory corrente;
	%\item Controllo ortografico: con il comando make aspell verrà invocato il programma aspell su tutti i documenti della directory corrente;
	%\item Eliminazione file errati o vecchi: con il comando make clean verranno eliminatati i file generati da compilazioni vecchio o file non necessari della directory corrente;
	\item Evidenziare glossario: con il comando "java glossary" verrà eseguito uno script che selezionerà nei documenti della directory corrente le parole contenute nella versione più recente del glossario e le evidenzierà con il simbolo "\G".

\end{itemize}

\paragraph{Pianificazione delle attività}
Per pianificare le attività e la gestione del progetto e delle risorse umane si è scelto di usare \gls{GanttProject}\footnote{\url{http://www.ganttproject.biz}}.

\paragraph{Diagrammi UML}

Per la realizzazione dei diagrammi \gls{UML} è stato scelto \gls{Astah} Professional Edition\footnote{\url{http://astah.net/download}}.
È possibile ricevere gratuitamente la licenza della versione professional inviando una richiesta al sito \url{http://astah.net/student-license-request}.
