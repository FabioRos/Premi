La responsabilità di gestione del progetto è attribuita al \textit{Responsabile di Progetto}. \\
Utilizzando tutti gli strumenti a disposizione il \textit{Responsabile di Progetto} avrà il compito di:
\begin{itemize}
	\item Pianificare le attività;
	\item Gestire le risorse;
	\item Analizzare e prevenire i rischi.
\end{itemize}

\subsubsection{Pianificazione delle attività}
Per pianificare le varie attività da svolgere il \textit{Responsabile di Progetto} dovrà utilizzare \gls{GanttProject} per realizzare i diagrammi di \gls{Gantt}.

\subsubsection{Coordinazione e gestione delle risorse}
Per gestire le risorse durante lo sviluppo del progetto, il \textit{Responsabile di Progetto} dovrà pianificare la quantità di ore che ogni risorsa dovrà dedicare a ciascuna attività.\\
Si è deciso di utilizzare il servizio di gestione dei \gls{task} e creazione di \gls{milestone} offerto da \gls{GitHub} spiegato nella sezione \ref{repository}. Questo permette di accentrare le informazioni in un solo ambiente.

\subsubsection{Analisi e prevenzione dei rischi}
Durante l'avanzamento del progetto, il \textit{Responsabile di Progetto} deve monitorare costantemente il verificarsi dei rischi descritti nel \textit{Piano di Progetto v2.0.0}. La valutazione dei rischi deve essere aggiornata al mutare delle situazioni di pericolo, in occasione di significative modifiche al processo produttivo oppure all'introduzione di nuove tecnologie. Una volta completata la valutazione dei rischi si dovrà redigere una relazione scritta, modificando e aggiornando il \textit{Piano di Progetto}, contenente le misure di prevenzione previste e il loro programma di attuazione.
