In questo capitolo si descrivono i vari standard adottati da \GRUPPO nella stesura, verifica e approvazione della documentazione da produrre.
\subsection{Template}
Per semplificare la redazione dei documenti è stato creato un template \LaTeX contenente tutte le impostazioni per l'aspetto grafico. \\
Ogni documento dovrà essere realizzato con il template \LaTeX presente nel \gls{repository}. Inoltre è stata scritta una piccola guida su come usare al meglio il template.
\subsection{Versionamento}
Per ogni documento è obbligatorio specificare la versione. Un numero di versione deve essere nella seguente forma:
\begin{center}
	\emph{X.Y.Z}
\end{center}
dove X, Y, Z sono interi non negativi e non devono contenere zeri iniziali. X è la versione \textit{Major}, Y è la versione \textit{Minor}, e Z e la versione \textit{Patch}.
\begin{itemize}
	\item X: la versione Major (X.y.z | X > 0) identifica la versione di rilascio. Deve essere incrementata se è stata introdotta qualsiasi modifica non retrocompatibile. Le versioni Patch e Minor devono essere reimpostate a 0 quando la versione Major è incrementata. 
	\item Y: la versione Minor (x.Y.z | x > 0) deve essere incrementata se è stata introdotta una nuova funzionalità. La versione Patch deve essere reimpostata a 0 quando la versione Minor è incrementata.
	\item Z: la versione Patch (x.y.Z | x > 0) deve essere incrementata solo se sono state introdotte correzioni retrocompatibili di bug. Una correzione di un bug è definita come una modifica interna che corregge un comportamento errato.
\end{itemize}
Una volta che un pacchetto versionato è stato rilasciato, i contenuti di quella versione \textbf{non devono} essere modificati. Qualsiasi modifica \textbf{deve} essere rilasciata come una nuova versione.

\subsection{Struttura dei documenti}
\subsubsection{Prima pagina}
Ogni documento deve avere in prima pagina le seguenti informazioni:
\begin{itemize}
	\item Nome del progetto;
	\item Logo del gruppo;
	\item Nome del gruppo;
	\item Nome del documento;
	\item La versione del documento;
	\item Cognome e nome dei redattori del documento;
	\item Cognome e nome dei verificatori del documento;
	\item Cognome e nome di chi ha approvato il documento;
	\item Tipo d'uso del documento;
	\item Lista di distribuzione del documento;
	\item Breve descrizione del documento.
\end{itemize}
\subsubsection{Diario delle modifiche}
Nella seconda pagina di ogni documento deve essere presente il diario delle modifiche.\\
Ogni riga del diario delle modifiche contiene:
\begin{itemize}
	\item Una breve descrizione sulle modifiche effettuate;
	\item Cognome e nome di chi ha effettuato la modifica;
	\item Data della modifica;
	\item Versione del documento dopo la modifica.
\end{itemize}

\subsubsection{Indice}
In ogni documento deve essere presente un indice delle sezioni.\\
Nel caso in cui il documento contenga immagini e/o tabelle devono essere presenti anche i relativi indici.

\subsubsection{Formattazione delle pagine}
L'intestazione di ogni pagina contiene:
\begin{itemize}
	\item Logo del gruppo;
	\item La sezione corrente all'interno del documento.
\end{itemize}
A piè di ogni pagina è presente:
\begin{itemize}
	\item Nome e versione del documento;
	\item Pagina corrente nel formato N di T, dove N è il numero della pagina corrente e T è il numero di pagine totali del documento.
\end{itemize}

\subsection{Classificazione documenti}
\subsubsection{Documenti informali}
Si definiscono documenti informali tutti i documenti in fase di sviluppo e devono ancora essere approvati dal \textit{Responsabile di Progetto}. I documenti sono da considerarsi esclusivamente per uso interno e non potranno essere divulgati prima di essere stati verificati ed approvati. 
\subsubsection{Documenti formali}
Si definiscono documenti formali tutti i documenti che sono stati approvati dal \textit{Responsabile di Progetto} e quindi sono pronti ad essere diffusi a terze parti.\\
I documenti pronti per il rilascio dovranno essere rinominati osservando le seguenti regole:
\begin{itemize}
	\item La prima lettera di ogni parola, che non sia una preposizione, deve essere maiuscola;
	\item Gli spazi devono essere sostituiti con il carattere "\_"(underscore);
	\item Tutte le parole devono essere prive di accenti;
	\item La versione del documento verrà aggiunta dopo il nome, e sarà preceduta dal carattere "-"(trattino) e da una "v".
\end{itemize}

\subsection{Norme tipografiche}
Questa sezione racchiude tutte le informazioni riguardanti l'ortografia, la tipografia e l'assunzione di uno stile uniforme in tutti i documenti allo scopo di evitare incoerenze tra le diverse parti del documento.

\subsubsection{Punteggiatura}
\begin{itemize}
	\item Punteggiatura: qualsiasi carattere di punteggiatura non può seguire un carattere di spazio;
	\item Lettere maiuscole: l'uso delle maiuscole è obbligatorio in una serie di casi:
	\begin{itemize}
		\item All'inizio di testo o di una sua parte (capitolo, paragrafo, ecc.);
		\item Dopo il punto, il punto esclamativo, il punto interrogativo;
		\item All'inizio di ogni elemento di un elenco puntato;
		\item Per i ruoli, di progetto, i nomi dei documenti, le fasi di progetto, revisioni di progetto oltre che dove previsto dalla lingua italiana.
	\end{itemize}
	\item Parentesi: il testo racchiuso dentro le parentesi non deve mai iniziare per un carattere di spazio e non deve mai terminare con un carattere di spazio o punteggiatura.
\end{itemize}	

\subsubsection{Stile del testo}
\begin{itemize}
	\item Maiuscolo: l'uso delle maiuscole è limitato alla trascrizione degli acronimi;
	\item Grassetto: l'uso del grassetto deve essere utilizzato nei seguenti casi: 
	\begin{itemize}
		\item Elenchi puntati: per evidenziare l'oggetto trattato;
		\item Altri casi: è possibile utilizzare il grassetto per evidenziare parole chiave.
	\end{itemize}
	\item Corsivo: l'uso del corsivo deve essere utilizzato nei seguenti casi:
	\begin{itemize}
		\item Citazioni: ogni citazione deve essere scritta in corsivo; 
		\item Documenti: ogni riferimento ad un documento deve essere scritto in corsivo (esempio: \textit{Glossario});
		\item Ruoli: ogni riferimento a figure particolari deve essere scritto in corsivo (esempio: \textit{Verificatore});
		\item Altri casi: è possibile utilizzare il corsivo per evidenziare parole particolarmente significative.
	\end{itemize}
	\item \LaTeX: ogni riferimento a \LaTeX va scritto tramite il comando \verb|\LaTex|.
\end{itemize}

\subsubsection{Composizione del testo}
\begin{itemize}
	\item Elenchi puntati: per redigere un elenco si qualificano gli elementi e si introducono mediante
	i due punti. Dopo i due punti si deve andare a capo. Ogni elemento dell'elenco termina con un punto e virgola, eccetto l'ultimo che termina con un punto semplice;
	\item Pedice "G": il pedice "G" viene utilizzato in presenza di termini presenti nel \textit{Glossario};
	\item Note a piè di pagina: ogni nota deve iniziare con la lettera della prima parola maiuscola e non deve essere preceduta da alcun carattere di spazio. Ogni nota deve terminare con un punto.
\end{itemize}

\subsubsection{Formati ricorrenti}
\begin{itemize}
	\item Nomi propri: l'utilizzo dei nomi propri deve seguire la seguente forma "Cognome Nome";
	\item Percorsi: per gli indirizzi email e gli indirizzi web deve essere usato esclusivamente il comando \LaTeX  \verb|\url|;
	\item Date:  ogni data deve seguire lo standard internazionale per date ed orari ISO 8601:
	\begin{center}
		AAAA.MM.GG
	\end{center}
	dove:
	\begin{itemize}
		\item AAAA: rappresenta il formato dell'anno scritto con quattro cifre;
		\item MM: rappresenta il formato del mese scritto con due cifre;
		\item GG: rappresenta il formato del giorno scritto con due cifre.
	\end{itemize}
	\item Riferimenti ai documenti: ci si riferirà ai vari documenti scrivendo in corsivo il nome del documento e mettendo una lettera maiuscolo per ogni parola che non sia un articolo. Nel caso in cui ci si deve riferire ad una versione specifica del documento, essa andrà indicata alla fine  del nome del documento (esempio: \textit{Norme di Progetto v1.0.0});
	\item Nome del gruppo: ci si riferirà al gruppo solo come "\textit{DazzleWorks}" con il nome del gruppo in corsivo. Per una scrittura corretta del nome è stato creato il comando \LaTeX apposito \verb|\GRUPPO|;
	\item Nome del progetto: ci si riferirà al nome del progetto solo come "\textbf{PREMI}". Per una scrittura corretta del nome è stato creato il comando \LaTeX \verb|\PROGETTO|.
\end{itemize}

\subsubsection{Sigle}
Le sigle potranno essere utilizzate esclusivamente all'interno delle tabelle e diagrammi. Sono presenti le seguente sigle: 
\begin{itemize}
	\item Documenti:
	\begin{itemize}
		\item NdP = \textit{Norme di Progetto};
		\item AdR = \textit{Analisi dei Requisiti};
		\item PdP = \textit{Piano di Progetto};
		\item PdQ = \textit{Piano di Qualifica};
		\item SdF = \textit{Studio di Fattibilità}.
	\end{itemize}
	\item Revisioni:
	\begin{itemize}
		\item RR = \textit{Revisione dei Requisiti};
		\item RP = \textit{Revisione di Progettazione};
		\item RQ = \textit{Revisione di Qualifica};
		\item RA = \textit{Revisione di Accettazione}.
	\end{itemize}
\end{itemize}

\subsection{Componenti grafiche}
\subsubsection{Immagini}
Tutte le immagini dovranno avere il formato \gls{PDF}. In modo da garantire una maggiore qualità dell'immagine in caso di ridimensionamento è consigliato usare il formato \gls{PDF} vettoriale. La conversione di un immagine in formato \gls{PDF} è possibile attraverso il software GIMP\footnote{GNU Image Manipulation Program è un programma per la creazione e modifica di immagini digitali.}, il quale è stato già usato da tutti  i componenti del gruppo. \\ Le immagini devono essere accompagnate da una didascalia che inizia con la parola "Figura" con la prima lettera maiuscola, seguita dal numero della figura, dal carattere "due punti" e da una breve descrizione della figura.
\subsubsection{Tabelle}
Ogni tabella deve essere accompagnata da una didascalia che inizia con la parola "Tabella" con la prima lettera maiuscola, seguita dal numero della tabella, dal carattere "due punti" e da una breve descrizione non banale della tabella.

\subsection{Procedure di avanzamento di un documento}
Quando il redattore ritiene di aver finito la stesura di un documento si devono seguire questi passi esattamente in quest'ordine:
\begin{enumerate}
	\item Il redattore ha il compito di contattare il \textit{Responsabile di Progetto} per informarlo della terminazione della stesura del documento;
	\item Il \textit{Responsabile di Progetto} provvederà a contattare ed assegnare la verifica del documento ad un verificatore disponibile, che non sia in conflitto di interessi;
	\item Il \textit{Verificatore} provvederà alla  verifica del documento;
	\item Nel caso in cui vengano trovati errori nel documento, il \textit{Verificatore} provvederà alla creazione di un ticket di correzione e lo assegnerà al redattore, il quale correggerà gli errori trovati e infine si tornerà di nuovo al passo uno;
	\item Se non sono stati trovati errori, il documento verrà approvato dal \textit{Responsabile di Progetto}. 
\end{enumerate}
