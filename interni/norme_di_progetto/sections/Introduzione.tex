\subsection{Scopo del Documento}
Questo documento ha lo scopo di definire le norme che tutti i membri del gruppo \GRUPPO\ dovranno seguire e rispettare durante lo svolgimento del progetto \PROGETTO. Tutti i membri sono tenuti a leggere il documento seguendo le norme per garantire uniformità al fine di ottenere una migliore efficacia ed efficienza delle attività.
I membri del gruppo potranno contattare l'\textit{Amministratore di Progetto} per eventuali suggerimenti, consigli, opinioni riguardanti le norme di progetto. 
L'\textit{Amministratore di Progetto}, dopo essersi consultato con gli altri membri del gruppo, avrà la responsabilità di accettare o rifiutare i suggerimenti proposti. 
Il documento, in particolare, pone l'accento sui seguenti punti:

\begin{itemize}
	\item Interazione tra i membri del gruppo e la comunicazione verso l'esterno;
	\item Le modalità di accesso e le norme di utilizzo del \gls{repository};
	\item Stesura documenti e convenzioni di scrittura utilizzate;
	\item Modalità di lavoro durante lo sviluppo del progetto;
	\item Organizzazione dell'ambiente di lavoro.
\end{itemize}

\subsection{Scopo del Progetto}
Lo scopo del progetto è la realizzazione di un software di presentazione di \textit{\gls{slide}}, sviluppato in tecnologia \gls{HTML5} che funzioni sia in ambiente desktop che mobile.
Si richiede agli sviluppatori di realizzare effetti grafici a supporto dello storytelling\footnote{Strumento riflessivo per la costruzione di significati interpretativi della realtà.} che siano di livello comparabile con \textit{Prezi}\footnote{È un servizio per la realizzazione di presentazioni.}.
Il software dovrà coprire i due momenti fondamentali per questo tipo di attività: la creazione da parte dell'autore e la presentazione al pubblico, sia in presenza diretta che via web.

\subsection{Glossario}
Per prevenire ed evitare qualsiasi dubbio e per permettere una maggiore chiarezza e comprensione del testo su termini ambigui, abbreviazioni e acronimi utilizzati nei vari documenti, essi sono stati raccolti nel \textit{Glossario v4.0.0} nel quale si possono trovare tutte le informazioni desiderate.
Al fine di rendere subito evidente un termine presente nel \textit{Glossario}, esso verrà marcato con il pedice \G.

\subsection{Riferimenti}

\subsubsection{Normativi}

\begin{itemize}
	\item \url{http://www.w3.org/standards/webdesign/htmlcss};
	\item \url{http://en.wikipedia.org/wiki/ISO_8601}; 	
	\item \url{http://www.pmi.it/impresa/normativa/news/51788/dipendenti-al-pc-obblighi-di\\-legge-per-i-datori-di-lavoro.html}.
\end{itemize}


