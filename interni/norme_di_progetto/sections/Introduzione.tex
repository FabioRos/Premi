\subsection{Scopo del Documento}
Questo documento ha lo scopo di definire le norme che tutti i membri del gruppo \GRUPPO\ dovranno seguire e rispettare durante lo svolgimento del progetto \PROGETTO. Tutti i membri sono tenuti a leggere il documento e a seguire le norme per garantire uniformità e una migliore efficacia ed efficienza delle attività.
I membri del gruppo potranno contattare l'\textit{Amministratore di Progetto} per eventuali suggerimenti, consigli, opinioni riguardanti le norme di progetto. 
L'\textit{Amministratore di Progetto}, dopo essersi consultato con gli altri membri del gruppo, avrà la responsabilità di accettare o rifiutare i suggerimento proposti. 
Il documento, in particolare, pone l'accento sui seguenti punti:

\begin{itemize}
	\item Iterazione tra i membri del gruppo e la comunicazione verso l'esterno;
	\item Le modalità di accesso e le norme di utilizzo del \gls{repository};
	\item Stesura documenti e convenzioni di scrittura utilizzate;
	\item Modalità di lavoro durante lo sviluppo del progetto;
	\item Organizzazione dell'ambiente di lavoro.
\end{itemize}

\subsection{Scopo del Progetto}
Lo scopo del progetto è la realizzazione di un software di presentazione di "slide", sviluppato in tecnologia \gls{HTML5} e che funzioni sia su desktop che su dispositivo mobile.
Si richiede agli sviluppatori di realizzare effetti grafici a supporto dello “storytelling”\footnote{Strumento riflessivo per la costruzione di significati interpretativi della realtà.} che siano di livello comparabile con \textit{Prezi}\footnote{È un servizio per la realizzazione di presentazioni.}.
Il software dovrà coprire i due momenti fondamentali per questo tipo di attività: la creazione da parte dell'autore e la presentazione al pubblico, sia in presenza diretta che via web.

\subsection{Glossario}
Al fine di evitare ogni ambiguità e permettere al lettore una migliore comprensione dei termini tecnici, acronimi, e termini che necessitano di essere chiariti,  sono riportate nel documento \textit{Glossario}. 
Ogni occorrenza  di un termine appartenente al \textit{Glossario} è marcata da una "G" in pedice.

\subsection{Riferimenti}

\subsubsection{Normativi}

\begin{itemize}
	\item http://www.w3.org/standards/webdesign/htmlcss
\end{itemize}
