Dopo la fase di \textbf{Analisi} si passerà alla fase di \textbf{Progettazione} dove i \textit{Progettisti} dovranno seguire le seguenti regole (organizzate per macro-aree).\\

\subsubsection{Diagrammi}

Dovrà essere utilizzato il linguaggio UML per la realizzazione dei seguenti diagrammi:

\begin{itemize}
	\item Diagrammi delle classi;
	\item Diagrammi dei package;
	\item Diagrammi di attività;
	\item Diagrammi di sequenza.
\end{itemize}

\subsubsection{Design Pattern}

I \textit{Progettisti} dovranno utilizzare il design pattern che ritengono più adatto al contesto per rendere l'applicazione più efficiente possibile.

\subsubsection{Classi di verifica}

Andranno create delle classi di verifica per testare che tutti i componenti abbiano un comportamento corretto.

\subsubsection{Stile di progettazione}

Durante la fase di \textbf{Progettazione} bisognerà fare attenzione a:

\begin{itemize}
\item \textbf{Ricorsione}: non dovrà essere utilizzata la ricorsione a meno che non sia strettamente necessaria. In quel caso dovrà essere fornita una dimostrazione induttiva sulla correttezza del metodo in questione;
\item \textbf{Annidamento di cicli}: all'interno di un metodo non dovranno esserci cicli annidati con una profondità maggiore a cinque.
\end{itemize}