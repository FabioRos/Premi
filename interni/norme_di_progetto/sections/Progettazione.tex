Dopo la fase di \textbf{Analisi} si passerà alla fase di \textbf{Progettazione} dove i \textit{Progettisti} dovranno seguire le seguenti regole (organizzate per macro-aree).\\
\subsubsection{Specifica Tecnica}
I \textit{Progettisti} dovranno descrivere la progettazione ad alto livello dell'architettura del software e dei singoli componenti nella \textit{Specifica Tecnica}.

\paragraph{Diagrammi UML}
Dovranno essere realizzati i seguenti diagrammi:
\begin{itemize}
	\item Diagrammi delle classi;
	\item Diagrammi dei package;
	\item Diagrammi delle attività;
	\item Diagrammi di sequenza.
\end{itemize}
La lingua utilizzata nella realizzazione dei diagrammi sarà l'\textbf{inglese} e lo standard \gls{UML} sarà 2.0.

\paragraph{Design Pattern}
I \textit{Progettisti} dovranno utilizzare il \gls{design pattern} che ritengono più adatto al contesto per rendere l'applicazione più efficiente possibile. Ogni \gls{design pattern} utilizzato verrà accompagnato da una breve descrizione e da un diagramma che ne esemplifica il funzionamento.

\paragraph{Classi di verifica}
Andranno create delle classi di verifica per testare che tutti i componenti abbiano un comportamento corretto.

\paragraph{Stile di progettazione}
Durante la fase di \textbf{Progettazione} bisognerà fare attenzione a:
\begin{itemize}
	\item \textbf{Ricorsione}: non dovrà essere utilizzata la ricorsione a meno che non sia strettamente necessaria. In quel caso dovrà essere fornita una dimostrazione induttiva sulla correttezza del metodo in questione;
	\item \textbf{Annidamento di cicli}: all'interno di un metodo non dovranno esserci cicli annidati con una profondità maggiore a cinque.
\end{itemize}




