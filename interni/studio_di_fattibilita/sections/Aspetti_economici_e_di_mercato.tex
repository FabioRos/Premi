L’intero progetto verrà reso disponibile sulla piattaforma GitHub (\url{https://github.com})  per soddisfare i requisiti che impongono la natura open-source del progetto software.
La scelta è ricaduta su questo servizio piuttosto che su sourceforge poiché maggiormente conosciuta da cinque membri del gruppo quindi riducendo lo sforzo di apprendimento.
Capitolato C4:\url{http://www.math.unipd.it/~tullio/IS-1/2014/Progetto/C4.pdf}, pag.7
	 “Costituirà titolo preferenziale nella valutazione delle proposte la pubblicazione del progetto sul sito “www.sourceforge.net” o altri repository pubblici, in conformità con i relativi requisiti di natura open-source, per favorire la continuità del prodotto risultante.”

Ad oggi, il mercato offre molti prodotti di questo tipo però tutte su assomigliano tra loro e presentano la stessa struttura. Quello che ci viene richiesto è di rompere la sequenzialità delle presentazioni attuali per realizzare una nuova modalità d’uso.
Capitolato C4:\url{http://www.math.unipd.it/~tullio/IS-1/2014/Progetto/C4.pdf}, pag.3
“Il progetto vuole essere fortemente sperimentale, indagando su nuove possibilità offerte dai sistemi moderni con tecnologie web sia nel campo degli effetti durante le presentazioni che sullo svolgimento non lineare delle stesse. Pur avendo scardinato la sequenza delle slide, Prezi non ha superato la linearità della presentazione.”
