L’intero progetto verrà reso disponibile sulla piattaforma GitHub per soddisfare i requisiti che impongono la natura open-source del progetto software.
La scelta è ricaduta su questo servizio piuttosto che su Sourceforge\footnote{\url{http://sourceforge.net}} poiché maggiormente conosciuto dai membri del gruppo riducendone lo sforzo di apprendimento.
Ad oggi, il mercato offre molti prodotti di questo tipo però tutti si assomigliano tra loro e presentano la stessa struttura. Quello che ci viene richiesto è di rompere la sequenzialità delle presentazioni attuali per realizzare una nuova modalità d’uso.