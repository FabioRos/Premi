\url{http://www.math.unipd.it/~tullio/IS-1/2014/Progetto/C3.pdf} \\*
Questo capitolato richiede di sviluppare un framework per esperti di dominio basato su Node.js, Express.js e socket.io che permetta di generare grafici provenienti da fonti generiche. \\*
Questo framework deve mettere a disposizione delle funzioni di aggiornamento dei dati sfruttando la tecnologia WebSocket.
\subsubsection{Rischi potenziali} 
Il gruppo teme di non riuscire a padroneggiare sufficientemente bene le nuove tecnologie con le quali il non è mai venuto a contatto rimanendo nei termini di tempo imposti.
\subsubsection{Aspetti positivi}
Il progetto è stato ritenuto interessantissimo dal gruppo ed a contenuto altamente formativo. \\*
Le tecnologie adottate, sebbene nuove e sconosciute a tutti i membri del gruppo, sono state ritenute altamente spendibili ed interessanti.
\subsubsection{Aspetti negativi}
Le tecnologie completamente nuove,seppur interessanti, sono tante ed alcune di queste hanno una curva di apprendimento che cresce di più lungo l’asse delle ascisse rispetto a quello delle ordinate.
\subsubsection{Valutazione del capitolato}
Al momento della scelta il capitolato non era disponibile.\\* C’è da dire che il gruppo, alla luce delle tempistiche ridotte, non lo avrebbe comunque scelto poiché il fattore tempo è stato ritenuto da subito una componente molto rilevante.
