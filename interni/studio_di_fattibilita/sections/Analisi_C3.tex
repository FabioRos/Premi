Questo capitolato richiede di sviluppare un \gls{framework} per esperti di dominio basato su Node.js\footnote{\url{https://nodejs.org}}, Express.js\footnote{\url{http://expressjs.com}} e socket.io\footnote{\url{http://socket.io}} che permetta di generare grafici a partire da dati provenienti da fonti generiche. \\
Questo \gls{framework} deve mettere a disposizione delle funzioni di aggiornamento dei dati sfruttando la tecnologia WebSocket\footnote{\url{http://it.wikipedia.org/wiki/WebSocket}}.
\subsubsection{Rischi potenziali} 
Il gruppo teme di non riuscire a padroneggiare sufficientemente bene le nuove tecnologie con le quali non è mai venuto a contatto rimanendo nei termini di tempo imposti.
\subsubsection{Aspetti positivi}
Il progetto è stato ritenuto interessantissimo dal gruppo ed a contenuto altamente formativo. \\
Le tecnologie adottate, sebbene nuove e sconosciute a tutti i membri del gruppo, sono state ritenute altamente spendibili ed interessanti.
\subsubsection{Aspetti negativi}
Le tecnologie completamente nuove, seppur interessanti, sono tante ed alcune di queste hanno una curva di apprendimento molto blanda, di conseguenza richiedono molto tempo per essere padroneggiate completamente.
\subsubsection{Valutazione del capitolato}
Al momento della scelta il capitolato non era disponibile.\\ C’è da dire che il gruppo, alla luce delle tempistiche ridotte, non lo avrebbe comunque scelto poiché il fattore tempo è stato ritenuto da subito una componente molto rilevante.

