L'obiettivo di questo capitolato è creare da zero una infrastruttura che permetta di interrogare big data dai social, \gls{Facebook}, Twitter e Instagram.
E’ proposto l'utilizzo della “Google Cloud Platform” e dei servizi web forniti dal pacchetto “Google Endpoints”. 

\subsubsection{Rischi potenziali}
I requisiti sono stati ritenuti troppo generici  rendendo difficile effettuare una buona analisi degli stessi anche in vista di possibili nuove istruzioni che potrebbero provocare un pivoting\footnote{Con il termine \textit{"Pivoting"} è inteso un cambio di strategia rispetto a quella precedentemente programmata, solitamente a seguito di un avvenimento oppure un risultato di mercato non atteso che richiede una strategia diversa da quella seguita fino ad allora.} in corso d'opera.

\subsubsection{Aspetti positivi}
Le tecnologie coinvolte in questo progetto hanno suscitato particolare interesse in tutti i membri del gruppo. In particolar modo l'utilizzo di una infrastruttura cloud completa quale la “Google Cloud Platform” è stata valutata dal team altamente formativa. Dal punto di vista del gruppo, i big data sono considerati un argomento altamente formativo e rivestiranno sicuramente una grande importanza anche nel futuro.

\subsubsection{Aspetti negativi}
Il gruppo ha ritenuto l'argomento cardine del progetto troppo vasto per essere studiato abbastanza approfonditamente da permettere una progettazione ottimale. Il poco tempo disponibile ed il modo di operare molto diverso da quello già conosciuto rende necessario uno sforzo che, sebbene
sarebbe pienamente giustificato, potrebbe non dare la certezza di fare un buon lavoro.\\
Come citato nell'analisi dei rischi, inoltre, la non specificità delle richieste  ha suscitato nei membri non poche preoccupazioni.


\subsubsection{Valutazione del capitolato} 

A seguito di questa analisi non avremmo scelto questo capitolato. La scelta, tuttavia, non sarebbe comunque stata possibile in quanto la capacità di assorbimento dei gruppi per il capitolato era già completo al momento della stesura di questo documento.

