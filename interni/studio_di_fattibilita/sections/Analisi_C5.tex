L'obiettivo di questo capitolato è la realizzazione di una piattaforma per lo sport e le attività ricreative in montagna utilizzando uno SmartWatch di costruzione del Proponente. Tale SmartWatch può collegarsi ad internet attraverso uno smartphone che, mediante un'applicazione dedicata e dei servizi Cloud, miri a ridurre al minimo le situazioni di rischio dovute alla scarsità di informazioni sulle zone esplorate da un utilizzatore di tale software.\\
Si richiede di realizzare un'applicazione Android\footnote{\url{https://www.android.com}, sistema operativo open-source sviluppato da Google appositamente per dispositivi portatili. Il codice sorgente è disponibile all'indirizzo \url{https://source.android.com/source/index.html}.}, quindi il linguaggio d'implementazione deve essere \gls{Java}\footnote{\url{https://www.\gls{java}.com}}. Le tecnologie coinvolte sono, inoltre, la geolocalizzazione e la condivisione di connessione\footnote{Si può indicare come modalità \textit{"Tethering"}, ovvero sfruttando la connessione di un device diverso da quello di utilizzo come Gateway per offrire accesso alla Rete.}.
\subsubsection{Rischi potenziali}
Lavorare con un hardware proprietario oltre che con tecnologie non note come l'SDK Android può rendere difficile il calcolo del tempo necessario all’autoformazione.
In virtù delle testimonianze di altri gruppi che hanno scelto il capitolato, il rapporto con il Proponente sembra essere difficoltoso ed eventuali momenti di stop dei lavori, con conseguente ritardo di consegna, potrebbero non consentire al gruppo di completare il lavoro nel tempo limite.
\subsubsection{Aspetti positivi}
Al momento le applicazioni per smartphone ed il Cloud computing sono argomenti che suscitano un fortissimo interesse da parte del mercato. In particolare dopo l'ultimo MWC\footnote{MWC sta per "Mobile World Congress" ed è la fiera internazionale specializzata su Software e dispositivi del mondo \textit{mobile} più grande in Europa.\url{http://www.mobileworldcongress.com/}.} di Barcellona, il mercato degli SmartWatch sembra aver subito una fortissima crescita.
\subsubsection{Aspetti negativi}
Un grosso punto critico è stato individuato nella scarsa spendibilità sul mercato del valore aggiunto che tale progetto può portare al gruppo poiché derivanti da un ambiente proprietario. Inoltre, sebbene il mercato degli SmartWatch sia in crescita, sia i sistemi operativi equipaggiati sia le funzionalità offerte stanno prendendo diverse direzioni.
\subsubsection{Valutazione del capitolato}
Dall'analisi dei punti precedenti il gruppo ha deciso di scartare questo capitolato.
Inoltre, ad oggi, il numero di team che hanno scelto questo progetto ha saturato la disponibilità del Committente.

