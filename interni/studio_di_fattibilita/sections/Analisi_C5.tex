\url{http://www.math.unipd.it/~tullio/IS-1/2014/Progetto/C5.pdf}
L’obiettivo di questo capitolato è la realizzazione di una piattaforma per lo sport e le attività ricreative in montagna utilizzando uno SmartWatch con eventuale connessione ad internet utilizzando la connessione di uno smartphone che mediante un’applicazione dedicata e dei servizi Cloud mira a ridurre al minimo per un utente le situazioni di rischio dovute alla scarsità di informazioni sulle zone esplorate.
\subsubsection{Rischi potenziali}
Lavorare con un hardware proprietario oltre che con tecnologie non note come l’SDK Android può rendere difficile il calcolo del tempo necessario all’autoformazione.
In virtù delle testimonianze dei nostri predecessori, il rapporto con il proponente sembra essere difficoltoso ed eventuali momenti di stop dei lavori, con conseguente ritardo di consegna, potrebbe non consentire al gruppo di completare il lavoro nel tempo limite.
\subsubsection{Aspetti positivi}
Il mercato degli SmartWatch sembra essere in crescita.
\subsubsection{Aspetti negativi}
Secondo il gruppo, un grosso punto critico di questo progetto è rappresentato dalla scarsa spendibilità sul mercato del valore aggiunto che ci elargisce poiché derivanti da un ambiente proprietario. Inoltre, sebbene il mercato degli SmartWatch sia in crescita, i sistemi operativi equipaggiati stanno prendendo diverse direzioni; stesso dicasi per le funzionalità offerte.
\subsubsection{Valutazione del capitolato}
Dall’analisi dei punti precedenti il gruppo ha deciso di scartare questo capitolato.
Inoltre, ad oggi, il numero di team che hanno scelto questo progetto ha saturato la disponibilità del committente.
