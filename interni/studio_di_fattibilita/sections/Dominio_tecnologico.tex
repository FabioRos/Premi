Per l'implementazione del prodotto richiesto, il gruppo avrà bisogno di utilizzare le seguenti tecnologie:
\begin{itemize}
	\item \textbf{\gls{HTML5}}
	\begin{description}
		\item[Riferimenti]\ 
		\noindent\begin{itemize}
			\item Standard: \url{http://www.w3.org/TR/html5};
			\item Capitolato C4.
		\end{itemize}
		
		\item[Vantaggi d'implementazione]\ \\
		\noindent Questa tecnologia permette, mediante il supporto a Canvas, la creazione di figure ed animazioni  utilizzando il linguaggio \gls{JavaScript}. Questo ci premetterà di avere una superficie dove disegnare la nostra presentazione e realizzare effetti grafici quando avviene il cambio da una \gls{slide} ad un'altra. \\ Un altro punto a favore di \gls{HTML5} è che è uno standard web\footnote{W3C Reccomandation, 28 Ottobre 2014, \url{http://www.w3.org/TR/html5}}, quindi abbiamo delle specifiche precise su cui operare. \\ Questa tecnologia è infine largamente diffusa sia sui dispositivi \textit{desktop} sia \textit{mobile} rendendo il codice estremamente portabile.

		\item[Conoscenza attuale]\ \\
		\noindent Le competenze del gruppo \GRUPPO{}  derivano dal corso di Tecnologie Web \footnote{\url{http://docenti.math.unipd.it/gaggi/tecweb/programma.html}} e costituiscono una solida base anche se il livello di conoscenza attuale andrà certamente integrato per lo svolgimento ottimale di questo progetto.
	\end{description}

	\item \textbf{\gls{JavaScript}}
	\begin{description}
		\item[Riferimenti]\ 
		\noindent\begin{itemize}
			\item \url{http://www.w3schools.com/js};
			\item Capitolato C4.
		\end{itemize}
		
		\item[Vantaggi d'implementazione]\ \\ 
		\noindent Questo linguaggio permette di implementare delle funzioni dinamiche su un codice statico quale l'\gls{HTML}. Per il progetto \PROGETTO{} questa caratteristica è indispensabile.
		
		\item[Conoscenza attuale]\ \\ 
		\noindent Al momento le nostre conoscenze derivano dal corso di Tecnologie Web. \\ Due membri del gruppo hanno una conoscenza superficiale della libreria jQuery.
	\end{description}
	
	\item \textbf{CSS3}
	\begin{description}
		\item[Riferimenti]\ 
		\noindent\begin{itemize}
			\item \url{http://www.w3schools.com/css}.
		\end{itemize}
		
		\item[Vantaggi d'implementazione]\ \\ 
		\noindent Permette di avere degli standard di riferimento garanti del fatto che, a fronte di una buona scrittura del codice, esso potrà essere interpretato in ogni dispositivo dotato di un \gls{browser} aggiornato ad una certa data o versione, dalla quale fisseremo la compatibilità in avanti\footnote{Compatibilità con quella specifica versione e con tutte le successive.}. \\ CSS3 Risulta inoltre uno strumento indispensabile per curare la grafica in modo ottimale.
		
		\item[Conoscenza attuale]\ \\ 
		\noindent Il gruppo ritiene di avere le competenze basilari per lavorare con questa tecnologia anche se ad oggi ha usato in prevalenza le features della versione 2 della stessa.
	\end{description}
	
	\item \textbf{Ajax}
	\begin{description}
		\item[Riferimenti]\ 
		\noindent\begin{itemize}
			\item \url{http://www.w3schools.com/Ajax}.
		\end{itemize}
		
		\item[Vantaggi d'implementazione]\ \\ 
		\noindent Utilizzare Ajax permette al gruppo di effettuare operazioni in modo asincrono, ovvero l'esecuzione di un operazione che provoca un cambio di stato della pagina non per forza deve causarne il completo ricaricamento ex-novo.
		
		\item[Conoscenza attuale]\ \\
		\noindent Il gruppo ritiene che le sue competenze in materia siano molto ridotte ed è consapevole di dover lavorare molto su questo fronte.
	\end{description}
	
	\item \textbf{\gls{reveal.js}}
	\begin{description}
		\item[Riferimenti]\ 
		\noindent\begin{itemize}
			\item \url{https://github.com/hakimel/reveal.js}.
		\end{itemize}
		
		\item[Vantaggi d'implementazione]\ \\ 
		\noindent L'utilizzo di questo \gls{framework} consente di avere già implementate alcune funzioni cardine del progetto, soprattutto per quanto riguarda la visualizzazione delle presentazioni.
		
		\item[Conoscenza attuale]\ \\
		\noindent Ad oggi, nessun membro del gruppo conosce bene questo strumento, ma tutti lo hanno visionato almeno una volta per comprenderne le caratteristiche principali.
	\end{description}
\end{itemize}

