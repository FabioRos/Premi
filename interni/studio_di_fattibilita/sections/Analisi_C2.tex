Il capitolato chiede di realizzare un software per il controllo qualità del vetro. Il linguaggio di programmazione con il quale va sviluppato questo prodotto è il \textit{C++}\footnote{\url{http://www.cplusplus.com}} attraverso l'utilizzo della libreria \textit{Qt}\footnote{\url{http://www.qt.io}}.
\subsubsection{Rischi potenziali}
I requisiti sono stati ritenuti troppo generici, nello specifico, le richieste comprendono facilità d'uso ed interfaccia user friendly che hanno criteri di giudizio non del tutto oggettivi.
Inoltre, tra i requisiti vi è la gestione di ricette per il controllo delle varie tipologie di non conformità, senza specificare quali nello specifico.
Il gruppo \GRUPPO{} ritiene molto influente il fatto che il dominio applicativo è assolutamente sconosciuto a tutti i membri. Questo rischierebbe di compromettere la buona riuscita del progetto.
\subsubsection{Aspetti positivi}
Questo capitolato rappresenta ad avviso del gruppo una opportunità di concretizzazione delle proprie conoscenze su un progetto non fine a se stesso ma utile a supporto di processi industriali. In particolare, l'attività nella quale il progetto avrebbe trovato posto sarebbe stata quella del controllo della qualità che ricopre un ruolo chiave e di importanza indiscutibile nell'industria moderna.
\subsubsection{Aspetti negativi}
La complessità del progetto ha suscitato nel gruppo il timore di non essere in grado di terminare lo stesso nei termini di tempo previsto in virtù del numero dei membri e della data di inizio dei lavori. \\
Di controparte è stato ritenuto il progetto meno interessante dal punto di vista dell'apprendimento di nuove tecnologie, anche se avrebbe dato molto al gruppo per quanto riguarda l'algoritmica.\\
Un ultimo aspetto che ha suscitato perplessità su questo capitolato è stata la spendibilità dello stesso al di fuori del mondo accademico poiché ritenuto molto specifico.
\subsubsection{Valutazione del capitolato}
Le tecnologie richieste fanno già parte, anche se non in modo completo, del bagaglio culturale di tutti i membri del gruppo, costituendo un grosso punto a favore di questo progetto. \\
D'altra parte, il gruppo ha ritenuto difficile individuare una metrica precisa per verificare in modo oggettivo ed esaustivo che un'applicazione sia facile da utilizzare ed a quali tipologie di cambiamenti avrebbe dovuto plasmarsi, scartando di conseguenza il capitolato in questione.
