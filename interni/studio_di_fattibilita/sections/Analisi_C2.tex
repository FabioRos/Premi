\url{http://www.math.unipd.it/~tullio/IS-1/2014/Progetto/C2.pdf}
Il capitolato chiede di realizzare un software per il controllo qualità del vetro. 
\subsubsection{Rischi potenziali}
\begin{itemize}
	\item I requisiti sono stati ritenuti troppo generici:
	“Il progetto richiede la	costruzione di una applicazione	/ utilità/	sistema	software,	applicabile specificamente al settore del	vetro, che dovrà avere le seguenti caratteristiche: 
	\item facilità d’uso(interfaccia user friendly);
 	\item gestione di “recipes”	(ricette)  per il controllo delle non conformità (varie tipologie); 
	\item Alerts	e Reporting (errori e reportistica); 
	\item Fruibilità via web;
“Lo scopo finale, dunque,  è un	prodotto flessibile ai cambiamenti e facile da utilizzare.”
\end{itemize}
\subsubsection{Aspetti positivi}
Questo capitolato rappresenta ad avviso del gruppo una opportunità di concretizzazione delle proprie conoscenze su un progetto non fine a se stesso ma utile a supporto di processi industriali. In particolare, l’attività nella quale il progetto avrebbe trovato posto sarebbe stata quella del controllo della qualità che ricopre un ruolo chiave e di importanza indiscutibile.
\subsubsection{Aspetti negativi}
La complessità del progetto ha suscitato nel gruppo il timore di non essere in grado di terminare lo stesso nei termini di tempo previsto in virtù del numero dei partecipanti e della data di inizio dei lavori. \\*
Di controparte è stato ritenuto il progetto meno interessante dal punto di vista dell’apprendimento di nuove tecnologie, anche se avrebbe dato tanto al gruppo per quanto riguarda l’algoritmica.\\*
Un ultimo aspetto che ha suscitato perplessità su questo capitolato è stata la spendibilità dello stesso al di fuori del mondo accademico poiché ritenuto molto specifico.
\subsubsection{Valutazione del capitolato}
Le tecnologie richieste fanno già parte, anche se non in modo completo, del bagaglio culturale di tutti i membri del gruppo, costituendo un grosso punto a favore di questo progetto. \\*
D’altra parte, il gruppo ha ritenuto difficile individuare una metrica precisa per verificare in modo oggettivo ed esaustivo che un’applicazione sia facile da utilizzare ed a quali tipologie di cambiamenti avrebbe dovuto plasmarsi scartando di conseguenza il capitolato in questione.
