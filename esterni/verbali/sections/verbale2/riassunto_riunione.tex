Il primo incontro con il Proponente è stato fissato per chiarire i dubbi e le problematiche emerse nella prima analisi del capitolato. In particolare, il principale dubbio che al gruppo premeva chiarire era la distinzione tra i due requisiti obbligatori \textit{presentare una presentazione} e \textit{presentare una presentazione sul \gls{browser}}, considerato che il software è da sviluppare interamente con tecnologie web. Dal confronto col Proponente è emerso che la richiesta di utilizzo del linguaggio \gls{HTML5} è dovuta proprio alla possibilità di sviluppare applicativi software che possano essere eseguiti anche offline che assicurano piena portabilità anche ai dispositivi \textit{mobile}. 

\noindent Per l'utilizzo del software \PROGETTO{} attraverso dispositivi \textit{mobile}, è emerso che è sufficiente assicurare la sola visualizzazione della presentazione, visto che su dispositivi con schermi di piccole dimensioni è complesso effettuare creazioni e modifiche di slide.

\noindent Si è concordato col Proponente la possibilità di utilizzare il framework \gls{reveal.js} che consente di avere già implementate alcune delle funzioni principali del progetto, tra cui la visualizzazione delle presentazioni.

\noindent Vista la grande presenza di software simili nel mercato il Proponente, prendendo spunto dal software HaikuDeck\footnote{\url{https://www.haikudeck.com/}}, ci ha fatto capire che basta anche una piccola innovazione per riuscire a creare qualcosa di originale e quindi vincente sul mercato. 

\noindent Infine la discussione si è spostata sulla realizzazione della \gls{UI}, da cui si è concordata la creazione di una interfaccia semplice e minimale, cercando di utilizzare elementi strutturali come ornamento, facendo proprio il principio dell'architetto tedesco \textit{Ludwig Mies van der Rohe}\footnote{\url{http://it.wikipedia.org/wiki/Ludwig_Mies_van_der_Rohe}} "\textit{Less is more}". 

