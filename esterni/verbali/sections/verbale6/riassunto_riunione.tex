In questo incontro abbiamo discusso con il Proponente le tecnologie più opportune da adottare nello sviluppo del progetto e abbiamo portato un prototipo con alcune funzionalità che verranno implementate nel programma.

\noindent In primo luogo, abbiamo proposto MongoDB come DBMS per il programma, scelta subito condivisa dal proponente e che quindi verrà utilizzata per sviluppare la base di dati dell'applicazione.

\noindent Per lo sviluppo della parte front-end la nostra idea era di usare Angular.js. Dopo averne parlato con il Proponente, è stato scelto di utilizzare Angular2.js, la versione successiva del framework da noi proposto.

\noindent Successivamente è stato discussa la tecnologia da utilizzare per sviluppare la parte di back-end. L'idea proposta dal nostro gruppo è stata quella di utilizzare PHP attraverso il framework Laravel. Il Proponente, inizialmente, non condivideva appieno l'utilizzo di PHP, in quanto più propenso all'utilizzo di tecnologie più moderne e innovative. Motivando la nostra scelta, però, e spiegando che è stato scelto PHP in quanto è un linguaggio che si adatta molto bene all'utilizzo richiesto e che è già conosciuto dai componenti del gruppo (fattore importante visto il poco tempo a disposizione per la codifica) ha acconsentito all'utilizzo di esso per lo sviluppo del progetto.

\noindent Dopo aver visto il prototipo con il Proponente, ci ha proposto l'utilizzo del framework Mustache.js come appoggio per la creazione delle infografiche. Dopo averci mostrato alcuni esempi e descritto brevemente le sue funzionalità, è stato deciso di utilizzarlo all'interno del progetto.