% importa variabili globali
% definizione variabili globali
\def\GRUPPO {\textit{DazzleWorks}}

\def\PROGETTO {\textbf{Premi}}

\def \PROPONENTE {Piccoli Gregorio, \textit{Zucchetti spa}}
\def\COMMITTENTE {Prof. Vardanega Tullio, \\ & Dr. Cardin Riccardo}

\def\PROPONENTE {Piccoli Gregorio, \textit{Zucchetti spa}}

\def\EMAIL {dazzleworksgroup@gmail.com}

\def\LOGO {../../template/img/logo.png}

\def\INTESTAZIONE {../../template/img/intestazione.png}
\def\PIEDIPAGINA {../../template/img/piedipagina.png}

\def\G {{\small $_G$}}


% definizione variabili locali
\def\DOCUMENTO{Manuale Utente}
\def\VERSIONE{2.0.0}

\def\DESCRIZIONE{Documento che facilita l'utilizzo dell'applicazione da parte dell'utente.}

\def\REDATTORE {Suierica Bogdan}
\def\VERIFICATORE {Ros Fabio}
\def\RESPONSABILE {Agostinetto Matteo}

\def\USO {Esterno}

\def\DISTRIBUZIONE {\GRUPPO{}\\ & \COMMITTENTE{}\\ & \PROPONENTE{}\\}


% abilita (true) / disabilita (false) indice, lista tabelle, lista figure
\def\INDICE	{true}
\def\TABELLE {true}
\def\FIGURE {true}


% importa struttura
\documentclass[a4paper]{article}

% ----- definizioni -----
\def\TITLE		{\mbox{\GRUPPO}}
\def\SUBTITLE	{\SIGLA, \PROGETTO}


% ----- nuovi comandi -----
% fornisce il caption per riferirsi ad una particolare sezione
\newcommand{\numref}[1]{\textsf{\textsl{``\nameref{#1}'' (\ref{#1})}}}


% ----- package -----
\usepackage[T1]{fontenc}   % codifica dei font in uscita
\usepackage[utf8]{inputenc}   % lettere accentate da tastiera
\usepackage[italian]{babel}   % lingua principale del documento
\usepackage[a4paper, top= 3cm, bottom= 3cm, left= 3cm, right= 3cm, bindingoffset= 5mm]{geometry} % impostazione margini

% ------- uso il quarto livello di section attraverso \paragraph{title} 
\usepackage{titlesec} 
\setcounter{secnumdepth}{4} % dichiaro il numero di livelli 
\titleformat{\paragraph}
{\normalfont\normalsize\bfseries}{\theparagraph}{1em}{}
\titlespacing*{\paragraph}
{0pt}{3.25ex plus 1ex minus .2ex}{1.5ex plus .2ex}
% -------

\usepackage{amssymb} %

\usepackage{booktabs} % comandi aggiuntivi per le tabelle

\usepackage{calc} % espressioni aritmetiche
\usepackage{caption} % descrizione figure, ecc
\usepackage{chapterbib} % inclusione delle bibliografie

\usepackage{datatool} % manipolazione dati
\usepackage{dcolumn} % array in tabular

\usepackage{epstopdf} % conversione eps--> pdf
\usepackage{enumitem} % personalizzazione liste
\usepackage{eurosym} % simbolo euro

\usepackage{fancyhdr}   %personalizzazione dello stile
\usepackage{float} % definizione di oggetti floating (es. figure, tabelle)
\usepackage[bottom]{footmisc} % personalizzazione note

\usepackage[]{glossaries}	% glossario
\usepackage{graphicx, subfigure} % pacchetto grafica testo
\usepackage{grffile} % estende gestione filename graphic

\usepackage[colorlinks=true, urlcolor=blue, citecolor=black, linkcolor=black, hyperindex, breaklinks]{hyperref} % gestione dei link

\usepackage{ifthen}	% costrutto ifthenelse

% \usepackage{listings} % inserimento pezzi di codice
\usepackage{longtable} % tabelle su più pagine

\usepackage{pgf} % grafica postscript e PDF
\usepackage{pgfplots}	% composizione di grafici
\pgfplotsset{/pgf/number format/use comma, compat=newest}	% opzioni per i grafici

\usepackage{multirow} % span multiriga

\usepackage{tabularx, array} % crea paragrafi a colonne
\usepackage{titlesec} % personalizzazione titoli
\usepackage{tikz} % gestione delle formule
\usepackage{totpages} % conta numero pagine

\usepackage{soul} % gestione letterspacing
\usepackage{subfigure} % gestione delle sottofigure

\usepackage{verbatim} % inserimento testo verbatim, non interpretato

\usepackage{wallpaper} % gestione background

\usepackage{xspace} % spazi automatici per le macro


% ----- posizione etichette -----
\captionsetup{tableposition=top, figureposition=bottom, font=small}


% ----- glossario -----
\loadglsentries{../../glossario/glossario.tex}
\renewcommand*{\glssymbolsgroupname}{Simboli}


% ----- stile pagina -----
\pagestyle{fancy}

	% header
	\fancypagestyle {firststyle} {	% definizione stile "firststyle"
		\fancyhf{}
	}

	% indentazione paragrafo
	%\setlength{\parindent} {0pt}
	\setlength{\headheight} {25pt}

	% intestazione
	\lhead{}
	\rhead{\nouppercase{\leftmark}}
	\renewcommand{\headrulewidth}{0pt}  % no linea sotto intestazione

	% piè di pagina
	\lfoot{\footnotesize{{\DOCUMENTO} \\ {\VERSIONE}}}
	\cfoot{}
	\rfoot{\thepage}
	\renewcommand{\footrulewidth}{0pt}   % no linea sopra piè di pagina


% ----- inizio documento -----

% ----- prima pagina -----
\begin{document}
\thispagestyle{firststyle}

\begin{center}

%   \vspace{7cm}
	\textbf{{\fontsize{40pt}{41pt}\selectfont \PROGETTO}} \\
	\rule{8cm}{3pt}
   
   \vspace{4cm}
   \includegraphics[height= 4cm] {\LOGO}
   
	\vspace{1cm}
   {\fontsize{30pt}{31pt}\selectfont \textbf{\GRUPPO}}
	
	\vspace{5cm}
	{\fontsize{18pt}{24pt}\selectfont \textbf{\DOCUMENTO}}
	
%	\vspace{1cm}
	\begin{center}
		\begin{tabular}{r|l}
				\textbf{Versione} & \VERSIONE \\
				\textbf{Redattori} & \REDATTORE \\
				\textbf{Verificatori} & \VERIFICATORE \\
				\textbf{Responsabili} & \RESPONSABILE \\
				\textbf{Uso} & \USO \\
				\textbf{Lista di distribuzione} & \DISTRIBUZIONE
		\end{tabular}
	\end{center}

	\vspace{1cm}
	\textbf{\DESCRIZIONE}

\end{center}


\newpage

% ----- pagine successive -----
\ULCornerWallPaper{1}{\INTESTAZIONE}
\LLCornerWallPaper{1}{\PIEDIPAGINA}

%\thispagestyle{empty}

\newpage

% diario delle modifiche


% numerazione pagine indici
\pagenumbering{Roman}


\newpage
\section*{Diario delle modifiche}

\begin{table}[h]
\centering
\begin{tabular}{|c|p{0.3\textwidth}|c|c|c|}
	\toprule
		\textbf{Versione} & \textbf{Modifiche} & \textbf{Autore} & \textbf{Ruolo} & \textbf{Data}\\
	\midrule
	\midrule

		v3.0.0 & \textit{Approvazione documento} & Suierica Bogdan & \textit{Responsabile di Progetto} & 2015-07-01 \\
	\midrule
		v2.1.0 & \textit{Eseguita verifica documento} & Agostinetto Matteo & \textit{Verificatore} & 2015-06-30 \\
	\midrule
		v2.0.2 & \textit{Eseguite correzioni a sottosezioni relative a UC9. Aggiunti requisiti di qualità.} & Crespan Emanuele & \textit{Analista} & 2015-05-09 \\
	\midrule
		v2.0.2 & \textit{Eseguite correzioni a sottosezioni relative a UC4, UC6 e UC7} & Crespan Emanuele & \textit{Analista} & 2015-05-08 \\
	\midrule
		v2.0.1 & \textit{Eseguite correzioni a sottosezioni relative a UC1 e UC2} & Crespan Emanuele & \textit{Analista} & 2015-05-07 \\
	\midrule
		v2.0.0 & \textit{Approvazione documento} & Carraro Nicola & \textit{Responsabile di Progetto} & 2015-05-05 \\
	\midrule
		v1.2.0 & \textit{Eseguita verifica documento} & Crespan Emanuele & \textit{Verificatore} & 2015-05-05 \\
	\midrule
		v1.1.2 & \textit{Eseguite correzioni a sottosezioni relative a UC9.2, UC9.3 e UC10} & Burlin Valerio & \textit{Analista} & 2015-05-04\\
	\midrule
		v1.1.1 & \textit{Eseguite correzioni a sottosezioni relative a UC2.3, UC2.5, UC4.1.3, UC6} & Suierica Bogdan & \textit{Analista} & 2015-05-04\\
	\midrule
		v1.1.0 & \textit{Eseguita verifica documento} & Crespan Emanuele & \textit{Verificatore} & 2015-05-04\\
	\midrule
		v1.0.13 & \textit{Eseguite modifiche a tabelle Requisiti Funzionali di sottosezione 4.1 e Requisiti di Vincolo di sottosezione 4.2} & Burlin Valerio & \textit{Analista} & 2015-05-04\\
	\midrule
		v1.0.12 & \textit{Eseguita modifica a sottosezione relativa a UC10 ed inserimento sottosezione relativa a UC13: Consultazione Manuale Utente} & Suierica Bogdan & \textit{Analista} & 2015-05-03\\
	\midrule
		v1.0.11 & \textit{Eseguita modifica a sottosezione relativa a UC9.9; aggiunte sottosezioni relative a UC9.16.6 e UC9.19} & Suierica Bogdan & \textit{Analista} & 2015-05-03\\

	\bottomrule
\end{tabular}
\end{table}
\newpage
\begin{table}[h]
\centering
\begin{tabular}{|c|p{0.3\textwidth}|c|c|c|}
	\toprule
		\textbf{Versione} & \textbf{Modifiche} & \textbf{Autore} & \textbf{Ruolo} & \textbf{Data}\\
	\midrule
	\midrule
		v1.0.10 & \textit{Eseguite modifiche a sottosezioni relative a UC9.2, 9.3 e 9.6; aggiunte sottosezioni relative ai casi d'uso da UC9.6.1 a UC9.6.5} & Burlin Valerio & \textit{Analista} & 2015-05-03\\
	\midrule
		v1.0.9 & \textit{Eliminata sottosezione relativa a UC8.1} & Suierica Bogdan & \textit{Analista} & 2015-05-02\\
	\midrule
		v1.0.8 & \textit{Eseguite modifiche a sottosezioni relative a UC6, UC6.1 e UC6.2} & Suierica Bogdan & \textit{Analista} & 2015-05-02\\
	\midrule
		v1.0.7 & \textit{Eseguite modifiche a sottosezioni relative a UC5.1 e UC5.2} & Suierica Bogdan & \textit{Analista} & 2015-05-02\\
	\midrule
		v1.0.6 & \textit{Inserimento sottosezione 3.26 relativa a UC4.1.3: Spostarsi tra Slide; eseguite modifiche a sottosezioni relative a UC4, UC4.1 e UC4.2} & Burlin Valerio & \textit{Analista} & 2015-05-01\\
	\midrule
		v1.0.5 & \textit{Eseguita modifica a sottosezioni relative a UC3, UC3.1 e UC3.2; eliminazione sottosezione relativa a UC3.3} & Burlin Valerio & \textit{Analista} & 2015-05-01\\
	\midrule
		v1.0.4 & \textit{Eseguita modifica a sottosezione relativa a UC2: Autenticazione; eliminazione sottosezioni relative a UC2.3, UC2.4 e UC2.5 ed inserito UC2.3: Recupero Dati} & Suierica Bogdan & \textit{Analista} & 2015-05-01\\
	\midrule
		v1.0.3 & \textit{Eseguita modifica a sottosezione relativa a UC1: Registrazione; eliminazione sottosezioni relative a UC1.3, UC1.4 e UC1.5} & Suierica Bogdan & \textit{Analista} & 2015-04-30\\
	\midrule
		v1.0.2 & \textit{Eseguita modifica a sottosezione riguardante UCP: Scenario Principale} & Burlin Valerio & \textit{Analista} & 2015-04-30\\
	\bottomrule
\end{tabular}
\end{table}
\newpage
\begin{table}[h]
\centering
\begin{tabular}{|c|p{0.3\textwidth}|c|c|c|}
	\toprule
		\textbf{Versione} & \textbf{Modifiche} & \textbf{Autore} & \textbf{Ruolo} & \textbf{Data}\\
	\midrule
	\midrule
		v1.0.1 & \textit{Eseguita modifica a sottosezioni 2.1 e 2.2 sulla base delle segnalazioni ricevute in sede di Revisione dei Requisiti} & Burlin Valerio & \textit{Analista} & 2015-04-30\\
	\midrule
		v1.0.0 & \textit{Approvazione documento} & Burlin Valerio & \textit{Responsabile di Progetto} & 2015-04-13\\
	\midrule
		v0.2.0 & \textit{Eseguita verifica documento} & Crespan Emanuele & \textit{Verificatore} & 2015-04-11\\
	\midrule
		v0.1.5 & \textit{Eseguite correzioni a sottosezione 3.88 relativa a UC11} & Ros Fabio & \textit{Analista} & 2015-04-11\\
	\midrule
		v0.1.4 & \textit{Eseguite correzioni a sottosezioni 3.48, 3.51, 3.54, 3.60, 3.68, 3.69, 3.70, 3.78  e 3.81 relative a UC9} & Ros Fabio & \textit{Analista} & 2015-04-10\\
	\midrule
		v0.1.3 & \textit{Eseguite correzioni a sottosezioni 3.44 e 3.45 relative a UC7} & Ros Fabio & \textit{Analista} & 2015-04-10\\
	\midrule
		v0.1.2 & \textit{Eseguite correzioni a sottosezioni 3.32 e 3.34 relative a UC4} & Agostinetto Matteo & \textit{Analista} & 2015-04-09\\
	\midrule
		v0.1.1 & \textit{Eseguite correzioni a sottosezioni 3.17, 3.18, 3.20 e 3.24 relative a UC2} & Agostinetto Matteo & \textit{Analista} & 2015-04-09\\
	\midrule
		v0.1.0 & \textit{Eseguita verifica documento} & Crespan Emanuele & \textit{Verificatore} & 2015-04-07\\
	\midrule
		v0.0.18 & \textit{Stesura sottosezione 4.4 relativa al tracciamento Fonti - Requisiti} & Ros Fabio & \textit{Analista} & 2015-04-02\\
	\midrule
		v0.0.17 & \textit{Stesura sottosezioni 4.2 e 4.3 relative a tabelle Requisiti Funzionali e Requisiti di Vincolo} & Ros Fabio & \textit{Analista} & 2015-03-29\\
	\midrule
		v0.0.16 & \textit{Inizio stesura sezione 4-Requisiti con sottosezione 4.1} & Ros Fabio & \textit{Analista} & 2015-03-26\\
	\bottomrule
\end{tabular}
\end{table}
\newpage
\begin{table}[h]
\centering
\begin{tabular}{|c|p{0.3\textwidth}|c|c|c|}
	\toprule
		\textbf{Versione} & \textbf{Modifiche} & \textbf{Autore} & \textbf{Ruolo} & \textbf{Data}\\
	\midrule
	\midrule
		v0.0.15 & \textit{Stesura sottosezioni 3.91, 3.92 e 3.93 relative a UC12: Eliminazione di un Progetto} & Carraro Nicola & \textit{Analista} & 2015-03-26\\
	\midrule
		v0.0.14 & \textit{Stesura sottosezioni 3.88, 3.89 e 3.90 relative a UC11: Salvataggio di un Progetto} & Carraro Nicola & \textit{Analista} & 2015-03-25\\
	\midrule
		v0.0.13 & \textit{Stesura sottosezioni 3.37, 3.38 e 3.39 relative a UC5: Generazione PDF Progetto} & Agostinetto Matteo & \textit{Analista} & 2015-03-24\\
	\midrule
		v0.0.12 & \textit{Stesura sottosezioni da 3.84 a 3.87 relative a UC10: Creazione dell'Infografica} & Carraro Nicola & \textit{Analista} & 2015-03-24\\
	\midrule
		v0.0.11 & \textit{Stesura sottosezioni da 3.48 a 3.83 relative a UC9: Modifica della Presentazione} & Carraro Nicola & \textit{Analista} & 2015-03-23\\
	\midrule
		v0.0.10 & \textit{Stesura sottosezioni da 3.32 a 3.36 relative a UC4: Visualizzazione} & Agostinetto Matteo & \textit{Analista} & 2015-03-22\\
	\midrule
		v0.0.9 & \textit{Stesura sottosezioni da 3.28 a 3.31 relative a UC3: Ricerca di un Progetto} & Agostinetto Matteo & \textit{Analista} & 2015-03-22\\
	\midrule
		v0.0.8 & \textit{Stesura sottosezioni 3.46 e 3.47 relative a UC8: Apertura di un Progetto} & Carraro Nicola & \textit{Analista} & 2015-03-22\\
	\midrule
		v0.0.7 & \textit{Stesura sottosezioni 3.43, 3.44 e 3.45 relative a UC7: Creazione di un Progetto} & Carraro Nicola & \textit{Analista} & 2015-03-22\\
	\midrule
		v0.0.6 & \textit{Stesura sottosezioni 3.40, 3.41 e 3.42 relative a UC6: Esportazione Progetto} & Carraro Nicola & \textit{Analista} & 2015-03-21\\
	\bottomrule
\end{tabular}
\end{table}
\newpage
\begin{table}[h]
	\centering
	\begin{tabular}{|c|p{0.3\textwidth}|c|c|c|}
		\toprule
		\midrule
		v0.0.5 & \textit{Stesura sottosezioni da 3.14 a 3.27 relative a UC2: Autenticazione} & Ros Fabio & \textit{Analista} & 2015-03-21\\
		\midrule
		v0.0.4 & \textit{Stesura sottosezioni da 3.2 a 3.13 relative a UC1: Registrazione} & Ros Fabio & \textit{Analista} & 2015-03-21\\
		\midrule
		v0.0.3 & \textit{Stesura sezione 2-Descrizione generale con relative sottosezioni} & Carraro Nicola & \textit{Analista} & 2015-03-21\\
		\midrule
		v0.0.2 & \textit{Inizio stesura sezione 3-Casi d'Uso con sottosezione 3.1 relativa a UCP: Scenario Principale} & Agostinetto Matteo & \textit{Analista} & 2015-03-20\\
		\midrule
		v0.0.1 & \textit{Creazione documento e stesura sezione 1-Introduzione con relative sottosezioni} & Agostinetto Matteo & \textit{Analista} & 2015-03-16\\
		\bottomrule
	\end{tabular}
\end{table}
\newpage

% importa indici
% definizione indice
\ifthenelse{\equal{\INDICE}{true}}
	{\tableofcontents \newpage}{}

% definizione lista tabelle
%\ifthenelse{\equal{\TABELLE}{true}} 
%	{\listoftables \newpage}{}

% definizione lista figure
\ifthenelse{\equal{\FIGURE}{true}}
	{\listoffigures \newpage}{}


% numerazione pagine
\pagenumbering{arabic}

	% formato visualizzazione
	\rfoot{\thepage ~di~\pageref{TotPages}}


% separatore
\iffalse
	AOjvdYTJD7mcIIYItfsNiYPbmTTogRSP9hrrb2XPE1laMyQ9NHrPgTCTxnW0eV1YcM3Wqh7t5qThjczeXWq3O5FJ7BBQjoWZovC5
\fi

\section{Introduzione}
\subsection{Scopo del documento}
Il presente documento ha lo scopo di aiutare l'utente ad orientarsi ed apprendere l'uso e il funzionamento del software\ped{G}.

\subsection{Scopo del prodotto}
Lo scopo del progetto è realizzare un software\ped{G} per un sistema di presentazione di slide\ped{G} sfruttando la tecnologia HTML5\ped{G}. Lo scopo principale è quello di creare un prodotto che sia di qualità comparabile, in prestazioni, funzionalità ed effetti visivi, ai maggiori concorrenti già presenti sul mercato (Prezi, Powerpoint, Keynote, Impress, ...).

\subsection{Prerequisiti}
L'utente deve possedere una connessione ad internet funzionante e un web browser (Google Chrome\ped{G} versione 41 o superiore, Mozilla Firefox\ped{G} 37 o superiore, Safari\ped{G} versione 8 o superiore, Opera\ped{G} versione 28 o superiore, Internet Explorer\ped{G} versione 9 o superiore).

\subsection{Accesso all'applicativo PREMI}
Per poter avere accesso al software è necessario recarsi all'indirizzo internet: \textbf{\textit{www.dazzleworks.it}}

\subsection{Glossario}
Per rendere chiaro e non ambiguo il contenuto del documento \textit{manuale utente} è stato realizzato un apposito glossario che contiene le definizioni per i termini tecnici, specifici e di dominio e acronimi, per rendere la documentazione il più possibile chiara ed univocamente interpretabile. Esso è consultabile nell'apposita sezione \textit{Glossario} posta alla fine di questo documento.

\noindent I vocaboli in questione sono facilmente riconoscibili poichè seguiti dal carattere '\ped{G}'.

\subsection{Riferimenti}
\subsubsection{Normativi}

\begin{itemize}
	\item Capitolato d'appalto C4: \PROGETTO: Software di presentazione "better than Prezi" \\ \url{http://www.math.unipd.it/~tullio/IS-1/2014/Progetto/C4.pdf}.
\end{itemize}

\newpage

\section{Premi}
Di seguito viene spiegato l'utilizzo delle principali funzioni dell'applicazione
\subsection{Registrazione}
Per creare un nuovo account selezionare il tasto \textbf{Sign In} posizionato in alto a destra dello schermo.

\noindent Ora bisogna compilare un piccolo form che richiede alcuni dati necessari per poter effettuare la registrazione: username, mail, first name, last name, password e verifica password.

%immagine form

\noindent Una volta compilati tutti i campi richiesti premere il bottone \textbf{Sign In} situato sotto il form.

\subsection{Autenticazione}
Se si possiede già un account è possibile effettuare l'autenticazione.

\noindent Per autenticarsi premere il pulsante \textbf{Log In} situato in alto a destra dello schermo accanto al pulsante di registrazione.

\noindent Ora sono richiesti i dati d'accesso.
%figura

\noindent Una volta inseriti premere il pulsante \textbf{Log In} situato sotto il form.

\subsection{Creazione di un Progetto}

Effettuata l'autenticazione è possibile cominciare a creare progetti.


\noindent Per creare un progetto selezionare dal menù in alto \textbf{New Project} e inserite un titolo per il progetto.

%figura


\subsection{Creazione di una Presentazione}

Una volta creato un progetto verrà automaticamente aperto l'editor che permette di creare le slide della presentazione.

\subsection{Creazione di un'Infografica}

Per creare un'infografica è sufficiente aprire una presentazione selezionare il pulsante \textbf{New Infographic} dal menù laterale e verrà aperto l'editor con cui è possibile modificare l'infografica.

\subsection{Ricerca di un progetto}

È possibile fare una ricerca tra i progetti salvati da altri utenti.

\noindent Per fare ciò è sufficiente compilare il filtro di ricerca, presente sulla propria home page, e premere il pulsante \textbf{Search}.

%figura

Verrà visualizzata una lista di progetti che rispettano i parametri inseriti.

%figura

\noindent A questo punto basta selezionarne uno e scegliere la modalità di visualizzazione.
\newpage

\section{Ricerca di un progetto}
È possibile fare ricerche di progetti utilizzando come filtro l'username di un utente oppure il nome di un progetto.
\newline
Per eseguire una ricerca è sufficiente recarsi sulla home page tramite l'utilizzo del pulsante \textbf{PREMI} posto nell'angolo superiore sinistro dello schermo, scrivere la chiave di ricerca nell'apposita casella di testo, selezionare il filtro che si vuole utilizzare (Users o Project) e infine premere il tasto \textbf{Search}.

\begin{figure}[h] 
	\centering 
	\includegraphics[scale=0.40] {img/ricerca.png}
	\caption{Ricerca} 
\end{figure}


\noindent I risultati della ricerca verranno mostrati sotto il pulsante \textbf{Search}, come mostrato nella figura sottostante.

\begin{figure}[h] 
	\centering 
	\includegraphics[scale=0.40] {img/ricercaris.png}
	\caption{Risultati ricerca} 
\end{figure}
\newpage

\section{Creazione di un progetto}
Per creare un progetto un utente deve essere iscritto ed autenticato. Per accedere alla pagina di creazione di un progetto l'utente deve premere il pulsante azzurro \textbf{My Project} posto in alto a destra sullo schermo. Una volta premuto si caricherà la pagina corrispondente. A questo punto l'utente deve premere il pulsante verde \textbf{New Project} posto in alto a sinistra; si aprirà un pop-up nel quale viene richiesto di inserire il nome del progetto che si vuole creare. Una volta inserito il nome basterà premere il pulsante \textbf{OK}.


\noindent Una volta premuto il tasto \textbf{OK} il nuovo progetto sarà creato e aggiunto alla lista dei progetti dell'utente (vedi figura sottostante).


\begin{figure}[H] 
	\centering 
	\includegraphics[scale=0.60] {img/projectlist.png}
	\caption{Lista dei progetti disponibili} 
\end{figure}

\section{Eliminazione di un progetto}
Per eliminare un progetto un utente deve essere iscritto ed autenticato. Per accedere alla pagina di eliminazione di un progetto l'utente deve premere il pulsante azzurro \textbf{My Project} posto in alto a destra sullo schermo. Una volta premuto si caricherà la pagina corrispondente. A questo punto l'utente deve selezionare il progetto da eliminare dalla lista dei progetti in alto a sinistra.
Una volta selezionato il progetto apparirà al centro dello schermo il titolo del progetto scelto,un'immagine di anteprima della prima slide\ped{G} del progetto e sotto a questa un menù. 

\begin{figure}[H] 
	\centering 
	\includegraphics[scale=0.40] {img/elimina_pro}
	\caption{Eliminazione di un progetto} 
\end{figure}

Selezionando dal menù la voce \textbf{Delete} apparirà il seguente pop-up\ped{G}:

\begin{figure}[H] 
	\centering 
	\includegraphics[scale=0.60] {img/del_project}
	\caption{Pop-up conferma eliminazione progetto} 
\end{figure}

\noindent Premendo il tasto \textbf{Yes} si confermerà l'eliminazione del progetto, premendo il tasto \textbf{No} il progetto non verrà eliminato.

\section{Rinominazione di un progetto}
Per rinominare un progetto un utente deve essere iscritto ed autenticato. Per accedere alla pagina di rinominazione di un progetto l'utente deve premere il pulsante azzurro \textbf{My Project} posto in alto a destra sullo schermo. Una volta premuto si caricherà la pagina corrispondente. A questo punto l'utente deve selezionare il progetto da rinominare dalla lista dei progetti in alto a sinistra.
Una volta selezionato il progetto apparirà al centro dello schermo il titolo del progetto scelto, un'immagine di anteprima della prima slide\ped{G} del progetto e sotto a questa un menù. 

\begin{figure}[H] 
	\centering 
	\includegraphics[scale=0.40] {img/rinomina_pro.png}
	\caption{Rinominazione di un progetto} 
\end{figure}

Selezionando dal menù la voce \textbf{Rename} apparirà il seguente pop-up\ped{G}:

\begin{figure}[H] 
	\centering 
	\includegraphics[scale=0.60] {img/rename_project.png}
	\caption{Pop-up di rinominazione di un progetto} 
\end{figure}

\noindent Per rinominare il progetto sarà sufficiente inserire nell'apposito spazio il nuovo nome che si desidera dare al progetto e successivamente confermare tramite il tasto \textbf{OK}. 

\section{Stampa ed esportazione di una presentazione}
Per stampare un progetto oppure per esportare il progetto in formato PDF\ped{G} un utente deve essere iscritto ed autenticato. Per accedere alla pagina di stampa ed esportazione di un progetto l'utente deve premere il pulsante azzurro \textbf{My Project} posto in alto a destra sullo schermo. Una volta premuto si caricherà la pagina corrispondente. A questo punto l'utente deve selezionare il progetto da stampare o esportare dalla lista dei progetti in alto a sinistra.
Una volta selezionato il progetto apparirà al centro dello schermo il titolo del progetto scelto, un'immagine di anteprima della prima slide\ped{G} del progetto e sotto a questa un menù. 

\begin{figure}[H] 
	\centering 
	\includegraphics[scale=0.40] {img/stampa_pro}
	\caption{Stampa ed esportazione di un progetto} 
\end{figure}

\noindent Selezionando dal menù la voce \textbf{Print} si aprirà una nuova scheda dove verrà visualizzata l'intera presentazione come un'unica pagina web. Le slide\ped{G} verranno impaginate verticalmente una sotto l'altra nell'ordine in cui vengono visualizzate nella presentazione.

\begin{figure}[H] 
	\centering 
	\includegraphics[scale=0.40] {img/print}
	\caption{Stampa ed esportazione di un progetto - Presentazione come un'unica pagina} 
\end{figure}

\subsection{Stampa di una presentazione}
\noindent Nel caso in cui si voglia stampare la presentazione è sufficiente utilizzare la funzione stampa presente nel browser\ped{G} che si sta utilizzando. Per ogni chiarimento in merito è consigliata la consultazione del manuale d'utilizzo del proprio browser\ped{G}. 

\subsection{Esportazione di una presentazione}
Nel caso in cui si voglia esportare la presentazione in formato PDF\ped{G} si dovrà utilizzare la funzionalità di stampa presente in Google Chrome\ped{G} e successivamente sfruttare la stampa su file. Di seguito verrà indicata la procedura da seguire, per questa guida è stata usata la versione 42 di Google Chrome\ped{G}. Una volta selezionata la funzionalità di stampa, apparirà la seguente schermata:

\begin{figure}[H] 
	\centering 
	\includegraphics[scale=0.40] {img/print_google}
	\caption{Stampa ed esportazione di un progetto - Esportazione in formato PDF con Google Chrome} 
\end{figure}

\noindent Nel menù laterale di sinistra, a fianco alla voce \textit{Destinazione} selezionare il pulsante \textbf{Modifica...} e successivamente selezionare la voce \textit{Salva come PDF\ped{G}} (vedi figura sotto): 

\begin{figure}[H] 
	\centering 
	\includegraphics[scale=0.40] {img/salvacome}
	\caption{Stampa ed esportazione di un progetto - Settaggio stampa PDF con Google Chrome} 
\end{figure}

\noindent Una volta seguiti questi passi sarà sufficiente premere il pulsante di colore blu \textbf{Salva} e scegliere la destinazione di salvataggio della presentazione. 


\section{Creazione di una presentazione}
Per creare una presentazione è necessario prima selezionare il progetto dalla lista dei progetti disponibili. Una volta selezionato il progetto apparirà al centro dello schermo il titolo del progetto scelto, un'immagine di anteprima della prima slide\ped{G} del progetto e sotto a questa un menù. Selezionando dal menù la voce \textbf{Edit} si accederà alla pagina relativa alla creazione della presentazione, con le relative funzioni.

\begin{figure}[H] 
	\centering 
	\includegraphics[scale=0.40] {img/presentazione.png}
	\caption{Creazione di una presentazione} 
\end{figure}
\newpage


\section{Editor presentazioni}
Di seguito vengono spiegati l'utilizzo degli strumenti di modifica delle presentazioni.

\subsection{Layout principale}
Una volta avviato l'editor delle presentazioni la pagina che appare si presenta semplice ed intuitiva. A sinistra si trova un menù verticale con tutti i pulsanti che permettono di modificare la slide\ped{G} corrente. Al centro dello schermo invece si trova lo spazio di gestione dei contenuti della slide\ped{G}, dove è possibile interagire con i contenuti. In alto a destra si trova il pulsante verde per l'avvio dell'aiuto a schermo.

\begin{figure}[H] 
	\centering 
	\includegraphics[scale=0.40] {img/layout_editor.png}
	\caption{Layout principale} 
\end{figure}

\subsection{Visual Help}
Il pulsante posto in alto a destra, di colore verde, \textbf{Visual Help} permette l'avvio di un breve tour guidato, che mostra mediante l'utilizzo di pop-up\ped{G} le varie funzionalità associate ai pulsanti presenti nella schermata di modifica delle presentazioni. Una volta premuto il bottone, apparirà sullo schermo un piccolo pop-up\ped{G} di colore nero con una breve spiegazione sulle funzionalità del pulsante indicato dal pop-up stesso (come il pulsante viene indicato dal pop-up\ped{G} è segnalato nell'immagine sottostante):

\begin{figure}[H] 
	\centering 
	\includegraphics[scale=0.80] {img/tour.png}
	\caption{Visual Help - Pop-up di aiuto del pulsante Style} 
\end{figure}


\subsection{Menù laterale di sinistra}
Di seguito verrà analizzato, dall'alto verso il basso, ciascun pulsante presente nel menù laterale di sinistra.

\begin{itemize}
 \item \textbf{Style}\\
    Il pulsante \textbf{Style} permette di scegliere gli effetti di transizione e il tema da applicare alla slide\ped{G}. Una volta scelte le modifiche si deve confermare con il tasto \textbf{OK}.	
    \begin{figure}[H] 
    	\centering 
    	\includegraphics[scale=0.40] {img/editor_style.png}
    	\caption{Menù laterale - Style} 
    \end{figure}
	
 \item \textbf{Salva}\\
	Il pulsante di colore verde è il pulsante che permette di salvare le modifiche apportate alla slide\ped{G}. 	
	\begin{figure}[H] 
		\centering 
		\includegraphics[scale=0.40] {img/editor_save.png}
		\caption{Menù laterale - Salva} 
	\end{figure}
	
	 \item \textbf{Elimina}\\
	 Il pulsante di colore rosso è il pulsante che permette di eliminare tutto il contenuto della slide\ped{G} corrente.  	
	 \begin{figure}[H] 
	 	\centering 
	 	\includegraphics[scale=0.40] {img/editor_del.png}
	 	\caption{Menù laterale - Elimina} 
	 \end{figure}
	
 \item \textbf{Text}\\
    Il pulsante \textbf{Text} permette di inserire del testo all'interno della slide\ped{G}. Una volta inserito il testo nell'apposita casella si deve confermare con il tasto \textbf{OK}.
    \begin{figure}[H] 
	\centering 
	\includegraphics[scale=0.40] {img/editor_text.png}
	\caption{Menù laterale - Text} 
    \end{figure}
    
    
 \item \textbf{Image}\\
    Il pulsante \textbf{Image} permette di aggiungere un'immagine alla slide\ped{G} corrente tra quelle già caricate o di caricarne un'altra presente sul file system\ped{G} dell'utente. Una volta scelta l'immagine, questa verrà inserita automaticamente nella slide\ped{G}.
   \begin{figure}[H] 
	\centering 
	\includegraphics[scale=0.40] {img/editor_img.png}
	\caption{Menù laterale - Image} 
    \end{figure}

  \item \textbf{Table}\\
  Questa funzionalità non è stata ancora implementata.
  
  \item \textbf{Chart}\\
  Questa funzionalità non è stata ancora implementata.
  
  \item \textbf{RealTime}\\
  Questa funzionalità non è stata ancora implementata.


  \item \textbf{Navigazione delle slide}\\
  I pulsanti freccia permettono di navigare tra le slide\ped{G} già create della presentazione. Ad ogni tasto corrisponde una direzione di spostamento.
  \begin{figure}[h] 
  	\centering 
  	\includegraphics[scale=0.80] {img/editor_move.png}
  	\caption{Menù laterale - Aggiunta di una slide} 
  \end{figure}
 
 
 
\item \textbf{Aggiunta di una slide}\\
 Il pulsante con il simbolo \textbf{+} permette di aggiungere una nuova slide\ped{G} nella direzione corrispondente al pulsante premuto.
 \begin{figure}[h] 
 	\centering 
 	\includegraphics[scale=0.80] {img/editor_add.png}
 	\caption{Menù laterale - Aggiunta di una slide} 
 \end{figure}

\end{itemize}


\newpage

\subsection{Modifica di un componente}
Per modificare un componente è sufficiente selezionarlo nella slide\ped{G} e modificarne gli attributi dal menù che comparirà sulla destra.

\subsubsection{Text}
La grandezza, la posizione e la rotazione della casella di testo possono essere modificate con il mouse tramite gli appositi punti di ancoraggio che compaiono una volta selezionato il testo. Il trascinamento in un angolo provoca una variazione della dimensione proporzionale tra larghezza ed altezza.

\begin{figure}[H] 
	\centering 
	\includegraphics[scale=0.80] {img/text_anchor.png}
	\caption{Modifica di un componente - Modifica testo con mouse} 
\end{figure}

\noindent In alternativa si possono modificare le proprietà del testo dal menù laterale di destra, nel dettaglio:
		
		\begin{itemize}
			\item \textbf{ScaleX}: modifica la larghezza della casella di testo;
			\item \textbf{ScaleY}: modifica l'altezza della casella di testo;
			\item \textbf{Freccia verso l'alto}: sposta la casella di testo di un livello verso l'alto;
			\item \textbf{Freccia verso il basso}: sposta la casella di testo di un livello verso il basso;
			\item \textbf{Delete}: elimina la casella di testo;
			\item \textbf{Text}: modifica il testo contenuto nella casella di testo;
			\item \textbf{Tasto B}: modifica lo stile del testo in BOLD;
			\item \textbf{Tasto I}: modifica lo stile del testo in ITALIC;
			\item \textbf{Tasto U}: modifica lo stile del testo in UNDERLINE;
			\item \textbf{Size}: modifica la grandezza del testo;
			\item \textbf{Select Font}: modifica il font del testo;
			\item \textbf{Color}: modifica il colore del testo.
		\end{itemize}
		 \begin{figure}[h] 
		    \centering 
		    \includegraphics[scale=0.40] {img/text_edit.png}
		    \caption{Slide Editor - Modifica testo da menù} 
		\end{figure}
		
		
\newpage 

\subsubsection{Image}
La grandezza, la posizione e la rotazione dell'immagine possono essere modificate con il mouse tramite gli appositi punti di ancoraggio che compaiono una volta selezionata l'immagine. Il trascinamento in un angolo provoca una variazione della dimensione proporzionale tra larghezza ed altezza.

\begin{figure}[H] 
	\centering 
	\includegraphics[scale=0.80] {img/img_anchor.png}
	\caption{Modifica di un componente - Modifica immagine con mouse} 
\end{figure}

\noindent In alternativa si possono modificare le proprietà dell'immagine dal menù laterale di destra, nel dettaglio:

		\begin{itemize}
			\item \textbf{ScaleX}: modifica la larghezza dell'immagine;
			\item \textbf{ScaleY}: modifica l'altezza dell'immagine;
			\item \textbf{Freccia verso l'alto}: sposta l'immagine di un livello verso l'alto;
			\item \textbf{Freccia verso il basso}: sposta l'immagine di un livello verso il basso;
			\item \textbf{Delete}: elimina l'immagine;
		\end{itemize}
		
\begin{figure}[H] 
	\centering 
	\includegraphics[scale=0.40] {img/img_edit.png}
	\caption{Slide Editor - Modifica immagine da menù} 
\end{figure}	



\newpage


\section{Visualizzazione presentazioni}
Per avviare una presentazione è necessario prima selezionare il progetto dalla lista dei progetti disponibili. Una volta selezionato il progetto apparirà al centro dello schermo il titolo del progetto scelto, un'immagine di default e sotto a questa un menù. Selezionando dal menù la voce \textbf{Play} si aprirà una nuova scheda nel browser dove sarà avviata la presentazione.

\begin{figure}[H] 
	\centering 
	\includegraphics[scale=0.40] {img/avv_pres.png}
	\caption{Avvio di una presentazione} 
\end{figure}

\subsection{Menù di navigazione}
\noindent Il menù di navigazione è composto da quattro frecce di colore diverso in base alle slide che sono disponibili nella direzione indicata dalla freccia. Se la freccia è di colore nero non ci sono slide disponibili in quella direzione, viceversa se la freccia è di colore blu ci si può spostare in quella direzione verso la prossima slide. Come esempio si prenda l'immagine sottostante, che indica la possibilità di visualizzare la prossima slide spostandosi a sinistra o in basso.

\begin{figure}[H] 
	\centering 
	\includegraphics[scale=0.70] {img/nav.png}
	\caption{Menù di navigazione in dettaglio} 
	\end{figure}

\subsection{Presentazione in modalità ascoltatore}

\noindent Una volta avviata una presentazione essa verrà avviata di default in modalità ascoltatore. Verrà visualizzata la prima slide del progetto e, in basso a destra, il menù di navigazione utilizzabile sia con il mouse sia con i tasti freccia della tastiera. 
\begin{figure}[H] 
	\centering 
	\includegraphics[scale=0.40] {img/sfondook.png}
	\caption{Presentazione in modalità ascoltatore} 
\end{figure}


\subsection{Presentazione in modalità presentatore}
\noindent Per avviare presentazione in modalità presentatore è necessario avviare prima una presentazione come ascoltatore. Una volta avviata è necessario premere il tasto \textbf{S} della tastiera per avviare la modalità presentatore.
\begin{figure}[H] 
	\centering 
	\includegraphics[scale=0.40] {img/note.png}
	\caption{Presentazione in modalità presentatore} 
\end{figure}

\noindent Come si può notare dall'immagine precedente, l'interfaccia grafica di questa modalità presenta delle funzionalità di supporto al presentatore, dividendo lo schermo in tre parti ben distinte:
\begin{itemize}
	\item \textbf{A}: in questa parte di schermo viene visualizzata la slide che attualmente si sta visualizzando;
	\item \textbf{B}: in questa parte di schermo viene visualizzata la prossima slide che verrà visualizzata;
	\item \textbf{C}: in questa parte di schermo vengono visualizzati gli aiuti al presentatore, cioè il tempo trascorso da quando la presentazione è partita, l'orario corrente e, sotto di essi, eventuali note che il presentatore può essersi scritto. 
\end{itemize}

\subsection{Scorciatoie da tastiera}
\noindent Una volta avviata una presentazione, in qualunque modalità, sono possibili le seguenti azioni:
\begin{itemize}
	\item \textbf{Tasto ESC}: la pressione di questo tasto permette la visualizzazione della matrice completa di tutte le slides che compongono la presentazione. Ripremendo il tasto \textbf{ESC} si ritorna alla visualizzazione normale delle slides;
	\begin{figure}[H] 
		\centering 
		\includegraphics[scale=0.40] {img/anteprima.png}
		\caption{Tasto ESC - Matrice delle slides} 
	\end{figure}
	\item \textbf{Tasto B}: la pressione di questo tasto permette di mettere in pausa la presentazione oscurando la slide corrente. Per riprendere la presentazione è sufficiente ripremere il tasto \textbf{B}.
	\begin{figure}[H] 
		\centering 
		\includegraphics[scale=0.40] {img/b.png}
		\caption{Tasto B - Pausa della presentazione} 
	\end{figure}
\end{itemize}

\newpage

\section{Segnalazione degli errori}
Per la segnalazione di eventuali errori l'utente dovrà spedire una mail all'indirizzo \textbf{\textit{support@dazzleworks.it}} con la seguente formattazione:
\begin{itemize}
	\item \textbf{Oggetto}: pagina o pulsante che provoca l'errore o un comportamento diverso da quanto descritto nel \textit{manuale utente};
	
	\item \textbf{Testo della mail}: una descrizione, il più precisa e dettagliata possibile sull'errore riscontrato e il nome utente del profilo utilizzato dall'utente;
	
	\item \textbf{Allegati}: se possibile allegare uno screenshot\ped{G} della pagina nella quale l'errore si verifica.
\end{itemize}

\noindent Ogni segnalazione verrà visionata e presa in carico nel minor tempo possibile, verrà infine spedita una mail di risposta all'utente con le eventuali procedure da adottare per evitare l'errore o con una segnalazione di avvenuta correzione di quest'ultimo.
\newpage


\section{Glossario}
\paragraph{B}
\begin{itemize}
	\item[] \textbf{Browser}: un browser è un programma che consente di visualizzare i contenuti delle pagine web e di interagire con esse.
\end{itemize}

\newpage


\paragraph{F}
\begin{itemize}
	\item[] \textbf{Facebook}: servizio di rete sociale lanciato nel febbraio del 2004. 

	\item[] \textbf{File system}: indica informalmente un meccanismo con il quale i file sono posizionati e organizzati o su un dispositivo di archiviazione o su una memoria di massa, come ad esempio un disco rigido.
\end{itemize}
\newpage

\paragraph{G}
\begin{itemize}
	\item[] \textbf{Google}: è un motore di ricerca per Internet il cui dominio è stato registrato il 15 settembre 1997.
	\item[] \textbf{Google Chrome}: è un browser\ped{G} sviluppato da Google\ped{G}.
\end{itemize}
\newpage

\paragraph{H}
\begin{itemize}
	\item[] \textbf{HTML5}: linguaggio di markup\ped{G} per la strutturazione delle pagine web, da ottobre 2014 pubblicato come W3C Recommendation.
\end{itemize}
\newpage

\paragraph{I}
\begin{itemize}
	\item[] \textbf{Internet Explorer}: è un browser web grafico proprietario sviluppato da Microsoft e incluso in Windows\ped{G} a partire dal 1995.
\end{itemize}
\newpage


\paragraph{L}
\begin{itemize}
	\item[] \textbf{Linguaggio di markup}: un linguaggio di markup è un insieme di regole che descrivono i meccanismi di rappresentazione (strutturali, semantici o presentazionali) di un testo che, utilizzando convenzioni standardizzate, sono utilizzabili su più supporti.
\end{itemize}
\newpage

\paragraph{M}
\begin{itemize}
	\item[] \textbf{Mozilla Firefox}: è un web browser open source multipiattaforma prodotto da Mozilla Foundation.
\end{itemize}
\newpage

\paragraph{O}
\begin{itemize}
	\item[] \textbf{Opera}: è un browser web freeware e multipiattaforma prodotto da Opera Software.
\end{itemize}
\newpage

\paragraph{P}
\begin{itemize}
	\item[] \textbf{PDF}:  è un formato di file basato su un linguaggio di descrizione di pagina sviluppato da Adobe Systems nel 1993 per rappresentare documenti in modo indipendente dal software utilizzato per generarli o per visualizzarli.
	\item[] \textbf{Pop-up}: sono degli elementi dell'interfaccia grafica, quali finestre o riquadri, che compaiono automaticamente durante l'uso di un'applicazione ed in determinate situazioni, per attirare l'attenzione dell'utente.
\end{itemize}
\newpage

\paragraph{S}
\begin{itemize}
	\item[] \textbf{Safari}: è un browser web sviluppato da Apple Inc.
	\item[] \textbf{Screenshot}: lo screenshot è il risultato della cattura (istantanea) di ciò che è visualizzato sul monitor del computer.
	\item[] \textbf{Slide}: diapositiva digitale.
	\item[] \textbf{Software}: è l'informazione o le informazioni utilizzate da uno o più sistemi informatici e memorizzate su uno o più supporti informatici. Tali informazioni possono essere quindi rappresentate da uno o più programmi, oppure da uno o più dati, oppure da una combinazione delle due.
\end{itemize}
\newpage


\paragraph{W}
\begin{itemize}
	\item[] \textbf{Windows}: è una famiglia di ambienti operativi e sistemi operativi dedicati ai personal computer, alle workstation, ai server e agli smartphone.
\end{itemize}
\newpage

\paragraph{Acronimi}
\begin{itemize}
	\item[] \textbf{PDF}: Portable Document Format.
	\item[] \textbf{W3C}: World Wide Web Consortium.
\end{itemize}
\newpage

%\newpage

% ...

%\printglossaries

\end{document}
