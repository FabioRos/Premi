% importa variabili globali
% definizione variabili globali
\def\GRUPPO {\textit{DazzleWorks}}

\def\PROGETTO {\textbf{Premi}}

\def \PROPONENTE {Piccoli Gregorio, \textit{Zucchetti spa}}
\def\COMMITTENTE {Prof. Vardanega Tullio, \\ & Dr. Cardin Riccardo}

\def\PROPONENTE {Piccoli Gregorio, \textit{Zucchetti spa}}

\def\EMAIL {dazzleworksgroup@gmail.com}

\def\LOGO {../../template/img/logo.png}

\def\INTESTAZIONE {../../template/img/intestazione.png}
\def\PIEDIPAGINA {../../template/img/piedipagina.png}

\def\G {{\small $_G$}}


% definizione variabili locali
\def\DOCUMENTO{Manuale Utente}
\def\VERSIONE{1.0.0}

\def\DESCRIZIONE{Documento che facilita l'utilizzo dell'applicazione da parte dell'utente.}

\def\REDATTORE {Agostinetto Matteo}
\def\VERIFICATORE {Crespan Emanuele}
\def\RESPONSABILE {Suierica Bogdan}

\def\USO {Esterno}

\def\DISTRIBUZIONE {\GRUPPO{}\\ & \COMMITTENTE{}\\ & \PROPONENTE{}\\}


% abilita (true) / disabilita (false) indice, lista tabelle, lista figure
\def\INDICE	{true}
\def\TABELLE {true}
\def\FIGURE {true}


% importa struttura
\documentclass[a4paper]{article}

% ----- definizioni -----
\def\TITLE		{\mbox{\GRUPPO}}
\def\SUBTITLE	{\SIGLA, \PROGETTO}


% ----- nuovi comandi -----
% fornisce il caption per riferirsi ad una particolare sezione
\newcommand{\numref}[1]{\textsf{\textsl{``\nameref{#1}'' (\ref{#1})}}}


% ----- package -----
\usepackage[T1]{fontenc}   % codifica dei font in uscita
\usepackage[utf8]{inputenc}   % lettere accentate da tastiera
\usepackage[italian]{babel}   % lingua principale del documento
\usepackage[a4paper, top= 3cm, bottom= 3cm, left= 3cm, right= 3cm, bindingoffset= 5mm]{geometry} % impostazione margini

% ------- uso il quarto livello di section attraverso \paragraph{title} 
\usepackage{titlesec} 
\setcounter{secnumdepth}{4} % dichiaro il numero di livelli 
\titleformat{\paragraph}
{\normalfont\normalsize\bfseries}{\theparagraph}{1em}{}
\titlespacing*{\paragraph}
{0pt}{3.25ex plus 1ex minus .2ex}{1.5ex plus .2ex}
% -------

\usepackage{amssymb} %

\usepackage{booktabs} % comandi aggiuntivi per le tabelle

\usepackage{calc} % espressioni aritmetiche
\usepackage{caption} % descrizione figure, ecc
\usepackage{chapterbib} % inclusione delle bibliografie

\usepackage{datatool} % manipolazione dati
\usepackage{dcolumn} % array in tabular

\usepackage{epstopdf} % conversione eps--> pdf
\usepackage{enumitem} % personalizzazione liste
\usepackage{eurosym} % simbolo euro

\usepackage{fancyhdr}   %personalizzazione dello stile
\usepackage{float} % definizione di oggetti floating (es. figure, tabelle)
\usepackage[bottom]{footmisc} % personalizzazione note

\usepackage[]{glossaries}	% glossario
\usepackage{graphicx, subfigure} % pacchetto grafica testo
\usepackage{grffile} % estende gestione filename graphic

\usepackage[colorlinks=true, urlcolor=blue, citecolor=black, linkcolor=black, hyperindex, breaklinks]{hyperref} % gestione dei link

\usepackage{ifthen}	% costrutto ifthenelse

% \usepackage{listings} % inserimento pezzi di codice
\usepackage{longtable} % tabelle su più pagine

\usepackage{pgf} % grafica postscript e PDF
\usepackage{pgfplots}	% composizione di grafici
\pgfplotsset{/pgf/number format/use comma, compat=newest}	% opzioni per i grafici

\usepackage{multirow} % span multiriga

\usepackage{tabularx, array} % crea paragrafi a colonne
\usepackage{titlesec} % personalizzazione titoli
\usepackage{tikz} % gestione delle formule
\usepackage{totpages} % conta numero pagine

\usepackage{soul} % gestione letterspacing
\usepackage{subfigure} % gestione delle sottofigure

\usepackage{verbatim} % inserimento testo verbatim, non interpretato

\usepackage{wallpaper} % gestione background

\usepackage{xspace} % spazi automatici per le macro


% ----- posizione etichette -----
\captionsetup{tableposition=top, figureposition=bottom, font=small}


% ----- glossario -----
\loadglsentries{../../glossario/glossario.tex}
\renewcommand*{\glssymbolsgroupname}{Simboli}


% ----- stile pagina -----
\pagestyle{fancy}

	% header
	\fancypagestyle {firststyle} {	% definizione stile "firststyle"
		\fancyhf{}
	}

	% indentazione paragrafo
	%\setlength{\parindent} {0pt}
	\setlength{\headheight} {25pt}

	% intestazione
	\lhead{}
	\rhead{\nouppercase{\leftmark}}
	\renewcommand{\headrulewidth}{0pt}  % no linea sotto intestazione

	% piè di pagina
	\lfoot{\footnotesize{{\DOCUMENTO} \\ {\VERSIONE}}}
	\cfoot{}
	\rfoot{\thepage}
	\renewcommand{\footrulewidth}{0pt}   % no linea sopra piè di pagina


% ----- inizio documento -----

% ----- prima pagina -----
\begin{document}
\thispagestyle{firststyle}

\begin{center}

%   \vspace{7cm}
	\textbf{{\fontsize{40pt}{41pt}\selectfont \PROGETTO}} \\
	\rule{8cm}{3pt}
   
   \vspace{4cm}
   \includegraphics[height= 4cm] {\LOGO}
   
	\vspace{1cm}
   {\fontsize{30pt}{31pt}\selectfont \textbf{\GRUPPO}}
	
	\vspace{5cm}
	{\fontsize{18pt}{24pt}\selectfont \textbf{\DOCUMENTO}}
	
%	\vspace{1cm}
	\begin{center}
		\begin{tabular}{r|l}
				\textbf{Versione} & \VERSIONE \\
				\textbf{Redattori} & \REDATTORE \\
				\textbf{Verificatori} & \VERIFICATORE \\
				\textbf{Responsabili} & \RESPONSABILE \\
				\textbf{Uso} & \USO \\
				\textbf{Lista di distribuzione} & \DISTRIBUZIONE
		\end{tabular}
	\end{center}

	\vspace{1cm}
	\textbf{\DESCRIZIONE}

\end{center}


\newpage

% ----- pagine successive -----
\ULCornerWallPaper{1}{\INTESTAZIONE}
\LLCornerWallPaper{1}{\PIEDIPAGINA}

%\thispagestyle{empty}

\newpage

% diario delle modifiche


% numerazione pagine indici
\pagenumbering{Roman}



% importa indici
% definizione indice
\ifthenelse{\equal{\INDICE}{true}}
	{\tableofcontents \newpage}{}

% definizione lista tabelle
%\ifthenelse{\equal{\TABELLE}{true}} 
%	{\listoftables \newpage}{}

% definizione lista figure
\ifthenelse{\equal{\FIGURE}{true}}
	{\listoffigures \newpage}{}


% numerazione pagine
\pagenumbering{arabic}

	% formato visualizzazione
	\rfoot{\thepage ~di~\pageref{TotPages}}


% separatore
\iffalse
	AOjvdYTJD7mcIIYItfsNiYPbmTTogRSP9hrrb2XPE1laMyQ9NHrPgTCTxnW0eV1YcM3Wqh7t5qThjczeXWq3O5FJ7BBQjoWZovC5
\fi

\section{Introduzione}
\subsection{Scopo del documento}
Il presente documento ha lo scopo di aiutare l'utente ad orientarsi ed apprendere l'uso e il funzionamento del software\ped{G}.

\subsection{Scopo del prodotto}
Lo scopo del progetto è realizzare un software\ped{G} per un sistema di presentazione di slide\ped{G} sfruttando la tecnologia HTML5\ped{G}. Lo scopo principale è quello di creare un prodotto che sia di qualità comparabile, in prestazioni, funzionalità ed effetti visivi, ai maggiori concorrenti già presenti sul mercato (Prezi, Powerpoint, Keynote, Impress, ...).

\subsection{Prerequisiti}
L'utente deve possedere una connessione ad internet funzionante e un web browser (Google Chrome\ped{G} versione 41 o superiore, Mozilla Firefox\ped{G} 37 o superiore, Safari\ped{G} versione 8 o superiore, Opera\ped{G} versione 28 o superiore, Internet Explorer\ped{G} versione 9 o superiore).

\subsection{Accesso all'applicativo PREMI}
Per poter avere accesso al software è necessario recarsi all'indirizzo internet: \textbf{\textit{www.dazzleworks.it}}

\subsection{Glossario}
Per rendere chiaro e non ambiguo il contenuto del documento \textit{manuale utente} è stato realizzato un apposito glossario che contiene le definizioni per i termini tecnici, specifici e di dominio e acronimi, per rendere la documentazione il più possibile chiara ed univocamente interpretabile. Esso è consultabile nell'apposita sezione \textit{Glossario} posta alla fine di questo documento.

\noindent I vocaboli in questione sono facilmente riconoscibili poichè seguiti dal carattere '\ped{G}'.

\subsection{Riferimenti}
\subsubsection{Normativi}

\begin{itemize}
	\item Capitolato d'appalto C4: \PROGETTO: Software di presentazione "better than Prezi" \\ \url{http://www.math.unipd.it/~tullio/IS-1/2014/Progetto/C4.pdf}.
\end{itemize}

\newpage

\section{Premi}
Di seguito viene spiegato l'utilizzo delle principali funzioni dell'applicazione
\subsection{Registrazione}
Per creare un nuovo account selezionare il tasto \textbf{Sign In} posizionato in alto a destra dello schermo.

\noindent Ora bisogna compilare un piccolo form che richiede alcuni dati necessari per poter effettuare la registrazione: username, mail, first name, last name, password e verifica password.

%immagine form

\noindent Una volta compilati tutti i campi richiesti premere il bottone \textbf{Sign In} situato sotto il form.

\subsection{Autenticazione}
Se si possiede già un account è possibile effettuare l'autenticazione.

\noindent Per autenticarsi premere il pulsante \textbf{Log In} situato in alto a destra dello schermo accanto al pulsante di registrazione.

\noindent Ora sono richiesti i dati d'accesso.
%figura

\noindent Una volta inseriti premere il pulsante \textbf{Log In} situato sotto il form.

\subsection{Creazione di un Progetto}

Effettuata l'autenticazione è possibile cominciare a creare progetti.


\noindent Per creare un progetto selezionare dal menù in alto \textbf{New Project} e inserite un titolo per il progetto.

%figura


\subsection{Creazione di una Presentazione}

Una volta creato un progetto verrà automaticamente aperto l'editor che permette di creare le slide della presentazione.

\subsection{Creazione di un'Infografica}

Per creare un'infografica è sufficiente aprire una presentazione selezionare il pulsante \textbf{New Infographic} dal menù laterale e verrà aperto l'editor con cui è possibile modificare l'infografica.

\subsection{Ricerca di un progetto}

È possibile fare una ricerca tra i progetti salvati da altri utenti.

\noindent Per fare ciò è sufficiente compilare il filtro di ricerca, presente sulla propria home page, e premere il pulsante \textbf{Search}.

%figura

Verrà visualizzata una lista di progetti che rispettano i parametri inseriti.

%figura

\noindent A questo punto basta selezionarne uno e scegliere la modalità di visualizzazione.
\newpage

\section{Ricerca di un progetto}
È possibile fare ricerche di progetti utilizzando come filtro l'username di un utente oppure il nome di un progetto.
\newline
Per eseguire una ricerca è sufficiente recarsi sulla home page tramite l'utilizzo del pulsante \textbf{PREMI} posto nell'angolo superiore sinistro dello schermo, scrivere la chiave di ricerca nell'apposita casella di testo, selezionare il filtro che si vuole utilizzare (Users o Project) e infine premere il tasto \textbf{Search}.

\begin{figure}[h] 
	\centering 
	\includegraphics[scale=0.40] {img/ricerca.png}
	\caption{Ricerca} 
\end{figure}


\noindent I risultati della ricerca verranno mostrati sotto il pulsante \textbf{Search}, come mostrato nella figura sottostante.

\begin{figure}[h] 
	\centering 
	\includegraphics[scale=0.40] {img/ricercaris.png}
	\caption{Risultati ricerca} 
\end{figure}
\newpage

\section{Creazione di un progetto}
Per creare un progetto un utente deve essere iscritto ed autenticato. Per accedere alla pagina di creazione di un progetto l'utente deve premere il pulsante azzurro \textbf{My Project} posto in alto a destra sullo schermo. Una volta premuto si caricherà la pagina corrispondente. A questo punto l'utente deve premere il pulsante verde \textbf{New Project} posto in alto a sinistra; si aprirà un pop-up nel quale viene richiesto di inserire il nome del progetto che si vuole creare. Una volta inserito il nome basterà premere il pulsante \textbf{OK}.


\noindent Una volta premuto il tasto \textbf{OK} il nuovo progetto sarà creato e aggiunto alla lista dei progetti dell'utente (vedi figura sottostante).


\begin{figure}[H] 
	\centering 
	\includegraphics[scale=0.60] {img/projectlist.png}
	\caption{Lista dei progetti disponibili} 
\end{figure}

\section{Eliminazione di un progetto}
Per eliminare un progetto un utente deve essere iscritto ed autenticato. Per accedere alla pagina di eliminazione di un progetto l'utente deve premere il pulsante azzurro \textbf{My Project} posto in alto a destra sullo schermo. Una volta premuto si caricherà la pagina corrispondente. A questo punto l'utente deve selezionare il progetto da eliminare dalla lista dei progetti in alto a sinistra.
Una volta selezionato il progetto apparirà al centro dello schermo il titolo del progetto scelto,un'immagine di anteprima della prima slide\ped{G} del progetto e sotto a questa un menù. 

\begin{figure}[H] 
	\centering 
	\includegraphics[scale=0.40] {img/elimina_pro}
	\caption{Eliminazione di un progetto} 
\end{figure}

Selezionando dal menù la voce \textbf{Delete} apparirà il seguente pop-up\ped{G}:

\begin{figure}[H] 
	\centering 
	\includegraphics[scale=0.60] {img/del_project}
	\caption{Pop-up conferma eliminazione progetto} 
\end{figure}

\noindent Premendo il tasto \textbf{Yes} si confermerà l'eliminazione del progetto, premendo il tasto \textbf{No} il progetto non verrà eliminato.

\section{Creazione di una presentazione}
Per creare una presentazione è necessario prima selezionare il progetto dalla lista dei progetti disponibili. Una volta selezionato il progetto apparirà al centro dello schermo il titolo del progetto scelto, un'immagine di anteprima della prima slide\ped{G} del progetto e sotto a questa un menù. Selezionando dal menù la voce \textbf{Edit} si accederà alla pagina relativa alla creazione della presentazione, con le relative funzioni.

\begin{figure}[H] 
	\centering 
	\includegraphics[scale=0.40] {img/presentazione.png}
	\caption{Creazione di una presentazione} 
\end{figure}
\newpage

\section{Editor presentazioni}
Di seguito vengono spiegati l'utilizzo degli strumenti di modifica delle presentazioni.

\subsection{Layout principale}
Una volta avviato l'editor delle presentazioni la pagina che appare si presenta semplice ed intuitiva. A sinistra si trova un menù verticale con tutti i pulsanti che permettono di modificare la slide\ped{G} corrente. Al centro dello schermo invece si trova lo spazio di gestione dei contenuti della slide\ped{G}, dove è possibile interagire con i contenuti. In alto a destra si trova il pulsante verde per l'avvio dell'aiuto a schermo.

\begin{figure}[H] 
	\centering 
	\includegraphics[scale=0.40] {img/layout_editor.png}
	\caption{Layout principale} 
\end{figure}

\subsection{Visual Help}
Il pulsante posto in alto a destra, di colore verde, \textbf{Visual Help} permette l'avvio di un breve tour guidato, che mostra mediante l'utilizzo di pop-up\ped{G} le varie funzionalità associate ai pulsanti presenti nella schermata di modifica delle presentazioni. Una volta premuto il bottone, apparirà sullo schermo un piccolo pop-up\ped{G} di colore nero con una breve spiegazione sulle funzionalità del pulsante indicato dal pop-up stesso (come il pulsante viene indicato dal pop-up\ped{G} è segnalato nell'immagine sottostante):

\begin{figure}[H] 
	\centering 
	\includegraphics[scale=0.80] {img/tour.png}
	\caption{Visual Help - Pop-up di aiuto del pulsante Style} 
\end{figure}


\subsection{Menù laterale di sinistra}
Di seguito verrà analizzato, dall'alto verso il basso, ciascun pulsante presente nel menù laterale di sinistra.

\begin{itemize}
 \item \textbf{Style}\\
    Il pulsante \textbf{Style} permette di scegliere gli effetti di transizione e il tema da applicare alla slide\ped{G}. Una volta scelte le modifiche si deve confermare con il tasto \textbf{OK}.	
    \begin{figure}[H] 
    	\centering 
    	\includegraphics[scale=0.40] {img/editor_style.png}
    	\caption{Menù laterale - Style} 
    \end{figure}
	
 \item \textbf{Salva}\\
	Il pulsante di colore verde è il pulsante che permette di salvare le modifiche apportate alla slide\ped{G}. 	
	\begin{figure}[H] 
		\centering 
		\includegraphics[scale=0.40] {img/editor_save.png}
		\caption{Menù laterale - Salva} 
	\end{figure}
	
	 \item \textbf{Elimina}\\
	 Il pulsante di colore rosso è il pulsante che permette di eliminare tutto il contenuto della slide\ped{G} corrente.  	
	 \begin{figure}[H] 
	 	\centering 
	 	\includegraphics[scale=0.40] {img/editor_del.png}
	 	\caption{Menù laterale - Elimina} 
	 \end{figure}
	
 \item \textbf{Text}\\
    Il pulsante \textbf{Text} permette di inserire del testo all'interno della slide\ped{G}. Una volta inserito il testo nell'apposita casella si deve confermare con il tasto \textbf{OK}.
    \begin{figure}[H] 
	\centering 
	\includegraphics[scale=0.40] {img/editor_text.png}
	\caption{Menù laterale - Text} 
    \end{figure}
    
    
 \item \textbf{Image}\\
    Il pulsante \textbf{Image} permette di aggiungere un'immagine alla slide\ped{G} corrente tra quelle già caricate o di caricarne un'altra presente sul file system\ped{G} dell'utente. Una volta scelta l'immagine, questa verrà inserita automaticamente nella slide\ped{G}.
   \begin{figure}[H] 
	\centering 
	\includegraphics[scale=0.40] {img/editor_img.png}
	\caption{Menù laterale - Image} 
    \end{figure}

  \item \textbf{Table}\\
  Questa funzionalità non è stata ancora implementata.
  
  \item \textbf{Chart}\\
  Questa funzionalità non è stata ancora implementata.
  
  \item \textbf{RealTime}\\
  Questa funzionalità non è stata ancora implementata.


  \item \textbf{Navigazione delle slide}\\
  I pulsanti freccia permettono di navigare tra le slide\ped{G} già create della presentazione. Ad ogni tasto corrisponde una direzione di spostamento.
  \begin{figure}[h] 
  	\centering 
  	\includegraphics[scale=0.80] {img/editor_move.png}
  	\caption{Menù laterale - Aggiunta di una slide} 
  \end{figure}
 
 
 
\item \textbf{Aggiunta di una slide}\\
 Il pulsante con il simbolo \textbf{+} permette di aggiungere una nuova slide\ped{G} nella direzione corrispondente al pulsante premuto.
 \begin{figure}[h] 
 	\centering 
 	\includegraphics[scale=0.80] {img/editor_add.png}
 	\caption{Menù laterale - Aggiunta di una slide} 
 \end{figure}

\end{itemize}


\newpage

\subsection{Modifica di un componente}
Per modificare un componente è sufficiente selezionarlo nella slide\ped{G} e modificarne gli attributi dal menù che comparirà sulla destra.

\subsubsection{Text}
La grandezza, la posizione e la rotazione della casella di testo possono essere modificate con il mouse tramite gli appositi punti di ancoraggio che compaiono una volta selezionato il testo. Il trascinamento in un angolo provoca una variazione della dimensione proporzionale tra larghezza ed altezza.

\begin{figure}[H] 
	\centering 
	\includegraphics[scale=0.80] {img/text_anchor.png}
	\caption{Modifica di un componente - Modifica testo con mouse} 
\end{figure}

\noindent In alternativa si possono modificare le proprietà del testo dal menù laterale di destra, nel dettaglio:
		
		\begin{itemize}
			\item \textbf{ScaleX}: modifica la larghezza della casella di testo;
			\item \textbf{ScaleY}: modifica l'altezza della casella di testo;
			\item \textbf{Freccia verso l'alto}: sposta la casella di testo di un livello verso l'alto;
			\item \textbf{Freccia verso il basso}: sposta la casella di testo di un livello verso il basso;
			\item \textbf{Delete}: elimina la casella di testo;
			\item \textbf{Text}: modifica il testo contenuto nella casella di testo;
			\item \textbf{Tasto B}: modifica lo stile del testo in BOLD;
			\item \textbf{Tasto I}: modifica lo stile del testo in ITALIC;
			\item \textbf{Tasto U}: modifica lo stile del testo in UNDERLINE;
			\item \textbf{Size}: modifica la grandezza del testo;
			\item \textbf{Select Font}: modifica il font del testo;
			\item \textbf{Color}: modifica il colore del testo.
		\end{itemize}
		 \begin{figure}[h] 
		    \centering 
		    \includegraphics[scale=0.40] {img/text_edit.png}
		    \caption{Slide Editor - Modifica testo da menù} 
		\end{figure}
		
		
\newpage 

\subsubsection{Image}
La grandezza, la posizione e la rotazione dell'immagine possono essere modificate con il mouse tramite gli appositi punti di ancoraggio che compaiono una volta selezionata l'immagine. Il trascinamento in un angolo provoca una variazione della dimensione proporzionale tra larghezza ed altezza.

\begin{figure}[H] 
	\centering 
	\includegraphics[scale=0.80] {img/img_anchor.png}
	\caption{Modifica di un componente - Modifica immagine con mouse} 
\end{figure}

\noindent In alternativa si possono modificare le proprietà dell'immagine dal menù laterale di destra, nel dettaglio:

		\begin{itemize}
			\item \textbf{ScaleX}: modifica la larghezza dell'immagine;
			\item \textbf{ScaleY}: modifica l'altezza dell'immagine;
			\item \textbf{Freccia verso l'alto}: sposta l'immagine di un livello verso l'alto;
			\item \textbf{Freccia verso il basso}: sposta l'immagine di un livello verso il basso;
			\item \textbf{Delete}: elimina l'immagine;
		\end{itemize}
		
\begin{figure}[H] 
	\centering 
	\includegraphics[scale=0.40] {img/img_edit.png}
	\caption{Slide Editor - Modifica immagine da menù} 
\end{figure}	



\newpage


\section{Visualizzazione presentazioni}
Per avviare una presentazione è necessario prima selezionare il progetto dalla lista dei progetti disponibili. Una volta selezionato il progetto apparirà al centro dello schermo il titolo del progetto scelto, un'immagine di default e sotto a questa un menù. Selezionando dal menù la voce \textbf{Play} si aprirà una nuova scheda nel browser dove sarà avviata la presentazione.

\begin{figure}[H] 
	\centering 
	\includegraphics[scale=0.40] {img/avv_pres.png}
	\caption{Avvio di una presentazione} 
\end{figure}

\subsection{Menù di navigazione}
\noindent Il menù di navigazione è composto da quattro frecce di colore diverso in base alle slide che sono disponibili nella direzione indicata dalla freccia. Se la freccia è di colore nero non ci sono slide disponibili in quella direzione, viceversa se la freccia è di colore blu ci si può spostare in quella direzione verso la prossima slide. Come esempio si prenda l'immagine sottostante, che indica la possibilità di visualizzare la prossima slide spostandosi a sinistra o in basso.

\begin{figure}[H] 
	\centering 
	\includegraphics[scale=0.70] {img/nav.png}
	\caption{Menù di navigazione in dettaglio} 
	\end{figure}

\subsection{Presentazione in modalità ascoltatore}

\noindent Una volta avviata una presentazione essa verrà avviata di default in modalità ascoltatore. Verrà visualizzata la prima slide del progetto e, in basso a destra, il menù di navigazione utilizzabile sia con il mouse sia con i tasti freccia della tastiera. 
\begin{figure}[H] 
	\centering 
	\includegraphics[scale=0.40] {img/sfondook.png}
	\caption{Presentazione in modalità ascoltatore} 
\end{figure}


\subsection{Presentazione in modalità presentatore}
\noindent Per avviare presentazione in modalità presentatore è necessario avviare prima una presentazione come ascoltatore. Una volta avviata è necessario premere il tasto \textbf{S} della tastiera per avviare la modalità presentatore.
\begin{figure}[H] 
	\centering 
	\includegraphics[scale=0.40] {img/note.png}
	\caption{Presentazione in modalità presentatore} 
\end{figure}

\noindent Come si può notare dall'immagine precedente, l'interfaccia grafica di questa modalità presenta delle funzionalità di supporto al presentatore, dividendo lo schermo in tre parti ben distinte:
\begin{itemize}
	\item \textbf{A}: in questa parte di schermo viene visualizzata la slide che attualmente si sta visualizzando;
	\item \textbf{B}: in questa parte di schermo viene visualizzata la prossima slide che verrà visualizzata;
	\item \textbf{C}: in questa parte di schermo vengono visualizzati gli aiuti al presentatore, cioè il tempo trascorso da quando la presentazione è partita, l'orario corrente e, sotto di essi, eventuali note che il presentatore può essersi scritto. 
\end{itemize}

\subsection{Scorciatoie da tastiera}
\noindent Una volta avviata una presentazione, in qualunque modalità, sono possibili le seguenti azioni:
\begin{itemize}
	\item \textbf{Tasto ESC}: la pressione di questo tasto permette la visualizzazione della matrice completa di tutte le slides che compongono la presentazione. Ripremendo il tasto \textbf{ESC} si ritorna alla visualizzazione normale delle slides;
	\begin{figure}[H] 
		\centering 
		\includegraphics[scale=0.40] {img/anteprima.png}
		\caption{Tasto ESC - Matrice delle slides} 
	\end{figure}
	\item \textbf{Tasto B}: la pressione di questo tasto permette di mettere in pausa la presentazione oscurando la slide corrente. Per riprendere la presentazione è sufficiente ripremere il tasto \textbf{B}.
	\begin{figure}[H] 
		\centering 
		\includegraphics[scale=0.40] {img/b.png}
		\caption{Tasto B - Pausa della presentazione} 
	\end{figure}
\end{itemize}

\newpage


%\newpage

% ...

%\printglossaries

\end{document}
