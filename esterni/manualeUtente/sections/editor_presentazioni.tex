\noindent
Di seguito vengono spiegati l'utilizzo degli strumenti di modifica delle presentazioni.

\subsection{Menù laterale}
\subsubsection{Text}
Premendo il pulsante del menù laterale \textbf{Text} verrà aggiunto alla slide corrente un campo di testo.
\subsubsection{Image}
Premendo il pulsante del menù laterale \textbf{Image} verrà visualizzata una finestra che permette di aggiungere un'immagine alla slide corrente tra quelle già caricate o di caricarne un'altra.
\subsubsection{Table}
Premendo il pulsante del menù laterale \textbf{Table} verrà aggiunta una tabella alla slide corrente.
\subsubsection{Chart}
Premendo il pulsante del menù laterale \textbf{Chart} verrà aggiunto un grafico alla slide corrente.
\subsubsection{RealTimeData}
Premendo il pulsante del menù laterale \textbf{RealTimeData} verrà aggiunto componente rappresentante i valori, presi in tempo reale, con un formato adeguato.
\subsubsection{Shape}
Premendo il pulsante del menù laterale \textbf{Shape} verrà visualizzata una finestra nella quale si può scegliere una forma da aggiungere alla slide corrente.
\subsubsection{Style}
\subsubsection{Settings}
Premendo il pulsante del menù laterale \textbf{Settings} verrà visualizzata una finestra nella quale si possono modificare alcune opzioni riguardante la presentazione.

\subsection{Modifica di un componente}
Per modificare un componente è sufficiente selezionarlo nella slide e modificarne gli attributi dal menù che comparirà sulla destra.
\begin{itemize}

	\item \textbf{Text:}
		\begin{itemize}
			\item \textbf{Dimensione font;}
			\item \textbf{Tipo di font;}
			\item \textbf{Dimensione capo di testo;}
			\item \textbf{Rotazione;}
			\item \textbf{Allineamento.}
		\end{itemize}
	
	\item \textbf{Image:}
		\begin{itemize}
			\item \textbf{Dimensione;}
			\item \textbf{Colore di sfondo;}
			\item \textbf{Opacità;}
			\item \textbf{Rotazione.}
		\end{itemize}

	\item \textbf{Table:}
		\begin{itemize}
			\item \textbf{Numero di colonne;}
			\item \textbf{Numero di righe;}
			\item \textbf{Dimensione;}
			\item \textbf{Rotazione;}
			\item \textbf{Colore di sfondo.}
		\end{itemize}
		
	\item \textbf{Chart:}
		\begin{itemize}
			\item \textbf{Dimensione;}
			\item \textbf{Tipo di grafico;}
			\item \textbf{Dati;}
			\item \textbf{Rotazione;}
			\item \textbf{Colore di sfondo.}
		\end{itemize}
	
	\item \textbf{RealTimeData:}
		\begin{itemize}
			\item \textbf{Dimensione;}
			\item \textbf{Rotazione;}
			\item \textbf{Indirizzo;}
			\item \textbf{Tipo di dati;}
		\end{itemize}
		
	\item \textbf{Shape:}
		\begin{itemize}
			\item \textbf{Dimensione;}
			\item \textbf{Rotazione;}
			\item \textbf{Colore;}
		\end{itemize}
\end{itemize}

\noindent Per modificare la dimensione di un componente è anche possibile selezionarlo e trascinare i bordi e gli angoli de riquadro.

\subsection{Aggiunta di una slide}
Per aggiungere una slide è sufficiente premere i simbolo \textbf{+} che si trovano a destra e sotto la slide corrente.

\subsection{Navigazione delle slide}
Per navigare tra una slide e l'altra bisogna utilizzare le quattro freccete situate nell'angolo in basso a destra della slide o le freccette della tastiera.