\subsection{Scopo del documento}
Il presente documento ha lo scopo di aiutare l'utente ad orientarsi ed apprendere l'uso e il funzionamento del software\ped{G}.

\subsection{Scopo del prodotto}
Lo scopo del progetto è realizzare un software\ped{G} per un sistema di presentazione di slide\ped{G} sfruttando la tecnologia HTML5\ped{G}. Lo scopo principale è quello di creare un prodotto che sia di qualità comparabile, in prestazioni, funzionalità ed effetti visivi, ai maggiori concorrenti già presenti sul mercato (Prezi, Powerpoint, Keynote, Impress, ...).

\subsection{Prerequisiti}
L'utente deve possedere una connessione ad internet funzionante e un web browser (Google Chrome\ped{G} versione 41 o superiore, Mozilla Firefox\ped{G} 37 o superiore, Safari\ped{G} versione 8 o superiore, Opera\ped{G} versione 28 o superiore, Internet Explorer\ped{G} versione 9 o superiore).

\subsection{Accesso all'applicativo PREMI}
Per poter avere accesso al software è necessario recarsi all'indirizzo internet: \textbf{\textit{www.dazzleworks.it}}

\subsection{Glossario}
Per rendere chiaro e non ambiguo il contenuto del documento \textit{manuale utente} è stato realizzato un apposito glossario che contiene le definizioni per i termini tecnici, specifici e di dominio e acronimi, per rendere la documentazione il più possibile chiara ed univocamente interpretabile. Esso è consultabile nell'apposita sezione \textit{Glossario} posta alla fine di questo documento.

\noindent I vocaboli in questione sono facilmente riconoscibili poichè seguiti dal carattere '\ped{G}'.

\subsection{Riferimenti}
\subsubsection{Normativi}

\begin{itemize}
	\item Capitolato d'appalto C4: \PROGETTO: Software di presentazione "better than Prezi" \\ \url{http://www.math.unipd.it/~tullio/IS-1/2014/Progetto/C4.pdf}.
\end{itemize}
