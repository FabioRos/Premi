\subsection{Scopo del documento}
Il presente documento ha lo scopo di aiutare l'utente ad orientarsi ed apprendere l'uso e il funzionamento del software.
\subsection{Scopo del prodotto}
Lo scopo del progetto è realizzare un software per un sistema di rappresentazione di slide sfruttando la tecnologia HTML5. Lo scopo principale è quello di creare un prodotto che sia di qualità comparabile, in prestazioni, funzionalità ed effetti visivi, ai maggiori concorrenti già presenti sul mercato (Prezi, Powerpoint, Keynote, Impress, ...).
\subsection{Glossario}
Per rendere chiaro e non ambiguo il contenuto dei documenti è stato realizzato un apposito Glossario (\textit{“Glossario v3.0.0”}) che contiene le definizioni per i termini tecnici, specifici di dominio, acronimi  per rendere la documentazione il più possibile chiara ed univocamente interpretabile. Esso è presente in singola copia al fine di evitare ridondanze per termini che compaiono in più documenti.
I vocaboli in questione sono facilmente riconoscibili poiché seguiti dal carattere '\ped{G}'.
\subsection{Riferimenti}
\subsubsection{Normativi}
\begin{itemize}
	\item Norme di Progetto: \textit{“Norme di Progetto v3.0.0”}.
	\item Capitolato d’appalto C4: \PROGETTO: Software di presentazione "better than Prezi" \\ \url{http://www.math.unipd.it/~tullio/IS-1/2014/Progetto/C4.pdf};
\end{itemize}
