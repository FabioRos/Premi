\noindent
Per entrare in modalità visualizzazione di una presentazione è necessario entrare nel progetto ad essa asciato e premere il bottone play. 

all'interno di una presentazione sono possibili le seguenti azioni:

\begin{itemize}
 \item \textbf{spostamento tra slides}: \\disponibile attraverso dei tasti freccia in basso a destra o mediante i tasti freccia della tastiera.
 \item \textbf{passare in modalità presentatore}: \\disponibile attraverso la pressione del tasto '\textbf{S}' della tastiera (S sta per Speaker View).
  \item \textbf{pausa esposizione}: \\disponibile attraverso la pressione del tasto '\textbf{B}' della tastiera (B sta per Blind). In questa modalità la presentazione si oscura, permettendo al presentatore di fare una pausa.
  \item \textbf{visualizzazione dell'indice}: \\disponibile attraverso la pressione del tasto '\textbf{esc}' della tastiera. In questa modalità è possibile vedere la matrice completa di tutte le slide che compongono la presentazione.

  \end{itemize}
