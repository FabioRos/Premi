\noindent
Di seguito vengono spiegati l'utilizzo degli strumenti di modifica delle infografiche.

\subsection{Menù laterale}
\subsubsection{Text}
Premendo il pulsante del menù laterale \textbf{Text} verrà aggiunto all'infografica un campo di testo.
\subsubsection{Image}
Premendo il pulsante del menù laterale \textbf{Image} verrà visualizzata una finestra che permette di aggiungere un'immagine all'infografica tra quelle già caricate o di caricarne un'altra.
\subsubsection{Table}
Premendo il pulsante del menù laterale \textbf{Table} verrà aggiunta una tabella all'infografica.
\subsubsection{Chart}
Premendo il pulsante del menù laterale \textbf{Chart} verrà aggiunto un grafico all'infografica.
\subsubsection{Shape}
Premendo il pulsante del menù laterale \textbf{Shape} verrà visualizzata una finestra nella quale si può scegliere una forma da aggiungere all'infografica.
\subsubsection{Template}
Premendo il pulsante del menù laterale \textbf{Template} verrà visualizzata una finestra nella quale si può scegliere uno stile predefinito di composizione dell'infografica.
\subsubsection{Slide}
Premendo il pulsante del menù laterale \textbf{Slide} verrà visualizzata una finestra nella quale si potrà scegliere da quali slide della presentazione ricavare le informazioni più rilevanti che verranno poi inserite nell'infografica secondo il template scelto.
\subsubsection{Settings}
Premendo il pulsante del menù laterale \textbf{Settings} verrà visualizzata una finestra nella quale si possono modificare alcune opzioni riguardante l'infografica.
\subsubsection{Save}
Premendo il pulsante del menù laterale \textbf{Save} verrà salvata l'infografica all'interno del progetto.
\subsubsection{Print}
Premendo il pulsante del menù laterale \textbf{Print} verrà aperta la finestra di stampa del browser per stampare l'infografica.

\subsection{Modifica di un componente}
Per modificare un componente è sufficiente selezionarlo nell'infografica e modificarne gli attributi dal menù che comparirà sulla destra.
\begin{itemize}

	\item \textbf{Text:}
		\begin{itemize}
			\item \textbf{Dimensione font;}
			\item \textbf{Tipo di font;}
			\item \textbf{Dimensione campo di testo;}
			\item \textbf{Rotazione;}
			\item \textbf{Allineamento.}
		\end{itemize}
	
	\item \textbf{Image:}
		\begin{itemize}
			\item \textbf{Dimensione;}
			\item \textbf{Colore di sfondo;}
			\item \textbf{Opacità;}
			\item \textbf{Rotazione.}
		\end{itemize}

	\item \textbf{Table:}
		\begin{itemize}
			\item \textbf{Numero di colonne;}
			\item \textbf{Numero di righe;}
			\item \textbf{Dimensione;}
			\item \textbf{Rotazione;}
			\item \textbf{Colore di sfondo.}
		\end{itemize}
		
	\item \textbf{Chart:}
		\begin{itemize}
			\item \textbf{Dimensione;}
			\item \textbf{Tipo di grafico;}
			\item \textbf{Dati;}
			\item \textbf{Rotazione;}
			\item \textbf{Colore di sfondo.}
		\end{itemize}
	
	\item \textbf{RealTimeData:}
		\begin{itemize}
			\item \textbf{Dimensione;}
			\item \textbf{Rotazione;}
			\item \textbf{Indirizzo;}
			\item \textbf{Tipo di dati;}
		\end{itemize}
		
	\item \textbf{Shape:}
		\begin{itemize}
			\item \textbf{Dimensione;}
			\item \textbf{Rotazione;}
			\item \textbf{Colore;}
		\end{itemize}
	
	\item \textbf{Slide:}
		\begin{itemize}
			\item \textbf{Dimensione;}
			\item \textbf{Colore di sfondo;}
			\item \textbf{Opacità;}
			\item \textbf{Rotazione.}
		\end{itemize}	
	
\end{itemize}

\noindent Per modificare la dimensione di un componente è anche possibile selezionarlo e trascinare i bordi e gli angoli del riquadro.