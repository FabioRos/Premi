\noindent
Di seguito vengono spiegati l'utilizzo degli strumenti di modifica delle infografiche.

\subsection{Menù laterale}
  \begin{itemize}
      \item \textbf{Slide}\\
	  Premendo il pulsante del menù laterale \textbf{Slide} verrà visualizzata una finestra nella quale si potrà scegliere da quali slide della presentazione ricavare le informazioni più rilevanti che verranno poi inserite nell'infografica secondo il template scelto.
      \item \textbf{Settings}\\
	  Premendo il pulsante del menù laterale \textbf{Settings} verrà visualizzata una finestra nella quale si possono modificare alcune opzioni riguardante l'infografica, tra le quali, dimensione della carta e template.
      \item \textbf{Save}\\
	  Premendo il pulsante del menù laterale \textbf{Save} verrà salvata l'infografica all'interno del progetto generando un file PNG ed un PDF a patire dalla superficie interattiva.
      \item \textbf{Print}\\
	  Premendo il pulsante del menù laterale \textbf{Print} verrà aperta la finestra di stampa del browser per stampare l'infografica.
      \item \textbf{Modifica di un componente}\\
	  Per modificare un componente (slide) è sufficiente trascinare una slide da quelle disponibili nel progetto corrente nella superfice interattiva.
  \end{itemize}