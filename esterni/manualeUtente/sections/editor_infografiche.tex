\noindent
Di seguito vengono spiegati l'utilizzo degli strumenti di modifica delle infografiche.

\subsection{Menù laterale}
  \begin{itemize}
      \item \textbf{\gls{Slide}}\\
	  Premendo il pulsante del menù laterale \textbf{\gls{Slide}} verrà visualizzata una finestra nella quale si potrà scegliere da quali \gls{slide} della presentazione ricavare le informazioni più rilevanti che verranno poi inserite nell'\gls{infografica} secondo il \gls{template} scelto.
      \item \textbf{Settings}\\
	  Premendo il pulsante del menù laterale \textbf{Settings} verrà visualizzata una finestra nella quale si possono modificare alcune opzioni riguardante l'\gls{infografica}, tra le quali, dimensione della carta e \gls{template}.
      \item \textbf{Save}\\
	  Premendo il pulsante del menù laterale \textbf{Save} verrà salvata l'\gls{infografica} all'interno del progetto generando un file \gls{PNG} ed un PDF a patire dalla superficie interattiva.
      \item \textbf{Print}\\
	  Premendo il pulsante del menù laterale \textbf{Print} verrà aperta la finestra di stampa del \gls{browser} per stampare l'\gls{infografica}.
      \item \textbf{Modifica di un componente}\\
	  Per modificare un componente (\gls{slide}) è sufficiente trascinare una \gls{slide} da quelle disponibili nel progetto corrente nella superfice interattiva.
  \end{itemize}
