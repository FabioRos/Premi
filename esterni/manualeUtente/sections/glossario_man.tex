\paragraph{B}
\begin{itemize}
	\item[] \textbf{Browser}: un browser è un programma che consente di visualizzare i contenuti delle pagine web e di interagire con esse.
\end{itemize}

\newpage


\paragraph{F}
\begin{itemize}
	\item[] \textbf{Facebook}: servizio di rete sociale lanciato nel febbraio del 2004. 

	\item[] \textbf{File system}: indica informalmente un meccanismo con il quale i file sono posizionati e organizzati o su un dispositivo di archiviazione o su una memoria di massa, come ad esempio un disco rigido.
\end{itemize}
\newpage

\paragraph{G}
\begin{itemize}
	\item[] \textbf{Google}: è un motore di ricerca per Internet il cui dominio è stato registrato il 15 settembre 1997.
	\item[] \textbf{Google Chrome}: è un browser\ped{G} sviluppato da Google\ped{G}.
\end{itemize}
\newpage

\paragraph{H}
\begin{itemize}
	\item[] \textbf{HTML5}: linguaggio di markup\ped{G} per la strutturazione delle pagine web, da ottobre 2014 pubblicato come W3C Recommendation.
\end{itemize}
\newpage

\paragraph{I}
\begin{itemize}
	\item[] \textbf{Internet Explorer}: è un browser web grafico proprietario sviluppato da Microsoft e incluso in Windows\ped{G} a partire dal 1995.
\end{itemize}
\newpage


\paragraph{L}
\begin{itemize}
	\item[] \textbf{Linguaggio di markup}: un linguaggio di markup è un insieme di regole che descrivono i meccanismi di rappresentazione (strutturali, semantici o presentazionali) di un testo che, utilizzando convenzioni standardizzate, sono utilizzabili su più supporti.
\end{itemize}
\newpage

\paragraph{M}
\begin{itemize}
	\item[] \textbf{Mozilla Firefox}: è un web browser open source multipiattaforma prodotto da Mozilla Foundation.
\end{itemize}
\newpage

\paragraph{O}
\begin{itemize}
	\item[] \textbf{Opera}: è un browser web freeware e multipiattaforma prodotto da Opera Software.
\end{itemize}
\newpage

\paragraph{S}
\begin{itemize}
	\item[] \textbf{Screenshot}: lo screenshot è il risultato della cattura (istantanea) di ciò che è visualizzato sul monitor del computer.
	\item[] \textbf{Software}: è l'informazione o le informazioni utilizzate da uno o più sistemi informatici e memorizzate su uno o più supporti informatici. Tali informazioni possono essere quindi rappresentate da uno o più programmi, oppure da uno o più dati, oppure da una combinazione delle due.
\end{itemize}
\newpage


\paragraph{P}
\begin{itemize}
	\item[] \textbf{PDF}:  è un formato di file basato su un linguaggio di descrizione di pagina sviluppato da Adobe Systems nel 1993 per rappresentare documenti in modo indipendente dall'hardware e dal software utilizzati per generarli o per visualizzarli.
	\item[] \textbf{Pop-up}: sono degli elementi dell'interfaccia grafica, quali finestre o riquadri, che compaiono automaticamente durante l'uso di un'applicazione ed in determinate situazioni, per attirare l'attenzione dell'utente.
\end{itemize}
\newpage

\paragraph{S}
\begin{itemize}
	\item[] \textbf{Safari}: è un browser web sviluppato da Apple Inc.
	\item[] \textbf{Slide}: diapositiva digitale.
\end{itemize}
\newpage

\paragraph{W}
\begin{itemize}
	\item[] \textbf{Windows}: è una famiglia di ambienti operativi e sistemi operativi dedicati ai personal computer, alle workstation, ai server e agli smartphone.
\end{itemize}
\newpage

\paragraph{Acronimi}
\begin{itemize}
	\item[] \textbf{PDF}: Portable Document Format;
	\item[] \textbf{W3C}: World Wide Web Consortium.
\end{itemize}