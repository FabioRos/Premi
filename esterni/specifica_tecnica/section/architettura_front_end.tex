\subsection{Premi::Front-End}
	\subsubsection*{Informazioni sul package}
		\begin{figure}[h]
			\centering
			\includegraphics[width=\linewidth]{img/front-end_package}
			\caption[Premi::Front-End]{Premi::Front-End}
		\end{figure}
		Il package contiene le varie componenti della parte di \gls{front-end} dell'applicazione.

		\subsubsection*{Package contenuti:}
			\begin{itemize}
				\item \textbf{Controllers:} Package contenente i controller della parte \gls{front-end} dell'applicazione.
				\item \textbf{Views:} Package contenente le views della componente \gls{front-end} dell'applicazione;
				\item \textbf{Model:} Package che definiscono la logica di business dell'applicazione;
				\item \textbf{Directives:} Package contenente le direttive che compongono la view;
				\item \textbf{Services:} Package contenente le classi per interfacciarsi con le API REST del \gls{back-end}. Permette di popolare il model del client e di richiedere l'esecuzione di operazioni al server.
			\end{itemize}

		\subsection{Premi::Front-End::Controllers}
			\subsubsection*{Informazioni sul package}
\begin{figure}[h]
	\centering
	\includegraphics[width=1.0\linewidth]{img/front-end_controllers}
	\caption[Premi::Front-End::Controllers]{Premi::Front-End::Controllers}
\end{figure}
\begin{itemize}
	\item \textbf{Descrizione}: Il package serve a far comunicare le view con il model in entrambe le direzioni ovvero rendere visibili gli aggiornamenti del model nella view e viceversa aggiornare il model con le informazioni provenienti dalla view.
	\item \textbf{Interazione con altri package}:
	\begin{itemize}
		\item Vi è un'interazione con il package Premi::Front-End::Views in quanto i controller sono utilizzati per iniziallizzare le view e per aggiornare il model del frontend con le modifiche avvenute in esse;
		\item Vi è un'interazione con il package Premi::Front-End::Model poiché in esso vengono depositati e prelevati i dati necessari alle view richieste;
		\item Vi è un'interazione con il package Premi::Front-End::Services responsabile del recupero e salvataggio delle informazioni e dell'invocazione di precise procedure dall'esterno mediante chiamate REST.
	\end{itemize}

\end{itemize}

\subsubsection*{Classi contenute}
\begin{itemize}

	\item Premi::Front-End::Controllers::ProjectsCtrl:
	\begin{itemize}
		\item \textbf{Descrizione}: classe con lo scopo di fornire le operazioni necessarie alla visualizzazione e gestione dei progetti relativi ad un utente.
		\item \textbf{Relazioni con altre classi}:
		\begin{itemize}
			\item Premi::Front-End::Services::ProjectService;
			\item Premi::Front-End::Views::Projects.
		\end{itemize}
	\end{itemize}
	\item  Premi::Front-End::Controllers::ProjectCtrl:
	\begin{itemize}
		\item \textbf{Descrizione}: classe con lo scopo di fornire le operazioni necessarie alla gestione di un progetto.
		\item \textbf{Relazioni con altre classi}:
		\begin{itemize}
			\item Premi::Front-End::Services::ProjectService;
			\item Premi::Front-End::Views::Project.
		\end{itemize}
	\end{itemize}
	\item  Premi::Front-End::Controllers::SlideEditorCtrl:
	\begin{itemize}
		\item \textbf{Descrizione}: classe di suporto all'editor di una presentazione fornendo le operazioni necessarie alla composizione e modifica di una slide. Essa permette di assemblare e modificare una \gls{slide} e i metodi che effettueranno le operazioni sui componenti.
		\item \textbf{Relazioni con altre classi}:
		\begin{itemize}
			\item Premi::Front-End::Views::SlideEditor;
			\item Premi::Front-End::Controllers::PresentationEditorCtrl;
			\item Premi::Front-End::Services::SlideService;
			\item Premi::Front-End::Controllers::ComponentCtrl;
			\item Premi::Front-End::Controllers::ChartCtrl;
			\item Premi::Front-End::Controllers::ImageCtrl;
			\item Premi::Front-End::Controllers::RealTimeDataCtrl;
			\item Premi::Front-End::Controllers::TableCtrl;
			\item Premi::Front-End::Controllers::TextCtrl.
		\end{itemize}
	\end{itemize}
	\item  Premi::Front-End::Controllers::PresentationEditorCtrl:
	\begin{itemize}
		\item \textbf{Descrizione}: classe con lo scopo di fornire le operazioni necessarie alla gestione dell'editor di una presentazione. Essa permette di assemblare e modificare una \gls{slide} e i metodi che effettueranno le operazioni sui componenti.
		\item \textbf{Relazioni con altre classi}:
		\begin{itemize}
			\item Premi::Front-End::Services::PresentationService;
			\item Premi::Front-End::Views::Presentation;
			\item Premi::Front-End::Controllers::SlideEditorCtrl.
		\end{itemize}
	\end{itemize}
	\item  Premi::Front-End::Controllers::InfographicEditorCtrl:
	\begin{itemize}
		\item \textbf{Descrizione}: classe con lo scopo di fornire le operazioni necessarie alla gestione dell'editor di una inforgrafica. Essa permette di assemblare e modificare una \gls{slide} e i metodi che effettueranno le operazioni sui componenti.
		\item \textbf{Relazioni con altre classi}:
		\begin{itemize}
			\item Premi::Front-End::Services::InfographicService;
			\item Premi::Front-End::Views::InfographicEditor.
		\end{itemize}
	\end{itemize}
	\item  Premi::Front-End::Controllers::ComponentCtrl:
	\begin{itemize}
		\item \textbf{Descrizione}: classe con lo scopo di fornire le operazioni necessarie alla gestione di un componente di una \gls{slide}.
		\item \textbf{Relazioni con altre classi}:
			\begin{itemize}
				\item Premi::Front-End::Model::Component;
				\item Premi::Front-End::Controllers::ChartCtrl;
				\item Premi::Front-End::Controllers::ImageCtrl;
				\item Premi::Front-End::Controllers::RealTimeDataCtrl;
				\item Premi::Front-End::Controllers::TableCtrl;
				\item Premi::Front-End::Controllers::TextCtrl.
			\end{itemize}
	\end{itemize}
	
	
	
	
	
	
	
	
	
	
	
	
	
	
	\item  Premi::Front-End::Controllers::ChartCtrl:
		\begin{itemize}
			\item \textbf{Descrizione}: classe con lo scopo di fornire le operazioni necessarie alla gestione di un grafico.
			\item \textbf{Relazioni con altre classi}:
				\begin{itemize}
					\item Premi::Front-End::Model::Component;
					\item Premi::Front-End::Model::Chart;
					\item Premi::Front-End::Controllers::ComponentCtrl;
					\item Premi::Front-End::Controllers::SlideEditorCtrl;
					\item Premi::Front-End::Services::ChartService.
				\end{itemize}
		\end{itemize}
		
	\item  Premi::Front-End::Controllers::ImageCtrl:
			\begin{itemize}
				\item \textbf{Descrizione}: classe con lo scopo di fornire le operazioni necessarie alla gestione di una immagine.
				\item \textbf{Relazioni con altre classi}:
					\begin{itemize}
						\item Premi::Front-End::Model::Component;
						\item Premi::Front-End::Model::Image;
						\item Premi::Front-End::Controllers::ComponentCtrl;
						\item Premi::Front-End::Controllers::SlideEditorCtrl;
						\item Premi::Front-End::Services::ImageService.
					\end{itemize}
			\end{itemize}
			
	\item  Premi::Front-End::Controllers::RealTimeDataCtrl:
			\begin{itemize}
				\item \textbf{Descrizione}: classe con lo scopo di fornire le operazioni necessarie alla gestione di un componente con aggiornamento in tempo reale.
				\item \textbf{Relazioni con altre classi}:
					\begin{itemize}
						\item Premi::Front-End::Model::Component;
						\item Premi::Front-End::Model::RealTimeData;
						\item Premi::Front-End::Controllers::ComponentCtrl;
						\item Premi::Front-End::Controllers::SlideEditorCtrl;
						\item Premi::Front-End::Services::RealTimeDataService.
					\end{itemize}
			\end{itemize}
			
	\item  Premi::Front-End::Controllers::TableCtrl:
			\begin{itemize}
				\item \textbf{Descrizione}: classe con lo scopo di fornire le operazioni necessarie alla gestione di una tabella.
				\item \textbf{Relazioni con altre classi}:
					\begin{itemize}
						\item Premi::Front-End::Model::Component;
						\item Premi::Front-End::Model::TableCtrl;
						\item Premi::Front-End::Controllers::ComponentCtrl;
						\item Premi::Front-End::Controllers::SlideEditorCtrl;
						\item Premi::Front-End::Services::TableService.
					\end{itemize}
			\end{itemize}
			
	\item  Premi::Front-End::Controllers::TextCtrl:
			\begin{itemize}
				\item \textbf{Descrizione}: classe con lo scopo di fornire le operazioni necessarie alla gestione di un componente di tipo testuale.
				\item \textbf{Relazioni con altre classi}:
					\begin{itemize}
						\item Premi::Front-End::Model::Component;
						\item Premi::Front-End::Model::TextCtrl;
						\item Premi::Front-End::Controllers::ComponentCtrl;
						\item Premi::Front-End::Controllers::SlideEditorCtrl;
						\item Premi::Front-End::Services::TextService.
					\end{itemize}
			\end{itemize}
	
	
	
	
	
	
	
	
	
	
	
	
	
	
	
	\item  Premi::Front-End::Controllers::PresentationCtrl:
	\begin{itemize}
		\item \textbf{Descrizione}: classe con lo scopo di fornire le operazioni necessarie alla gestione della visualizzazione di una presentazione. Essa permette di assemblare e configurare una presentazione e di rendere disponibile la visualizzazione in modalità classica o presentatore.
		\item \textbf{Relazioni con altre classi}:
		\begin{itemize}
			\item Premi::Front-End::Services::PresentationService;
			\item Premi::Front-End::Views::Presentation.
		\end{itemize}
	\end{itemize}
	\item  Premi::Front-End::Controllers::AuthenticationCtrl
		\begin{itemize}
		\item \textbf{Descrizione}: classe con lo scopo di fornire le operazioni necessarie all'autenticazione nel sistema, registrazione,accredito e recupero password di un utente.
		\item \textbf{Relazioni con altre classi}:
		\begin{itemize}
			\item Premi::Front-End::Services::AuthenticationService;
			\item Premi::Front-End::Views::InfographicEditor;
			\item Premi::Front-End::Views::PresentationEditor;
			\item Premi::Front-End::Views::SignUpEditor;
			\item Premi::Front-End::Views::LogIn;
			\item Premi::Front-End::Views::Project;
			\item Premi::Front-End::Model::User.
		\end{itemize}
	\end{itemize}
	\item  Premi::Front-End::Controllers::HomePageCtrl:
	\begin{itemize}
		\item \textbf{Descrizione}: classe con lo scopo di fornire le operazioni necessarie al corretto funzionamento della Home Page. In particolare si occupa della gestione delle funzionalità di ricerca di progetti e presentazioni mediante il titolo oppure il nome dell'utente proprietario.
		\begin{itemize}
			\item Premi::Front-End::Services::ProjectService;
			\item Premi::Front-End::Views::HomePage.
		\end{itemize}
	\end{itemize}
	\item  Premi::Front-End::Controllers::HelpCtrl:
	\begin{itemize}
		\item \textbf{Descrizione}: 
			classe con lo scopo di fornire le operazioni necessarie alla gestione dell'aiuto agli utenti sottoforma di tour\footnote{Un tour è una esplorazione guidata delle funzionalità basilari del sistema al fine di permettere all'utente di familiarizzare velocemente con esso.} e suggerimenti.
		\item \textbf{Relazioni con altre classi}:
			\begin{itemize}
				\item Premi::Front-End::Views::SlideEditor;
				\item Premi::Front-End::Views::InfographicEditor;
				\item Premi::Front-End::Views::Projects;
				\item Premi::Front-End::Views::Search.
			\end{itemize}
			
	\end{itemize}
	\item  Premi::Front-End::Controllers::MyAccountCtrl:
		\begin{itemize}
			\item \textbf{Descrizione}: 
				classe con lo scopo di fornire le operazioni necessarie alla visualizzazione e gestione del profilo di un utente che si è autenticato nel sistema.
			\item \textbf{Relazioni con altre classi}:	
			\begin{itemize}
				\item Premi::Front-End::Services::AuthenticationService::UserService.
				\item Premi::Front-End::Views::MyAccount.
			\end{itemize}
		\end{itemize}
\end{itemize}
\newpage

			\newpage

		\subsection{Premi::Front-End::Views}
			\subsubsection*{Informazioni sul package}
\begin{figure}[h]
	\centering
	\includegraphics[width=0.9\linewidth]{img/front-end_views}
	\caption[Premi::Front-End::Views]{Premi::Front-End::Views}
\end{figure}
Il package contiene gli elementi per creare la parte grafica del \gls{front-end}, la visualizzazione delle pagine e dell'editor del progetto.

\subsubsection*{Classi contenute:}
\begin{itemize}

	\item Premi::Front-End::Views::HomePageView:
	\begin{itemize}
		\item \textbf{Descrizione}: classe per la gestione della pagina iniziale del sito.
	\end{itemize}

	\item Premi::Front-End::Views::LoginView:
	\begin{itemize}
		\item \textbf{Descrizione}: classe per la gestione della pagina per l'accesso al sito e il recupero delle proprie credenziali.
	\end{itemize}

	\item Premi::Front-End::Views::SignUpView:
	\begin{itemize}
		\item \textbf{Descrizione}: classe per la gestione della pagina per la registrazione al sito.
	\end{itemize}

	\item Premi::Front-End::Views::ProjectsView:
	\begin{itemize}
		\item \textbf{Descrizione}: classe per la gestione della pagina contenente i progetti creati da un utente.
	\end{itemize}

	\item Premi::Front-End::Views::HelpView:
	\begin{itemize}
		\item \textbf{Descrizione}: classe per la gestione della guida all'applicazione.
	\end{itemize}

	\item Premi::Front-End::Views::PresentationView:
	\begin{itemize}
		\item \textbf{Descrizione}: classe per la gestione della visualizzazione della presentazione.
	\end{itemize}

	\item Premi::Front-End::Views::PresentationEditorView:
	\begin{itemize}
		\item \textbf{Descrizione}: classe per la gestione della pagina dell'editor della presentazione.
	\end{itemize}

	\item Premi::Front-End::Views::SlideEditorView:
	\begin{itemize}
		\item \textbf{Descrizione}: classe per la gestione della pagina dell'editor di una \gls{slide};
		\item \textbf{Relazioni con altre classi}:
		\begin{itemize}
			\item Premi::Front-End::Views::ComponentView.
		\end{itemize}
	\end{itemize}

	\item Premi::Front-End::Views::ComponentsView:
	\begin{itemize}
		\item \textbf{Descrizione}: classe per la gestione grafica degli elementi che è possibile includere in una \gls{slide}.
	\end{itemize}

	\item Premi::Front-End::Views::InfographicView:
	\begin{itemize}
		\item \textbf{Descrizione}: classe per la gestione dell pagina per l'\gls{infografica}.
	\end{itemize}

\end{itemize}

			\newpage
					
		\subsection{Premi::Front-End::Model}
				\subsubsection*{Informazioni sul package}
		\begin{figure}[h]
			\centering
			\includegraphics[width=0.9\linewidth]{img/front-end_model}
			\caption[Premi::Front-End::Model]{Premi::Front-End::Model}
		\end{figure}
		Il package serve per mantenere i dati relativi al \textit{\gls{front-end}} e tutta la loro logica di \gls{business}.

	\subsubsection*{Classi contenute}
		\begin{itemize}
		 \item Premi::Front-End::Model::Projects:
			\begin{itemize}
				\item \textbf{Descrizione}: classe per la gestione di una collezione di progetti.Un progetto racchiude una presentazione e zero o più infografiche.
				\item \textbf{Relazioni con altre classi}:
				\begin{itemize}
					\item Premi::Front-End::Model::Project.
				\end{itemize}
			\end{itemize}
		\item  Premi::Front-End::Model::Project: 
			 \begin{itemize}
				\item \textbf{Descrizione}: classe per la gestione di un progetto.
				\item \textbf{Relazioni con altre classi}:
				\begin{itemize}
					\item Premi::Front-End::Model::Presentation;
					\item Premi::Front-End::Model::Infographic.
				\end{itemize}
			\end{itemize}
		 \item  Premi::Front-End::Model::Infographic:
			\begin{itemize}
				\item \textbf{Descrizione}: classe per la gestione di una \gls{infografica}. Un'\gls{infografica} ha il compito di raggruppare più \gls{slide} in un \gls{template} grafico scelto dall'utente in un ordine impostabile di volta in volta.
				\item \textbf{Relazioni con altre classi}:
				\begin{itemize}
					\item Premi::Front-End::Model::Slide.
				\end{itemize}
			\end{itemize}
		 \item   Premi::Front-End::Model::Presentation:
			\begin{itemize}
				\item \textbf{Descrizione}: classe per la gestione di una presentazione. Una presentazione raggruppa più \gls{slide}. Per la visualizzazione delle presentazioni è stato scelto di utilizzare il \gls{framework} \gls{Reveal.js} che permette di avere una visualizzazione a griglia, di conseguenza una presentazione deve memorizzare anche le coordinate delle sue \gls{slide}.
				\item \textbf{Relazioni con altre classi}:
				\begin{itemize}
					\item Premi::Front-End::Model::Slide.
				\end{itemize}
			\end{itemize}
		 \item Premi::Front-End::Model::Slide: Classe per la gestione di una \gls{slide}.
			\begin{itemize}
				\item \textbf{Descrizione}: classe per la gestione di una \gls{slide}.
				\item \textbf{Relazioni con altre classi}:
				\begin{itemize}
					\item Premi::Front-End::Model::Component.
				\end{itemize}
			\end{itemize}
		 \item  Premi::Front-End::Model::Component: 
			\begin{itemize}
				\item \textbf{Descrizione}: Classe astratta concretizzata ed estesa dalle varie componenti implementando il pattern \textit{composite} per fare si che elementi foglia e collezione vengano trattati allo stesso modo. Nello specifico, una tabella rappresenta un aggregato di altre componenti.
			\end{itemize}
		 \item  Premi::Front-End::Model::Text:
			\begin{itemize}
				\item \textbf{Descrizione}: classe per la gestione di un elemento testuale e delle sue proprietà di formattazione. Concretizza ed estende Premi::Front-End::Model::Component.
				\item \textbf{Relazioni con altre classi}:
				\begin{itemize}
					\item Premi::Front-End::Model::Component.
				\end{itemize}
			\end{itemize}
		 \item  Premi::Front-End::Model::Image:
			\begin{itemize}
				\item \textbf{Descrizione}: lasse per la gestione di un elemento di tipo immagine.
				\item \textbf{Relazioni con altre classi}:
				\begin{itemize}
					\item Premi::Front-End::Model::Component.
				\end{itemize}
			 \end{itemize}
		 \item  Premi::Front-End::Model::Table:
			\begin{itemize}
				\item \textbf{Descrizione}: classe per la gestione di una tabella. Una tabella può contenere altre componenti che concretizzano la classe Premi::Front-End::Model::Component.
				\item \textbf{Relazioni con altre classi}:
				\begin{itemize}
					\item Premi::Front-End::Model::Component.
				\end{itemize}
			 \end{itemize}
		 \item  Premi::Front-End::Model::RealTimeData:
			\begin{itemize}
				\item \textbf{Descrizione}: classe per la gestione di componenti che si aggiornano in tempo reale con cadenza personalizzabile.
				\item \textbf{Relazioni con altre classi}:
				\begin{itemize}
					\item Premi::Front-End::Model::Component.
				\end{itemize}
			 \end{itemize}
		 \item  Premi::Front-End::Model::Chart: 
			\begin{itemize}
				\item \textbf{Descrizione}: classe per la gestione dei dati necessari per disegnare un grafico.
				\item \textbf{Relazioni con altre classi}:
				\begin{itemize}
					\item Premi::Front-End::Model::Component.
				\end{itemize}
			 \end{itemize}
		 \item  Premi::Front-End::Model::User:
			\begin{itemize}
				\item \textbf{Descrizione}: classe per la gestione degli utenti.
			 \end{itemize}

		 \end{itemize}

			\newpage

		\subsection{Premi::Front-End::Directives}
			\subsubsection*{Informazioni sul package}
\begin{figure}[h]
	\centering
	\includegraphics[width=0.9\linewidth]{img/front-end_directives}
	\caption[Premi::Front-End::Directives]{Premi::Front-End::Directives}
\end{figure}
Il package contiene gli elementi per creare le direttive del \gls{front-end}, ovvero per creare gli oggetti che consentono di richiamare le view e di collegarle ai rispettivi controller.

\subsubsection*{Classi contenute}
\begin{itemize}
	\item Premi::Front-End::Directives::HomePage:
	\begin{itemize}
		\item \textbf{Descrizione}: classe per la gestione della componente grafica della home page e del rispettivo controller, permettendo di raggiungere i link per la ricerca di un progetto, il login al sito e la registrazione ad esso.
	\end{itemize}

    \item Premi::Front-End::Directives::Projects:
	\begin{itemize}
		\item \textbf{Descrizione}: classe per la gestione della componente grafica della pagina relativa ai progetti creati da un utente e del rispettivo controller. Permette di aprire il progetto desiderato.
	\end{itemize}

    \item Premi::Front-End::Directives::Help:
	\begin{itemize}
		\item \textbf{Descrizione}: classe per la gestione della componente grafica dell'help al sito e del rispettivo controller, permettendo di visualizzare la guida all'utilizzo dell'applicazione
	\end{itemize}

    \item Premi::Front-End::Directives::Project:
	\begin{itemize}
		\item \textbf{Descrizione}: classe per la gestione della componente grafica della pagina relativa al progetto aperto e del rispettivo controller.
	\end{itemize}

    \item Premi::Front-End::Directives::Presentation:
	\begin{itemize}
		\item \textbf{Descrizione}: classe per la gestione della visualizzazione della presentazione una volta che è stata creata e del rispettivo controller. Permette poi di interagire con il sistema fornendo tutti i comandi necessari allo scopo.
	\end{itemize}

    \item Premi::Front-End::Directives::PresentationEditor:
	\begin{itemize}
		\item \textbf{Descrizione}: classe per la gestione della componente grafica relativa alla modifica di una presentazione e del rispettivo controller. Richiama al suo interno la direttiva per la modifica della singola slide.
	\end{itemize}

    \item Premi::Front-End::Directives::SlideEditor:
	\begin{itemize}
		\item \textbf{Descrizione}: classe per la gestione della componente grafica relativa alla modifica di una slide e del rispettivo controller.
	\end{itemize}

    \item Premi::Front-End::Directives::SlideEditorPanel:
	\begin{itemize}
		\item \textbf{Descrizione}: classe per la gestione della componente grafica relativa al pannello di modifica di un componente durante la modifica di una slide e del rispettivo controller. Dà accesso alle proprietà di ogni componente.
	\end{itemize}

    \item Premi::Front-End::Directives::Infographic:
	\begin{itemize}
		\item \textbf{Descrizione}: classe per la gestione della componente grafica relativa alla visualizzazione di un'\gls{infografica} e del rispettivo controller. Permette di accedere al tool di modifica dell'\gls{infografica}.
	\end{itemize}

    \item Premi::Front-End::Directives::InfographicEditor:
	\begin{itemize}
		\item \textbf{Descrizione}: classe per la gestione della componente grafica relativa alla creazione e modifica di un'\gls{infografica} e del rispettivo controller.
	\end{itemize}

\end{itemize}

			\newpage

		\subsection{Premi::Front-End::Services}
			\subsubsection*{Informazioni sul package}
\begin{figure}[h]
	\centering
	\includegraphics[width=0.9\linewidth]{img/front-end_services}
	\caption[Premi::Front-End::Services]{Premi::Front-End::Services}
\end{figure}
Il package contiene gli elementi per creare i service del \gls{front-end}, ovvero per creare gli oggetti che consentono di interfacciarsi con le API \gls{REST} del \gls{back-end}, popolare il model del client e richiedere l'esecuzione di altre operazioni necessarie al server.

\subsubsection*{Classi contenute}
\begin{itemize}

    \item Premi::Front-End::Services::AuthenticationService:
	\begin{itemize}
		\item \textbf{Descrizione}: classe per la gestione dell'autenticazione e della registrazione di un utente.
	\end{itemize}

    \item Premi::Front-End::Services::ProjectService:
	\begin{itemize}
		\item \textbf{Descrizione}: classe per la gestione di un progetto e dei suoi componenti (presentazione e infografiche).
	\end{itemize}

    \item Premi::Front-End::Services::PresentationService:
	\begin{itemize}
		\item \textbf{Descrizione}: classe per la gestione di una presentazione e dei suoi dati.
	\end{itemize}

    \item Premi::Front-End::Services::SlideService:
	\begin{itemize}
		\item \textbf{Descrizione}: classe per la gestione di una slide, della sua modifica e degli elementi presenti in essa.
	\end{itemize}

    \item Premi::Front-End::Services::InfographicService:
	\begin{itemize}
		\item \textbf{Descrizione}: classe per la gestione di una infografica e della sua modifica.
	\end{itemize}

\end{itemize}

			\newpage
