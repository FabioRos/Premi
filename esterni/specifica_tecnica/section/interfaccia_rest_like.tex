\newpage
\subsection{Interfaccia REST-like}
Si é scelto di utilizzare uno stile \gls{REST-like} per quanto riguarda l'interfaccia della componente \gls{back-end} dell'applicativo \PROGETTO, cioè basato sullo stile \gls{REST} ma modificato per permettere l'autenticazione e l'utilizzo di determinate operazioni. Più precisamente il comportamento dell'interfaccia con cui si accede agli elementi della collection può considerarsi \gls{REST} all'interno di una sessione utente, ovvero dall'operazione di login fino a quella di logout. Le motivazioni di tale scelta sono le seguenti:
\begin{itemize}
	\item Semplicità di utilizzo;
	\item Semplicità di integrazione con i \gls{framework} esistenti (\gls{Angular.js});
	\item Indipendenza dal \gls{linguaggio di programmazione} utilizzato.
	\end{itemize}
	\gls{REST} utilizza un aggregato di dati con un nome (\gls{URI}) e una rappresentazione su cui é possibile invocare operazioni \gls{CRUD} tramite la seguente configurazione:
	
	\begin{table}[h]
		\begin{tabular}{|p{0.2\textwidth}|p{0.35\textwidth}|p{0.35\textwidth}|}
			\toprule
			
			\textbf{Risorsa} & \textbf{URI della collection} \smallbreak
			es. http://site.com/users  & \textbf{URI di un utente} \smallbreak
			es. http:/site.com/users/\{id\} \\
			
			\midrule
			\textbf{GET} & Fornisce informazioni sui membri della collection. & Fornisce una rappresentazione dell'elemento della collection indicato. \\ \midrule
			\textbf{PUT} & Non utilizzata. & Sostituisce l'elemento della collection indicato. Se non esiste lo crea. \\  \midrule
			\textbf{POST} & Crea un nuovo elemento nella collection. La \gls{URI} del nuovo elemento é generata in automatico e solitamente viene restituita dall'operazione. & Non utilizzata. \\ \midrule
			\textbf{DELETE} & Non utilizzata. & Cancella l'elemento della collection indicato.  \\ \midrule
			
			
			\end{tabular}\\
			\caption{Tabella configurazioni REST}
			
			\end{table}
			
Si é deciso di scegliere il formato \gls{JSON} come formato di rappresentazione dei dati poiché si integra perfettamente con i \gls{framework} utilizzati e con il linguaggio \gls{Javascript}.
