\subsection{Interfaccia REST}
Di seguito sono elencate le risorse REST associate al tipo di metodo che è possibile richiedere su di esse e i permessi richiesti per poter effettuare la richiesta.
Le tipologie di permessi sono:
\begin{itemize}
	\item Utente non autenticato: risorsa che può essere richiesta da qualsiasi utente;
	\item Utente autenticato: risorsa che può essere richiesta solo da utenti autenticati;
	\item Utente proprietario: risorsa che può essere richiesta solo da utenti proprietari di un progetto.
\end{itemize}

\begin{table}[h]
	\begin{tabular}{|p{0.5\textwidth}|p{0.15\textwidth}|p{0.35\textwidth}|}
		\toprule
		
		\textbf{Chiamata} & \textbf{Tipo risorsa}  & \textbf{Tipo utente} \\
		\bottomrule
	\end{tabular}\\	
\end{table}

\begin{table}[h]
	\begin{tabular}{|p{0.5\textwidth}|p{0.15\textwidth}|p{0.35\textwidth}|}
		\toprule
		\textbf{/login}	& \textbf{POST} & \textbf{Utente non autenticato} \\ \midrule
		\multicolumn{3}{|c|}{Crea una nuova sessione associata all'utente, corrisponde all'azione di login.} \\
		\bottomrule
	\end{tabular}\\
	\par\bigskip
	
	\begin{tabular}{|p{0.5\textwidth}|p{0.15\textwidth}|p{0.35\textwidth}|}
		\toprule
		\textbf{/logout} & \textbf{DELETE} & \textbf{Utente autenticato, Utente proprietario} \\ \midrule
		\multicolumn{3}{|c|}{Elimina la sessione associata all'utente, corrisponde all'azione di logout.} \\
		\bottomrule
	\end{tabular}\\
	\par\bigskip
	
	\begin{tabular}{|p{0.5\textwidth}|p{0.15\textwidth}|p{0.35\textwidth}|}
		\toprule
		\textbf{/register} & \textbf{POST} & \textbf{Utente non autenticato} \\ \midrule
		\multicolumn{3}{|c|}{Crea una richiesta di registrazione.} \\
		\bottomrule
	\end{tabular}\\
	\par\bigskip
	
	\begin{tabular}{|p{0.5\textwidth}|p{0.15\textwidth}|p{0.35\textwidth}|}
		\toprule
		\textbf{/forgotpassword} & \textbf{POST} & \textbf{Utente non autenticato} \\ \midrule
		\multicolumn{3}{|c|}{Crea una richiesta di recupero password.} \\
		\bottomrule
	\end{tabular}\\
	\par\bigskip
	
	\begin{tabular}{|p{0.5\textwidth}|p{0.15\textwidth}|p{0.35\textwidth}|}
		\toprule
		\textbf{/search} & \textbf{GET} & \textbf{Utente non autenticato, Utente autenticato, Utente proprietario} \\ \midrule
		\multicolumn{3}{|c|}{Restituisce i risultati di una ricerca.} \\
		\bottomrule
	\end{tabular}\\
	\par\bigskip
	
	\begin{tabular}{|p{0.5\textwidth}|p{0.15\textwidth}|p{0.35\textwidth}|}
		\toprule
		\textbf{/projects} & \textbf{GET} & \textbf{Utente proprietario} \\ \midrule
		\multicolumn{3}{|c|}{Restituisce la lista di tutti i progetti del relativo utente.} \\
		\bottomrule
		\textbf{/projects} & \textbf{POST} & \textbf{Utente proprietario} \\ \midrule
		\multicolumn{3}{|c|}{Crea un nuovo progetto.} \\
		\bottomrule
	\end{tabular}
\end{table}
\newpage

\begin{table}[H]
	\begin{tabular}{|p{0.5\textwidth}|p{0.15\textwidth}|p{0.35\textwidth}|}
		\toprule
		\textbf{/projects/project\{id\}} & \textbf{GET} & \textbf{Utente proprietario} \\ \midrule
		\multicolumn{3}{|c|}{Restituisce i dati del progetto con id project\{id\}.} \\ \midrule
		\textbf{/projects/project\{id\}} & \textbf{PUT} & \textbf{Utente proprietario} \\ \midrule
		\multicolumn{3}{|c|}{Modifica i dati del progetto con id project\{id\}.} \\ \midrule
		\textbf{/projects/project\{id\}} & \textbf{DELETE} & \textbf{Utente proprietario} \\ \midrule
		\multicolumn{3}{|c|}{Elimina il progetto con id project\{id\}.} \\
		\bottomrule
	\end{tabular}
	\\ \par\bigskip
	
	\begin{tabular}{|p{0.5\textwidth}|p{0.15\textwidth}|p{0.35\textwidth}|}
		\toprule
		\textbf{/projects/project\{id\}/infographic} & \textbf{GET} & \textbf{Utente proprietario} \\ \midrule
		\multicolumn{3}{|c|}{Restituisce i dati dell'infografica del progetto con id project\{id\}.} \\
		\bottomrule
		\textbf{/projects/project\{id\}/infographic} & \textbf{POST} & \textbf{Utente proprietario} \\ \midrule
		\multicolumn{3}{|c|}{Crea un'infografica relativa al progetto con id project\{id\}.} \\
		\bottomrule
		\textbf{/projects/project\{id\}/infographic} & \textbf{DELETE} & \textbf{Utente proprietario} \\ \midrule
		\multicolumn{3}{|c|}{Elimina l'infografica relativa al progetto con id project\{id\}.} \\
		\bottomrule
	\end{tabular}\\
	\par\bigskip
	
	\begin{tabular}{|p{0.5\textwidth}|p{0.15\textwidth}|p{0.35\textwidth}|}
		\toprule
		\textbf{/projects/project\{id\}/presentation} & \textbf{GET} & \textbf{Utente proprietario} \\ \midrule
		\multicolumn{3}{|c|}{Restituisce i dati della presentazione del progetto con id project\{id\}.} \\
		\bottomrule
		\textbf{/projects/project\{id\}/presentation} & \textbf{POST} & \textbf{Utente proprietario} \\ \midrule
		\multicolumn{3}{|c|}{Crea una presentazione relativa al progetto con id project\{id\}.} \\
		\bottomrule
		\textbf{/projects/project\{id\}/presentation} & \textbf{DELETE} & \textbf{Utente proprietario} \\ \midrule
		\multicolumn{3}{|c|}{Elimina la presentazione relativa al progetto con id project\{id\}.} \\
		\bottomrule
	\end{tabular}\\
	\par\bigskip
	
	\begin{tabular}{|p{0.5\textwidth}|p{0.15\textwidth}|p{0.35\textwidth}|}
		\toprule
		\textbf{/projects/project\{id\}/presentation
			/slide\{id\}/components} & \textbf{GET} & \textbf{Utente proprietario} \\ \midrule
		\multicolumn{3}{|p{1.0\textwidth}|}{Restituisce la lista dei componenti della slide con id slide\{id\} nella presentazione del progetto con id project\{id\}.} \\
		\bottomrule
	\end{tabular}\\
	\par\bigskip
	\begin{tabular}{|p{0.5\textwidth}|p{0.15\textwidth}|p{0.35\textwidth}|}
		\toprule
		\textbf{/projects/project\{id\}/presentation
			/slide\{id\}/components/component\{id\}} & \textbf{POST} & \textbf{Utente proprietario} \\ \midrule
		\multicolumn{3}{|p{1.0\textwidth}|}{Crea la componente di id component\{id\} nella slide con id slide\{id\} della presentazione relativa al progetto con id project\{id\}.} \\
		\bottomrule
		\textbf{/projects/project\{id\}/presentation
			/slide\{id\}/components/component\{id\}} & \textbf{PUT} & \textbf{Utente proprietario} \\ \midrule
		\multicolumn{3}{|p{1.0\textwidth}|}{Modifica la componente di id component\{id\} nella slide con id slide\{id\} della presentazione relativa al progetto con id project\{id\}.} \\
		\bottomrule
		\textbf{/projects/project\{id\}/presentation
			/slide\{id\}/components/component\{id\}} & \textbf{DELETE} & \textbf{Utente proprietario} \\ \midrule
		\multicolumn{3}{|p{1.0\textwidth}|}{Elimina la componente di id component\{id\} nella slide con id slide\{id\} della presentazione relativa al progetto con id project\{id\}.} \\
		\bottomrule
	\end{tabular}	
\end{table}