	\subsubsection*{Informazioni sul package}
		\begin{figure}[h]
			\centering
			\includegraphics[width=0.9\linewidth]{img/front-end_model}
			\caption[Premi::Front-End::Model]{Premi::Front-End::Model}
		\end{figure}
		Il package serve per mantenere i dati relativi al \textit{\gls{front-end}} e tutta la loro logica di \gls{business}.

	\subsubsection*{Classi contenute}
		\begin{itemize}
		 \item Premi::Front-End::Model::Projects:
			\begin{itemize}
				\item \textbf{Descrizione}: classe per la gestione di una collezione di progetti.Un progetto racchiude una presentazione e zero o più infografiche.
				\item \textbf{Relazioni con altre classi}:
				\begin{itemize}
					\item Premi::Front-End::Model::Project.
				\end{itemize}
			\end{itemize}
		\item  Premi::Front-End::Model::Project: 
			 \begin{itemize}
				\item \textbf{Descrizione}: classe per la gestione di un progetto.
				\item \textbf{Relazioni con altre classi}:
				\begin{itemize}
					\item Premi::Front-End::Model::Presentation;
					\item Premi::Front-End::Model::Infographic.
				\end{itemize}
			\end{itemize}
		 \item  Premi::Front-End::Model::Infographic:
			\begin{itemize}
				\item \textbf{Descrizione}: classe per la gestione di una \gls{infografica}. Un'\gls{infografica} ha il compito di raggruppare più \gls{slide} in un \gls{template} grafico scelto dall'utente in un ordine impostabile di volta in volta.
				\item \textbf{Relazioni con altre classi}:
				\begin{itemize}
					\item Premi::Front-End::Model::Slide.
				\end{itemize}
			\end{itemize}
		 \item   Premi::Front-End::Model::Presentation:
			\begin{itemize}
				\item \textbf{Descrizione}: classe per la gestione di una presentazione. Una presentazione raggruppa più \gls{slide}. Per la visualizzazione delle presentazioni è stato scelto di utilizzare il \gls{framework} \gls{Reveal.js} che permette di avere una visualizzazione a griglia, di conseguenza una presentazione deve memorizzare anche le coordinate delle sue \gls{slide}.
				\item \textbf{Relazioni con altre classi}:
				\begin{itemize}
					\item Premi::Front-End::Model::Slide.
				\end{itemize}
			\end{itemize}
		 \item Premi::Front-End::Model::Slide: Classe per la gestione di una \gls{slide}.
			\begin{itemize}
				\item \textbf{Descrizione}: classe per la gestione di una \gls{slide}.
				\item \textbf{Relazioni con altre classi}:
				\begin{itemize}
					\item Premi::Front-End::Model::Component.
				\end{itemize}
			\end{itemize}
		 \item  Premi::Front-End::Model::Component: 
			\begin{itemize}
				\item \textbf{Descrizione}: Classe astratta concretizzata ed estesa dalle varie componenti implementando il pattern \textit{composite} per fare si che elementi foglia e collezione vengano trattati allo stesso modo. Nello specifico, una tabella rappresenta un aggregato di altre componenti.
			\end{itemize}
		 \item  Premi::Front-End::Model::Text:
			\begin{itemize}
				\item \textbf{Descrizione}: classe per la gestione di un elemento testuale e delle sue proprietà di formattazione. Concretizza ed estende Premi::Front-End::Model::Component.
				\item \textbf{Relazioni con altre classi}:
				\begin{itemize}
					\item Premi::Front-End::Model::Component.
				\end{itemize}
			\end{itemize}
		 \item  Premi::Front-End::Model::Image:
			\begin{itemize}
				\item \textbf{Descrizione}: lasse per la gestione di un elemento di tipo immagine.
				\item \textbf{Relazioni con altre classi}:
				\begin{itemize}
					\item Premi::Front-End::Model::Component.
				\end{itemize}
			 \end{itemize}
		 \item  Premi::Front-End::Model::Table:
			\begin{itemize}
				\item \textbf{Descrizione}: classe per la gestione di una tabella. Una tabella può contenere altre componenti che concretizzano la classe Premi::Front-End::Model::Component.
				\item \textbf{Relazioni con altre classi}:
				\begin{itemize}
					\item Premi::Front-End::Model::Component.
				\end{itemize}
			 \end{itemize}
		 \item  Premi::Front-End::Model::RealTimeData:
			\begin{itemize}
				\item \textbf{Descrizione}: classe per la gestione di componenti che si aggiornano in tempo reale con cadenza personalizzabile.
				\item \textbf{Relazioni con altre classi}:
				\begin{itemize}
					\item Premi::Front-End::Model::Component.
				\end{itemize}
			 \end{itemize}
		 \item  Premi::Front-End::Model::Chart: 
			\begin{itemize}
				\item \textbf{Descrizione}: classe per la gestione dei dati necessari per disegnare un grafico.
				\item \textbf{Relazioni con altre classi}:
				\begin{itemize}
					\item Premi::Front-End::Model::Component.
				\end{itemize}
			 \end{itemize}
		 \item  Premi::Front-End::Model::User:
			\begin{itemize}
				\item \textbf{Descrizione}: classe per la gestione degli utenti.
			 \end{itemize}

		 \end{itemize}
