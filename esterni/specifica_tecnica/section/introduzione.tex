\subsection{Scopo del documento}
	Il documento ha lo scopo di definire l'architettura generale e i \gls{design pattern} da utilizzare secondo i quali verrà sviluppato il software del progetto Premi.
	
\subsection{Scopo del prodotto}
Lo scopo del progetto è realizzare un software per un sistema di rappresentazione di \gls{slide} sfruttando la tecnologia  \gls{HTML5}. Lo scopo principale è quello di creare un prodotto che sia di qualità comparabile, in prestazioni, funzionalità ed effetti visivi, ai maggiori concorrenti già presenti sul mercato (Prezi, Powerpoint, Keynote, Impress, ...).

\subsection{Glossario}
Per prevenire ed evitare qualsiasi dubbio e per permettere una maggiore chiarezza e comprensione del testo su termini ambigui, abbreviazioni e acronimi utilizzati nei vari documenti, essi sono stati raccolti nel \textit{Glossario v2.0.0} nel quale si possono trovare tutte le informazioni desiderate.
Al fine di rendere subito evidente un termine presente nel \textit{Glossario}, esso verrà marcato con il pedice \G\footnote{Per le istruzioni si rimanda al documento \textit{Norme di Progetto v2.0.0} .}.

\subsection{Riferimenti}

\subsubsection{Normativi}
	\begin{itemize}
		\item \textbf{Analisi dei Requisiti:} \textit{Analisi dei Requisiti v2.0.0};
		\item \textbf{Norme di Progetto:} \textit{Norme di Progetto v2.0.0}.
	\end{itemize}
	
\subsubsection{Informativi}
