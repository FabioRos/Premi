In questa sezione verranno descritte le tecnologie che saranno usate nel progetto.


\subsubsection{AJAX}
È una tecnica di sviluppo software che permette di realizzare applicazioni web interattive tramite chiamate asincrone tra client e server.
La scelta di AJAX è dovuta alla necessità di scambiare dati con il server e di aggiornare porzioni di pagine web senza dover ricaricare l'intera pagina.

\subsubsection{HTML5}
È un linguaggio di markup per la strutturazione delle pagine web, pubblicato come W3C Recommendation da ottobre 2014.
I vantaggi offerti da questa tecnologia sono il supporto canvas per la creazione di figure e animazioni utilizzando il linguaggio JavaScript. Ciò permette di avere una superficie dove disegnare la nostra presentazione e di realizzare effetti grafici. Inoltre è ampiamente diffuso sia in ambiente \textit{desktop} sia in ambiente \textit{mobile}.

\subsubsection{PHP}
È un linguaggio di programmazione interpretato, il quale interprete è disponibile con licenza open-source.
L'uso di questa tecnologia è motivato dalla necessità di avere un linguaggio lato server che ci permette di interfacciarci con il database e di generare degli script di web-scraping da utilizzare per l'estrapolazione e l'elaborazione di dati ottenuti in real-time da sorgenti esterne in quanto, a causa della same-origin-policy, sarebbe impossibile ottenerli dal lato client. Inoltre, PHP è particolarmente adatto a rispondere alle chiamate in AJAX da parte del client.

\subsubsection{Angular.js}
È un framework che permette di realizzare applicazioni web con l'obiettivo di semplificare lo sviluppo e il test delle applicazioni favorendo un approccio dichiarativo allo sviluppo client-side. Basato sul pattern MVC, Angular funziona attraverso l'inclusione di tag e attributi addizionali che vengono interpretati come delle direttive.
Il vantaggio di utilizzare questo framework è quindi quello di semplificare lo sviluppo delle pagine web.

\subsubsection{Reveal.js}
È un framework che permette di creare delle presentazioni interattive in HTML.
L'utilizzo di Reveal.js è stato scelto per la sua strutturazione della presentazione che si adatta alla nostra visione del progetto, soprattutto per quanto riguarda il modo di effettuare la presentazione.

\subsubsection{Mustache}

\subsubsection{JQuery}
È una libreria di funzioni Javascript per le applicazioniweb, che si propone come obiettivo quello di semplificare la manipolazione, la gestione degli eventi e l'animazione delle pagine HTML. La sintassi di JQuery è studiata per semplificare la navigazione dei documenti, la selezione degli elementi DOM, creare animazioni, gestire eventi e implementare funzionalità AJAX.

\subsubsection{MongoDB}

\subsubsection{CSS3}

\subsubsection{Bootstrap}