In questa sezione verranno descritte le tecnologie che saranno usate nel progetto.

\subsection{AJAX}
È una tecnica di sviluppo software che permette di realizzare applicazioni web interattive tramite chiamate asincrone tra client e server.
La scelta di \gls{AJAX} è dovuta alla necessità di scambiare dati con il server e di aggiornare porzioni di pagine web senza dover ricaricare l'intera pagina.

\subsection{HTML5}
È un \gls{linguaggio di markup} per la strutturazione delle pagine web, pubblicato come \gls{W3C} Recommendation da ottobre 2014.
I vantaggi offerti da questa tecnologia sono il supporto canvas per la creazione di figure e animazioni utilizzando il linguaggio \gls{JavaScript}. Ciò permette di avere una superficie dove disegnare la nostra presentazione e di realizzare effetti grafici. Inoltre è ampiamente diffuso sia in ambiente \textit{desktop} sia in ambiente \textit{mobile}.

\subsection{PHP}
È un \gls{linguaggio di programmazione} interpretato, il quale interprete è disponibile con licenza open-source.
L'uso di questa tecnologia è motivato dalla necessità di avere un linguaggio lato server che ci permette di interfacciarci con il \gls{database} e di generare degli script di \gls{web-scraping} da utilizzare per l'estrapolazione e l'elaborazione di dati ottenuti in \gls{real-time} da sorgenti esterne in quanto, a causa della \gls{same-origin-policy}, sarebbe impossibile ottenerli dal lato client. Inoltre, \gls{PHP} è particolarmente adatto a rispondere alle chiamate in \gls{AJAX} da parte del client.

\subsection{Angular.js}
È un \gls{framework} che permette di realizzare applicazioni web con l'obiettivo di semplificare lo sviluppo e il test delle applicazioni favorendo un approccio dichiarativo allo sviluppo client-side. Basato sul pattern \gls{MVC}, \gls{Angular} funziona attraverso l'inclusione di tag e attributi addizionali che vengono interpretati come delle direttive.
Il vantaggio di utilizzare questo \gls{framework} è quindi quello di semplificare lo sviluppo delle pagine web.

\subsection{Reveal.js}
È un \gls{framework} che permette di creare delle presentazioni interattive in \gls{HTML}.
L'utilizzo di \gls{Reveal.js} è stato scelto per la sua strutturazione della presentazione che si adatta alla nostra visione del progetto, soprattutto per quanto riguarda il modo di effettuare la presentazione.

\subsection{Mustache}
È un sistema di \gls{template} web implementato per diversi linguaggi, tra i quali anche \gls{PHP} e \gls{Javascript}. È un sistema logic-less, in quanto non prevede l'utilizzo di strutture di controllo.
È stato scelto per la creazione delle \gls{infografiche} in seguito al suggerimento da parte del proponente.

\subsection{JQuery}
È una libreria di funzioni \gls{Javascript} per le applicazioni web, che si propone come obiettivo quello di semplificare la manipolazione, la gestione degli eventi e l'animazione delle pagine \gls{HTML}. La sintassi di \gls{JQuery} è studiata per semplificare la navigazione dei documenti, la selezione degli elementi DOM, creare animazioni, gestire eventi e implementare funzionalità \gls{AJAX}.

\subsection{MongoDB}
È un DBMS non relazionale, orientato ai documenti. Non utilizza tabelle, ma documenti in \gls{JSON} con schema dinamico, rendendo l'integrazione di dati più semplice e veloce.
È stato scelto di utilizzare questo \gls{DBMS} in quanto il modo in cui crea il \gls{database} permette di adattarsi molto bene alla struttura che avranno le \gls{slide} della presentazione.

\subsection{CSS3}
È un linguaggio usato per definire la formattazione di documenti \gls{HTML}, XHTML e XML ad esempio i siti web e relative pagine web, pubblicato come \gls{W3C} Recommendation.
È stato scelto in quanto è il linguaggio standard per la formattazione dello stile delle pagine web.

\subsection{Bootstrap}
È una raccolta di strumenti per la creazione di siti e applicazioni web. Contiene modelli di progettazione basati su \gls{HTML} e \gls{CSS}, sia per la tipografia che per le varie componenti dell'interfaccia.
È stato scelto di usare questi strumenti per lo sviluppo dell'applicazione per lo stile offerto e per il supporto fornito nella creazione della grafica di \gls{front-end}.

\subsection{Laravel}
Laravel è un framework per applicazioni web con una sintassi espressiva. Lo scopo di Laravel è proprio quello di facilitare ed automatizzare tutte quelle operazioni che, normalmente, sono tra le più ripetitive: dall'autenticazione al routing, passando per gestione delle sessioni e caching.
