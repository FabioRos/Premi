In AngularJS la variabile \$scope è utilizzata per indicare il contesto in cui vengono salvati i dati di un’applicazione (il model) ed in cui vengono valutate le espressioni utilizzate nella view.
Per la natura di AngularJS, in cui esiste una stretta correlazione fra view e model, ogni elemento di una view è associato ad un oggetto di scope .
Sono qui di seguito riportati i principali esempi di oggetti di \$scope.
\begin{itemize}
    \item \textbf{ \$scope.canvas: }
	\begin{itemize}
	      \item Variabile che contiene l'oggetto dalla libreria fabric.js per la gestione di un canvas interattivo.
	\end{itemize}
    \item \textbf{ \$scope.components: }
	\begin{itemize}
	      \item Variabile che contiene una lista di tutte le tipologie di oggetti che possono essere inseriti nelle slide.
	\end{itemize}
    \item \textbf{ \$scope.user: }
	\begin{itemize}
	      \item Variabile che contiene le informazioni relative all'utente che sta utilizzando il sistema.
	\end{itemize}
    \item \textbf{ \$scope.objectSelected: }
	\begin{itemize}
	      \item Variabile che contiene l'oggetto selezionato all'interno della slide in modalità editor presentazioni. Può assumere il valore  "\textit{group}" per permettere la selezione multipla. 
	\end{itemize}
    \item \textbf{ \$scope.update(): }
	\begin{itemize}
	      \item Funzione con il compito di aggiornare il canvas a seguito di un cambiamento nel model.
	\end{itemize}
    \item \textbf{ \$scope.toggleBold(\textit{obj}): }
	\begin{itemize}
	      \item Funzione con il compito di impostare la proprietà "grassetto" dell'oggetto "\textit{obj}" passato per argomento.
	\end{itemize}
     \item \textbf{ \$scope.togglItalic(\textit{obj}): }
	\begin{itemize}
	      \item Funzione con il compito di impostare la proprietà "corsivo" dell'oggetto "\textit{obj}" passato per argomento.
	\end{itemize}
     \item \textbf{ \$scope.toggleUnderlined(\textit{obj}): }
	\begin{itemize}
	      \item Funzione con il compito di impostare la proprietà "sottolineato" dell'oggetto "\textit{obj}" passato per argomento.
	\end{itemize}
     \item \textbf{ \$scope.addText(\textit{text}): }
	\begin{itemize}
	      \item Funzione con il compito di aggiungere in una posizione di default del canvas un oggetto di tipo testuale inizializzato con il valore "\textit{text}" passato per argomento.
	\end{itemize}
      \item \textbf{ \$scope.availableFonts: }
	\begin{itemize}
	      \item Variabile che contiene una lista di tutte le tipologie di carattere utilizzabili per gli oggetti di tipo testuale.
	\end{itemize}
    \item \textbf{ \$scope.addImage(\textit{path}): }
	\begin{itemize}
	      \item Funzione con il compito di aggiungere in una posizione di default del canvas un'immagine raggiungibile mediante il percorso "\textit{path}" passato per argomento.
	\end{itemize}
    \item \textbf{ \$scope.updateColor(\textit{path}): }
	\begin{itemize}
	      \item Funzione con il compito di aggiornare nel canvas il colore dell'oggetto testuale selezionato a seguito di un aggiornamento del model.
	\end{itemize}
    \item \textbf{ \$scope.slidesSVG:}
	\begin{itemize}
	      \item Variabile che contiene le slides della presentazione in formato SVG, ognuna con le rispettive coordinate x e y.
	\end{itemize}
    \item \textbf{ \$scope.slidesList:}
	\begin{itemize}
	      \item Variabile che contiene la lista delle slides di una infografica in modalità editor.\footnote{Un'infografica è costruita a partire da un template.}.
	\end{itemize}
    \item \textbf{ \$scope.template:}
	\begin{itemize}
	      \item Variabile che contiene le informazioni di un template di una infografica. La dimesione di un template appartiene ai formati di carta internazionale  epuò accogliere una lista di slide stabilendo i punti in cui andranno a posizionarsi.
	\end{itemize}
\end{itemize}
