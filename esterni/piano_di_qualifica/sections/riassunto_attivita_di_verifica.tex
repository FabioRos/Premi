Prima di effettuare la consegna dei documenti, allo scadere delle \gls{milestone}, vengono effettuate le attività di verifica dei documenti e dei processi.

\subsection{Revisione dei Requisiti}

\noindent I documenti sono stati verificati secondo le procedure descritte nella sezione \ref{sezione 3.6}.
L'analisi statica dei documenti è stata effettuata applicando la tecnica di \textit{\gls{walkthrough}} per controllare la presenza di errori. Una volta riscontrati gli errori si è poi provveduto a segnalarli e correggerli. Gli errori più frequenti sono stati riportati nella sezione lista di controllo presente nelle \textit{Norme di Progetto v2.0.0}. Abbiamo utilizzato la tecnica \textit{\gls{inspection}}, attraverso l'utilizzo della lista di controllo precedentemente compilata. Una volta corretti gli errori è stato applicato il ciclo \gls{PDCA} per migliorare i processi che hanno manifestato gli errori. Infine è stato eseguito il tracciamento Requisiti-Fonti tramite il tracker\footnote{L'utilizzo di tale strumento è descritto nelle \textit{Norme di Progetto v2.0.0 .}} creato. 

\noindent Sono state calcolate le metriche, descritte nella sezione \ref{sezione 3.8}, per i documenti e sono presentati in seguito in appendice \ref{appendice 1}. 

\noindent I processi sono stati verificati controllando l'avanzamento e calcolando le relative metriche, applicando la procedura descritta nella sezione \ref{sezione 3.7} e presentati in seguito in appendice \ref{appendice 2}.

\subsection{Revisione di Progettazione}

Dai risultati emersi dalla correzione della Revisione dei Requisiti sono stati riscontrati diversi errori che si è provveduto a correggere.
Sono state aggiunte alcune sezioni e riorganizzato alcuni documenti e si è redatto il documento riguardante la \textit{Specifica Tecnica}.
Dopo aver apportato le opportune modifiche è applicata la tecnica \textit{\gls{walkthrough}} per l'analisi statica dei documenti ed in modo parziale la tecnica \textit{\gls{inspection}} sui documenti già redatti e solamente incrementati, per controllare la presenza di errori. Tali analisi hanno portato alla rilevazione di alcuni errori che sono stati gestiti e risolti secondo le modalità descritte in questo documento. Una volta effettuata la correzione su tutti i documenti è stato applicato il ciclo \gls{PDCA} per migliorare i processi che hanno manifestato gli errori riscontrati e si è controllato che tutte le metriche calcolate fossero nel range di accettazione. Infine è stato eseguito il tracciamento Requisiti-Componenti. I risultati di tali verifiche sono presentati in seguito in appendice \ref{appendice 3}.

\noindent I processi sono stati verificati controllando l'avanzamento e calcolando le relative metriche, applicando la procedura descritta nella sezione \ref{sezione 3.7} e presentati in seguito in appendice \ref{appendice 4}.

