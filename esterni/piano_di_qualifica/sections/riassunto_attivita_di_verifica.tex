\subsection{Riassunto attività di verifica}

Prima di effettuare la consegna dei documenti, allo scadere della milestone, sono state effettuate le attività di verifica dei documenti e dei processi.

I documenti sono stati verificati secondo le procedure descritte nella sezione 2.6.
L`analisi statica dei documenti \'e stata effettuata applicando la tecnica di \textit{walkthrough} per controllare la presenza di errori. Una volta riscontrati gli errori si \'e poi provveduto a segnalarle correggerli. Gli errori pi\'u frequenti sono stati riportati nella sezione lista di controllo presente nelle Norme di Progetti v1.0.0 relativa alla tecnica \textit{inspection}, si \'e poi applicato il ciclo PDCA per migliorare il processo di verifica.
Si \'e poi provveduto ad utilizzare la tecnica \textit{inspection} utilizzando la lista di controllo precedentemente compilata.
Si sono poi calcolate le metriche, descritte nella sezione 2.7.1, per i documenti.
I processi sono stati verificati applicando la procedura descritta nella sezione 2.9.1.

\subsection{Documenti}

Di seguito viene riportata una tabella con gli indici di Gulpease calcolati per ogni documento terminata la fase di verifica.
Ogni documento deve rispettare le metriche descritte nella sezione 2.7.1.\\

\hspace{1cm}

\begin{center}
\begin{tabular}{|c|c|c|}
\hline 
\textbf{Documento} & \textbf{Valore Indice} & \textbf{Risultato} \\ 
\hline
\textit{Norme di Progetto v.1.0.0} & 74 & \textcolor{green}{\textit{Superato}} \\ 
\textit{Studio di Fattibilit\'a v.1.0.0} & 70 & \textcolor{green}{\textit{Superato}} \\ 
\textit{Piano di Progetto v1.0.0} & 69 & \textcolor{green}{\textit{Superato}} \\ 
\textit{Piano di Qualifica v1.0.0} & 70 & \textcolor{green}{\textit{Superato}} \\ 
\textit{Analisi dei Requisiti v1.0.0} & • & \textcolor{green}{\textit{Superato}} \\ 
\textit{Glossario v.1.0.0} & • & \textcolor{green}{\textit{Superato}} \\ 
\hline 
\end{tabular}

\end{center}