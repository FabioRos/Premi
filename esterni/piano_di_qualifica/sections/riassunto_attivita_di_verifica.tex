Prima di effettuare la consegna dei documenti, allo scadere delle \gls{milestone}, vengono effettuate le attività di verifica dei documenti e dei processi.

\subsection{Revisione dei Requisiti}

\noindent I documenti sono stati verificati secondo le procedure descritte nella sezione 2.6.
L'analisi statica dei documenti è stata effettuata applicando la tecnica di \textit{\gls{walkthrough}} per controllare la presenza di errori. Una volta riscontrati gli errori si è poi provveduto a segnalarli e correggerli. Gli errori più frequenti sono stati riportati nella sezione lista di controllo presente nelle \textit{Norme di Progetto v1.0.0} relativa alla tecnica \textit{\gls{inspection}}, si è poi applicato il ciclo \gls{PDCA} per migliorare il processo di verifica. Si è poi provveduto ad utilizzare la tecnica \textit{\gls{inspection}} utilizzando la lista di controllo precedentemente compilata. Si sono infine calcolate le metriche, descritte nella sezione 2.7.1, per i documenti. I processi sono stati verificati applicando la procedura descritta nella sezione 2.9.1.

\subsection{Revisione di Progettazione}

Dai risultati emersi dalla correzione della Revisione dei Requisiti sono emersi diversi errori che si è provveduto a correggere.
Sono state aggiunte alcune sezioni e riorganizzato alcuni documenti e si è redatto il documento riguardante la specifica tecnica.
Dopo aver apportato le opportune modifiche si è applicata la tecnica \textit{\gls{inspection}} per controllare la presenza di errori.

\subsection{Documenti}

Di seguito viene riportata una tabella con gli indici di \gls{Gulpease} calcolati per ogni documento terminata la fase di verifica. Ogni documento deve rispettare le metriche descritte nella sezione 2.7.1 .\\

\hspace{1cm}

\begin{center}
\begin{tabular}{|c|c|c|}
\hline 
\textbf{Documento} & \textbf{Valore Indice} & \textbf{Risultato} \\ 
\hline
\textit{Norme di Progetto v.1.0.0} & 74 & \textcolor{green}{\textit{Superato}} \\ 
\textit{Studio di Fattibilità v.1.0.0} & 70 & \textcolor{green}{\textit{Superato}} \\ 
\textit{Piano di Progetto v1.0.0} & 69 & \textcolor{green}{\textit{Superato}} \\ 
\textit{Piano di Qualifica v1.0.0} & 70 & \textcolor{green}{\textit{Superato}} \\ 
\textit{Analisi dei Requisiti v1.0.0} & 79 & \textcolor{green}{\textit{Superato}} \\ 
\textit{Glossario v.1.0.0} & 66 & \textcolor{green}{\textit{Superato}} \\ 
\hline 
\end{tabular}

\end{center}

\subsection{Processi}

Di seguito  viene riportata una tabella con i valori di Schedule Variance e Budget Variance calcolati per ogni fase.

\vspace{3mm}

\begin{center}

\begin{tabular}{|c|c|c|}
\toprule
\multicolumn{3}{c}{\textbf{Fase1}} \\
\hline
\textbf{Documento} & \textbf{SV} & \textbf{BV} \\ 
\hline
\textit{Norme di Progetto v2.0.0} & 0 & 0 \\ 
\textit{Studio di Fattibilità v2.0.0} & 0 & 0 \\ 
\textit{Piano di Progetto v2.0.0} & 0 & 0 \\ 
\textit{Piano di Qualifica v2.0.0} & 0 & 0 \\ 
\textit{Analisi dei Requisiti v2.0.0} & 0 & 0 \\ 
\textit{Glossario v2.0.0} & 0 & 0 \\ 
\hline
\end{tabular}
\end{center}
\vspace{3mm}
\begin{center}
\begin{tabular}{|c|c|c|}
\toprule
\multicolumn{3}{c}{\textbf{Fase2}} \\
\hline 
\textbf{Documento} & \textbf{SV} & \textbf{BV} \\ 
\hline
\textit{Norme di Progetto v2.0.0} & 0 & 0 \\ 
\textit{Studio di Fattibilità v2.0.0} & 0 & 0 \\ 
\textit{Piano di Progetto v2.0.0} & 0 & 0 \\ 
\textit{Piano di Qualifica v2.0.0} & 0 & 0 \\ 
\textit{Analisi dei Requisiti v2.0.0} & 0 & 0 \\ 
\textit{Glossario v2.0.0} & 0 & 0 \\ 
\hline
\end{tabular}
\end{center}
\vspace{3mm}
\begin{center}
\begin{tabular}{|c|c|c|}
\toprule
\multicolumn{3}{c}{\textbf{Fase3}} \\
\hline
\textbf{Documento} & \textbf{SV} & \textbf{BV} \\ 
\hline
\textit{Norme di Progetto v2.0.0} & 0 & 0 \\ 
\textit{Studio di Fattibilità v2.0.0} & 0 & 0 \\ 
\textit{Piano di Progetto v2.0.0} & 0 & 0 \\ 
\textit{Piano di Qualifica v2.0.0} & 0 & 0 \\ 
\textit{Analisi dei Requisiti v2.0.0} & 0 & 0 \\ 
\textit{Glossario v2.0.0} & 0 & 0 \\ 
\hline
\end{tabular}
\end{center}
\vspace{3mm}
\begin{center}
\begin{tabular}{|c|c|c|}
\toprule
\multicolumn{3}{c}{\textbf{Fase4}} \\
\hline
\textbf{Documento} & \textbf{SV} & \textbf{BV} \\ 
\hline
\textit{Norme di Progetto v2.0.0} & 0 & 0 \\ 
\textit{Studio di Fattibilità v2.0.0} & 0 & 0 \\ 
\textit{Piano di Progetto v2.0.0} & 0 & 0 \\ 
\textit{Piano di Qualifica v2.0.0} & 0 & 0 \\ 
\textit{Analisi dei Requisiti v2.0.0} & 0 & 0 \\ 
\textit{Glossario v2.0.0} & 0 & 0 \\ 
\hline
\end{tabular}
\end{center}
\vspace{3mm}
\begin{center}
\begin{tabular}{|c|c|c|}
\toprule
\multicolumn{3}{c}{\textbf{Fase5}} \\
\hline
\textbf{Documento} & \textbf{SV} & \textbf{BV} \\ 
\hline
\textit{Norme di Progetto v2.0.0} & 0 & 0 \\ 
\textit{Studio di Fattibilità v2.0.0} & 0 & 0 \\ 
\textit{Piano di Progetto v2.0.0} & 0 & 0 \\ 
\textit{Piano di Qualifica v2.0.0} & 0 & 0 \\ 
\textit{Analisi dei Requisiti v2.0.0} & 0 & 0 \\ 
\textit{Glossario v2.0.0} & 0 & 0 \\ 
\hline
\end{tabular}

\end{center}