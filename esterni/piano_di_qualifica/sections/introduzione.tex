\subsection{Scopo del documento}
Il \textit{Piano di Qualifica} ha lo scopo di fissare le strategie che il gruppo intende adottare, al fine di perseguire gli obbiettivi di qualità, sia di processo che di prodotto. Per questo motivo è necessaria una costante verifica sulle attività svolte. Così facendo si permette di trovare possibili incongruenze e anomalie per poter intervenire in maniera tempestiva ed efficace.
\subsection{Scopo del prodotto}
Lo scopo del prodotto è la realizzazione di un software di presentazione di \textit{\gls{slide}}, utilizzando il linguaggio \gls{HTML5}, che funzioni sia su dispositivi desktop che mobile. Si richiede la realizzazione di effetti grafici a supporto dello \textit{storytelling}\footnote{L'arte del raccontare storie impiegata come strategia di comunicazione.} e che sia di livello comparabile con Prezi\footnote{Software di presentazioni.}.
\subsection{Glossario}
Per prevenire ed evitare qualsiasi dubbio e per permettere una maggiore chiarezza e comprensione del testo su termini ambigui, abbreviazioni e acronimi utilizzati nei vari documenti, essi sono stati raccolti nel \textit{Glossario v2.0.0} nel quale si possono trovare tutte le informazioni desiderate.
Al fine di rendere subito evidente un termine presente nel \textit{Glossario}, esso verrà marcato con il pedice \G\footnote{Per le istruzioni si rimanda al documento \textit{Norme di Progetto v2.0.0} .}.
\subsection{Riferimenti}
	\subsubsection{Normativi}
	\begin{itemize}
		\item \textbf{Norme di Progetto:} \textit{Norme di Progetto v2.0.0};
		\item \textbf{Capitolato d'appalto C4:} Premi: software di presentazione "better than Prezzi" \url{http://www.math.unipd.it/~tullio/IS-1/2014/Progetto/C4p.svg#1_0}.
	\end{itemize}
	\subsubsection{Informativi}
	\begin{itemize}
		\item \textbf{Piano di Progetto:} \textit{Piano di Progetto v2.0.0};
		\item \textbf{\gls{Slide} Ingegneria del Software 2014/2015:} \url{http://www.math.unipd.it/~tullio/IS-1/2014/};
		\item \textbf{Indice {\gls{Gulpease}}:} \url{http://it.wikipedia.org/wiki/Indice_Gulpease};
		\item \textbf{Standard ISO/IEC 9126:} \url{http://it.wikipedia.org/wiki/ISO/IEC_9126};
		\item \textbf{Standard ISO/IEC 15504:} \url{http://en.wikipedia.org/wiki/ISO/IEC_15504}.
	\end{itemize}

