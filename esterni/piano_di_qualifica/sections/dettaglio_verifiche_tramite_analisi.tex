\subsection{Analisi}
\subsubsection{Documenti}
\label{appendice 1}

Di seguito viene riportata una tabella con gli indici di \gls{Gulpease} calcolati per ogni documento, una volta terminata la fase di verifica. Ogni documento deve rispettare le metriche descritte nella sezione \ref{sezione 3.8} .\\

\hspace{1cm}

\begin{table}[h]
	\begin{tabular}{|c|c|c|}
		\hline 
		\textbf{Documento} & \textbf{Valore Indice} & \textbf{Risultato} \\ 
		\hline
		\textit{Norme di Progetto v1.0.0} & 74 & \textcolor{green}{\textit{Superato}} \\ 
		\textit{Studio di Fattibilità v1.0.0} & 70 & \textcolor{green}{\textit{Superato}} \\ 
		\textit{Piano di Progetto v1.0.0} & 69 & \textcolor{green}{\textit{Superato}} \\ 
		\textit{Piano di Qualifica v10.0} & 70 & \textcolor{green}{\textit{Superato}} \\ 
		\textit{Analisi dei Requisiti v1.0.0} & 79 & \textcolor{green}{\textit{Superato}} \\ 
		\textit{Glossario v1.0.0} & 66 & \textcolor{green}{\textit{Superato}} \\ 
		\hline 
	\end{tabular}
\caption{Indice Gulpease, Analisi}
\end{table}


\subsubsection{Processi}
\label{appendice 2}
\vspace{3mm}

\begin{table}[h]
	\begin{tabular}{|c|c|c|}
		\toprule
			\textbf{Documento} & \textbf{SV€} & \textbf{BV€} \\ 
		\midrule
		\midrule
			\textit{Norme di Progetto v1.0.0} & -15 & 0 \\ 
			\textit{Studio di Fattibilità v1.0.0} & & -5 \\ 
			\textit{Piano di Progetto v1.0.0} & -15 & -30 \\ 
			\textit{Piano di Qualifica v1.0.0} & -45 & 0 \\ 
			\textit{Analisi dei Requisiti v1.0.0} & 35 & 60 \\ 
			\textit{Glossario v1.0.0} & 0 & 0 \\ 
		\bottomrule
	\end{tabular}
	\caption{Esiti verifica processi, Analisi}
\end{table}

\noindent Complessivamente sono stati registrati:
\begin{itemize}
	\item \textbf{Schedule Variance:} -40 €;
	\item \textbf{Budget Variance:} 25 €.
\end{itemize}

\noindent Da tali valori si può dedurre che i periodi di slack pianificati non erano sufficienti ad avere una schedule variance positiva, al contrario l'organizzazione del gruppo ha portato ad un costo minore in termini di budget variance. Nonostante lo schedule variance sia negativo, esso rimane comunque al di sopra del minimo accettabile di € -142.

\newpage

\subsection{Progettazione Architetturale}
\subsubsection{Documenti}
\label{appendice 3}

Di seguito viene riportata una tabella con gli indici di \gls{Gulpease} calcolati per ogni documento, una volta terminata la fase di verifica. Ogni documento deve rispettare le metriche descritte nella sezione \ref{sezione 3.8} .\\

\hspace{1cm}

\begin{table}[h]
	\begin{tabular}{|c|c|c|}
		\hline 
		\textbf{Documento} & \textbf{Valore Indice} & \textbf{Risultato} \\ 
		\hline
		\textit{Norme di Progetto v2.0.0} & 74 & \textcolor{green}{\textit{Superato}} \\  
		\textit{Piano di Progetto v2.0.0} & 68 & \textcolor{green}{\textit{Superato}} \\ 
		\textit{Piano di Qualifica v2.0.0} & 71 & \textcolor{green}{\textit{Superato}} \\ 
		\textit{Analisi dei Requisiti v2.0.0} & 79 & \textcolor{green}{\textit{Superato}} \\
		\textit{Specifica Tecnica v1.0.0} & 73 & \textcolor{green}{\textit{Superato}} \\ 
		\textit{Glossario v2.0.0} & 80 & \textcolor{green}{\textit{Superato}} \\ 
		\hline 
\end{tabular}
\caption{Indice Gulpease, Progettazione Architetturale}
\end{table}

\subsubsection{Processi}
\label{appendice 4}
\vspace{3mm}

\begin{table}[h]
	\begin{tabular}{|c|c|c|}
		\toprule
			\textbf{Documento} & \textbf{SV€} & \textbf{BV€} \\ 
		\midrule
		\midrule
			\textit{Norme di Progetto v2.0.0} & 0 & 0 \\  
			\textit{Piano di Progetto v2.0.0} & 0 & -11 \\ 
			\textit{Piano di Qualifica v2.0.0} & 20 & 0 \\ 
			\textit{Analisi dei Requisiti v2.0.0} & -50 & -25 \\
			\textit{Specifica Tecnica v1.0.0} & 60 & 25 \\ 
			\textit{Glossario v2.0.0} & 0 & 0 \\ 
		\bottomrule
	\end{tabular}
\caption{Esiti verifica processi, Analisi}
\end{table}

\noindent Complessivamente sono stati registrati:
\begin{itemize}
	\item \textbf{Schedule Variance:} 20 €;
	\item \textbf{Budget Variance:} -11 €.
\end{itemize}

\noindent Da tali valori si può dedurre che i periodi di slack pianificati abbiano aiutato ad avere una schedule variance positiva, al contrario l'inesperienza ad affrontare una progettazione architetturale di un sistema software ha portato ad un costo maggiore in termini di budget variance. Nonostante quest'ultima sia negativa, essa rimane comunque al di sopra del minimo accettabile di € -346.