\subsection{Scopo del documento}
Il \textit{Piano di Qualifica} ha lo scopo di fissare le strategia che il gruppo intende adottare, al fine di perseguire gli obbiettivi di qualità, di processo e di prodotto. Per questo motivo è necessario una costante verifica sulle attività svolte. Cosi facendo si permette di trovare possibili incongruenze e anomalie e intervenire in maniera tempestiva ed efficace.
\subsection{Scopo del prodotto}
Lo scopo del prodotto è la realizzazione di un software di presentazione di "slide", sviluppato in tecnologia HTML5 e che funzioni sia su desktop che su dispositivo mobile. Si richiede di realizzare effetti grafici a supporto dello "storytelling"\footnote{L'arte del raccontare storie impiegata come strategia di comunicazione.} che sia di livello comparabile con Prezzi\footnote{Software di presentazioni.}.
\subsection{Glossario}
Al fine di evitare ogni ambiguità e permettere al lettore una migliore comprensione dei termini tecnici, acronimi, e termini che necessitano di essere chiariti,  essi sono riportati nel documento \textit{Glossario}. 
Ogni occorrenza  di un termine appartenente al \textit{Glossario} è marcata da una "G" in pedice.
\subsection{Riferimenti}
	\subsubsection{Normativi}
	\begin{itemize}
		\item \textbf{Norme di Progetto:} \textit{Norme di Progetto v1.0.0};
		\item \textbf{Capitolato d'appalto C4:} Premi: software di presentazione "better than Prezzi" \url{http://www.math.unipd.it/~tullio/IS-1/2014/Progetto/C4p.svg#1_0}.
	\end{itemize}
	\subsubsection{Informativi}
	\begin{itemize}
		\item \textbf{Piano di Progetto:} \textit{Piano di Progetto v1.0.0};
		\item \textbf{Slide Ingegneria del Software 2014/2015:} \url{http://www.math.unipd.it/~tullio/IS-1/2014/}
		\item \textbf{Indice Gulpease:} \url{http://it.wikipedia.org/wiki/Indice_Gulpease}
		\item \textbf{Standard ISO/IEC 9126:} \url{http://it.wikipedia.org/wiki/ISO/IEC_9126}
		\item \textbf{Standard ISO/IEC 15504:} \url{http://en.wikipedia.org/wiki/ISO/IEC_15504}
	\end{itemize}