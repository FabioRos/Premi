\subsection{Definizione obiettivi}
\subsubsection{Qualità di processo}
Affinchè venga garantita la qualità del prodotto è necessario perseguire la qualità dei processi che lo definiscono. Per garantire questo è stato deciso di adottare lo standard ISO/IEC 15504\footnote{Per approfondimenti consultare la sezione \ref{15504} in appendice A.} denominato SPICE(Software Process Improvement and Capability Determination), il quale fornisce gli strumenti necessari a valutare l'idoneità di questi ultimi.

\noindent Per applicare correttamente questo modello è necessario utilizzare il ciclo di Deming\footnote{Per approfondimenti consultare la sezione \ref{PDCA} in appendice A.}, detto ciclo PDCA, il quale definisce una metodologia di controllo dei processi durante il loro ciclo di vita che permette di migliorarne in modo continuo la qualità.

\subsubsection{Qualità di prodotto}
Al fine di garantire il corretto funzionamento di un prodotto software ed aumentarne il valore commerciale è necessario fissare degli obiettivi qualitativi e garantire che questi vengano effettivamente rispettati.

\noindent Lo standard ISO/IEC 9126\footnote{Per approfondimenti consultare la sezione \ref{9126} in appendice A.} è stato scritto allo scopo di descrivere questi obiettivi e delineare delle metriche capaci di misurare il raggiungimento di tali obiettivi.

\subsection{Procedure di controllo di qualità di processo}
La qualità dei processi sarà garantita dall'applicazione del principio PDCA, descritto nella sezione \ref{PDCA}. Tramite questo principio sarà possibile garantire un miglioramento continuo della qualità di tutti processi, inclusa la verifica, e come conseguenza si otterrà un miglioramento dei prodotti risultanti.

\noindent Per avere il controllo dei processi, e conseguentemente qualità è necessario che:
\begin{itemize}
	\item I processi siano pianificati dettagliatamente;
	\item Nella pianificazione siano ripartite in modo chiaro le risorse;
	\item Vi sia controllo sui processi;
\end{itemize}
\noindent L'attuazione di tali punti è descritta nel dettaglio nel \textit{Piano di Progetto v4.0.0}.

\noindent La qualità dei processi viene inoltre monitorata attraverso l'analisi costante della qualità di prodotto. Un prodotto di bassa qualità indica un processo da migliorare.

\noindent Per migliorare la qualità dei processi si possono utilizzare le metriche descritte nella sezione \ref{sezione 3.7}.

\subsection{Procedure di controllo di qualità di prodotto}
Il controllo di qualità di prodotto sarà garantito da:
\begin{itemize}
	\item \textbf{Quality assurance:} insieme di attività realizzate per garantire il raggiungimento degli obiettivi di qualità. Prevede l'attuazione di tecniche di analisi statica e dinamica, descritte nelle \textit{Norme di Progetto v4.0.0};
	\item \textbf{Verifica:} processo che determina se l'output di una fase è consistente, completo e corretto. La verifica andrà eseguita costantemente durante l'intera durata del progetto. I risultati delle attività di verifica eseguite nei vari periodi di svolgimento del progetto sono riportati in appendice \ref{verifica}.
	\item \textbf{Validazione:} conferma oggettivamente che il sistema risponda ai requisiti.
\end{itemize}

\subsection{Organizzazione}
L'organizzazione della strategia di verifica si basa sull'attuazione di attività di verifica per ogni processo attuato. Per ogni processo realizzato ne viene verificata la qualità oltre a che quella relativa al prodotto ottenuto da esso.\\
l processo di verifica, diverso nelle varie fasi del progetto descritte nel \textit{Piano di Progetto v4.0.0}, sarà il seguente:
\begin{itemize}
	\item \textbf{Analisi:} in questa fase si devono seguire i metodi di verifica, descritti nelle \textit{Norme di Progetto v4.0.0}, sui documenti prodotti;
	\item \textbf{Analisi di Dettaglio:} in questa fase si devono verificare i processi che hanno portato ad un incremento nel \gls{versionamento} dei documenti, prodotti nella fase precedente, seguendo le procedure descritte nelle \textit{Norme di Progetto v4.0.0};
	\item \textbf{Progettazione architetturale:} in questa fase, oltre a verificare i processi che hanno portato ad un incremento nel \gls{versionamento} dei documenti, si andranno a verificare i prodotti ed i processi attuati per l'attività di progettazione architetturale;
	\item \textbf{Progettazione di Dettaglio e Codifica:} in questa fase, oltre a verificare i processi che hanno portato ad un incremento nel \gls{versionamento} dei documenti, si andrà a verificare che ogni requisito sia rintracciabile da uno dei componenti emersi durante la fase di progettazione;
	\item \textbf{Verifica e Validazione:} in questa fase, oltre a verificare i processi che hanno portato ad un incremento nel \gls{versionamento} dei documenti, verrà effettuato il collaudo del prodotto garantendone la correttezza.
\end{itemize} 

\subsection{Pianificazione strategica e temporale}
Gli obiettivi fissati possono essere raggiunti solamente attraverso una buona pianificazione in base alla quale si dovrà agire. Con l'obiettivo di rispettare le scadenze fissate nel \textit{Piano di Progetto v4.0.0} è fondamentale un'attività di verifica ben organizzata e sistematica, pertanto è essenziale, prima di iniziare qualsiasi attività, capirne la struttura ed i contenuti. È inoltre necessaria un'attenta lettura delle \textit{Norme di Progetto v4.0.0} in cui sono descritte le varie metodologie da seguire per l'individuazione e la correzione degli errori. Ogni attività di redazione dei documenti e di codifica deve essere preceduta da uno studio preliminare sulla struttura e sui contenuti degli stessi. Tale attività ha lo scopo di ridurre la possibilità di commettere imprecisioni di natura concettuale e tecnica favorendo l'attività verifica, dove saranno necessari minori interventi di correzione.

\subsection{Responsabilità}
Le responsabilità di tutte le attività di verifica e validazione sono a carico del \textit{Responsabile di Progetto} e dei \textit{Verificatori}. Questi ruoli, durante le varie fasi di progetto, saranno ricoperti da diversi componenti del gruppo, come descritto nel \textit{Piano di Progetto v4.0.0}. Per questo motivo è necessario che tutti i componenti del gruppo siano motivati e incoraggiati ad assumersi le responsabilità per il lavoro svolto e per sviluppare nuovi approcci atti al miglioramento della qualità.

\subsection{Risorse necessarie}
Per assicurarsi che gli obiettivi siano raggiunti e monitorare costantemente lo sviluppo è necessario l'utilizzo sia di risorse umane che tecnologiche. I ruoli che hanno una responsabilità maggiore per l'attività di verifica e validazione sono il \textit{Responsabile di Progetto} e il \textit{Verificatore}. Per una descrizione dettagliata dei ruoli e delle loro responsabilità si rimanda alle \textit{Norme di Progetto v4.0.0}. Inoltre nelle \textit{Norme di Progetto v4.0.0} vengono descritte le risorse tecnologiche, ossia gli strumenti hardware e software necessari alle attività di verifica. Affinché il lavoro delle persone implicate venga agevolato, in particolare del \textit{Verificatore}, si sono predisposti numerosi strumenti automatici garantendo un controllo più semplice e corretto.

\subsection{Attività chiave}
\begin{itemize}
	\item \textbf{Garanzia della qualità:} tutte le attività che sono realizzate all'interno di un sistema e che hanno l'obiettivo di perseguire e soddisfare requisiti di qualità di un servizio. Si tratta di misurazione sistematica, di confronto con uno o più standard di monitoraggio dei processi e \gls{feedback} che garantiscono la prevenzione degli errori; attività a fronte delle quali questi ultimi dovrebbero essere eliminati;
	\item \textbf{Pianificazione della qualità del prodotto:} la selezione di procedure standard appropriate per questo sistema, adattandole per uno specifico progetto software;
	\item \textbf{Controllo della qualità:} la definizione dei processi che assicurano che il team  di sviluppo software segua le procedure e gli standard adottati nel progetto.
\end{itemize}
	
\subsection{Misure e Metriche}
Per garantire il raggiungimento degli obiettivi è necessario fissare delle metriche sulla base delle quali poter misurare i risultati ottenuti dalle varie attività di verifica. È dunque di fondamentale importanza saper quantificare, attraverso delle metriche stabilite a priori, il processo di verifica. Grazie al ciclo di vita adottato, descritto nel \textit{Piano di Progetto v4.0.0}, le metriche incerte ed approssimate si potranno migliorare in modo incrementale.\\
Ogni metrica avrà due caratteristiche fondamentali:
\begin{itemize}
	\item \textbf{Range di accettazione:} intervallo entro il quale il prodotto si può ritenere soddisfacente;
	\item \textbf{Range ottimale:} valore entro il quale dovrebbe arrivare la misurazione.
\end{itemize}

\subsubsection{Metriche per i processi}
\label{sezione 3.7}
Per misurare lo stato di avanzamento dei processi si è scelto di utilizzare indici che ne analizzino i costi e
tempi.
\paragraph{2.9.1.1 Schedule Variance (SV)}
Lo Schedule Variance è un indicatore di efficacia e serve a controllare se si è in linea, in anticipo o in ritardo rispetto alla pianificazione temporale delle attività nella baseline.
Se SV G > 0 significa che il gruppo di lavoro sta producendo con maggior velocità rispetto a quanto pianificato, viceversa se negativo.
Essendo stati inseriti slack durante la pianificazione della baseline dei processi, il valore di tale indice è inizialmente positivo.
Parametri utilizzati:
\begin{itemize}
	\item \textbf{Range-accettazione}: $\left[  \geq - \: costo \: preventivo \: fase * 5 \% \right]$
	\item \textbf{Range-ottimale}: $\left[\geq 0\right]$.
\end{itemize}
\paragraph{2.9.1.2 Budget Variance (BV)}
Il Budget Variance è un indicatore che ha un valore contabile e finanziario e che serve a controllare se alla data corrente si è speso di più o di meno rispetto a quanto pianificato.
Se BV G > 0 significa che l'attuazione del progetto sta consumando il proprio budget con minor velocità rispetto a quanto pianificato, viceversa se negativo.
Parametri utilizzati:
\begin{itemize}
	\item \textbf{Range-accettazione}: $\left[  \geq - \: costo \: preventivo \: fase * 10 \% \right]$
	\item \textbf{Range-ottimale}: $\left[\geq 0\right]$.
\end{itemize}

\subsubsection{Metriche per la documentazione}
\label{sezione 3.8}
Come metrica per i documenti redatti si è deciso di utilizzare l'indice di leggibilità Gulpease. Rispetto ad altri ha il vantaggio di utilizzare la lunghezza delle parole in lettere anziché in sillabe, semplificandone il calcolo automatico. L'indice è tarato sulla lingua italiana e considera due variabili linguistiche: la lunghezza della parola e la lunghezza della frase rispetto al numero delle lettere. \\
\noindent L'indice è calcolato attraverso la seguente formula:\\
\begin{center}
	$89+ \frac{300*\left(numero\:\ delle\:\ frasi \right)-10*\left(numero\:\ delle\:\ lettere\right)}{numero\:\ delle\:\ parole}$
\end{center}
I risultati sono compresi tra 0 e 100, dove 100 indica la leggibilità più alta e 0 la leggibilità più bassa. In generale risulta che i testi con indice:
\begin{itemize}
	\item Inferiore a 80 sono difficili da leggere per chi ha la licenza elementare;
	\item Inferiore a 60 sono difficili da leggere per chi ha la licenza media;
	\item Inferiore a 40 sono difficile da leggere per chi ha il diploma superiore.
\end{itemize}
Basandoci su queste considerazioni è stato scelto di utilizzare:
\begin{itemize}
	\item \textbf{Range di accettazione:} [40 - 100];
	\item \textbf{Range ottimale:} [50 - 100].
\end{itemize}

\subsubsection{Metriche per il software}
\label{sezione 3.9}
Di seguito saranno descritte le metriche che riguardano il software. Data l'inesperienza del gruppo, questa sezione sarà soggetta a modifiche nelle prossime revisioni.
\begin{itemize}
	\item \textbf{Complessità ciclomatica:} misura direttamente il numero di cammini linearmente indipendenti attraverso il grafo di controllo di flusso. Essa viene calcolata utilizzando il grafo di controllo di flusso del programma: i nodi del grafo corrispondono a gruppi indivisibili di istruzioni, mentre gli archi connettono due nodi se il secondo gruppo di istruzioni può essere eseguito immediatamente dopo il primo gruppo. La complessità ciclomatica può, inoltre, essere applicata a singole funzioni, moduli, metodi o classi di un programma.
	\begin{itemize}
		\item \textbf{Range di accettazione:} [1 - 15];
		\item \textbf{Range ottimale:} [1 - 10].
	\end{itemize}
	\item \textbf{Attributi per classe:} un numero elevato di attributi all'interno della classe potrebbe indicare il bisogno di suddividere la classe in più sotto classi.
	\begin{itemize}
		\item \textbf{Range di accettazione:} [0 - 16];
		\item \textbf{Range ottimale:} [3 - 8].
	\end{itemize}
	\item \textbf{Numero di parametri per metodo:} indica il numero di parametri formali di un metodo. Un numero elevato di parametri potrebbe indicare la necessità di ridurre le funzionalità associate a tale metodo.
	\begin{itemize}
		\item \textbf{Range di accettazione:} [0 - 8];
		\item \textbf{Range ottimale:} [0 - 4].
	\end{itemize}
	\item \textbf{Linee di codice per linee di commento:} indica il rapporto tra il numero di linee di commento e il numero di linee di codice. Il codice poco commentato comporta una difficile manutenibilità.
	\begin{itemize}
		\item \textbf{Range di accettazione:} [>20];
		\item \textbf{Range ottimale:} [>30].
	\end{itemize}
	\item \textbf{Numero di livelli di annidamento:} indica il numero di livelli di annidamento dei metodi. Un numero elevato rappresenta un'alta complessità del codice riducendone il livello di astrazione.
	\begin{itemize}
		\item \textbf{Range di accettazione:} [1 - 6];
		\item \textbf{Range ottimale:} [1 - 3].
	\end{itemize}
	\item \textbf{Copertura del codice:} valore che indica la percentuale di codice che viene eseguito durante i test.
	Un valore elevato indica un'alta probabilità che i moduli testati abbiano pochi errori.
	\begin{itemize}
		\item \textbf{Range di accettazione:} [ $\geq$ 40\%];
		\item \textbf{Range ottimale:} [ $\geq$ 60\%].
	\end{itemize}
\end{itemize}

\subsection{Tecniche di analisi}
\subsubsection{Analisi statica}
L'analisi statica è una tecnica di analisi che permette di verificare quanto è stato prodotto evidenziando errori o anomalie. Essa è applicabile sia ai documenti che al codice e può essere eseguita in due modalità distinte ma complementari.

\paragraph{2.10.1.1 Walkthrough}
Si effettua eseguendo una lettura critica a largo spettro. Questa tecnica di norma è utilizzata nelle prime attività del progetto, quando i membri del gruppo non hanno ancora acquisito la necessaria esperienza che permette loro di attuare una verifica più precisa. Tramite l'utilizzo di tale tecnica il \textit{Verificatore} potrà creare una lista di controllo con gli errori che si verificano più frequentemente, in modo da 	migliorare il controllo di tali attività nelle fasi future del progetto.
Per distribuirne il carico di lavoro e affinchè tale attività sia efficace ed efficiente è necessaria la collaborazione di più membri del gruppo. Dopo una fase iniziale di lettura e individuazione degli errori, segue una fase di discussione finalizzata ad esaminare gli errori riscontrati e a proporre strategie adeguate per risolverli. L'ultima fase consiste nell'attuare le correzioni elaborate e a stilare un rapporto delle modifiche effettuate.

\paragraph{2.10.1.2 Inspection}
Questa tecnica consiste nell'analisi mirata di alcune parti di documenti o di codice che sono ritenute essere le maggiori fonti di errore. Deve essere redatta anticipatamente una lista di controllo da parte dei verificatori, creata tramite l'esperienza ottenuta attraverso la tecnica di walkthrough, che deve essere seguita al fine di rendere efficace questo processo. L'inspection è una tecnica molto più rapida del walkthrough poichè non richiede la lettura integrale del codice o di un documento ma solo delle parti ritenute critiche e riportate nella lista di controllo. A differenza della walkthrough questa tecnica è svolta solamente dai verificatori, che si occupano di individuare e correggere gli errori e di scrivere un rapporto che tenga traccia delle modifiche effettuate.

\subsubsection{Analisi dinamica}
L'analisi dinamica si applica solamente al prodotto software e si attua durante l'esecuzione del codice tramite l'uso di test predisposti a verificarne il corretto funzionamento e alla rilevazione eventuali malfunzionamenti o difetti di implementazione.
Condizione strettamente necessaria affinchè questa attività sia utile e generi risultati attendibili è che tali test siano ripetibili: infatti un test che, dato un certo input, produce sempre lo stesso output in uno specifico ambiente è capace di evidenziare problemi e verificare la correttezza del software.
Di conseguenza è necessario definire a priori:
\begin{itemize}
	\item \textbf{Ambiente}: consiste sia nel sistema hardware sia in quello software sui quali si è pianificato l'utilizzo del software prodotto;
	\item \textbf{Specifica}: consiste nello specificare quali sono gli input necessari affinchè si possa effettuare il test e quali output sono attesi;
	\item \textbf{Procedure}: consiste nello specificare come e in che ordine devono essere effettuati i test e come devono essere analizzati i risultati ottenuti.
\end{itemize}

\noindent Sono definibili 4 tipi di test: \textit{test di unità}, \textit{test di sistema}, \textit{test di integrazione}, \textit{test di accettazione}.

\paragraph{2.10.2.1 Test di unità}
Consiste nella verifica di ogni singola unità del software prodotto tramite l'utilizzo di appositi software dedicati. Per unità si intende la più piccola quantità di software che è utile verificare singolarmente e che viene prodotta da un singolo programmatore. Con questi test si vuole accertare il corretto funzionamento dei moduli che compongono il sistema al fine di individuare ed eliminare possibili errori di implementazione commessi dai programmatori.

\paragraph{2.10.2.2 Test di sistema}
Consiste nel validare il software prodotto una volta che lo si ritiene giunto ad una versione definitiva. Tali test deve verificare che la copertura dei requisiti software stabiliti nella fase di Analisi di Dettaglio sia totale.

\paragraph{2.10.2.3 Test di integrazione}
Consiste nella verifica dei componenti di sistema che vengono aggiunti in modo incrementale al software precedentemente prodotto e si prefigge di verificare che la combinazione di più componenti funzioni secondo le aspettative. Questo tipo di test serve per individuare possibili errori rimasti dalla realizzazione dei singoli moduli, dalle modifiche delle interfacce o errori presenti nelle componenti software preesistenti forniti da terze parti che non si conoscono ottimamente. Nel caso in cui manchino delle componenti necessarie all'esecuzione di tali test è necessario aggiungere delle componenti fittizie, in modo da non influenzare negativamente l'esito delle analisi.

\paragraph{2.10.2.4 Test di accettazione}
Si tratta del collaudo finale del software prodotto da parte del proponente. Se il collaudo viene positivamente superato si può procedere al rilascio ufficiale del prodotto sviluppato.
