\subsection{Organizzazione}
L'attività di verifica verrà istanziata per ogni processo attuato. La finalità di ogni attività di verifica è garantire la qualità per ogni processo e prodotto.\\
Il processo di verifica, diverso nelle varie fasi del progetto descritte nel \textit{Piano di Progetto v2.0.0}, sarà il seguente:
\begin{itemize}
	\item \textbf{Analisi:} in questa fase si devono seguire i metodi di verifica, descritti nelle \textit{Norme di Progetto v2.0.0}, sui documenti prodotti;
	\item \textbf{Analisi di Dettaglio:} in questa fase si devono verificare i processi che hanno portato ad un incremento nel \gls{versionamento} dei documenti, prodotti nella fase precedente, seguendo le procedure descritte nelle \textit{Norme di Progetto v2.0.0};
	\item \textbf{Progettazione architetturale:} in questa fase, oltre a verificare i processi che hanno portato ad un incremento nel \gls{versionamento} dei documenti, si andranno a verificare i prodotti ed i processi attuati per l'attività di progettazione architetturale;
	\item \textbf{Progettazione di Dettaglio e Codifica:} in questa fase, oltre a verificare i processi che hanno portato ad un incremento nel \gls{versionamento} dei documenti, si andrà a verificare che ogni requisito sia rintracciabile da uno dei componenti emersi durante la fase di progettazione;
	\item \textbf{Verifica e Validazione:} in questa fase, oltre a verificare i processi che hanno portato ad un incremento nel \gls{versionamento} dei documenti, verrà effettuato il collaudo del prodotto garantendone la correttezza.
\end{itemize} 

\subsection{Pianificazione strategica e temporale}
Gli obiettivi fissati possono essere raggiunti solamente attraverso una buona pianificazione in base alla quale si dovrà agire. Con l'obiettivo di rispettare le scadenze fissate nel \textit{Piano di Progetto v2.0.0} è fondamentale un'attività di verifica ben organizzata e sistematica, pertanto è essenziale, prima di iniziare qualsiasi attività, capirne la struttura ed i contenuti. È inoltre necessaria un'attenta lettura delle \textit{Norme di Progetto v2.0.0} in cui sono descritte le varie metodologie da seguire per l'individuazione e la correzione degli errori.

\subsection{Responsabile}
Le responsabilità di tutte le attività di verifica e validazione sono a carico del \textit{Responsabile di Progetto} e dei \textit{Verificatori}. Questi ruoli, durante le varie fasi di progetto, saranno ricoperti da diversi componenti del gruppo, come descritto nel \textit{Piano di Progetto v2.0.0}. Per questo motivo è necessario che tutti i componenti del gruppo siano motivati e incoraggiati ad assumersi le responsabilità per il lavoro svolto e per sviluppare nuovi approcci atti al miglioramento della qualità.

\subsection{Risorse necessarie}
Per assicurarsi che gli obiettivi siano raggiunti e monitorare costantemente lo sviluppo è necessario l'utilizzo sia di risorse umane che tecnologiche. I ruoli che hanno una responsabilità maggiore per l'attività di verifica e validazione sono il \textit{Responsabile di Progetto} e il \textit{Verificatore}. Per una descrizione dettagliata dei ruoli e delle loro responsabilità si rimanda alle \textit{Norme di Progetto v2.0.0}. Inoltre nelle \textit{Norme di Progetto v2.0.0} vengono descritte le risorse tecnologiche, ossia gli strumenti hardware e software necessari alle attività di verifica. Affinché il lavoro delle persone implicate venga agevolato, in particolare del \textit{Verificatore}, si sono predisposti numerosi strumenti automatici garantendo un controllo più semplice e corretto.

\subsection{Attività chiave}
\begin{itemize}
	\item \textbf{Garanzia della qualità:} tutte le attività che sono realizzate all'interno di un sistema e che hanno l'obiettivo di perseguire e soddisfare requisiti di qualità di un servizio. Si tratta di misurazione sistematica, di confronto con uno o più standard di monitoraggio dei processi e \gls{feedback} che garantiscono la prevenzione degli errori; attività a fronte delle quali questi ultimi dovrebbero essere eliminati;
	\item \textbf{Pianificazione della qualità del prodotto:} la selezione di procedure standard appropriate per questo sistema, adattandole per uno specifico progetto software;
	\item \textbf{Controllo della qualità:} la definizione dei processi che assicurano che il team  di sviluppo software segua le procedure e gli standard adottati nel progetto.
\end{itemize}
Per garantire la qualità dei processi e per garantire una corretta pianificazione delle attività descritte in precedenza, si è deciso di aderire allo standard ISO/IEC 15504 conosciuto anche come SPICE(Software Process Improvement and Capability Determination)\footnote{Per una descrizione dettagliata si rimanda all'appendice \ref{15504} .}.
	
\subsection{Misure e Metriche}
\label{sezione 3.6}
Per garantire il raggiungimento degli obiettivi è necessario fissare delle metriche sulla base delle quali poter misurare i risultati ottenuti dalle varie attività di verifica. È dunque di fondamentale importanza saper quantificare, attraverso delle metriche stabilite a priori, il processo di verifica. Grazie al ciclo di vita adottato, descritto nel \textit{Piano di Progetto v2.0.0}, le metriche incerte ed approssimate si potranno migliorare in modo incrementale.\\
Ogni metrica avrà due caratteristiche fondamentali:
\begin{itemize}
	\item \textbf{Range di accettazione:} intervallo entro il quale il prodotto si può ritenere soddisfacente;
	\item \textbf{Range ottimale:} valore entro il quale dovrebbe arrivare la misurazione.
\end{itemize}


\subsection{Metriche per i processi}
\label{sezione 3.7}
Per misurare lo stato di avanzamento dei processi si è scelto di utilizzare indici che ne analizzino i costi e
tempi.
\subsubsection{Schedule Variance (SV)}
Lo Schedule Variance è un indicatore di efficacia e serve a controllare se si è in linea, in anticipo o in ritardo rispetto alla pianificazione temporale delle attività nella \gls{baseline}.
Se SV G > 0 significa che il gruppo di lavoro sta producendo con maggior velocità rispetto a quanto pianificato, viceversa se negativo.
Essendo stati inseriti \gls{slack} durante la pianificazione della \gls{baseline} dei processi, il valore di tale indice è inizialmente positivo.
Parametri utilizzati:
\begin{itemize}
	\item \textbf{Range-accettazione}: $\left[  \geq - \: costo \: preventivo \: fase * 5 \% \right]$
	\item \textbf{Range-ottimale}: $\left[\geq 0\right]$.
\end{itemize}
\subsubsection{Budget Variance (BV)}
Il Budget Variance è un indicatore che ha un valore contabile e finanziario e che serve a controllare se alla data corrente si è speso di più o di meno rispetto a quanto pianificato.
Se BV G > 0 significa che l'attuazione del progetto sta consumando il proprio budget con minor velocità rispetto a quanto pianificato, viceversa se negativo.
Parametri utilizzati:
\begin{itemize}
	\item \textbf{Range-accettazione}: $\left[  \geq - \: costo \: preventivo \: fase * 10 \% \right]$
	\item \textbf{Range-ottimale}: $\left[\geq 0\right]$.
\end{itemize}

\subsection{Metriche per la documentazione}
\label{sezione 3.8}
Come metrica per i documenti redatti si è deciso di utilizzare l'indice di leggibilità \gls{Gulpease}. Rispetto ad altri ha il vantaggio di utilizzare la lunghezza delle parole in lettere anziché in sillabe, semplificandone il calcolo automatico. L'indice è tarato sulla lingua italiana e considera due variabili linguistiche: la lunghezza della parola e la lunghezza della frase rispetto al numero delle lettere. \\
\noindent L'indice è calcolato attraverso la seguente formula:\\
\begin{center}
	$89+ \frac{300*\left(numero\:\ delle\:\ frasi \right)-10*\left(numero\:\ delle\:\ lettere\right)}{numero\:\ delle\:\ parole}$
\end{center}
I risultati sono compresi tra 0 e 100, dove 100 indica la leggibilità più alta e 0 la leggibilità più bassa. In generale risulta che i testi con indice:
\begin{itemize}
	\item Inferiore a 80 sono difficili da leggere per chi ha la licenza elementare;
	\item Inferiore a 60 sono difficili da leggere per chi ha la licenza media;
	\item Inferiore a 40 sono difficile da leggere per chi ha il diploma superiore.
\end{itemize}
Basandoci su queste considerazioni è stato scelto di utilizzare:
\begin{itemize}
	\item \textbf{Range di accettazione:} [40 - 100];
	\item \textbf{Range ottimale:} [50 - 100].
\end{itemize}
	
\subsection{Metriche per il software}
Di seguito saranno descritte le metriche che riguardano il software. Data l'inesperienza del gruppo, questa sezione sarà soggetta a modifiche nelle prossime revisioni.
\begin{itemize}
	\item \textbf{Complessità ciclomatica:} misura direttamente il numero di cammini linearmente indipendenti attraverso il grafo di controllo di flusso. Essa viene calcolata utilizzando il grafo di controllo di flusso del programma: i nodi del grafo corrispondono a gruppi indivisibili di istruzioni, mentre gli archi connettono due nodi se il secondo gruppo di istruzioni può essere eseguito immediatamente dopo il primo gruppo. La complessità ciclomatica può, inoltre, essere applicata a singole funzioni, moduli, metodi o classi di un programma.
	\begin{itemize}
		\item \textbf{Range di accettazione:} [1 - 15];
		\item \textbf{Range ottimale:} [1 - 10].
	\end{itemize}
	\item \textbf{Attributi per classe:} un numero elevato di attributi all'interno della classe potrebbe indicare il bisogno di suddividere la classe in più sotto classi.
	\begin{itemize}
		\item \textbf{Range di accettazione:} [0 - 16];
		\item \textbf{Range ottimale:} [3 - 8].
	\end{itemize}
	\item \textbf{Numero di parametri per metodo:} indica il numero di parametri formali di un metodo. Un numero elevato di parametri potrebbe indicare la necessità di ridurre le funzionalità associate a tale metodo.
	\begin{itemize}
		\item \textbf{Range di accettazione:} [0 - 8];
		\item \textbf{Range ottimale:} [0 - 4].
	\end{itemize}
	\item \textbf{Linee di codice per linee di commento:} indica il rapporto tra il numero di linee di commento e il numero di linee di codice. Il codice poco commentato comporta una difficile manutenibilità.
	\begin{itemize}
		\item \textbf{Range di accettazione:} [>20];
		\item \textbf{Range ottimale:} [>30].
	\end{itemize}
	\item \textbf{Numero di livelli di annidamento:} indica il numero di livelli di annidamento dei metodi. Un numero elevato rappresenta un'alta complessità del codice riducendone il livello di astrazione.
	\begin{itemize}
		\item \textbf{Range di accettazione:} [1 - 6];
		\item \textbf{Range ottimale:} [1 - 3].
	\end{itemize}
	\item \textbf{Copertura del codice:} valore che indica la percentuale di codice che viene eseguito durante i test.
	Un valore elevato indica un'alta probabilità che i moduli testati abbiano pochi errori.
	\begin{itemize}
		\item \textbf{Range di accettazione:} [ $\geq$ 40\%];
		\item \textbf{Range ottimale:} [ $\geq$ 60\%].
	\end{itemize}
\end{itemize}

