\subsection{Organizzazione}
L'attività di verifica verrà istanziata per ogni processo attuato. La finalità di ogni attività di verifica è garantire la qualità per ogni processo e prodotto.\\
Il processo di verifica, diverso nelle varie fasi del progetto descritte nel \textit{Piano di Progetto v2.0.0}, sarà il seguente:
\begin{itemize}
	\item \textbf{Analisi:} in questa fase si devono seguire i metodi di verifica, descritti nelle \textit{Norme di Progetto v2.0.0}, sui documenti prodotti;
	\item \textbf{Analisi di Dettaglio:} in questa fase si devono verificare i processi che hanno portato ad un incremento nel \gls{versionamento} dei documenti, prodotti nella fase precedente, seguendo le procedure descritte nelle \textit{Norme di Progetto v2.0.0};
	\item \textbf{Progettazione architetturale:} in questa fase, oltre a verificare i processi che hanno portato ad un incremento nel \gls{versionamento} dei documenti, si andranno a verificare i prodotti ed i processi attuati per l'attività di progettazione architetturale;
	\item \textbf{Progettazione di Dettaglio e Codifica:} in questa fase, oltre a verificare i processi che hanno portato ad un incremento nel \gls{versionamento} dei documenti, si andrà a verificare che ogni requisito sia rintracciabile da uno dei componenti emersi durante la fase di progettazione;
	\item \textbf{Verifica e Validazione:} in questa fase, oltre a verificare i processi che hanno portato ad un incremento nel \gls{versionamento} dei documenti, verrà effettuato il collaudo del prodotto garantendone la correttezza.
\end{itemize} 

\subsection{Pianificazione strategica e temporale}
Gli obiettivi fissati possono essere raggiunti solamente attraverso una buona pianificazione in base alla quale si dovrà agire. Con l'obiettivo di rispettare le scadenze fissate nel \textit{Piano di Progetto v2.0.0} è fondamentale un'attività di verifica ben organizzata e sistematica, pertanto è essenziale, prima di iniziare qualsiasi attività, capirne la struttura ed i contenuti. È inoltre necessaria un'attenta lettura delle \textit{Norme di Progetto v2.0.0} in cui sono descritte le varie metodologie da seguire per l'individuazione e la correzione degli errori.

\subsection{Responsabile}
Le responsabilità di tutte le attività di verifica e validazione sono a carico del \textit{Responsabile di Progetto} e dei \textit{Verificatori}. Questi ruoli, durante le varie fasi di progetto, saranno ricoperti da diversi componenti del gruppo, come descritto nel \textit{Piano di Progetto v2.0.0}. Per questo motivo è necessario che tutti i componenti del gruppo siano motivati e incoraggiati ad assumersi le responsabilità per il lavoro svolto e per sviluppare nuovi approcci atti al miglioramento della qualità.

\subsection{Risorse necessarie}
Per assicurarsi che gli obiettivi siano raggiunti e monitorare costantemente lo sviluppo è necessario l'utilizzo sia di risorse umane che tecnologiche. I ruoli che hanno una responsabilità maggiore per l'attività di verifica e validazione sono il \textit{Responsabile di Progetto} e il \textit{Verificatore}. Per una descrizione dettagliata dei ruoli e delle loro responsabilità si rimanda alle \textit{Norme di Progetto v2.0.0}. Inoltre nelle \textit{Norme di Progetto v2.0.0} vengono descritte le risorse tecnologiche, ossia gli strumenti hardware e software necessari alle attività di verifica. Affinché il lavoro delle persone implicate venga agevolato, in particolare del \textit{Verificatore}, si sono predisposti numerosi strumenti automatici garantendo un controllo più semplice e corretto.

\subsection{Attività chiave}
\begin{itemize}
	\item \textbf{Garanzia della qualità:} tutte le attività che sono realizzate all'interno di un sistema e che hanno l'obiettivo di perseguire e soddisfare requisiti di qualità di un servizio. Si tratta di misurazione sistematica, di confronto con uno o più standard di monitoraggio dei processi e \gls{feedback} che garantiscono la prevenzione degli errori; attività a fronte delle quali questi ultimi dovrebbero essere eliminati;
	\item \textbf{Pianificazione della qualità del prodotto:} la selezione di procedure standard appropriate per questo sistema, adattandole per uno specifico progetto software;
	\item \textbf{Controllo della qualità:} la definizione dei processi che assicurano che il team  di sviluppo software segua le procedure e gli standard adottati nel progetto.
\end{itemize}
Per garantire la qualità dei processi e per garantire una corretta pianificazione delle attività descritte in precedenza, si è deciso di aderire allo standard ISO/IEC 15504 conosciuto anche come SPICE(Software Process Improvement and Capability Determination)\footnote{Per una descrizione dettagliata si rimanda all'appendice \ref{15504} .}.
	
\subsection{Misure e Metriche}
\label{sezione 3.6}
Per garantire il raggiungimento degli obiettivi è necessario fissare delle metriche sulla base delle quali poter misurare i risultati ottenuti dalle varie attività di verifica. È dunque di fondamentale importanza saper quantificare, attraverso delle metriche stabilite a priori, il processo di verifica. Grazie al ciclo di vita adottato, descritto nel \textit{Piano di Progetto v2.0.0}, le metriche incerte ed approssimate si potranno migliorare in modo incrementale.\\
Ogni metrica avrà due caratteristiche fondamentali:
\begin{itemize}
	\item \textbf{Range di accettazione:} intervallo entro il quale il prodotto si può ritenere soddisfacente;
	\item \textbf{Range ottimale:} valore entro il quale dovrebbe arrivare la misurazione.
\end{itemize}

