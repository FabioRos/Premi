Di seguito verranno descritti tutti i test di validazione, di sistema e di integrazione che sono stati previsti. Si prevede in futuro l'aggiornamento dei test di unit\'a. Per le tempistiche di aggiornamento dei test si faccia riferimento al \textit{Piano di Progetto v2.0.0?}. La dicitura \textbf{N.A.} contenuta nella colonna \textit{Stato} nelle tabelle sottostanti indica che il test non \'e ancora stato applicato in quanto verr\'a applicato solo successivamente, come descritto nel \textit{Piano di Progetto v2.0.0?}.


\subsection{Test di sistema}
In questa sezione vengono descritti i test di sistema che permettono di verificare il comportamento dinamico del sistema nella sua interezza rispetto ai requisiti descritti nell'\textit{Analisi dei Requisiti v2.0.0?}.
I test di sistema descritti sotto, sono quelli che si riferiscono ai requisisti software individuati e quindi meritevoli di un test.

\subsubsection{Descrizione dei test di sistema}
\begin{center}
	\begin{table}[h]
		\begin{tabular}{|l|p{0.7\textwidth}|l|c|}
			\toprule
			
			\textbf{Test} & \textbf{Descrizione} & \textbf{Stato} & \textbf{Requisito} \\
			
			\midrule
			TS1 & Viene verificato che il sistema registri correttamente un utente & N.A. & R[OBB][F]1\\ \midrule
			TS1.1 & Viene verificato che il sistema sia portabile e correttamente visualizzato su desktop, mobile e tablet & N.A & R[OBB][V]-1.1 \\ \midrule
			TS1.6 & Viene verificato che il sistema registri correttamente un utente attraverso un social network & N.A. & R[OPZ][F]1.6\\ \midrule
			TS2 & Viene verificato che il sistema autentichi correttamente un utente & N.A. & R[OBB][F]2\\  \midrule
			TS3	& Viene verificato che il sistema permetta di effettuare una ricerca di un progetto con le modalità implementate & N.A. & R[OBB][F]3\\ \midrule
			TS4	& Viene verificato che il sistema apra un progetto & N.A. & R[OBB][F]4\\ \midrule
			TS5 & Viene verificato che il sistema generi il PDF del progetto & N.A. & R[OBB][F]5\\ \midrule
			TS6 & Viene verificato che il sistema generi un pacchetto stand-alone per la visualizzazione offline & N.A. & R[OBB][F]6\\ \midrule
			TS7 & Viene verificato che il sistema crei un nuovo progetto & N.A. & R[OBB][F]7\\ \midrule
			TS8 & Viene verificato che il sistema apra un progetto precedentemente creato & N.A. & R[OBB][F]8\\ \midrule
			TS9 & Viene verificato che il sistema permetta la modifica del contenuto di una \gls{slide} & N.A. & R[OBB][F]9\\ \midrule
			TS10 & Viene verificato che il sistema crei un'\gls{infografica} & N.A. & R[DES][F]10\\ \midrule
			TS11 & Viene verificato che il sistema permetta di salvare un progetto & N.A. & R[OBB][F]11\\
			
						
			\bottomrule
			
		\end{tabular}
		\caption{Tabella di tracciamento test di sistema/requisiti}
		
	\end{table}
	
\end{center}

\newpage

\subsection{Test di integrazione}
In questa sezione vengono descritti i test di integrazione da usare per testare le componenti descritte nella progettazione ad alto livello, tali test permettono di verificare la corretta implementazione ed il corretto flusso dei dati all'interno del sistema. \'E stato scelto di utilizzare una strategia di integrazione incrementale, che permette lo sviluppo e la verifica delle componenti in parallelo. Unendo infatti le varie componenti per via incrementale, nella maggior parte dei casi, gli errori riscontrati da un test sono da attribuirsi all'ultima parte aggiunta, inoltre con questa strategia è possibile retrocedere e tornare ad uno stato noto e funzionante del sistema. Si \'e deciso di utilizzare il metodo \gls{bottom-up} per integrare prima le componenti con minori dipendenze funzionali e maggiori funzionalit\'a (che corrispondono alle componenti per garantire i requisiti obbligatori) al fine di ottenere il prima possibile una versione funzionante delle parti obbligatorie del sistema. Agendo in questo modo le componenti strettamente legate a parti obbligatorie vengono testate molte volte contribuendo cos\'i ad abbassare il numero di errori in esse contenute. Si procede poi risalendo l'albero delle dipendenze fino ad arrivare alla componente di alto livello alla quale saranno dedicati gli ultimi test.

\begin{figure}[h]
\centering
\includegraphics[width=1\linewidth]{img/Integrazione_componenti.png}
\caption[Sequenza d'integrazione delle componenti]{Sequenza d'integrazione delle componenti}
\label{fig:Integrazione_componenti}
\end{figure}

\subsection{Descrizione dei test di integrazione}


\begin{center}
	\begin{table}[h]
		\begin{tabular}{|l|p{0.7\textwidth}|l|c|}
		\toprule
			\textbf{Test} & \textbf{Descrizione} & \textbf{Componente} & \textbf{Stato} \\
			\midrule
			TI1 & Test di controllo finale lato utente & Front-End & N.A\\
			\midrule
			TI2 & Test che verifica la corretta visualizzazione delle interfacce & View & N.A\\
			\midrule
			TI3 & Test che verifica la corretta interazione tra i comandi & Controller & N.A\\
			\midrule
			TI4 & Test che verifica il corretto comportamento tra client e server & Back-End & N.A\\
			\midrule
			TI5 & Test che verifica il corretto funzionamento del server & Logic-tier & N.A\\
			\midrule
			TI6 & Test che verifica la corretta interazione tra i comandidel server & Controller & N.A\\
			\midrule
			TI7 & Test che verifica la correttezza dei dati & Back-End::Data-tier & N.A\\
		\bottomrule
		\end{tabular}
		\caption{Tabella di tracciamento test di integrazione}
	\end{table}
\end{center}

\subsection{Tracciamento componenti–test di integrazione}

\begin{center}
	\begin{table}[h]
		\begin{tabular}{|c|c|}
		\toprule
			\textbf{Componente} & \textbf{Test}\\
			\midrule
			Front-End & TI1\\
			\midrule
			Front-End::View & TI2\\
			\midrule
			Front-End::Controller & TI3\\
			\midrule
			Back-End & TI4\\
			\midrule
			Back-End::Logic-tier & TI5\\
			\midrule
			Back-End::Logic-tier::Controller & TI6\\
			\midrule
			Back-End::Data-tier & TI7\\
		\bottomrule
		\end{tabular}
		\caption{Tabella di tracciamento test di integrazione}
	\end{table}
\end{center}

\newpage

\subsection{Test di validazione}
In questa sezione si descrivono i test di validazione che servono per accertarsi che il prodotto realizzato sia conforme alle aspettative.
Per ogni test vengono descritti i vari passi che un utente deve eseguire per testare i requisiti ad esso associati. Per quanto riguarda il tracciamento test di validazione/requisiti \'e riportato in \textit{Analisi dei Requisiti v2.0.0?}.

\subsection{Test TV1}
L'utente vuole verificare che ci si possa iscrivere. \newline
All'utente \'e richiesto di:
\begin{itemize}
	\item Cliccare il bottone per la registrazione (TV1.1);
	\item Inserire un nome utente univoco (TV1.2);
	\item Inserire una password di almeno 8 caratteri (TV1.3);
	\item Reinserire la password (TV1.4);
	\item Inserire il proprio nome (TV1.5);
	\item Inserire il proprio cognome (TV1.6);
	\item Inserire il proprio indirizzo email (TV1.7);
	\item Confermare i dati inseriti cliccando il bottone "Registrati" (TV1.8);
	\item Controllare di aver ricevuto una email di conferma di avvenuta registrazione (TV1.9).
\end{itemize}

\subsection{Test TV2}
L'utente vuole verificare che ci si possa autenticare. \newline
All'utente \'e richiesto di:
\begin{itemize}
	\item Cliccare il bottone per effettuare il login (TV2.1);
	\item Inserire il proprio nome utente (TV2.2);
	\item Inserire la propria password (TV2.3);
	\item Confermare i dati inseriti cliccando il bottone "Accedi" (TV2.4);
	\item Controllare di poter accedere alla propria pagina personale (TV2.5).
\end{itemize}

\subsection{Test TV3}
L'utente vuole verificare di riuscire a visualizzare i risultati di una ricerca. \newline
All'utente \'e richiesto di:
\begin{itemize}
	\item Effettuare una ricerca usando come chiave un nome utente (TV3.1);
	\item Effettuare una ricerca usando come chiave un titolo di un progetto (TV3.2);
	\item Verificare che compaia la lista dei risultati della ricerca (TV3.3).
\end{itemize}

\subsection{Test TV4}
L'utente vuole verificare di riuscire a visualizzare una presentazione. \newline
All'utente \'e richiesto di:
\begin{itemize}
	\item Aprire una presentazione (TV4.1);
	\item Scegliere l'opzione per la visualizzazione come ascoltatore (TV4.2);
	\item Scegliere l'opzione per la visualizzazione come presentatore (TV4.3);
	\item Verificare che la presentazione sia visualizzata correttamente e che ci si possa muovere liberamente tra le slide (TV4.4).
\end{itemize}

\subsection{Test TV5}
L'utente vuole verificare che il PDF creato sia corretto rispetto al progetto. \newline
All'utente \'e richiesto di:
\begin{itemize}
	\item Aprire un progetto (TV5.1);
	\item Scegliere la funzione di generazione del PDF dall'apposito men\'u (TV5.2);
	\item Verificare che il PDF creato corrisponda al progetto selezionato (TV5.3).
\end{itemize}

\subsection{Test TV6}
L'utente vuole verificare che il progetto esportato sia utilizzabile offline. \newline
All'utente \'e richiesto di:
\begin{itemize}
	\item Aprire un progetto (TV6.1);
	\item Scegliere la funzione di esportazione del progetto dall'apposito men\'u (TV6.2);
	\item Scegliere la posizione in cui salvare il progetto in locale (TV6.3);
	\item Avviare la presentazione in locale del progetto salvato (TV6.4);
	\item Verificare che la presentazione funzioni senza errori e che corrisponda al progetto selezionato (TV6.5);
\end{itemize}

\subsection{Test TV7}
L'utente vuole verificare che sia possibile editare un progetto precedentemente creato. \newline
All'utente \'e richiesto di:
\begin{itemize}
	\item Selezionare un progetto (TV7.1);
	\item Selezionare la slide che si vuole modificare (TV7.2);
	\item Scegliere l'elemento che si desidera modificare o inserirne uno di nuovo (TV7.3)
	\item[] All'utente \'e richiesto di:
	\begin{itemize}
		\item Inserire una nuova slide (TV7.3.1);
		\item Inserire un'immagine (TV7.3.2);
		\item Inserire una casella di testo (TV7.3.3);
		\item Inserire dati \gls{real time} (TV7.3.4);
		\item Inserire una tabella (TV7.3.5);
		\item Inserire un grafico (TV7.3.6);
		\item Selezionare un effetto di transizione (TV7.3.7);
		\item Cambiare la dimensione di un elemento (TV7.3.8);
		\item Cambiare la posizione di un elemento (TV7.3.9);
		\item Ruotare un elemento (TV7.3.10);
		\item Rimuovere un elemento (TV7.3.11);
		\item Modificare una tabella o un grafico (TV7.3.12);
		\item Inserire note/parole chiave (TV7.3.13);
	\end{itemize}
	\item Verificare che le modifiche effettuate siano state applicate al progetto tramite visualizzazione dello stesso (TV7.4).
\end{itemize}

\subsection{Test TV8}
L'utente vuole verificare che si possa salvare un progetto. \newline
All'utente \'e richiesto di:
\begin{itemize}
	\item Creare un nuovo progetto o modificare un progetto già esistente (TV8.1);
	\item Selezionare la funzione di salvataggio (TV8.2);
	\item Dare un nome al progetto (TV8.3);
	\item Verificare che il progetto sia stato salvato, con il nome scelto, nell'apposita sezione riservata ai propri progetti (TV8.4).
\end{itemize}

\subsection{Test TV9}
L'utente vuole verificare che si possa eliminare un progetto. \newline
All'utente \'e richiesto di:
\begin{itemize}
	\item Selezionare il progetto che si vuole eliminare (TV9.1);
	\item Selezionare la funzione di eliminazione del progetto (TV9.2);
	\item Confermare l'eliminazione (TV9.3);
	\item Verificare che il progetto eliminato non sia più presente nell'apposita sezione riservata ai propri progetti (TV9.4).
\end{itemize}