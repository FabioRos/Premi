\subsection{Scopo del Documento}
Dopo aver analizzato il capitolato C4 Premi e a seguito dell'incontro con il proponente si sono evidenziati i requisiti
necessari che saranno inseriti nel prodotto. Tale documento ha l'obiettivo di elencarli e descriverli in modo dettagliato.

\subsection{Scopo del Prodotto}
Lo scopo del progetto è realizzare un software per un sistema di rappresentazione di slide sfruttando la tecnologia \gls{HTML5}.
Lo scopo principale è quello di creare un prodotto che sia di qualità e comparabile, in prestazioni, funzionalità ed 
effetti visivi, ai maggiori concorrenti già presenti sul mercato (Prezi, Powerpoint, Keynote, Impress, ...).

\subsection{Glossario}
Per prevenire ed evitare qualsiasi dubbio e per permettere una maggiore chiarezza e
comprensione del testo su termini ambigui, abbreviazioni e acronimi utilizzati nei vari documenti,
essi sono stati raccolti nel \textit{Glossario v1.0.0} nel quale si possono trovare tutte le informazioni desiderate.
Al fine di rendere subito evidente un termine presente nel Glossario, esso verrà marcato con una "G" in pedice.

\subsection{Riferimenti}
\subsubsection{Normativi}
\begin{itemize}
	\item Capitolato d'appalto C4: Premi: Software di presentazione "better than Prezi"
	\newline \url{http://www.math.unipd.it/~tullio/IS-1/2014/Progetto/C4p.svg}.
	\item Verbali esterni:
	\begin{itemize}
		\item Verbale di incontro con il proponente in data 2015-03-06 \textit{(Verbale2)};
		\item Verbale di scambio di mail con il proponente in data 2015-03-18 \textit{(Verbale4)};
		\item Verbale di incontro con il proponente in data 2015-04-08 \textit{(Verbale5)}.
	\end{itemize}
	\item Verbali interni:
	\begin{itemize}
		\item Verbale di incontro tra i membri del gruppo in data 2015-03-09 \textit{(Verbale3)}.
	\end{itemize}
	\item Norme di progetto: \textit{Norme di progetto v1.0.0} .
\end{itemize}
\subsubsection{Informativi}
\begin{itemize}
	\item Slide del corso Ingegneria del Software:
	\begin{itemize}
		\item Ingegneria dei requisiti: \url{http://www.math.unipd.it/~tullio/IS-1/2014/Dispense/L08.pdf};
		\item Diagrammi dei casi d'uso: \url{http://www.math.unipd.it/~tullio/IS-1/2014/Dispense/E1b.pdf}.
	\end{itemize}
	\item Software Engineering - Ian Sommerville - 9th Edition (2010):
	\begin{itemize}
		\item Chapter 4: Requirements engineering.
	\end{itemize}
	\item \gls{UML} Distilled - Martin Fowler - 4a Edizione (2010):
	\begin{itemize}
		\item Capitolo 9: Casi d’uso.
	\end{itemize}
	\item IEEE 830-1998: Recommended Practice for Software Requirements Specifications
	\begin{itemize}
		\item \url{http://en.wikipedia.org/wiki/Software_requirements_specification}.
	\end{itemize}
\end{itemize}
