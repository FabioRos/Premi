\subsection{Caso d'uso UC9: Modifica della presentazione}
\begin{figure}[h] 
	\centering 
	\includegraphics[scale=0.35] {img/UC9.png} 
	\caption{UC9 - Modifica della presentazione} 
\end{figure}

\begin{itemize}
	\item \textbf{Attori:} Proprietario;
	\item \textbf{Scopo e descrizione:} L'utente sta lavorando su una presentazione per inserire delle nuove slide o per modificare quelle già create. Può quindi inserire i seguenti elementi: slide, immagini, caselle di testo, dati real time, tabelle. Ha inoltre la possibilità di modificare gli elementi già inseriti, oppure di eliminarli. Può infine scegliere l'effetto di transizione tra una slide e l'altra;
	\item \textbf{Precondizione:} L'utente ha creato una presentazione nella fase di creazione di un progetto;
	\item \textbf{Flusso principale degli eventi:}
	\begin{enumerate}
		
		\item L'utente sceglie il template [UC9.1];
		
		\item L'utente inserisce una nuova slide [UC9.2];
		\item L'utente rimuove una slide [UC9.3];
		
		\item L'utente inserisce un'immagine [UC9.4];
		\item L'utente carica un file per inserire l'immagine [UC9.14];
		
		\item L'utente inserisce una casella di testo [UC9.5];
		\item L'utente sceglie la formattazione del testo [UC9.15];
		
		\item L'utente inserisce dati real time [UC9.6];
		
		\item L'utente inserisce una tabella [UC9.7];
		\item L'utente modifica una tabella [UC9.16];
		
		\item L'utente inserisce un grafico [UC9.8];
		\item L'utente modifica un grafico [UC9.17];
		
		\item L'utente sceglie un effetto di transizione [UC9.9];
		
		\item L'utente cambia la dimensione di un elemento [UC9.10];
		
		\item L'utente cambia la posizione di un elemento [UC9.11];
		
		\item L'utente ruota un elemento [UC9.12];
		
		\item L'utente rimuove un elemento [UC9.13].
	\end{enumerate}
	\item \textbf{Postcondizione:} Il sistema esegue le operazioni effettuate dall'utente.
\end{itemize}


\subsection{Caso d'uso UC9.1: Scegliere un template}
\begin{itemize}
	\item \textbf{Attori:} Proprietario;
	\item \textbf{Scopo e descrizione:} L'utente sceglie un template da utilizzare per la propria presentazione;
	\item \textbf{Precondizione:} Il sistema è in attesa che l'utente selezioni il template da utilizzare;
	\item \textbf{Postcondizione:} Il sistema imposta il template selezionato per la presentazione.
\end{itemize}


\subsection{Caso d'uso UC9.2: Inserire una nuova slide}
\begin{itemize}
	\item \textbf{Attori:} Proprietario;
	\item \textbf{Scopo e descrizione:} L'utente crea una nuova slide nella presentazione per poter inserire del contenuto;
	\item \textbf{Precondizione:} Il sistema è in attesa che l'utente crei una nuova slide;
	\item \textbf{Postcondizione:} Il sistema ha creato la nuova slide.
\end{itemize}


\subsection{Caso d'uso UC9.3: Rimuovere una slide}
\begin{itemize}
	\item \textbf{Attori:} Proprietario;
	\item \textbf{Scopo e descrizione:} L'utente vuole rimuovere una slide della presentazione precedentemente creata;
	\item \textbf{Precondizione:} Il sistema è in attesa che l'utente selezioni una slide e il comando per rimuoverla;
	\item \textbf{Postcondizione:} Il sistema ha rimosso la slide selezionata.
\end{itemize}


\subsection{Caso d'uso UC9.4: Inserire un'immagine}
\begin{itemize}
\item \textbf{Attori:} Proprietario;
\item \textbf{Scopo e descrizione:} L'utente deve inserire un'immagine in una slide. Seleziona il comando e sceglie l'immagine da inserire nella slide corrente;
\item \textbf{Precondizione:} Il sistema è in attesa che l'utente selezioni il comando per inserire un'immagine;
\item \textbf{Postcondizione:} Il sistema ha inserito l'immagine selezionata dall'utente nella slide.
\end{itemize}


\subsection{Caso d'uso UC9.5: Inserire una casella di testo}
\begin{itemize}
\item \textbf{Attori:} Proprietario;
\item \textbf{Scopo e descrizione:} L'utente deve inserire una casella di testo nella slide. Seleziona il comando e inserisce la casella di testo nel punto della slide desiderato;
\item \textbf{Precondizione:} Il sistema è in attesa che l'utente selezioni il comando per inserire una casella di testo;
\item \textbf{Postcondizione:} Il sistema ha inserito la casella di testo nella slide.
\end{itemize}


\subsection{Caso d'uso UC9.6: Inserire dati real time}
\begin{itemize}
	\item \textbf{Attori:} Proprietario;
	\item \textbf{Scopo e descrizione:} L'utente deve inserire dei dati real time;
	\item \textbf{Precondizione:} Il sistema è in attesa che l'utente inserisca i dati real time;
	\item \textbf{Postcondizione:} Il sistema ha inserito i dati real time.
\end{itemize}


\subsection{Caso d'uso UC9.7: Inserire una tabella}
\begin{figure}[h]
	\centering 
	\includegraphics[scale=0.45] {img/UC9.7.png} 
	\caption{UC9.7 - Inserire una tabella} 
\end{figure}

\begin{itemize}
	\item \textbf{Attori:} Utente;
	\item \textbf{Scopo e descrizione:} L'utente deve inserire una tabella. Seleziona il numero di righe e di colonne e inserisce i dati;
	\item \textbf{Precondizione:} Il sistema è in attesa che l'utente selezioni il comando per inserire una tabella nella slide;
	\item \textbf{Flusso principale degli eventi:}
	\begin{enumerate}
		\item L'utente inserisce il numero di righe e di colonne [UC9.7.1];
		\item L'utente inserisce i dati all'interno della tabella [UC9.7.2];
	\end{enumerate}
	\item \textbf{Postcondizione:} Il sistema ha inserito la tabella nella slide.
\end{itemize}

	\subsection{Caso d'uso UC9.7.1: Inserire il numero di righe e colonne}
	\begin{itemize}
		\item \textbf{Attori:} Utente;
		\item \textbf{Scopo e descrizione:} L'utente deve inserire il numero di righe e di colonne per la tabella da inserire;
		\item \textbf{Precondizione:} Il sistema è in attesa che l'utente inserisca il numero di righe e di colonne;
		\item \textbf{Postcondizione:} Il sistema registra l'inserimento dell'utente.
	\end{itemize}
	
	\subsection{Caso d'uso UC9.7.2: inserire contenuto tabella}
	\begin{itemize}
		\item \textbf{Attori:} Utente;
		\item \textbf{Scopo e descrizione:} L'utente deve inserire il contenuto nelle celle della tabella;
		\item \textbf{Precondizione:} Il sistema è in attesa che l'utente inserisca il contenuto desiderato;
		\item \textbf{Postcondizione:} Il sistema salva il contenuto inserito dall'utente.
	\end{itemize}


\subsection{Caso d'uso UC9.8: Inserire un grafico}
\begin{figure}[h] 
	\centering 
	\includegraphics[scale=0.45] {img/UC9.8.png} 
	\caption{UC9.8 - Inserire grafico} 
\end{figure}

\begin{itemize}
	\item \textbf{Attori:} Utente;
	\item \textbf{Scopo e descrizione:} L'utente deve inserire un grafico. Seleziona il tipo di grafico e inserisce i dati;
	\item \textbf{Precondizione:} Il sistema è in attesa che l'utente selezioni il comando per inserire un grafico;
	\item \textbf{Flusso di eventi:}
	\begin{enumerate}
		\item L'utente sceglie la tipologia di grafico da inserire [UC9.8.1];
		\item L'utente inserisce i dati da inserire nel grafico [UC9.8.2];
	\end{enumerate}
	\item \textbf{Postcondizione:} Il sistema ha creato il grafico.
\end{itemize}

	\subsection{Caso d'uso UC9.8.1: Scegliere la tipologia del grafico}
	\begin{itemize}
		\item \textbf{Attori:} Utente;
		\item \textbf{Scopo e descrizione:} L'utente deve scegliere il tipo di grafico da inserire;
		\item \textbf{Precondizione:} Il sistema ha aperto la finestra di dialogo per la scelta del tipo di grafico;
		\item \textbf{Postcondizione:} Il sistema ha registrato la scelta dell'utente.
	\end{itemize}
	
	\subsection{Caso d'uso UC9.8.2: Inserire i dati del grafico}
	\begin{itemize}
		\item \textbf{Attori:} Utente;
		\item \textbf{Scopo e descrizione:} L'utente deve inserire i dati per il grafico da inserire;
		\item \textbf{Precondizione:} Il sistema è in attesa che l'utente inserisca ii dati;
		\item \textbf{Postcondizione:} Il sistema salva i dati inseriti.
	\end{itemize}


\subsection{Caso d'uso UC9.9: Scegliere un effetto di transizione}
\begin{itemize}
	\item \textbf{Attori:} Proprietario;
	\item \textbf{Scopo e descrizione:} L'utente deve scegliere l'effetto di transizione da dare alla slide;
	\item \textbf{Precondizione:} Il sistema è in attesa che l'utente selezioni l'effetto desiderato;
	\item \textbf{Postcondizione:} Il sistema ha inserito l'effetto di transizione.
\end{itemize}


\subsection{Caso d'uso UC9.10: Ridimensionamento di un elemento}
\begin{itemize}
	\item \textbf{Attori:} Proprietario;
	\item \textbf{Scopo e descrizione:} L'utente cambia la grandezza dell'elemento selezionato della slide;
	\item \textbf{Precondizione:} Il sistema mostra l'elemento selezionato che si intende ridimensionare;
	\item \textbf{Postcondizione:} Il sistema ha ridimensionato l'elemento della slide.
\end{itemize}


\subsection{Caso d'uso UC9.11: Spostamento di un elemento}
\begin{itemize}
	\item \textbf{Attori:} Proprietario;
	\item \textbf{Scopo e descrizione:} L'utente cambia la posizione dell'elemento selezionato della slide;
	\item \textbf{Precondizione:} Il sistema mostra l'elemento selezionato che si intende spostare;
	\item \textbf{Postcondizione:} Il sistema ha spostato l'elemento della slide.
\end{itemize}


\subsection{Caso d'uso UC9.12: Rotazione di un elemento}
\begin{itemize}
	\item \textbf{Attori:} Proprietario;
	\item \textbf{Scopo e descrizione:} L'utente ruota l'elemento selezionato della slide;
	\item \textbf{Precondizione:} Il sistema mostra l'elemento selezionato che si intende ruotare;
	\item \textbf{Postcondizione:} Il sistema ha ruotato l'elemento della slide.
\end{itemize}


\subsection{Caso d'uso UC9.13: Rimozione di un elemento}
\begin{itemize}
	\item \textbf{Attori:} Proprietario;
	\item \textbf{Scopo e descrizione:} L'utente elimina l'elemento selezionato dalla slide;
	\item \textbf{Precondizione:} Il sistema mostra l'elemento selezionato che si intende cancellare;
	\item \textbf{Postcondizione:} Il sistema ha cancellato l'elemento dalla slide.
\end{itemize}


\subsection{Caso d'uso UC9.14: Caricamento di un file}
\begin{itemize}
	\item \textbf{Attori:} Proprietario;
	\item \textbf{Scopo e descrizione:} L'utente deve caricare un file da utilizzare nella presentazione;
	\item \textbf{Precondizione:} Il sistema è in attesa che l'utente selezioni il file;
	\item \textbf{Postcondizione:} Il sistema ha caricato il file selezionato dall'utente e lo ha inserito nella slide.
\end{itemize}


\subsection{Caso d'uso UC9.15: Scegliere la formattazione del testo}
\begin{figure}[h] 
	\centering 
	\includegraphics[scale=0.45] {img/UC9.15.png} 
	\caption{UC9.15 - Scegliere la formattazione del testo}
\end{figure}

\begin{itemize}
	\item \textbf{Attori:} Proprietario;
	\item \textbf{Scopo e descrizione:} L'utente può modificare l'aspetto del testo contenuto in una casella di testo. L'utente seleziona il testo e poi sceglie che modifiche effettuare;
	\item \textbf{Precondizione:} Il sistema è in attesa che l'utente selezioni la modifica da apportare al testo e il testo da modificare è selezionato;
	\item \textbf{Flusso principale degli eventi:}
	\begin{enumerate}
		\item L'utente può cambiare la grandezza del testo [UC9.15.1];
		\item L'utente può cambiare il colore del testo [UC9.15.2];
		\item L'utente può cambiare il font del testo [UC9.15.3];
		\item L'utente può abilitare o disabilitare il testo in corsivo [UC9.15.4];
		\item L'utente può abilitare o disabilitare il testo in grassetto [UC9.15.5];
		\item L'utente può spostare il testo in una nuova posizione [UC9.15.6].
	\end{enumerate}
	\item \textbf{Postcondizione:} Il sistema ha apportato le modifiche scelte al testo.
\end{itemize}

\subsection{Caso d'uso UC9.15.1: Scegliere la grandezza}
\begin{itemize}
	\item \textbf{Attori:} Proprietario;
	\item \textbf{Scopo e descrizione:} L'utente può cambiare la grandezza del testo;
	\item \textbf{Precondizione:} Il testo da modificare è selezionato;
	\item \textbf{Postcondizione:} Il testo è stato ingrandito o rimpicciolito secondo la scelta dell'utente.
\end{itemize}

\subsection{Caso d'uso UC9.15.2: Scegliere il colore}
\begin{itemize}
	\item \textbf{Attori:} Proprietario;
	\item \textbf{Scopo e descrizione:} L'utente può cambiare il colore del testo;
	\item \textbf{Precondizione:} Il testo da modificare è selezionato;
	\item \textbf{Postcondizione:} Il testo è stato colorato secondo la scelta dell'utente.
\end{itemize}

\subsection{Caso d'uso UC9.15.3: Scegliere il font}
\begin{itemize}
	\item \textbf{Attori:} Proprietario;
	\item \textbf{Scopo e descrizione:} L'utente può cambiare il font del testo;
	\item \textbf{Precondizione:} Il testo da modificare è selezionato;
	\item \textbf{Postcondizione:} Il testo ha cambiato font secondo la scelta dell'utente.
\end{itemize}

\subsection{Caso d'uso UC9.15.4: Abilitare/disabilitare corsivo}
\begin{itemize}
	\item \textbf{Attori:} Proprietario;
	\item \textbf{Scopo e descrizione:} L'utente può abilitare o disabilitare la scrittura in corsivo;
	\item \textbf{Precondizione:} Il testo da modificare è selezionato oppure è stata selezionata la casella di testo nella quale poter scrivere;
	\item \textbf{Postcondizione:} Il testo è stato modificato secondo la scelta dell'utente.
\end{itemize}

\subsection{Caso d'uso UC9.15.5: Abilitare/Disabilitare grassetto}
\begin{itemize}
	\item \textbf{Attori:} Proprietario;
	\item \textbf{Scopo e descrizione:} L'utente può abilitare o disabilitare la scrittura in grassetto;
	\item \textbf{Precondizione:} Il testo da modificare è selezionato oppure è stata selezionata la casella di testo nella quale poter scrivere;
	\item \textbf{Postcondizione:} Il testo è stato modificato secondo la scelta dell'utente.
\end{itemize}

\subsection{Caso d'uso UC9.15.6: Posizionare il testo}
\begin{itemize}
	\item \textbf{Attori:} Proprietario;
	\item \textbf{Scopo e descrizione:} L'utente può spostare una casella di testo in una nuova posizione;
	\item \textbf{Precondizione:} La casella di testo da spostare è stata selezionata;
	\item \textbf{Postcondizione:} La casella di testo è stata spostata secondo la scelta dell'utente.
\end{itemize}

\subsection{Caso d'uso UC9.16: Modificare una tabella}
\begin{figure}[h] 
	\centering 
	\includegraphics[scale=0.45] {img/UC9.16.png} 
	\caption{UC9.16 - Modificare una tabella} 
\end{figure}

\begin{itemize}
	\item \textbf{Attori:} Proprietario;
	\item \textbf{Scopo e descrizione:} L'utente può modificare l'aspetto della tabella e del suo contenuto. L'utente seleziona la tabella o il testo e poi sceglie che modifiche effettuare;
	\item \textbf{Precondizione:} Il sistema è in attesa che l'utente selezioni la modifica da apportare alla tabella e la tabella o il testo da modificare sono selezionati;
	\item \textbf{Flusso principale degli eventi:}
	\begin{enumerate}
		\item L'utente può modificare il numero di righe e di colonne [UC9.16.1]
		\item L'utente può cambiare la grandezza della tabella [UC9.16.2];
		\item L'utente può cambiare il colore di sfondo della tabella [UC9.16.3];
		\item L'utente può cambiare l'allineamento del testo [UC9.16.4];
		\item L'utente può cambiare la formattazione del testo [UC9.15];
	\end{enumerate}
	\item \textbf{Postcondizione:} Il sistema ha apportato le modifiche scelte alla tabella.
\end{itemize}

\subsection{Caso d'uso UC9.16.1: Modificare il numero di righe e colonne della tabella}
\begin{itemize}
	\item \textbf{Attori:} Proprietario;
	\item \textbf{Scopo e descrizione:} L'utente può modificare la grandezza della tabella;
	\item \textbf{Precondizione:} La tabella da modificare è stata selezionata;
	\item \textbf{Postcondizione:} La tabella è stata modificata nelle sue dimensioni secondo la scelta dell'utente.
\end{itemize}

\subsection{Caso d'uso UC9.16.2: Scegliere grandezza della tabella}
\begin{itemize}
	\item \textbf{Attori:} Proprietario;
	\item \textbf{Scopo e descrizione:} L'utente può modificare la grandezza della tabella;
	\item \textbf{Precondizione:} La tabella da modificare è stata selezionata;
	\item \textbf{Postcondizione:} La tabella è stata modificata nelle sue dimensioni secondo la scelta dell'utente.
\end{itemize}

\subsection{Caso d'uso UC9.16.3: Scegliere colore di sfondo della tabella}
\begin{itemize}
	\item \textbf{Attori:} Proprietario;
	\item \textbf{Scopo e descrizione:} L'utente può modificare il colore di sfondo della tabella;
	\item \textbf{Precondizione:} La tabella o le celle da modificare sono state selezionate;
	\item \textbf{Postcondizione:} Lo sfondo della tabella o delle celle è stato modificato secondo la scelta dell'utente.
\end{itemize}

\subsection{Caso d'uso UC9.16.4: Scegliere allineamento del testo}
\begin{itemize}
	\item \textbf{Attori:} Proprietario;
	\item \textbf{Scopo e descrizione:} L'utente può modificare l'allineamento del testo della tabella;
	\item \textbf{Precondizione:} La tabella o le celle da modificare sono state selezionate;
	\item \textbf{Postcondizione:} L'allineamento del testo della tabella o delle celle è stato modificato secondo la scelta dell'utente.
\end{itemize}


\subsection{Caso d'uso UC9.17: Personalizzare un grafico}
\begin{figure}[h] 
	\centering 
	\includegraphics[scale=0.45] {img/UC9.17.png} 
	\caption{UC9.17 - Personalizzare un grafico} 
\end{figure}

\begin{itemize}
	\item \textbf{Attori:} Proprietario;
	\item \textbf{Scopo e descrizione:} L'utente può modificare la tipologia e l'aspetto del grafico o del suo contenuto. L'utente seleziona il grafico e poi sceglie che modifiche effettuare;
	\item \textbf{Precondizione:} Il sistema è in attesa che l'utente selezioni la modifica da apportare al grafico e il grafico da modificare è selezionato;
	\item \textbf{Flusso principale degli eventi:}
	\begin{enumerate}
		\item L'utente può cambiare la tipologia del grafico [UC9.17.1];
		\item L'utente può cambiare la dimensione del grafico [UC9.17.2]
		\item L'utente può cambiare i colori del grafico [UC9.17.3];
	\end{enumerate}
	\item \textbf{Postcondizione:} Il sistema ha apportato le modifiche scelte al grafico.
\end{itemize}

\subsection{Caso d'uso UC9.17.1: Scegliere la tipologia del grafico}
\begin{itemize}
	\item \textbf{Attori:} Proprietario;
	\item \textbf{Scopo e descrizione:} L'utente deve modificare la tipologia del grafico. Seleziona il grafico e il comando per cambiare il tipo;
	\item \textbf{Precondizione:} Il grafico da modificare è stata selezionato e il sistema è in attesa che l'utente selezioni il comando;
	\item \textbf{Postcondizione:} La tipologia del grafico è stata modificata secondo la scelta dell'utente.
\end{itemize}

\subsection{Caso d'uso UC9.17.2: Scegliere la grandezza del grafico}
\begin{itemize}
	\item \textbf{Attori:} Proprietario;
	\item \textbf{Scopo e descrizione:} L'utente deve modificare la grandezza del grafico;
	\item \textbf{Precondizione:} Il grafico da modificare è stata selezionato;
	\item \textbf{Postcondizione:} Il grafico è stata modificato nelle sue dimensioni secondo la scelta dell'utente.
\end{itemize}

\subsection{Caso d'uso UC9.17.3: Scegliere i colori del grafico}
\begin{itemize}
	\item \textbf{Attori:} Proprietario;
	\item \textbf{Scopo e descrizione:} L'utente deve modificare il set di colori del grafico;
	\item \textbf{Precondizione:} Il grafico da modificare è stata selezionato e il sistema è in attesa che l'utente selezioni il comando;
	\item \textbf{Postcondizione:} I colori del grafico sono stati modificati secondo la scelta dell'utente.
\end{itemize}