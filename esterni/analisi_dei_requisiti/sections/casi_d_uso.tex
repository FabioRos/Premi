Verrano descritti di seguito i casi d'uso individuati per il progetto \PROGETTO. Ogni \gls{caso d'uso} ha un codice che lo indentifica nella forma:
\begin{center}
	UC[codice univoco del padre].[codice progressivo di livello]
\end{center}
Il codice progressivo di livello potrà contenere diversi altri livelli di gerarchia, essi saranno separati da un punto. Per comodità e chiarezza lo scenario principale verrà individuato attraverso il codice UCP. Successivamente verranno descritti i figli senza però riportare il codice del padre.

\subsection{Caso d'uso UCP: Scenario principale}
\begin{figure}[h] 
	\centering 
	\includegraphics[scale=0.5] {img/UCP.png} 
	\caption{UCP - Scenario principale} 
\end{figure}

\newpage

\begin{itemize}
	\item \textbf{Attori:} Utente;
	\item \textbf{Scopo e descrizione:} Dopo aver avviato il programma, l'utente potrà effettuare diverse operazioni. Potrà creare una nuova presentazione o caricarne una precedentemente creata. Analogamente potrà creare una nuova \gls{infografica} oppure caricarne una già creata. Se è aperta una presentazione o una \gls{infografica} l'utente potrà modificare e salvare i nuovi cambiamenti o esportare l'intera presentazione (o \gls{infografica}). Inoltre l'utente avrà a disposizione una funzione di stampa per stampare le slide della presentazione oppure l'\gls{infografica};
	\item \textbf{Precondizione:} Il programma è stato correttamente avviato ed è pronto all'uso;
	\item \textbf{Flusso degli eventi:}
	\begin{enumerate}
		\item L'utente può creare una nuova presentazione [UC1];
		\item L'utente può aprire una presentazione [UC2];
		\item L'utente può modificare una presentazione [UC3];
		\item L'utente può salvare una presentazione [UC4];
		\item L'utente può esportare una presentazione [UC5];
		\item L'utente può stampare una presentazione [UC6];
		\item L'utente può creare una nuova \gls{infografica} [UC7];
		\item L'utente può aprire una \gls{infografica} [UC8];
		\item L'utente può salvare una \gls{infografica} [UC9];
		\item L'utente può stampare una \gls{infografica} [UC10].
	\end{enumerate}
	\item \textbf{Postcondizione:} Il sistema ha ottenuto le informazioni sulle operazioni che l’utente desidera eseguire.
\end{itemize}

\subsection{Caso d'uso UC1: Creazione nuova presentazione}
\begin{figure}[h] 
	\centering 
	\includegraphics[scale=0.45] {img/UC1.png} 
	\caption{UC1 - Creazione nuova presentazione} 
\end{figure}

\begin{itemize}
	\item \textbf{Attori:} Utente;
	\item \textbf{Scopo e descrizione:} L'utente sta creando una nuova presentazione di slide. Per completare l'operazione deve inserire almeno una slide vuota. Potrà anche inserire una o più: immagine, casella di testo, dato \gls{real time}. Potrà essere scelto anche l'effetto di transizione da una slide ad un'altra;
	\item \textbf{Precondizione:} Il sistema mostra la schermata di creazione di una presentazione e l'utente vuole creare una nuova slide;
	\item \textbf{Flusso degli eventi:}
	\begin{enumerate}
		\item L'utente crea una nuova slide [UC1.1];
		\item L'utente può inserire un'immagine o più [UC1.2];
		\item L'utente carica un file per inserire l'immagine [UC1.6];
		\item L'utente può inserire una casella di testo o più [UC1.3];
		\item L'utente sceglie la formattazione del testo [UC1.7];
		\item L'utente può inserire dei dati \gls{real time} [UC1.4];
		\item L'utente può scegliere un effetto di transizione [UC1.5];
	\end{enumerate}
	\item \textbf{Postcondizione:} Il sistema mostra le operazioni effettuate dall'utente.
\end{itemize}

\subsection{Caso d'uso UC1.1: Inserire nuova slide}
\begin{itemize}
	\item \textbf{Attori:} Utente;
	\item \textbf{Scopo e descrizione:} L'utente crea una nuova slide nella presentazione per poter inserire del contenuto;
	\item \textbf{Precondizione:} Il sistema è in attesa che l'utente crei una nuova slide;
	\item \textbf{Postcondizione:} Il sistema ha creato la nuova slide.
\end{itemize}

\subsection{Caso d'uso UC1.2: Inserire un'immagine}
\begin{itemize}
\item \textbf{Attori:} Utente;
\item \textbf{Scopo e descrizione:} L'utente deve inserire l'immagine da mettere nella slide;
\item \textbf{Precondizione:} Il sistema è in attesa che l'utente selezioni l'immagine;
\item \textbf{Postcondizione:} Il sistema ha caricato l'immagine selezionata dall'utente.
\end{itemize}

\subsection{Caso d'uso UC1.3: Inserire casella di testo}
\begin{itemize}
\item \textbf{Attori:} Utente;
\item \textbf{Scopo e descrizione:} L'utente deve inserire una casella di testo nella slide;
\item \textbf{Precondizione:} Il sistema è in attesa che l'utente crei una casella di testo;
\item \textbf{Postcondizione:} Il sistema ha creato la casella di testo.
\end{itemize}

\subsection{Caso d'uso UC1.4: Inserire dati real time}
\begin{itemize}
	\item \textbf{Attori:} Utente;
	\item \textbf{Scopo e descrizione:} L'utente deve inserire dei dati \gls{real time};
	\item \textbf{Precondizione:} Il sistema è in attesa che l'utente inserisca i dati \gls{real time};
	\item \textbf{Postcondizione:} Il sistema ha inserito i dati \gls{real time}.
\end{itemize}

\subsection{Caso d'uso UC1.5: Scegliere effetto di transizione}
\begin{itemize}
	\item \textbf{Attori:} Utente;
	\item \textbf{Scopo e descrizione:} L'utente deve scegliere l'effetto di transizione da dare alla slide;
	\item \textbf{Precondizione:} Il sistema è in attesa che l'utente selezioni l'effetto desiderato;
	\item \textbf{Postcondizione:} Il sistema ha inserito l'effetto di transizione.
\end{itemize}

\subsection{Caso d'uso UC1.6: Caricare file}
\begin{figure}[h] 
	\centering 
	\includegraphics[scale=0.45] {img/UC1.6.png} 
	\caption{UC1.6 - Caricare file} 
\end{figure}

\begin{itemize}
	\item \textbf{Attori:} Utente;
	\item \textbf{Scopo e descrizione:} L'utente deve caricare un file. Naviga il \gls{filesystem} cercando il file desiderato, lo seleziona e conferma la selezione caricando il file;
	\item \textbf{Precondizione:} Il sistema è in attesa che l'utente selezioni il file;
	\item \textbf{Flusso degli eventi:}
	\begin{enumerate}
		\item L'utente naviga il \gls{filesystem} alla ricerca del file desiderato [UC1.6.1];
		\item L'utente seleziona il file [UC1.6.2];
		\item L'utente conferma il file selezionato [UC1.6.3].
	\end{enumerate}
	\item \textbf{Postcondizione:} Il sistema ha caricato il file selezionato dall'utente e lo ha inserito nella slide.
\end{itemize}

\subsection{Caso d'uso UC1.6.1: Navigare il filesystem}
\begin{itemize}
	\item \textbf{Attori:} Utente;
	\item \textbf{Scopo e descrizione:} L'utente può navigare il \gls{filesystem} per selezionare la cartella dentro la quale è contenuto il file desiderato;
	\item \textbf{Precondizione:} Il sistema è in attesa che l'utente selezioni una cartella;
	\item \textbf{Postcondizione:} Il sistema ha aggiornato la cartella corrente con quella scelta dall'utente.
\end{itemize}

\subsection{Caso d'uso UC1.6.2: Selezionare il file}
\begin{itemize}
	\item \textbf{Attori:} Utente;
	\item \textbf{Scopo e descrizione:} L'utente deve selezionare il file che intende caricare;
	\item \textbf{Precondizione:} Il sistema mostra i file contenuti nella cartella precedentemente selezionata;
	\item \textbf{Postcondizione:} Il sistema evidenzia il file scelto dall'utente.
\end{itemize}

\subsection{Caso d'uso UC1.6.3: Confermare selezione}
\begin{itemize}
	\item \textbf{Attori:} Utente;
	\item \textbf{Scopo e descrizione:} L'utente conferma che il file selezionato in precedenza è quello corretto;
	\item \textbf{Precondizione:} Il sistema ha selezionato il file indicato dall'utente;
	\item \textbf{Postcondizione:} Il sistema ha caricato il file scelto precedentemente dall'utente.
\end{itemize}

\subsection{Caso d'uso UC1.7: Scegliere formattazione del testo}
\begin{figure}[h] 
	\centering 
	\includegraphics[scale=0.45] {img/UC1.7.png} 
	\caption{UC1.7 - Scegliere formattazione del testo} 
\end{figure}

\begin{itemize}
	\item \textbf{Attori:} Utente;
	\item \textbf{Scopo e descrizione:} L'utente può modificare l'aspetto del testo contenuto in una casella di testo. L'utente seleziona il testo e poi sceglie che modifiche effettuare;
	\item \textbf{Precondizione:} Il sistema è in attesa che l'utente selezioni la modifica da apportare al testo e il testo da modificare è selezionato;
	\item \textbf{Flusso degli eventi:}
	\begin{enumerate}
		\item L'utente può cambiare la grandezza del testo [UC1.7.1];
		\item L'utente può cambiare il colore del testo [UC1.7.2];
		\item L'utente può cambiare il \gls{font} del testo [UC1.7.3];
		\item L'utente può abilitare o disabilitare il testo in corsivo [UC1.7.4];
		\item L'utente può abilitare o disabilitare il testo in grassetto [UC1.7.5];
		\item L'utente può spostare il testo in una nuova posizione [UC1.7.6].
	\end{enumerate}
	\item \textbf{Postcondizione:} Il sistema ha apportato le modifiche scelte al testo.
\end{itemize}

\subsection{Caso d'uso UC1.7.1: Scegliere grandezza}
\begin{itemize}
	\item \textbf{Attori:} Utente;
	\item \textbf{Scopo e descrizione:} L'utente può cambiare la grandezza del testo;
	\item \textbf{Precondizione:} Il testo da modificare è selezionato;
	\item \textbf{Postcondizione:} Il testo è stato ingrandito o rimpicciolito secondo la scelta dell'utente.
\end{itemize}

\subsection{Caso d'uso UC1.7.2: Scegliere colore}
\begin{itemize}
	\item \textbf{Attori:} Utente;
	\item \textbf{Scopo e descrizione:} L'utente può cambiare il colore del testo;
	\item \textbf{Precondizione:} Il testo da modificare è selezionato;
	\item \textbf{Postcondizione:} Il testo è stato colorato secondo la scelta dell'utente.
\end{itemize}

\subsection{Caso d'uso UC1.7.3: Scegliere font}
\begin{itemize}
	\item \textbf{Attori:} Utente;
	\item \textbf{Scopo e descrizione:} L'utente può cambiare il \gls{font} del testo;
	\item \textbf{Precondizione:} Il testo da modificare è selezionato;
	\item \textbf{Postcondizione:} Il testo ha cambiato \gls{font} secondo la scelta dell'utente.
\end{itemize}

\subsection{Caso d'uso UC1.7.4: Abilitare/Disabilitare corsivo}
\begin{itemize}
	\item \textbf{Attori:} Utente;
	\item \textbf{Scopo e descrizione:} L'utente può abilitare o disabilitare la scrittura in corsivo;
	\item \textbf{Precondizione:} Il testo da modificare è selezionato oppure è stata selezionata la casella di testo nella quale poter scrivere;
	\item \textbf{Postcondizione:} Il testo è stato modificato secondo la scelta dell'utente.
\end{itemize}

\subsection{Caso d'uso UC1.7.5: Abilitare/Disabilitare grassetto}
\begin{itemize}
	\item \textbf{Attori:} Utente;
	\item \textbf{Scopo e descrizione:} L'utente può abilitare o disabilitare la scrittura in grassetto;
	\item \textbf{Precondizione:} Il testo da modificare è selezionato oppure è stata selezionata la casella di testo nella quale poter scrivere;
	\item \textbf{Postcondizione:} Il testo è stato modificato secondo la scelta dell'utente.
\end{itemize}

\subsection{Caso d'uso UC1.7.6: Posizionare testo}
\begin{itemize}
	\item \textbf{Attori:} Utente;
	\item \textbf{Scopo e descrizione:} L'utente può spostare una casella di testo in una nuova posizione;
	\item \textbf{Precondizione:} La casella di testo da spostare è stata selezionata;
	\item \textbf{Postcondizione:} La casella di testo è stata spostata secondo la scelta dell'utente.
\end{itemize}

\newpage
\subsection{Caso d'uso UC2: Autenticazione}
\begin{figure}[h] 
	\centering 
	\includegraphics[scale=0.45] {img/UC2.png} 
	\caption{UC2 - Autenticazione} 
\end{figure}

\begin{itemize}
	\item \textbf{Attori:} Utente non autenticato;
	\item \textbf{Scopo e descrizione:} L'utente è già iscritto e vuole avviare la procedura di autenticazione al sito per accedere ai propri file;
	\item \textbf{Precondizione:} L'utente ha selezionato la voce "accedi" presente sul sito;
	\item \textbf{Flusso principale degli eventi:}
	\begin{enumerate}
		\item L'utente inserisce le proprie credenziali [UC2.1];
		\item L'utente conferma l'inserimento dei dati selezionando la voce "login" [UC2.2];
		\item Si può verificare un errore di accesso [UC2.3];
		\item L'utente viene reindirizzato alla propria pagina personale [UC2.4].
	\end{enumerate}
	\item \textbf{Postcondizione:} Il sistema verifica le credenziali inserite e permette all'utente di accedere alla sua pagina personale.
\end{itemize}

\newpage

\subsection{Caso d'uso UC2.1: Inserimento credenziali}
\begin{figure}[h] 
	\centering 
	\includegraphics[scale=0.45] {img/UC2.1.png} 
	\caption{UC2.1 - Inserimento credenziali} 
\end{figure}
\begin{itemize}
	\item \textbf{Attori:} Utente non autenticato;
	\item \textbf{Scopo e descrizione:} L'utente inserisce nome utente e password per poter accedere al sito;
	\item \textbf{Precondizione:} L'utente visualizza la schermata di inserimento dei dati richiesti per l'accesso;
	\item \textbf{Flusso principale degli eventi:}
	\begin{enumerate}
		\item L'utente inserisce il proprio nome utente [UC2.1.1];
		\item L'utente inserisce la propria password [UC2.1.2];
	\end{enumerate}
	\item \textbf{Postcondizione:} Tutti i campi richiesti sono stati compilati correttamente.
\end{itemize}

\subsection{Caso d'uso UC2.1.1: Inserire nome utente}
\begin{itemize}
	\item \textbf{Attori:} Utente non autenticato;
	\item \textbf{Scopo e descrizione:} L'utente inserisce il proprio nome utente;
	\item \textbf{Precondizione:} La casella dove inserire il nome utente è vuota;
	\item \textbf{Postcondizione:} La casella è stata compilata con il nome utente inserito dall'utente.
\end{itemize}

\subsection{Caso d'uso UC2.1.2: Inserire password}
\begin{itemize}
	\item \textbf{Attori:} Utente non autenticato;
	\item \textbf{Scopo e descrizione:} L'utente inserisce la propria password;
	\item \textbf{Precondizione:} La casella dove inserire la password è vuota;
	\item \textbf{Postcondizione:} La casella è stata compilata con la password inserita dall'utente.
\end{itemize}

\subsection{Caso d'uso UC2.2: Accesso}
\begin{itemize}
	\item \textbf{Attori:} Utente non autenticato;
	\item \textbf{Scopo e descrizione:} L'utente conferma le credenziali inserite scegliendo di effettuare l'accesso tramite il tasto di login;
	\item \textbf{Precondizione:} Nome utente e password sono stati inseriti;
	\item \textbf{Postcondizione:} Il sistema verifica i dati inseriti.
\end{itemize}

\subsection{Caso d'uso UC2.3: Errore di autenticazione}
\begin{itemize}
	\item \textbf{Attori:} Sistema;
	\item \textbf{Scopo e descrizione:} L'utente ha inserito delle credenziali errate e il sistema blocca l'accesso, riportando l'utente alla schermata di accesso;
	\item \textbf{Precondizione:} Nome utente e password sono stati inseriti in modo errato;
	\item \textbf{Postcondizione:} Il sistema riporta l'utente alla schermata di accesso.
\end{itemize}

\subsection{Caso d'uso UC2.4: Dati inseriti correttamente}
\begin{itemize}
	\item \textbf{Attori:} Sistema;
	\item \textbf{Scopo e descrizione:} Il sistema ha verificato che le credenziali inserite sono corrette;
	\item \textbf{Precondizione:} Nome utente e password sono stati inseriti;
	\item \textbf{Postcondizione:} Il sistema ha verificato che i dati inseriti sono corretti.
\end{itemize}

\subsection{Caso d'uso UC2.5: Reindirizzamento a pagina personale}
\begin{itemize}
	\item \textbf{Attori:} Sistema;
	\item \textbf{Scopo e descrizione:} Il sistema dopo aver consentito l'accesso all'utente mostra la sua pagina personale;
	\item \textbf{Precondizione:} Nome utente e password sono stati inseriti correttamente;
	\item \textbf{Postcondizione:} Il sistema mostra la pagina personale dell'utente.
\end{itemize}

\newpage

\subsection{Caso d'uso UC3: Ricerca di un progetto}
\begin{figure}[h] 
	\centering 
	\includegraphics[scale=0.45] {img/UC3.png} 
	\caption{UC3 - Ricerca di un progetto} 
\end{figure}

\begin{itemize}
	\item \textbf{Attori:} Utente non autenticato, utente autenticato;
	\item \textbf{Scopo e descrizione:} L'utente può cercare un progetto utilizzando come chiave di ricerca il nome utente o il titolo del progetto;
	\item \textbf{Precondizione:} L'utente sta visualizzando la schermata di ricerca;
	\item \textbf{Postcondizione:} Il sistema mostra all'utente il risultato della ricerca.
\end{itemize}

\subsection{Caso d'uso UC3.1: Ricerca tramite nome utente}
\begin{itemize}
	\item \textbf{Attori:} Utente non autenticato, utente autenticato;
	\item \textbf{Scopo e descrizione:} L'utente inserisce nella casella di ricerca il nome utente di un creatore di un progetto;
	\item \textbf{Precondizione:} L'utente sta visualizzando la schermata di ricerca. Estende il caso d'uso UC3;
	\item \textbf{Postcondizione:} L'utente ha inserito un nome utente nella casella di ricerca ed ha avviato la ricerca.
\end{itemize}

\subsection{Caso d'uso UC3.2: Ricerca tramite titolo del progetto}
\begin{itemize}
	\item \textbf{Attori:} Utente non autenticato, utente autenticato;
	\item \textbf{Scopo e descrizione:} L'utente inserisce nella casella di ricerca il titolo di un progetto;
	\item \textbf{Precondizione:} L'utente sta visualizzando la schermata di ricerca. Estende il caso d'uso UC3;
	\item \textbf{Postcondizione:} L'utente ha inserito un titolo nella casella di ricerca ed ha avviato la ricerca.
\end{itemize}


\subsection{Caso d'uso UC4: Salvataggio presentazione}
\begin{figure}[h] 
	\centering 
	\includegraphics[scale=0.45] {img/UC4.png} 
	\caption{UC4 - Salvataggio presentazione} 
\end{figure}

\begin{itemize}
	\item \textbf{Attori:} Utente;
	\item \textbf{Scopo e descrizione:} L'utente ha creato una presentazione slide e vuole salvarla in una specifica cartella;
	\item \textbf{Precondizione:} Il sistema è in attesa che l'utente selezioni la funzione salva;
	\item \textbf{Flusso degli eventi:}
	\begin{enumerate}
		\item L'utente seleziona la funzione salva [UC4.1];
		\item L'utente seleziona la cartella nella quale salvare la presentazione [UC4.2];
		\item L'utente scrive il nome del file [UC4.3];
		\item L'utente può selezionare un file già esistente da sovrascrivere [UC4.4];
		\item L'utente conferma il salvataggio [UC4.5].
	\end{enumerate}
	\item \textbf{Postcondizione:} Il sistema salvato la pesentazione nella cartella selezionata con il nome indicato.
\end{itemize}

\subsection{Caso d'uso UC4.1: Selezionare funzione salva}
\begin{itemize}
	\item \textbf{Attori:} Utente;
	\item \textbf{Scopo e descrizione:} L'utente seleziona dall'apposito menù la funzione di salvataggio per salvare la presentazione;
	\item \textbf{Precondizione:} Il sistema è in attesa che l'utente selezioni la funzione salva;
	\item \textbf{Postcondizione:} Il sistema apre la finestra di dialogo per il salvataggio.
\end{itemize}

\subsection{Caso d'uso UC4.2: Navigazione nel filesystem}
\begin{itemize}
	\item \textbf{Attori:} Utente;
	\item \textbf{Scopo e descrizione:} L'utente naviga il filesystem per selezionare posizione di salvataggio della presentazione;
	\item \textbf{Precondizione:} Il sistema è in attesa che l'utente selezioni una cartella;
	\item \textbf{Postcondizione:} Il sistema aggiorna il percorso con quello scelto dall'utente.
\end{itemize}

\subsection{Caso d'uso UC4.3: Scrittura nome file}
\begin{itemize}
	\item \textbf{Attori:} Utente;
	\item \textbf{Scopo e descrizione:} L'utente deve inserire un nome valido con il quale salvare la presentazione;
	\item \textbf{Precondizione:} Il sistema permette all'utente di selezionare il nome della presentazione da salvare;
	\item \textbf{Postcondizione:} È stato inserito un nome valido per la presentazione che l'utente desidera salvare.
\end{itemize}

\subsection{Caso d'uso UC4.4: Seleziona file già esistente}
\begin{itemize}
	\item \textbf{Attori:} Utente;
	\item \textbf{Scopo e descrizione:} L'utente può selezionare una presentazione già esistente da sovrascrivere con il salvataggio;
	\item \textbf{Precondizione:} Il sistema contiene già una presentazione con il nome che l'utente desidera utilizzare;
	\item \textbf{Postcondizione:} La presentazione già esistente è stata selezionata dall'utente.
\end{itemize}

\subsection{Caso d'uso UC4.5: Conferma salvataggio}
\begin{itemize}
	\item \textbf{Attori:} Utente;
	\item \textbf{Scopo e descrizione:} L'utente conferma il salvataggio della presentazione;
	\item \textbf{Precondizione:} Il sistema ha ricevuto la richiesta di salvataggio della presentazione;
	\item \textbf{Postcondizione:} Il sistema ha salvato la presentazione selezionata dall'utente.
\end{itemize}
\newpage

\subsection{Caso d'uso UC5: Generazione del file PDF del progetto}
\begin{figure}[h] 
	\centering 
	\includegraphics[scale=0.45] {img/UC5.png} 
	\caption{UC5 - Stampa del progetto} 
\end{figure}

\begin{itemize}
	\item \textbf{Attori:} Utente autenticato, proprietario;
	\item \textbf{Scopo e descrizione:} L'utente ha aperto un progetto e vuole generare il PDF della presentazione o dell'infografica;
	\item \textbf{Precondizione:} L'utente ha un progetto aperto e il sistema è in attesa che selezioni la funzione di generazione del PDF;
	\item \textbf{Flusso principale degli eventi:}
	\begin{enumerate}
		\item L'utente seleziona la funzione di generazione del file PDF della presentazione [UC5.1];
		\item L'utente seleziona la funzione di generazione del file PDF dell'infografica [UC5.2];
	\end{enumerate}
	\item \textbf{Postcondizione:} Il sistema ha mandato in stampa la parte di progetto selezionata.
\end{itemize}


	\subsection{Caso d'uso UC5.1: Generare il PDF della presentazione}
	\begin{itemize}
		\item \textbf{Attori:} Utente autenticato, proprietario;
		\item \textbf{Scopo e descrizione:} L'utente ha aperto un progetto e vuole generare il PDF della presentazione;
		\item \textbf{Precondizione:} L'utente ha un progetto aperto e il sistema è in attesa che selezioni la funzione di generazione del PDF della presentazione;
		\item \textbf{Flusso principale degli eventi:}
		\begin{enumerate}
			\item L'utente seleziona la funzione di generazione del PDF [UC5.1.1];
			\item L'utente seleziona quali slide includere nel PDF [UC5.1.2];
		\end{enumerate}
		\item \textbf{Postcondizione:} Il sistema ha generato il PDF per l'utente e lo rende disponibile nel browser.
	\end{itemize}
	
	
		\subsection{Caso d'uso UC5.1.1: Selezionare la funzione}
		\begin{itemize}
			\item \textbf{Attori:} Utente autenticato, proprietario;
			\item \textbf{Scopo e descrizione:} L'utente seleziona la funzione di generazione del PDF;
			\item \textbf{Precondizione:} C'è una presentazione attiva e il sistema è in attesa che l'utente selezioni la funzione desiderata;
			\item \textbf{Postcondizione:} Il sistema apre la finestra di dialogo per la scelta delle slide.
		\end{itemize}
		
		\subsection{Caso d'uso UC5.1.2.: Selezionare le slide}
		\begin{itemize}
			\item \textbf{Attori:} Utente autenticato, proprietario;
			\item \textbf{Scopo e descrizione:} L'utente seleziona quali slide includere nel file da generare;
			\item \textbf{Precondizione:} Il sistema è in attesa che l'utente selezioni le silde;
			\item \textbf{Postcondizione:} Il sistema registra la scelta fatta dall'utente e genera il file PDF per renderlo disponibile nel browser.
		\end{itemize}


	\subsection{Caso d'uso UC5.2: Generare il PDF dell'infografica}
	\begin{itemize}
		\item \textbf{Attori:} Utente autenticato, proprietario;
		\item \textbf{Scopo e descrizione:} L'utente ha aperto un progetto e vuole generare il PDF dell'infografica attraverso l'apposito comando;
		\item \textbf{Precondizione:} L'utente ha un progetto aperto di cui aveva creato l'infografica e il sistema è in attesa che selezioni la funzione di generazione del PDF dell'infografica;
		\item \textbf{Postcondizione:} Il sistema ha generato il PDF per l'utente e lo rende disponibile nel browser.
	\end{itemize}

\subsection{Caso d'uso UC6: Esportazione del progetto}
	\begin{figure}[h]
		\centering
		\includegraphics[scale=0.45] {img/UC6.png}
		\caption{UC6 - Esportazione del progetto}
	\end{figure}

	\begin{itemize}
		\item \textbf{Attori:} Utente autenticato, Proprietario;
		\item \textbf{Scopo e descrizione:} L'utente ha aperto un progetto e vuole esportarlo in locale per poterlo visualizzare offline;
		\item \textbf{Precondizione:} Il sistema è in attesa che l'utente selezioni la funzione esporta;
		\item \textbf{Flusso principale degli eventi:}
		\begin{enumerate}
			\item L'utente seleziona dal menù la funzione esporta [UC6.1];
			\item L'utente salva il pacchetto in locale [UC6.2];
		\end{enumerate}
		\item \textbf{Postcondizione:} L'utente ha esportato il progetto e l'ha salvato in locale.
	\end{itemize}


\subsection{Caso d'uso UC6.1: Selezionare la funzione esporta}
	\begin{itemize}
		\item \textbf{Attori:} Utente autenticato, Proprietario;
		\item \textbf{Scopo e descrizione:} L'utente seleziona dall'apposito menù la funzione di esportazione per esportare il progetto;
		\item \textbf{Precondizione:} L'utente ha un progetto aperto e il sistema è in attesa che l'utente selezioni la funzione esporta;
		\item \textbf{Postcondizione:} Il sistema inizia la procedura di esportazione.
	\end{itemize}


\subsection{Caso d'uso UC6.2: Salvataggio del pacchetto in locale}
	\begin{itemize}
		\item \textbf{Attori:} Utente autenticato, Proprietario;
		\item \textbf{Scopo e descrizione:} L'utente deve salvare il pacchetto del progetto in locale.
		\item \textbf{Precondizione:} Il sistema ha preparato il pacchetto, il quale è pronto per essere scaricato;
		\item \textbf{Postcondizione:} Il pacchetto è stato scaricato e salvato in locale dall'utente.
	\end{itemize}

\subsection{Caso d'uso UC7: Creazione nuova infografica}
\begin{figure}[h] 
	\centering 
	\includegraphics[scale=0.45] {img/UC7.png} 
	\caption{UC7 - Creazione nuova infografica} 
\end{figure}

\begin{itemize}
	\item \textbf{Attori:} Utente;
	\item \textbf{Scopo e descrizione:} L'utente sta creando una nuova infografica. Per completare l'operazione deve scegliere un template e inserire almeno un elemento grafico o testuale. Potrà anche inserire: caselle di testo, immagini, dati \gls{real time}, tabelle, grafici. Potrà scegliere un template predefinito da utilizzare;
	\item \textbf{Precondizione:} Il sistema mostra la schermata di creazione di un'infografica e l'utente vuole scegliere il template;
	\item \textbf{Flusso degli eventi:}
	\begin{enumerate}
		\item L'utente sceglie il template [UC7.1];
		\item L'utente cambia il template [UC7.7];
		\item L'utente può inserire un'immagine o più [UC7.2];
		\item L'utente carica un file per inserire l'immagine [UC7.8];
		\item L'utente può inserire una casella di testo o più [UC7.3];
		\item L'utente sceglie la formattazione del testo [UC7.9];
		\item L'utente può inserire dei dati \gls{real time} [UC7.4];
		\item L'utente può inserire una tabella o più [UC7.5];
		\item L'utente personalizza la tabella [UC7.10];
		\item L'utente può inserire uno o più grafici [UC7.6];
		\item L'utente personalizza il grafico [UC7.11];		
	\end{enumerate}
	\item \textbf{Postcondizione:} Il sistema mostra le operazioni effettuate dall'utente.
\end{itemize}


\subsection{Caso d'uso UC7.1: Scegliere template}
\begin{itemize}
	\item \textbf{Attori:} Utente;
	\item \textbf{Scopo e descrizione:} L'utente sceglie il template con cui creare l'infografica;
	\item \textbf{Precondizione:} Il sistema è in attesa che l'utente scelga il template;
	\item \textbf{Postcondizione:} Il sistema carica il template selezionato.
\end{itemize}


\subsection{Caso d'uso UC7.2: Inserire immagine}
\begin{itemize}
\item \textbf{Attori:} Utente;
\item \textbf{Scopo e descrizione:} L'utente deve inserire l'immagine da mettere nell'infografica;
\item \textbf{Precondizione:} Il sistema è in attesa che l'utente selezioni l'immagine;
\item \textbf{Postcondizione:} Il sistema ha caricato l'immagine selezionata dall'utente.
\end{itemize}


\subsection{Caso d'uso UC7.3: Inserire casella di testo}
\begin{itemize}
\item \textbf{Attori:} Utente;
\item \textbf{Scopo e descrizione:} L'utente deve inserire una casella di testo nella slide;
\item \textbf{Precondizione:} Il sistema è in attesa che l'utente crei una casella di testo;
\item \textbf{Postcondizione:} Il sistema ha creato la casella di testo.
\end{itemize}


\subsection{Caso d'uso UC7.4: Inserire dati real time}
\begin{itemize}
	\item \textbf{Attori:} Utente;
	\item \textbf{Scopo e descrizione:} L'utente deve inserire dei dati \gls{real time};
	\item \textbf{Precondizione:} Il sistema è in attesa che l'utente inserisca i dati \gls{real time};
	\item \textbf{Postcondizione:} Il sistema ha inserito i dati \gls{real time}.
\end{itemize}


\subsection{Caso d'uso UC7.5: Inserire tabella}
\begin{figure}[h] 
	\centering 
	\includegraphics[scale=0.45] {img/UC7.5.png} 
	\caption{UC7.5 - Inserire tabella} 
\end{figure}

\begin{itemize}
	\item \textbf{Attori:} Utente;
	\item \textbf{Scopo e descrizione:} L'utente deve inserire una tabella. Seleziona il tipo di tabella, il numero di righe e di colonne e inserisce i dati;
	\item \textbf{Precondizione:} Il sistema è in attesa che l'utente crei una tabella;
	\item \textbf{Flusso di eventi:}
	\begin{enumerate}
		\item L'utente sceglie la tipologia di tabella da inserire [UC7.5.1];
		\item L'utente inserisce il numero di righe e di colonne [UC7.5.2];
		\item L'utente inserisce i dati all'interno della tabella [UC7.5.3];
	\end{enumerate}
	\item \textbf{Postcondizione:} Il sistema ha creato la tabella.
\end{itemize}

\subsection{Caso d'uso UC7.5.1: Scegliere tipologia tabella}
\begin{itemize}
	\item \textbf{Attori:} Utente;
	\item \textbf{Scopo e descrizione:} L'utente può scegliere il tipo di tabella da inserire;
	\item \textbf{Precondizione:} Il sistema è in attesa che l'utente selezioni il tipo di tabella;
	\item \textbf{Postcondizione:} Il sistema registra la scelta dell'utente.
\end{itemize}

\subsection{Caso d'uso UC7.5.2: Inserire numero di righe e colonne}
\begin{itemize}
	\item \textbf{Attori:} Utente;
	\item \textbf{Scopo e descrizione:} L'utente deve inserire il numero di righe e di colonne per la tabella da inserire;
	\item \textbf{Precondizione:} Il sistema è in attesa che l'utente inserisca il numero di righe e di colonne;
	\item \textbf{Postcondizione:} Il sistema registra l'inserimento dell'utente.
\end{itemize}

\subsection{Caso d'uso UC7.5.3: inserire contenuto tabella}
\begin{itemize}
	\item \textbf{Attori:} Utente;
	\item \textbf{Scopo e descrizione:} L'utente deve inserire il contenuto nelle celle della tabella;
	\item \textbf{Precondizione:} Il sistema è in attesa che l'utente inserisca il contenuto desiderato;
	\item \textbf{Postcondizione:} Il sistema salva il contenuto inserito dall'utente.
\end{itemize}


\subsection{Caso d'uso UC7.6: Inserire grafico}
\begin{figure}[h] 
	\centering 
	\includegraphics[scale=0.45] {img/UC7.6.png} 
	\caption{UC7.6 - Inserire grafico} 
\end{figure}

\begin{itemize}
	\item \textbf{Attori:} Utente;
	\item \textbf{Scopo e descrizione:} L'utente deve inserire un grafico. Seleziona il tipo di grafico e inserisce i dati;
	\item \textbf{Precondizione:} Il sistema è in attesa che l'utente crei un grafico;
	\item \textbf{Flusso di eventi:}
	\begin{enumerate}
		\item L'utente sceglie la tipologia di grafico da inserire [UC7.6.1];
		\item L'utente inserisce i dati da inserire nel grafico [UC7.6.2];
	\end{enumerate}
	\item \textbf{Postcondizione:} Il sistema ha creato il grafico.
\end{itemize}

\subsection{Caso d'uso UC7.6.1: Scegliere tipologia grafico}
\begin{itemize}
	\item \textbf{Attori:} Utente;
	\item \textbf{Scopo e descrizione:} L'utente può scegliere il tipo di grafico da inserire;
	\item \textbf{Precondizione:} Il sistema è in attesa che l'utente selezioni il tipo di grafico;
	\item \textbf{Postcondizione:} Il sistema registra la scelta dell'utente.
\end{itemize}

\subsection{Caso d'uso UC7.6.2: Inserire dati grafico}
\begin{itemize}
	\item \textbf{Attori:} Utente;
	\item \textbf{Scopo e descrizione:} L'utente deve inserire i dati per il grafico da inserire;
	\item \textbf{Precondizione:} Il sistema è in attesa che l'utente inserisca ii dati;
	\item \textbf{Postcondizione:} Il sistema salva i dati inseriti.
\end{itemize}


\subsection{Caso d'uso UC7.7: Cambiare template}
\begin{itemize}
	\item \textbf{Attori:} Utente;
	\item \textbf{Scopo e descrizione:} L'utente può cambiare template dell'infografica;
	\item \textbf{Precondizione:} Il sistema è in attesa che l'utente selezioni il template;
	\item \textbf{Postcondizione:} Il sistema ha cambiato il template con quello scelto dall'utente.
\end{itemize}


\subsection{Caso d'uso UC7.8: Caricare file}
\begin{figure}[h] 
	\centering 
	\includegraphics[scale=0.45] {img/UC7.8.png} 
	\caption{UC7.8 - Caricare file} 
\end{figure}

\begin{itemize}
	\item \textbf{Attori:} Utente;
	\item \textbf{Scopo e descrizione:} L'utente deve caricare un file. Naviga il \gls{filesystem} cercando il file desiderato, lo seleziona e conferma la selezione caricando il file;
	\item \textbf{Precondizione:} Il sistema è in attesa che l'utente selezioni il file;
	\item \textbf{Flusso degli eventi:}
	\begin{enumerate}
		\item L'utente naviga il \gls{filesystem} alla ricerca del file desiderato [UC7.8.1];
		\item L'utente seleziona il file [UC7.8.2];
		\item L'utente conferma il file selezionato [UC7.8.3].
	\end{enumerate}
	\item \textbf{Postcondizione:} Il sistema ha caricato il file selezionato dall'utente e lo ha inserito nella slide.
\end{itemize}

\subsection{Caso d'uso UC7.8.1: Navigare il filesystem}
\begin{itemize}
	\item \textbf{Attori:} Utente;
	\item \textbf{Scopo e descrizione:} L'utente può navigare il \gls{filesystem} per selezionare la cartella dentro la quale è contenuto il file desiderato;
	\item \textbf{Precondizione:} Il sistema è in attesa che l'utente selezioni una cartella;
	\item \textbf{Postcondizione:} Il sistema ha aggiornato la cartella corrente con quella scelta dall'utente.
\end{itemize}

\subsection{Caso d'uso UC7.8.2: Selezionare il file}
\begin{itemize}
	\item \textbf{Attori:} Utente;
	\item \textbf{Scopo e descrizione:} L'utente deve selezionare il file che intende caricare;
	\item \textbf{Precondizione:} Il sistema mostra i file contenuti nella cartella precedentemente selezionata;
	\item \textbf{Postcondizione:} Il sistema evidenzia il file scelto dall'utente.
\end{itemize}

\subsection{Caso d'uso UC7.8.3: Confermare selezione}
\begin{itemize}
	\item \textbf{Attori:} Utente;
	\item \textbf{Scopo e descrizione:} L'utente conferma che il file selezionato in precedenza è quello corretto;
	\item \textbf{Precondizione:} Il sistema ha selezionato il file indicato dall'utente;
	\item \textbf{Postcondizione:} Il sistema ha caricato il file scelto precedentemente dall'utente.
\end{itemize}


\subsection{Caso d'uso UC7.9: Scegliere formattazione del testo}
\begin{figure}[h] 
	\centering 
	\includegraphics[scale=0.45] {img/UC7.9.png} 
	\caption{UC7.9 - Scegliere formattazione del testo} 
\end{figure}

\begin{itemize}
	\item \textbf{Attori:} Utente;
	\item \textbf{Scopo e descrizione:} L'utente può modificare l'aspetto del testo contenuto in una casella di testo. L'utente seleziona il testo e poi sceglie che modifiche effettuare;
	\item \textbf{Precondizione:} Il sistema è in attesa che l'utente selezioni la modifica da apportare al testo e il testo da modificare è selezionato;
	\item \textbf{Flusso degli eventi:}
	\begin{enumerate}
		\item L'utente può cambiare la grandezza del testo [UC7.9.1];
		\item L'utente può cambiare il colore del testo [UC7.9.2];
		\item L'utente può cambiare il \gls{font} del testo [UC7.9.3];
		\item L'utente può abilitare o disabilitare il testo in corsivo [UC7.9.4];
		\item L'utente può abilitare o disabilitare il testo in grassetto [UC7.9.5];
		\item L'utente può spostare il testo in una nuova posizione [UC7.9.6].
	\end{enumerate}
	\item \textbf{Postcondizione:} Il sistema ha apportato le modifiche scelte al testo.
\end{itemize}

\subsection{Caso d'uso UC7.9.1: Scegliere grandezza}
\begin{itemize}
	\item \textbf{Attori:} Utente;
	\item \textbf{Scopo e descrizione:} L'utente può cambiare la grandezza del testo;
	\item \textbf{Precondizione:} Il testo da modificare è selezionato;
	\item \textbf{Postcondizione:} Il testo è stato ingrandito o rimpicciolito secondo la scelta dell'utente.
\end{itemize}

\subsection{Caso d'uso UC7.9.2: Scegliere colore}
\begin{itemize}
	\item \textbf{Attori:} Utente;
	\item \textbf{Scopo e descrizione:} L'utente può cambiare il colore del testo;
	\item \textbf{Precondizione:} Il testo da modificare è selezionato;
	\item \textbf{Postcondizione:} Il testo è stato colorato secondo la scelta dell'utente.
\end{itemize}

\subsection{Caso d'uso UC7.9.3: Scegliere font}
\begin{itemize}
	\item \textbf{Attori:} Utente;
	\item \textbf{Scopo e descrizione:} L'utente può cambiare il \gls{font} del testo;
	\item \textbf{Precondizione:} Il testo da modificare è selezionato;
	\item \textbf{Postcondizione:} Il testo ha cambiato \gls{font} secondo la scelta dell'utente.
\end{itemize}

\subsection{Caso d'uso UC7.9.4: Abilitare/Disabilitare corsivo}
\begin{itemize}
	\item \textbf{Attori:} Utente;
	\item \textbf{Scopo e descrizione:} L'utente può abilitare o disabilitare la scrittura in corsivo;
	\item \textbf{Precondizione:} Il testo da modificare è selezionato oppure è stata selezionata la casella di testo nella quale poter scrivere;
	\item \textbf{Postcondizione:} Il testo è stato modificato secondo la scelta dell'utente.
\end{itemize}

\subsection{Caso d'uso UC7.9.5: Abilitare/Disabilitare grassetto}
\begin{itemize}
	\item \textbf{Attori:} Utente;
	\item \textbf{Scopo e descrizione:} L'utente può abilitare o disabilitare la scrittura in grassetto;
	\item \textbf{Precondizione:} Il testo da modificare è selezionato oppure è stata selezionata la casella di testo nella quale poter scrivere;
	\item \textbf{Postcondizione:} Il testo è stato modificato secondo la scelta dell'utente.
\end{itemize}

\subsection{Caso d'uso UC7.9.6: Posizionare testo}
\begin{itemize}
	\item \textbf{Attori:} Utente;
	\item \textbf{Scopo e descrizione:} L'utente può spostare una casella di testo in una nuova posizione;
	\item \textbf{Precondizione:} La casella di testo da spostare è stata selezionata;
	\item \textbf{Postcondizione:} La casella di testo è stata spostata secondo la scelta dell'utente.
\end{itemize}


\subsection{Caso d'uso UC7.10: Personalizzare tabella}
\begin{figure}[h] 
	\centering 
	\includegraphics[scale=0.45] {img/UC7.10.png} 
	\caption{UC7.10 - Personalizzare tabella} 
\end{figure}

\begin{itemize}
	\item \textbf{Attori:} Utente;
	\item \textbf{Scopo e descrizione:} L'utente può modificare l'aspetto della tabella e del suo contenuto. L'utente seleziona la tabella o il testo e poi sceglie che modifiche effettuare;
	\item \textbf{Precondizione:} Il sistema è in attesa che l'utente selezioni la modifica da apportare alla tabella e la tabella o il testo da modificare sono selezionati;
	\item \textbf{Flusso degli eventi:}
	\begin{enumerate}
		\item L'utente può cambiare la grandezza della tabella [UC7.10.1];
		\item L'utente può cambiare il colore di sfondo della tabella [UC7.10.2];
		\item L'utente può cambiare l'allineamento del testo [UC7.10.3];
		\item L'utente può cambiare la formattazione del testo [UC7.9];
	\end{enumerate}
	\item \textbf{Postcondizione:} Il sistema ha apportato le modifiche scelte alla tabella.
\end{itemize}

\subsection{Caso d'uso UC7.10.1: Scegliere grandezza tabella}
\begin{itemize}
	\item \textbf{Attori:} Utente;
	\item \textbf{Scopo e descrizione:} L'utente può modificare la grandezza della tabella;
	\item \textbf{Precondizione:} La tabella da modificare è stata selezionata;
	\item \textbf{Postcondizione:} La tabella è stata modificata nelle sue dimensioni secondo la scelta dell'utente.
\end{itemize}

\subsection{Caso d'uso UC7.10.2: Scegliere colore di sfondo tabella}
\begin{itemize}
	\item \textbf{Attori:} Utente;
	\item \textbf{Scopo e descrizione:} L'utente può modificare il colore di sfondo della tabella;
	\item \textbf{Precondizione:} La tabella o le celle da modificare sono state selezionate;
	\item \textbf{Postcondizione:} Lo sfondo della tabella o delle celle è stato modificato secondo la scelta dell'utente.
\end{itemize}

\subsection{Caso d'uso UC7.10.3: Scegliere allineamento del testo}
\begin{itemize}
	\item \textbf{Attori:} Utente;
	\item \textbf{Scopo e descrizione:} L'utente può modificare l'allineamento del testo della tabella;
	\item \textbf{Precondizione:} La tabella o le celle da modificare sono state selezionate;
	\item \textbf{Postcondizione:} L'allineamento del testo della tabella o delle celle è stato modificato secondo la scelta dell'utente.
\end{itemize}


\subsection{Caso d'uso UC7.11: Personalizzare grafico}
\begin{figure}[h] 
	\centering 
	\includegraphics[scale=0.45] {img/UC7.11.png} 
	\caption{UC7.11 - Personalizzare grafico} 
\end{figure}

\begin{itemize}
	\item \textbf{Attori:} Utente;
	\item \textbf{Scopo e descrizione:} L'utente può modificare la tipologia e l'aspetto del grafico o del suo contenuto. L'utente seleziona il grafico e poi sceglie che modifiche effettuare;
	\item \textbf{Precondizione:} Il sistema è in attesa che l'utente selezioni la modifica da apportare al grafico e il grafico da modificare è selezionato;
	\item \textbf{Flusso degli eventi:}
	\begin{enumerate}
		\item L'utente può cambiare la tipologia del grafico [UC7.11.1];
		\item L'utente può cambiare la dimensione del grafico [UC7.11.2]
		\item L'utente può cambiare i colori del grafico [UC7.11.3];
	\end{enumerate}
	\item \textbf{Postcondizione:} Il sistema ha apportato le modifiche scelte al grafico.
\end{itemize}

\subsection{Caso d'uso UC7.11.1: Scegliere tipologia grafico}
\begin{itemize}
	\item \textbf{Attori:} Utente;
	\item \textbf{Scopo e descrizione:} L'utente può modificare la tipologia del grafico;
	\item \textbf{Precondizione:} Il grafico da modificare è stata selezionato;
	\item \textbf{Postcondizione:} La tipologia del grafico è stata modificata secondo la scelta dell'utente.
\end{itemize}

\subsection{Caso d'uso UC7.11.2: Scegliere grandezza grafico}
\begin{itemize}
	\item \textbf{Attori:} Utente;
	\item \textbf{Scopo e descrizione:} L'utente può modificare la grandezza del grafico;
	\item \textbf{Precondizione:} Il grafico da modificare è stata selezionato;
	\item \textbf{Postcondizione:} Il grafico è stata modificato nelle sue dimensioni secondo la scelta dell'utente.
\end{itemize}

\subsection{Caso d'uso UC7.11.3: Scegliere colori grafico}
\begin{itemize}
	\item \textbf{Attori:} Utente;
	\item \textbf{Scopo e descrizione:} L'utente può modificare il set di colori del grafico;
	\item \textbf{Precondizione:} Il grafico da modificare è stata selezionato;
	\item \textbf{Postcondizione:} I colori del grafico sono stati modificati secondo la scelta dell'utente.
\end{itemize}

\subsection{Caso d'uso UC8: Apertura di un progetto}
\begin{figure}[h] 
	\centering 
	\includegraphics[scale=0.45] {img/UC8.png} 
	\caption{UC8 - Apertura di un progetto} 
\end{figure}

\begin{itemize}
	\item \textbf{Attori:} Proprietario;
	\item \textbf{Scopo e descrizione:} L'utente vuole aprire un progetto esistente;
	\item \textbf{Precondizione:} L'utente ha già creato e salvato un progetto in precedenza;
	\item \textbf{Flusso degli eventi:}
	\begin{enumerate}
		\item L'utente seleziona il progetto da aprire [UC8.1].
	\end{enumerate}
	\item \textbf{Postcondizione:} Il sistema apre la schermata di modifica del progetto.
\end{itemize}


\subsection{Caso d'uso UC8.1: Selezionare il progetto da aprire}
\begin{itemize}
	\item \textbf{Attori:} Proprietario;
	\item \textbf{Scopo e descrizione:} L'utente deve selezionare il progetto che vuole aprire tra quelli selezionati in precedenza;
	\item \textbf{Precondizione:} Il sistema mostra i progetti creati dall'utente;
	\item \textbf{Postcondizione:} Il sistema registra la scelta dell'utente e apre la finestra di modifica.
\end{itemize}



\subsection{Caso d'uso UC9: Creazione dell'infografica}
\begin{figure}[h] 
	\centering 
	\includegraphics[scale=0.45] {img/UC9.png} 
	\caption{UC9 - Creazione dell'infografica} 
\end{figure}

\begin{itemize}
	\item \textbf{Attori:} Proprietario;
	\item \textbf{Scopo e descrizione:} L'utente ha creato una presentazione e da questa vuole creare un'infografica. Con l'opportuno tool, può scegliere il template da usare e le slide da includere in essa. Una volta completata la procedura verrà creata l'infografica corrispondente;
	\item \textbf{Precondizione:} L'utente ha creato una presentazione e ha selezionato il comando per creare l'infografica;
	
	\item \textbf{Flusso principale degli eventi:}
	\begin{enumerate}
		\item L'utente sceglie un template [UC9.1];
		\item L'utente sceglie le slide da inserire nell'infografica [UC9.2];
		\item L'utente conferma la creazione dell'infografica [UC9.3];
	\end{enumerate}
	\item \textbf{Postcondizione:} Il sistema ha creato l'infografica a seconda delle scelte dell'utente.
\end{itemize}


\subsection{Caso d'uso UC9.1: Scegliere un template}
\begin{itemize}
	\item \textbf{Attori:} Proprietario;
	\item \textbf{Scopo e descrizione:} L'utente sceglie il template con cui creare l'infografica;
	\item \textbf{Precondizione:} Il sistema ha aperto la finestra di dialogo per la scelta del template;
	\item \textbf{Postcondizione:} Il sistema registra la scelta del template fatta dall'utente.
\end{itemize}


\subsection{Caso d'uso UC9.2: Scegliere le slide}
\begin{itemize}
\item \textbf{Attori:} Proprietario;
\item \textbf{Scopo e descrizione:} L'utente sceglie quali slide del suo progetto inserire nell'infografica;
\item \textbf{Precondizione:} L'utente ha scelto il template con cui creare l'infografica;
\item \textbf{Postcondizione:} Il sistema registra la scelta delle slide fatta dall'utente.
\end{itemize}


\subsection{Caso d'uso UC9.3: Conferma della creazione dell'infografica}
\begin{itemize}
\item \textbf{Attori:} Proprietario;
\item \textbf{Scopo e descrizione:} L'utente vuole confermare la creazione dell'infografica;
\item \textbf{Precondizione:} L'utente ha scelto il template e le slide con cui creare l'infografica;
\item \textbf{Postcondizione:} Il sistema ha creato l'infografica.
\end{itemize}

\subsection{Caso d'uso UC10: Stampa infografica}
\begin{figure}[h] 
	\centering 
	\includegraphics[scale=0.45] {img/UC10.png} 
	\caption{UC10 - Stampa infografica} 
\end{figure}

\begin{itemize}
	\item \textbf{Attori:} Utente;
	\item \textbf{Scopo e descrizione:} L'utente ha creato un'infografica e vuole stamparla;
	\item \textbf{Precondizione:} Il sistema è in attesa che l'utente selezioni la funzione stampa;
	\item \textbf{Flusso degli eventi:}
	\begin{enumerate}
		\item L'utente seleziona la funzione stampa [UC10.1];
		\item L'utente seleziona le impostazioni di stampa [UC10.2];
		\item L'utente conferma la stampa [UC10.3].
	\end{enumerate}
	\item \textbf{Postcondizione:} Il sistema ha mandato in stampa l'infografica.
\end{itemize}

\subsection{Caso d'uso UC10.1: Selezionare la funzione di stampa}
\begin{itemize}
	\item \textbf{Attori:} Utente;
	\item \textbf{Scopo e descrizione:} L'utente seleziona dall'apposito menù la funzione di stampa per stampare l'infografica;
	\item \textbf{Precondizione:} Il sistema è in attesa che l'utente selezioni la funzione stampa;
	\item \textbf{Postcondizione:} Il sistema apre la finestra di dialogo per la stampa.
\end{itemize}

\subsection{Caso d'uso UC10.2: Selezionare le impostazioni di stampa}
\begin{figure}[h] 
	\centering 
	\includegraphics[scale=0.45] {img/UC10.2.png} 
	\caption{UC10.2 - Selezione impostazioni di stampa} 
\end{figure}

\begin{itemize}
	\item \textbf{Attori:} Utente;
	\item \textbf{Scopo e descrizione:} L'utente deve selezionare le impostazioni per la stampa dell'infografica;
	\item \textbf{Precondizione:} Il sistema permette all'utente di selezionare le impostazioni desiderate;
	\item \textbf{Flusso di eventi:}
	\begin{enumerate}
		\item L'utente seleziona quale stampante usare [UC10.2.1]
		\item L'utente seleziona le preferenze di stampa fornite dalla stampante[UC10.2.2]
	\end{enumerate}
	\item \textbf{Postcondizione:} Il sistema registra tutte le impostazioni selezionate dall'utente.
\end{itemize}
	
	\subsection{Caso d'uso UC10.2.1: Selezionare la stampante}
	\begin{itemize}
		\item \textbf{Attori:} Utente;
		\item \textbf{Scopo e descrizione:} L'utente seleziona quale stampante installata nel sistema usare;
		\item \textbf{Precondizione:} Il sistema è in attesa che l'utente selezioni la stampante da usare;
		\item \textbf{Postcondizione:} Il sistema registra la scelta fatta dall'utente.
	\end{itemize}
	
	\subsection{Caso d'uso UC10.2.2: Selezionare le preferenze di stampa}
	\begin{itemize}
		\item \textbf{Attori:} Utente;
		\item \textbf{Scopo e descrizione:} L'utente seleziona le preferenze di stampa fornite dai driver della stampante stessa;
		\item \textbf{Precondizione:} Il sistema è in attesa che l'utente selezioni le preferenze da usare;
		\item \textbf{Postcondizione:} Il sistema registra la scelte fatte dall'utente.
	\end{itemize}

\subsection{Caso d'uso UC10.3: Conferma di stampa}
\begin{itemize}
	\item \textbf{Attori:} Utente;
	\item \textbf{Scopo e descrizione:} L'utente conferma la stampa dell'infografica;
	\item \textbf{Precondizione:} Il sistema ha ricevuto la richiesta di stampa dell'infografica;
	\item \textbf{Postcondizione:} Il sistema ha mandato in stampa l'infografica.
\end{itemize}

\newpage