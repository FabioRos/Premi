\subsection{Prospettive d'uso del prodotto}
Il prodotto ha l'obiettivo di fornire uno strumento in grado di creare una presentazione di slide, sviluppato in tecnologia \gls{HTML5}, che risulti utilizzabile sia da piattaforme desktop che da mobile. Inoltre il prodotto garantirà l'inserimento di dati real time, ad esempio indici di borsa oppure link a immagini e video, rendendo le slide sempre aggiornate. Il software sarà disponibile per i maggiori sistemi operativi: Windows, Linux, Mac OS X.

\subsection{Funzioni del prodotto}
Il prodotto sfrutterà il browser come GUI. Il programma in sè non andrà interpretato come una pagina web, ma come un software che per l'occasione utilizzi il linguaggio JavaScript e le librerie contenute nel browser.
Una volta avviato il programma sarà possibile:
\begin{itemize}
	\item Creare un nuovo progetto o caricare un progetto già esistente:
	\begin{itemize}
		\item Gestire una slide;
		\item Scegliere gli effetti visivi di transizione;
		\item Inserire e posizionare del testo;
		\item Selezionare colore, font, e grandezza del testo;
		\item Inserire e posizionare delle immagini;
		\item Inserire dati real time;
		\item Rimuovere una slide.
	\end{itemize}
	\item Aprire un progetto;
	\item Salvare il progetto;
	\item Esportare il progetto;
	\item Stampare la presentazione creata:
	\begin{itemize}
		\item Stampare tutte le slide o solo alcune.
	\end{itemize}
	\item Creare un'infografica o caricare un'infografica già creata:
	\begin{itemize}
		\item Scegliere il template per l'infografica;
		\item Inserire il contenuto nell'infografica.
	\end{itemize}
	\item Stampare un'infografica.
\end{itemize}

\subsection{Caratteristiche dell'utente}
Non esiste una categoria di utenti definita ai quali sia rivolto il software, al giorno d'oggi infatti la presentazione tramite slide viene utilizzata da studenti, insegnanti, politici, rappresentanti e molti altri. Bisogna quindi essere in grado di fornire un prodotto di semplice utilizzo e intuitivo che sia adatto a tutti.

\subsection{Vincoli Generali}
Il software potrà essere eseguito su qualunque sistema operativo, ma avrà dei vincoli sul browser da utilizzare, che sono i seguenti:
\begin{itemize}
	\item Google Chrome v39;
	\item Mozilla Firefox v34;
	\item Internet Explorer v11;
	\item Opera v26;
	\item Safari v8.
\end{itemize}
