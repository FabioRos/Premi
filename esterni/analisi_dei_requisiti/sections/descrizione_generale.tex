\subsection{Prospettive d'uso del prodotto}
Il prodotto ha l'obbiettivo di fornire uno strumento in grado di creare una presentazione di slide, sviluppato in tecnologia HTML5 GLS??????, che risulti utilizzabile sia da piattaforme desktop che da mobile. Inoltre il prodotto garantirà l'inserimento di dati in \textit{real time}, ad esempio indici di borsa oppure link a immagini e video, rendendo le slide sempre aggiornate. Il software sarà disponibile per i maggiori sistemi operativi: Windows GLS????, Linux GLS????, MAC OS X GLS???.

\subsection{Funzioni del prodotto}
Il prodotto sfrutterà il browser GLS??? come GUI GLS???. Il programma in sè non andrà interpretato come una pagina web, ma come un software che per l'occasione utilizzi il linguaggio Javascript GLS???? e le librerie contenute nel browser GLS???.
Una volta avviato il programma sarà possibile:
\begin{itemize}
	\item Creare un nuovo progetto o caricare un progetto già esistente:
	\begin{itemize}
		\item Creare nuove slide;
		\item Modificare le slide precedentemente create;
		\item Cancellare slide;
		\item Scegliere effetti visivi di transizione;
		\item Inserire e posizionare del testo;
		\item Selezionare colore, font GLS???, e grandezza del testo;
		\item Inserire e posizionare una o più immagini;
		\item Inserire dati real time GLS????.
	\end{itemize}
	\item Salvare il progetto;
	\item Esportare il progetto;
	\item Stampare la presentazione creata:
	\begin{itemize}
		\item Stampare tutte le slide o solo alcune.
	\end{itemize}
	\item Creare un'infografica:
	\begin{itemize}
		\item Scegliere il layout GLS??? per l'infografica;
		\item Inserire il contenuto nell'infografica.
	\end{itemize}
	\item Stampare un'infografica.
\end{itemize}

\subsection{Caratteristiche dell'utente}
Non esiste una categoria di utenti definita ai quali sia rivolto il software, al giorno d'oggi infatti la presentazione tramite slide viene utilizzata da studenti, insegnanti, politici, rappresentanti, dottori e molti altri. Bisogna quindi essere in grado di fornire un prodotto di semplice utilizzo e intuitivo.

\subsection{Vincoli Generali}
Il software potrà essere eseguito su qualunque sistema operativo, ma avrà dei vincoli sul browser GLS??? da utilizzare, che sono i seguenti:
\begin{itemize}
	\item Google Chrome v??????
	\item Mozilla Firefox v?????????
	\item Internet Explorer v??????
	\item Opera v????????
	\item Safari v?????????
\end{itemize}
