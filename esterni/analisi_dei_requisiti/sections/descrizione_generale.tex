\subsection{Prospettive d'uso del prodotto}
Il prodotto ha l'obiettivo di fornire uno strumento in grado di creare una presentazione di \gls{slide}, sviluppato in tecnologia \gls{HTML5}, che risulti utilizzabile sia da piattaforme desktop che mobile. Il prodotto, inoltre, permette l'inserimento di dati \gls{real time}, ad esempio indici economici, meteo, rendendo le \gls{slide} sempre aggiornate e la creazione dell'\gls{infografica} relativa alla presentazione creata, cioè di creare un'immagine esplicativa con i contenuti della presentazione stessa. Il software sarà disponibile per i maggiori sistemi operativi: \gls{Windows}, \gls{Linux}, \gls{Mac OsX}.

\subsection{Funzioni del prodotto}
Il prodotto sfrutterà il \gls{browser} come \gls{GUI}. Il programma in sè non andrà interpretato come una pagina web, ma come un software che per l'occasione utilizzi il linguaggio \gls{JavaScript} e le librerie contenute nel \gls{browser}.
Una volta avviato il programma sarà possibile:
\begin{itemize}
	\item Registrarsi al sito;
	\item Eseguire l'accesso se è già stata fatta la registrazione;
	\item Ricercare un progetto esistente;
	\item Visualizzare un progetto esistente;
	\item Creare un nuovo progetto o caricare un progetto già esistente:
	\begin{itemize}
		\item Gestire una \gls{slide};
		\item Scegliere gli effetti visivi di transizione;
		\item Inserire e posizionare del testo;
		\item Selezionare colore, \gls{font} e grandezza del testo;
		\item Inserire e posizionare delle immagini;
		\item Inserire dati \gls{real time};
		\item Rimuovere una \gls{slide}.
	\end{itemize}
	\item Aprire un progetto;
	\item Salvare il progetto;
	\item Esportare il progetto;
	\item Stampare la presentazione creata:
	\begin{itemize}
		\item Stampare tutte le \gls{slide} o solo alcune.
	\end{itemize}
	\item Creare un'\gls{infografica} o caricare un'\gls{infografica} già creata:
	\begin{itemize}
		\item Scegliere il \gls{template} per l'\gls{infografica};
		\item Inserire il contenuto nell'\gls{infografica}.
	\end{itemize}
	\item Stampare un'\gls{infografica}.
\end{itemize}

\subsection{Caratteristiche dell'utente}
Non esiste una categoria di utenti definita ai quali sia rivolto il software, al giorno d'oggi infatti la presentazione tramite \gls{slide} viene utilizzata da studenti, insegnanti, politici, rappresentanti e molti altri. Bisogna quindi essere in grado di fornire un prodotto di semplice utilizzo e intuitivo che sia adatto a tutti.

\noindent Sono state individuate, però, tre principali tipologie di attori che andranno ad utilizzare il prodotto:
\begin{itemize}
	\item \textbf{Utente non autenticato}: è l'utente che esplora il sito senza autenticarsi e che avrà accesso a un numero ridotto di funzionalità;
	\item \textbf{Utente autenticato}: è l'utente che si è registrato al sito e che ha eseguito l'accesso. Avrà accesso a un numero più elevato di funzionalità;
	\item \textbf{Utente proprietario}: è un'estensione dell'utente autenticato in quanto ha eseguito l'accesso al sito e vuole creare o ha già creato un progetto. Ha accesso a tutte le funzionalità messe a disposizione in quanto può anche modificare i propri progetti.
\end{itemize}

\subsection{Vincoli Generali}
Il software potrà essere eseguito nei principali sistemi operativi, ma avrà dei vincoli sul \gls{browser} da utilizzare, che sono i seguenti:
\begin{itemize}
	\item Google Chrome versione 41 o superiore;
	\item Mozilla Firefox versione 37 o superiore;
	\item Internet Explorer versione 9 o superiore;
	\item Opera v28 o superiore;
	\item Safari versione 8 o superiore.
\end{itemize}

