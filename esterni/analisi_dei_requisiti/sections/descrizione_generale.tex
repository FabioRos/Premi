\subsection{Prospettive d'uso del prodotto}
Il prodotto ha l'obbiettivo di fornire uno strumento in grado di creare una presentazione di slide, sviluppato in tecnologia \gls{HTML5}, che risulti utilizzabile sia da piattaforme desktop che da mobile. Inoltre il prodotto garantirà l'inserimento di dati \gls{real time}, ad esempio indici di borsa oppure link a immagini e video, rendendo le slide sempre aggiornate. Il software sarà disponibile per i maggiori sistemi operativi: \gls{Windows}, \gls{Linux}, \gls{Mac OS X}.

\subsection{Funzioni del prodotto}
Il prodotto sfrutterà il \gls{browser} come \gls{GUI}. Il programma in sè non andrà interpretato come una pagina web, ma come un software che per l'occasione utilizzi il linguaggio \gls{JavaScript} e le librerie contenute nel \gls{browser}.
Una volta avviato il programma sarà possibile:
\begin{itemize}
	\item Creare un nuovo progetto o caricare un progetto già esistente:
	\begin{itemize}
		\item Creare nuove slide;
		\item Modificare le slide precedentemente create;
		\item Cancellare slide;
		\item Scegliere effetti visivi di transizione;
		\item Inserire e posizionare del testo;
		\item Selezionare colore, \gls{font}, e grandezza del testo;
		\item Inserire e posizionare una o più immagini;
		\item Inserire dati \gls{real time}.
	\end{itemize}
	\item Salvare il progetto;
	\item Esportare il progetto;
	\item Stampare la presentazione creata:
	\begin{itemize}
		\item Stampare tutte le slide o solo alcune.
	\end{itemize}
	\item Creare un'\gls{infografica} o caricare un'\gls{infografica} già creata:
	\begin{itemize}
		\item Scegliere il \gls{layout} per l'\gls{infografica};
		\item Inserire il contenuto nell'\gls{infografica}.
	\end{itemize}
	\item Stampare un'\gls{infografica}.
\end{itemize}

\subsection{Caratteristiche dell'utente}
Non esiste una categoria di utenti definita ai quali sia rivolto il software, al giorno d'oggi infatti la presentazione tramite slide viene utilizzata da studenti, insegnanti, politici, rappresentanti e molti altri. Bisogna quindi essere in grado di fornire un prodotto di semplice utilizzo e intuitivo che sia adatto a tutti.

\subsection{Vincoli Generali}
Il software potrà essere eseguito su qualunque sistema operativo, ma avrà dei vincoli sul \gls{browser} da utilizzare, che sono i seguenti:
\begin{itemize}
	\item \gls{Google Chrome} v??????
	\item \gls{Mozilla Firefox} v?????????
	\item \gls{Internet Explorer} v??????
	\item \gls{Opera} v????????
	\item \gls{Safari} v?????????
\end{itemize}
