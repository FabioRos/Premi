Di seguito verranno riportati i requisiti del sistema, utilizzando la seguente notazione: R\{importanza\}\{tipo\}\{codice\}

\begin{itemize}
	\item L'importanza può assumere i seguenti valori:
	\begin{itemize}
		\item \{OBB\}: Requisito obbligatorio;
		\item \{DES\}: Requisito desiderabile;
		\item \{OPZ\}: Requisito opzionale.
	\end{itemize}
	
	\item Tipo può assumere i seguenti valori:
	\begin{itemize}
		\item {F} : Requisito funzionale;
		\item {Q} : Requisito di qualità;
		\item {P} : Requisito di prestazione;
		\item {V} : Requisito di vincolo.
	\end{itemize}
	
	\item Codice rappresenta il codice univoco di ogni requisito in forma gerarchica.
\end{itemize}
\newpage

\subsection{Tabella dei requisiti funzionali}
\begin{table}[h]
	\begin{tabular}{|p{0.2\textwidth}|p{0.55\textwidth}|p{0.2\textwidth}|}
		\toprule
		\textbf{Requisito} & \textbf{Descrizione} & \textbf{Fonti} \\ 
		\midrule 
		R[OBB][F]1 & Il sistema deve permettere ad un utente la
		registrazione & UC1 Interna\\ \midrule 
		R[OBB][F]1.1 & Il sistema deve permettere la scelta di un nome
		utente & UC1.1.4 Interna\\ \midrule 
		R[OBB][F]1.1.1 & Il nome utente deve contenere almeno un carattere & UC1.1.5 Interna\\ \midrule 
		R[OBB][F]1.1.2 & Il nome utente deve essere univoco & UC1.1.5 Interna\\ \midrule 
		R[OBB][F]1.2 & Il sistema deve permettere la scelta di una password & UC1.1.5 Interna\\ \midrule 
		R[OBB][F]1.2.1 & La password deve contenere almeno 8 caratteri & UC1.1.6 Interna\\ \midrule 
		R[OBB][F]1.2.2 & La password deve essere criptata per garantire la
		confidenzialità dei dati & Interna\\ \midrule 
		R[OBB][F]1.3 & Il sistema deve permettere l'inserimento del nome dell'utente & UC1.1.1 Interna\\ \midrule 
		R[OBB][F]1.3.1 & Il nome deve contenere solo caratteri alfabetici & Interna\\ \midrule 
		R[OBB][F]1.4 & Il sistema deve permettere l'inserimento del cognome dell'utente & UC1.1.2 Interna\\ \midrule 
		R[OBB][F]1.4.1 & Il cognome deve contenere solo caratteri alfabetici & Interna\\ \midrule 
		R[OBB][F]1.5 & Il sistema deve permettere l'inserimento di una email & UC1.1.3 Interna\\ \midrule 
		R[OBB][F]1.5.1 & La mail deve essere formata da una stringa seguita
		dal carattere @ e una stringa dopo & UC1.1.4 Interna\\ \midrule 
		R[OPZ]F]1.6 & Il sistema deve permettere la registrazione tramite Social Network & Interna\newline Verbale3\\ \midrule 
		R[OPZ]F]1.6.1 & Il sistema deve permettere la registrazione attraverso
		Facebook & Interna\newline Verbale3\\ \midrule 
		R[OPZ]F]1.6.1.1 & Il sistema deve permettere la visualizzazione di un
		messaggio di errore nel caso di mancata registrazione
		tramite Facebook & Interna\newline Verbale3\\ \midrule 
		R[OPZ]F]1.6.2 & Il sistema deve permettere la registrazione attraverso
		Twitter & Interna\newline Verbale3\\ \midrule 
		R[OPZ]F]1.6.2.1 & Il sistema deve permettere la visualizzazione di un
		messaggio di errore nel caso di mancata registrazione
		tramite Twitter & Interna\newline Verbale3\\ \midrule 

	\end{tabular}
	\end{table}
	\newpage

	\begin{table}[H]
			\begin{tabular}{|p{0.2\textwidth}|p{0.55\textwidth}|p{0.2\textwidth}|}
			\midrule
			R[OPZ]F]1.6.3 & Il sistema deve permettere la registrazione attraverso Google+ & Interna\newline Verbale3\\ \midrule 
			R[OPZ]F]1.6.3.1 & Il sistema deve permettere la visualizzazione di un
			messaggio di errore nel caso di mancata registrazione
			tramite Google+ & Interna\newline Verbale3\\ \midrule 
			R[OBB][F]1.7 & Il sistema deve permettere la visualizzazione di un messaggio di errore se l'indirizzo mail non è conforme & UC1.1.7 Interna\\ \midrule 
			R[OBB][F]1.7.1 & Il sistema deve permettere la visualizzazione di un
			messaggio di errore se l'indirizzo mail non è presente nel database & UC1.1.7 Interna\\ \midrule 
			R[OBB][F]1.7.2 & Il sistema deve permettere la visualizzazione di un
			messaggio di errore se l'indirizzo mail è già presente & UC1.1.7 Interna\\ \midrule 
			R[OBB][F]2 & Il sistema deve permettere l'autenticazione dell'utente & UC2 UC2.2 Interna\\ \midrule 
			R[OBB][F]2.1 & Il sistema deve permettere l'inserimento del nome
			utente & UC2.1 Interna\\ \midrule 
			R[OBB][F]2.2 & Il sistema deve permettere l'inserimento della
			password & UC2.1 Interna\\ \midrule 
			R[OBB][F]2.3 & Il sistema deve segnalare l'errato inserimento delle credenziali & UC2.4 Interna\\ \midrule 
			R[OBB][F]2.4 & Il sistema deve permettere il reindirizzamento alla pagina personale in caso di avvenuta autenticazione & Interna\\ \midrule 
			R[OBB][F]2.5 & Il sistema deve permettere la reimpostazione della password in caso di dimenticanza & Interna\\ \midrule 
			R[OBB][F]2.5.1 & Il sistema deve permettere l'inserimento delle email
			per reimpostare la password & Interna\\ \midrule 
			R[OBB][F]3 & Il sistema deve permettere la ricerca di un progetto & UC3 Interna\\ \midrule 
			R[OBB][F]3.1 & Il sistema deve permettere la ricerca attraverso
			l'inserimento del nome utente & UC3.1 UC3.2 Interna\\ \midrule 
			R[OBB][F]3.2 & Il sistema deve permettere la ricerca attraverso
			l'inserimento del titolo del progetto & UC3.1 UC3.2 Interna\\ \midrule 
			R[OBB][F]3.3 & Il sistema deve permettere la visualizzazione del risultato della ricerca & UC3 Interna\\ \midrule 
			R[OBB][F]3.4 & Il sistema deve permettere il reinserimento di un nome utente diverso per la ricerca & Interna\\ \midrule 
			R[OBB][F]3.5 & Il sistema deve permettere il reinserimento di un titolo diverso per la ricerca & Interna\\ \midrule 
		\end{tabular}
	\end{table}
	\newpage

	\begin{table}[H]
		\begin{tabular}{|p{0.2\textwidth}|p{0.55\textwidth}|p{0.2\textwidth}|}
			\midrule
			R[OBB][F]4 & Il sistema deve permettere all'utente l'apertura del progetto in modalità visualizzazione & UC4 Capitolato\\ \midrule 
			R[OBB][F]4.1 & Il sistema deve permettere all'utente l'apertura di
			una presentazione & UC4.1 Capitolato\\ \midrule 
			R[OBB][F]4.1.1 & Il sistema deve permettere all'utente autenticato
			la visualizzazione di una presentazione in modalità
			presentatore & UC4.3 Capitolato\\ \midrule 
			R[OBB][F]4.1.1.1 & Il sistema deve poter visualizzare la slide corrente
			per il presentatore & UC4.3 Interna\\ \midrule 
			R[OBB][F]4.1.1.2 & Il sistema deve poter visualizzare le slide successive
			per il presentatore & UC4.3 Interna\\ \midrule 
			R[OBB][F]4.1.1.3 & Il sistema deve poter visualizzare le note della slide per il presentatore & UC4.3 Interna\\ \midrule 
			R[OBB][F]4.1.1.4 & Il sistema deve poter visualizzare il tempo trascorso per il presentatore & UC4.3 Interna\\ \midrule 
			R[OBB][F]4.1.2 & Il sistema deve permettere all'utente autenticato
			la visualizzazione di una presentazione in modalità
			ascoltatore & UC4.4 Capitolato\\ \midrule 
			R[OBB][F]4.1.2.1 & Il sistema deve poter visualizzare la slide corrente
			per l'ascoltatore & UC4.4 Interna\\ \midrule 
			R[OBB][F]4.1.3 & Il sistema deve poter visualizzare i comandi di navigazione per l'ascoltatore & UC4.1.1 Interna\\ \midrule 
			R[OPZ]F]4.2 & Il sistema deve permettere all'utente la visualizzazione
			di un'infografica & UC4.2 Interna\newline Verbale4\\ \midrule 
			R[OBB][F]5 & Il sistema deve permettere all'utente di generare il PDF del progetto & UC5 Capitolato\\ \midrule 
			R[OBB][F]5.1 & Il sistema deve permettere all'utente di generare il
			PDF della presentazione & UC5.1 Capitolato\\ \midrule 
			R[OBB][F]5.2 & Il sistema deve permettere all'utente di generare il
			PDF dell'infografica & UC5.2 Interna\newline Verbale4\\ \midrule 
			R[OBB][F]6 & L'utente deve poter esportare un pacchetto standalone per la visualizzazione offline & UC6 Interna\\ \midrule 
			R[OBB][F]6.1 & L'utente deve poter scegliere il percorso per il
			salvataggio del pacchetto & UC6.1 Interna\\ \midrule 
		\end{tabular}
	\end{table}
	\newpage

	\begin{table}[h]
		\begin{tabular}{|p{0.2\textwidth}|p{0.55\textwidth}|p{0.2\textwidth}|}
			\midrule
			R[OBB][F]7 & L'utente deve poter creare un nuovo progetto & UC7 Capitolato\\ \midrule 
			R[OBB][F]7.1 & L'utente deve poter inserire il titolo del progetto & UC7.1 Capitolato\\ \midrule 
			R[OBB][F]8 & Il proprietario deve poter aprire un progetto da lui precedentemente creato & UC8 Capitolato\\ \midrule 
			R[OBB][F]8.1 & Il proprietario deve poter scegliere il progetto da
			aprire & Capitolato\\ \midrule 
			R[OBB][F]9 & Il proprietario deve poter modificare il contenuto di una slide & UC9 Capitolato\\ \midrule 
			R[OBB][F]9.1 & Il proprietario deve poter modificare il template di
			stile della presentazione & UC9.1 Capitolato\newline Verbale2\\ \midrule 
			R[OBB][F]9.2 & Il proprietario deve poter inserire una nuova slide & UC9.2 Capitolato\\ \midrule 
			R[OBB][F]9.2.1 & Il proprietario deve poter inserire una nuova slide
			a destra & UC9.2 Interna\\ \midrule 
			R[OBB][F]9.2.2 & Il proprietario deve poter inserire una nuova slide
			a sinistra & UC9.2 Interna\\ \midrule 
			R[OBB][F]9.3 & Il proprietario deve poter rimuovere una slide & UC9.3 Capitolato\\ \midrule 
			R[OBB][F]9.4 & Il proprietario deve poter inserire un immagine & UC9.4 Capitolato\\ \midrule 
			R[OBB][F]9.4.1 & Il sistema deve accettare in input immagini di
			formato JPG & UC9.4 Capitolato\\ \midrule 
			R[OBB][F]9.4.2 & Il sistema deve accettare in input immagini di
			formato Portable Network Graphics (PNG) & UC9.4 Capitolato\\ \midrule 
			R[OBB][F]9.4.3 & Il sistema deve accettare in input immagini di formato GIF & UC9.4 Capitolato\\ \midrule 
			R[OBB][F]9.5 & Il proprietario deve poter inserire elementi di testo & UC9.5 Capitolato\\ \midrule 
			R[OPZ]F]9.6 & Il proprietario deve poter inserire dati real time & UC9.6 Interna\newline Verbale4\newline Verbale5\\ \midrule 
			R[OPZ]F]9.6.1 & Il proprietario deve poter caricare un file di dati & UC9.6.1 Interna\\ \midrule 
			R[OPZ]F]9.6.2 & Il proprietario deve poter caricare un file per la
			funzione di estrapolazione dati & UC9.6.2 Interna\\ \midrule 
		\end{tabular}
	\end{table}
	\newpage

	\begin{table}[h]
		\begin{tabular}{|p{0.2\textwidth}|p{0.55\textwidth}|p{0.2\textwidth}|}
			\midrule

			R[OPZ]F]9.6.3 & Il proprietario deve poter caricare un file per la funzione di elaborazione di dati dall'array & UC9.6.3 Interna\\ \midrule 
			R[OPZ]F]9.6.4 & Il proprietario deve poter scegliere il tipo di dati real time da inserire & UC9.6.4 Interna\\ \midrule 
			R[OPZ]F]9.6.5 & Il proprietario deve poter scegliere la sorgente di dati real time & UC9.6.5 Interna\\ \midrule 
			R[OPZ]F]9.7 & Il proprietario deve poter inserire le tabelle & UC9.7 Interna\\ \midrule 
			R[OPZ]F]9.7.1 & Il proprietario deve poter scegliere il numero di righe della tabella & UC9.7 Interna\\ \midrule 
			R[OPZ]F]9.7.2 & Il proprietario deve poter scegliere il numero di
			colonne della tabella & UC9.7 Interna\\ \midrule 
			R[OPZ]F]9.8 & Il proprietario deve poter inserire un grafico & UC9.8 Interna\\ \midrule 
			R[OPZ]F]9.8.1 & Il proprietario deve poter scegliere il modello di
			grafico & UC9.8.1 Interna\\ \midrule 
			R[OPZ]F]9.8.1.1 & Il proprietario deve poter scegliere il modello di
			grafico istogramma & UC9.8.1 Interna\\ \midrule 
			R[OPZ]F]9.8.1.2 & Il proprietario deve poter scegliere il modello di
			grafico a torta & UC9.8.1 Interna\\ \midrule 
			R[OPZ]F]9.8.1.3 & Il proprietario deve poter scegliere il modello di grafico a linea & UC9.8.1 Interna\\ \midrule 
			R[OPZ]F]9.8.2 & Il proprietario deve poter inserire il numero di
			elementi che avrà il grafico & UC9.8.2 Interna\\ \midrule 
			R[OPZ]F]9.8.3 & Il proprietario deve poter inserire il nome degli elementi del grafico & UC9.8.2 Interna\\ \midrule 
			R[OPZ]F]9.8.4 & Il proprietario deve poter inserire il valore degli elementi del grafico & UC9.8.2 Interna\\ \midrule 
			R[OBB][F]9.9 & Il proprietario deve poter scegliere un effetto di transizione delle slide & UC9.9 Capitolato\newline Verbale2\\ \midrule 
			R[OBB][F]9.9.1 & Il proprietario deve poter scegliere l'effetto di
			transizione a scorrimento & UC9.9 Capitolato\newline Verbale2\\ \midrule 
			R[OBB][F]9.9.2 & Il proprietario deve poter scegliere l'effetto di
			transizione a dissolvenza & UC9.9 Capitolato\newline Verbale2\\ \midrule 
			R[OBB][F]9.9.3 & Il proprietario deve poter scegliere l'effetto di transizione a convesso & UC9.9 Capitolato\newline Verbale2\\ \midrule

		\end{tabular}
	\end{table}
	\newpage

	\begin{table}[H]
		\begin{tabular}{|p{0.2\textwidth}|p{0.55\textwidth}|p{0.2\textwidth}|}
			\midrule

			R[OBB][F]9.9.4 & Il proprietario deve poter scegliere l'effetto di transizione a concavo & UC9.9 Capitolato\newline Verbale2\\ \midrule 
			R[OBB][F]9.9.5 & Il proprietario deve poter scegliere nessun effetto di transizione & UC9.9 Capitolato\newline Verbale2\\ \midrule 
			R[OBB][F]9.10 & Il proprietario deve poter ridimensionare un elemento selezionato & UC9.10 Interna\\ \midrule 
			R[OBB][F]9.11 & Il proprietario deve poter spostare un elemento selezionato & UC9.11 Interna\\ \midrule 
			R[OBB][F]9.12 & Il proprietario deve poter ruotare un elemento selezionato & UC9.12 Interna\\ \midrule 
			R[OBB][F]9.13 & Il proprietario deve poter rimuovere un elemento selezionato & UC9.13 Interna\\ \midrule 
			R[OBB][F]9.14 & Il proprietario deve poter caricare un file da inserire nella slide & UC9.14 Interna\\ \midrule 
			R[OBB][F]9.15 & Il proprietario deve poter modificare la formattazione del testo & UC9.15 Capitolato\newline Verbale2\\ \midrule 
			R[OBB][F]9.15.1 & Il proprietario deve poter cambiare la grandezza del
			testo & UC9.15.1 Capitolato\newline Verbale2\\ \midrule 
			R[OBB][F]9.15.2 & Il proprietario deve poter cambiare il colore del testo & UC9.15.2 Capitolato\newline Verbale2\\ \midrule 
			R[OBB][F]9.15.3 & Il proprietario deve poter cambiare il font del testo & UC9.15.3 Capitolato\newline Verbale2\\ \midrule 
			R[OBB][F]9.15.4 & Il proprietario deve poter impostare lo stile del testo normale & UC9.15.4 Capitolato\newline Verbale2\\ \midrule 
			R[OBB][F]9.15.5 & Il proprietario deve poter impostare lo stile del testo corsivo & UC9.15.5 Capitolato\newline Verbale2\\ \midrule 
			R[OBB][F]9.15.6 & Il proprietario deve poter impostare lo stile del testo grassetto & UC9.15.6 Capitolato\newline Verbale2\\ \midrule 
			R[OBB][F]9.15.7 & Il proprietario deve poter cambiare la posizione del testo all'interno della slide & UC9.15.7 Capitolato\newline Verbale2\\ \midrule 
			R[OBB][F]9.16 & Il proprietario deve poter modificare una tabella & UC9.16 Interna\\ \midrule 
			R[OBB][F]9.16.1 & Il proprietario deve poter cambiare il contenuto di
			una cella della tabella & UC9.16.1 Interna\\ \midrule 

		\end{tabular}
	\end{table}
	\newpage

	\begin{table}[h]
		\begin{tabular}{|p{0.2\textwidth}|p{0.55\textwidth}|p{0.2\textwidth}|}
			\midrule

			R[OBB][F]9.16.2 & Il proprietario deve poter aggiungere righe alla tabella & UC9.16.2 \\ \midrule
			R[OBB][F]9.16.3 & Il proprietario deve poter cancellare righe alla tabella & UC9.16.2 \\ \midrule
			R[OBB][F]9.16.4 & Il proprietario deve poter aggiungere colonne alla tabella & UC9.16.3 \\ \midrule
			R[OBB][F]9.16.5 & Il proprietario deve poter cancellare colonne alla tabella & UC9.16.3 \\ \midrule
			R[OBB][F]9.16.6 & Il proprietario deve poter modificare la grandezza della tabella & UC9.16.4 \\ \midrule
			R[OPZ][F]9.17 & Il proprietario deve poter modificare un grafico & UC9.17, Interna \\ \midrule
			R[OPZ][F]9.17.1 & Il proprietario deve poter aggiungere un elemento al grafico & UC9.17, Interna \\ \midrule
			R[OPZ][F]9.17.2 & Il proprietario deve poter cancellare un elemento dal grafico & UC9.17, Interna \\ \midrule
			R[OPZ][F]9.17.3 & Il proprietario deve poter modificare il nome di un elemento del grafico & UC9.17, Interna \\ \midrule
			R[OPZ][F]9.17.4 & Il proprietario deve poter modificare il valore di un elemento del grafico & UC9.17, Interna \\ \midrule
			R[OPZ][F]9.17.5 & Il proprietario deve poter modificare il modello di grafico & UC9.17.1, Interna \\ \midrule
			R[OPZ][F]9.17.5.1 & Il proprietario deve poter modificare il modello di grafico a istogramma & UC9.17.1, Interna \\ \midrule
			R[OPZ][F]9.17.5.2 & Il proprietario deve poter modificare il modello di grafico a torta & UC9.17.1, Interna \\ \midrule
			R[OPZ][F]9.17.5.3 & Il proprietario deve poter modificare il modello di grafico a linea & UC9.17.1, Interna \\ \midrule
			R[OPZ][F]9.17.6 & Il proprietario deve poter modificare la grandezza del grafico & UC9.17.2, Interna \\ \midrule
			R[OPZ][F]9.17.7 & Il proprietario deve poter modificare il set di colori del grafico & UC9.17.3, interna \\ \midrule
			R[OPZ][F]9.18 & Il proprietario deve poter modificare le note di una slide & UC9.18 \\ \midrule
			R[OPZ][F]9.19 & Il proprietario deve poter spostare una slide & UC9.19 \\ \midrule

	\end{tabular}
	\end{table}
	\newpage

	\begin{table}[h]
		\begin{tabular}{|p{0.2\textwidth}|p{0.55\textwidth}|p{0.2\textwidth}|}
			\midrule

			R[DES][F]10 & Il proprietario deve poter creare un'\gls{infografica} & UC10, Verbale4 \\ \midrule
			R[DES][F]10.1 & Il proprietario deve poter scegliere il \gls{template} dell'\gls{infografica} & UC10.1, Verbale4 \\ \midrule
			R[DES][F]10.2 & Il proprietario deve poter scegliere le \gls{slide} da inserire nell'\gls{infografica} & UC10.2, Verbale4 \\ \midrule
			R[OBB][F]11 & Il proprietario deve poter salvare il progetto & UC11, Capitolato  \\ \midrule
			R[OBB][F]11.1 & Il sistema deve essere previsto di una funziona salva & UC11.1, Capitolato \\ \midrule
			R[OBB][F]11.2 & Il proprietario deve poter modificare il nome inserito alla creazione del progetto & UC11.2, Capitolato \\ \midrule
			R[OBB][F]12 & Il proprietario deve poter eliminare un progetto & UC12, Capitolato \\ \midrule
			R[OBB][F]12.1 & Il proprietario deve poter selezionare il progetto da eliminare & UC12.1, Capitolato \\ \midrule
			R[OBB][F]12.1.1 & Il sistema deve permettere la visualizzazione di un messaggio per la conferma eleminazione & UC12.2, Capitolato \\ \midrule
			R[OBB][F]12.1.2 & Il proprietario deve poter confermare l'eliminazione del progetto  & UC12.2, Capitolato \\ \midrule
			R[OBB][F]13 & Il proprietario deve poter consultare il manuale utente & UC13 \\

			\bottomrule

		\end{tabular}
		\caption{Tabella dei requisiti funzionali}
	\end{table}
	\newpage

\newpage

\subsection{Tabella dei requisiti di vincolo}

\begin{center}
	\begin{table}[h]
	\begin{tabular}{|l|p{0.7\textwidth}|c|}
		\toprule
		
		\textbf{Requisito} & \textbf{Descrizione} & \textbf{Fonti} \\
		
		\midrule
		R[OBB][V]-1 & Il sistema deve essere portabile & Capitolato \\ \midrule
		R[OBB][V]-1.1 &	Portabilità desktop, tablet, mobile	& Capitolato\\  \midrule
		R[OBB][V]-1.1.1	& Il software deve funzionare su Internet Explorer versione 9 o superiore & Capitolato \\ \midrule
		R[OBB][V]-1.1.2	& Il software deve funzionare su Chrome 41 o superiore & Capitolato \\ \midrule
		R[OBB][V]-1.1.3 & Il software deve funzionare su Mozilla Firefox 37 o superiore & Capitolato  \\ \midrule
		R[OBB][V]-1.1.4	& Il software deve funzionare su Opera 28 o superiore & Capitolato \\

		\bottomrule
		
	\end{tabular}
	\caption{Tabella dei requisiti di vincolo}
	
	\end{table}
	
\end{center}

\newpage
\newpage

\subsection{Tabella dei requisiti di qualità}

\begin{center}
	\begin{table}[h]
	\begin{tabular}{|l|p{0.7\textwidth}|c|}
		\toprule
		
		\textbf{Requisito} & \textbf{Descrizione} & \textbf{Fonti} \\
		
		\midrule
		R[OBB][Q]-1 & Il sistema viene sviluppato secondo le \textit{Norme di Progetto v1.0.0} & Interna \\
		
		\bottomrule

	\end{tabular}
	\caption{Tabella dei requisiti di qualità}
	
	\end{table}
\end{center}
\newpage


\newpage