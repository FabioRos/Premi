\subsection{Caso d'uso UC5: Generazione del file PDF del progetto}
\begin{figure}[h] 
	\centering 
	\includegraphics[scale=0.45] {img/UC5.png} 
	\caption{UC5 - Stampa del progetto} 
\end{figure}

\begin{itemize}
	\item \textbf{Attori:} Utente autenticato, proprietario;
	\item \textbf{Scopo e descrizione:} L'utente ha aperto un progetto e vuole generare il PDF della presentazione o dell'infografica;
	\item \textbf{Precondizione:} L'utente ha un progetto aperto e il sistema è in attesa che selezioni la funzione di generazione del PDF;
	\item \textbf{Flusso principale degli eventi:}
	\begin{enumerate}
		\item L'utente seleziona la funzione di generazione del file PDF della presentazione [UC5.1];
		\item L'utente seleziona la funzione di generazione del file PDF dell'infografica [UC5.2];
	\end{enumerate}
	\item \textbf{Postcondizione:} Il sistema ha mandato in stampa la parte di progetto selezionata.
\end{itemize}


	\subsection{Caso d'uso UC5.1: Generare il PDF della presentazione}
	\begin{itemize}
		\item \textbf{Attori:} Utente autenticato, proprietario;
		\item \textbf{Scopo e descrizione:} L'utente ha aperto un progetto e vuole generare il PDF della presentazione;
		\item \textbf{Precondizione:} L'utente ha un progetto aperto e il sistema è in attesa che selezioni la funzione di generazione del PDF della presentazione;
		\item \textbf{Flusso principale degli eventi:}
		\begin{enumerate}
			\item L'utente seleziona la funzione di generazione del PDF [UC5.1.1];
			\item L'utente seleziona quali slide includere nel PDF [UC5.1.2];
		\end{enumerate}
		\item \textbf{Postcondizione:} Il sistema ha generato il PDF per l'utente e lo rende disponibile nel browser.
	\end{itemize}
	
	
		\subsection{Caso d'uso UC5.1.1: Selezionare la funzione}
		\begin{itemize}
			\item \textbf{Attori:} Utente autenticato, proprietario;
			\item \textbf{Scopo e descrizione:} L'utente seleziona la funzione di generazione del PDF;
			\item \textbf{Precondizione:} C'è una presentazione attiva e il sistema è in attesa che l'utente selezioni la funzione desiderata;
			\item \textbf{Postcondizione:} Il sistema apre la finestra di dialogo per la scelta delle slide.
		\end{itemize}
		
		\subsection{Caso d'uso UC5.1.2.: Selezionare le slide}
		\begin{itemize}
			\item \textbf{Attori:} Utente autenticato, proprietario;
			\item \textbf{Scopo e descrizione:} L'utente seleziona quali slide includere nel file da generare;
			\item \textbf{Precondizione:} Il sistema è in attesa che l'utente selezioni le silde;
			\item \textbf{Postcondizione:} Il sistema registra la scelta fatta dall'utente e genera il file PDF per renderlo disponibile nel browser.
		\end{itemize}


	\subsection{Caso d'uso UC5.2: Generare il PDF dell'infografica}
	\begin{itemize}
		\item \textbf{Attori:} Utente autenticato, proprietario;
		\item \textbf{Scopo e descrizione:} L'utente ha aperto un progetto e vuole generare il PDF dell'infografica attraverso l'apposito comando;
		\item \textbf{Precondizione:} L'utente ha un progetto aperto di cui aveva creato l'infografica e il sistema è in attesa che selezioni la funzione di generazione del PDF dell'infografica;
		\item \textbf{Postcondizione:} Il sistema ha generato il PDF per l'utente e lo rende disponibile nel browser.
	\end{itemize}