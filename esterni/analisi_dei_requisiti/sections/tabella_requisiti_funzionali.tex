\subsection{Tabella dei requisiti funzionali}

	\begin{longtable}{|l|p{0.7\textwidth}|c|}
%	\begin{table}[h]
%		\begin{tabular}{|l|p{0.7\textwidth}|c|}
		\caption{Tabella dei requisiti funzionali}
		
		\toprule
		\textbf{Requisito} & \textbf{Descrizione} & \textbf{Fonti} \\
		\midrule
		\endfirsthead
		
		\toprule
		%\textbf{Requisito} & \textbf{Descrizione} & \textbf{Fonti} \\
		%\midrule
		\endhead
		
		\midrule
		\endfoot
		
		\bottomrule
		\endlastfoot

%		\midrule
%		\endfirsthead
		
%		\midrule
%		\toprule
		\textbf{Requisito} & \textbf{Descrizione} & \textbf{Fonti} \\ \midrule 	\midrule
%		\endhead
		
%		\midrule
%		\endfoot
		
%		\midrule
%		\endlastfoot
		
		R[OBB][F]1 & Il sistema deve permettere ad un utente la registrazione & UC1 \\ \midrule
		R[OBB][F]1.1 & Il sistema deve permettere la scelta di un nome utente univoco & UC1.1.4 \\ \midrule
		R[OBB][F]1.1.1 & Il nome utente deve contenere almeno un carattere & UC1.1.5 \\ \midrule
		R[OBB][F]1.2 & Il sistema deve permettere la scelta di una password & UC1.1.5 \\ \midrule
		R[OBB][F]1.2.1 & La password deve contenere almeno 8 caratteri & UC1.1.6 \\ \midrule
		R[OBB][F]1.3 & Il sistema deve permettere l'inserimento del nome dell'utente & UC1.1.1 \\ \midrule
		R[OBB][F]1.4 & Il sistema deve permettere l'inserimento del cognome dell'utente & UC1.1.2 \\ \midrule
		R[OBB][F]1.5 & Il sistema deve permettere l'inserimento di una email & UC1.1.3 \\ \midrule
		R[OBB][F]1.5.1 & La mail deve essere formata da una stringa seguita dal carattere @ e una stringa dopo & UC1.1.4 \\ \midrule
		R[OPZ][F]1.6 & Il sistema deve permettere la registrazione attraverso social network & \\ \midrule
		R[OPZ][F]1.6.1 & Il sistema deve permettere la registrazione attraverso Facebook & \\ \midrule
		R[OPZ][F]1.6.2 & Il sistema deve permettere la registrazione attraverso Twitter & \\ \midrule
		R[OPZ][F]1.6.3 & Il sistema deve permettere la registrazione attraverso Google+ & \\ \midrule 
		R[OPZ][F]1.6.1.1 & Il sistema deve permettere la visualizzazione di un messaggio di errore nel caso di mancata registrazione tramite Facebook & \\ \midrule
		R[OPZ][F]1.6.2.1 & Il sistema deve permettere la visualizzazione di un messaggio di errore nel caso di mancata registrazione tramite Twitter & \\ \midrule
		R[OPZ][F]1.6.3.1 & Il sistema deve permettere la visualizzazione di un messaggio di errore nel caso di mancata registrazione tramite Google+ & \\ \midrule
		R[OPZ][F]1.6.5 & Il sistema deve permettere la visualizzazione di avvenuta registrazione & \\ \midrule
		R[OBB][F]1.7 & Il sistema deve permettere la visualizzazione di un messaggio di errore se i campi richiesto risultani sbagliati & UC1.3 \\ \midrule
		
		R[OBB][F]2.4 & Il sistema deve permettere il reindirizzamento alla pagina personale in caso di avvenuta autenticazione & UC2.4, UC2.5 \\ \midrule
		R[OBB][F]2.5 & Il sistema deve permettere il recupero della password e del nome utente in caso di dimenticanza & \\ \midrule
		R[OBB][F]2.5.1 & Il sistema deve permettere l'inserimento delle email per il recupero della password e del nome utente & \\ \midrule
		
		R[OBB][F]3 & Il sistema deve permettere la ricerca di un progetto  & UC3 \\ \midrule
		R[OBB][F]3.1.1 & Il sistema deve permettere la ricerca attraverso l'inserimento del nome utente & UC3.1, UC3.2 \\ \midrule
		R[OBB][F]3.1.2 & Il sistema deve permettere la ricerca attraverso l'inserimento del titolo del progetto & UC3.1, UC3.2 \\ \midrule
		R[OBB][F]3.2 & Il sistema deve permettere la visualizzazione del risultato della ricerca & UC3.3 \\ \midrule
		R[OBB][F]3.3.1 & Il sistema deve permettere il reinserimento del nome utente per la ricerca &  \\ \midrule
		R[OBB][F]3.3.2 & Il sistema deve permettere il reinserimento della password per la ricerca &  \\ \midrule
		
		R[OBB][F]4 & Il sistema deve permettere all'utente l'apertura del progetto in modalità visualizzazione & UC4 \\ \midrule
		R[OBB][F]4.1 & Il sistema deve permettere all'utente l'apertura di una presentazione & UC4.1 \\ \midrule
		R[OBB][F]4.1.1 & Il sistema deve permettere all'utente autenticato la visualizzazione di una presentazione in modalità presentatore & UC4.1.1 \\ \midrule
		R[OBB][F]4.1.2 & Il sistema deve permettere all'utente autenticato la visualizzazione di una presentazione in modalità ascoltatore & UC4.1.2 \\ \midrule
		R[OPZ][F]4.2 & Il sistema deve permettere all'utente la visualizzazione di un'infografica & UC4.2 \\ \midrule
		
		R[OBB][F]5 & Il sistema deve permettere all'utente di generare il PDF del progetto & UC5 \\ \midrule
		R[OBB][F]5.1 & Il sistema deve permettere all'utente di generare il PDF della presentazione & UC5.1 \\ \midrule
		R[OBB][F]5.1.1 & Il sistema deve permettere all'utente di scegliere le slide con le quali generare il PDF della presentazione & UC5.1.1 \\ \midrule
		R[OBB][F]5.2 & Il sistema deve permettere all'utente di generare il PDF dell'infografica & UC5.2 \\ \midrule
		
		R[OBB][F]6 & L'utente deve poter esportare un pacchetto stand-alone per la visualizzazione offline & UC6 \\ \midrule
		R[OBB][F]6.2 & L'utente deve poter scegliere il percorso per il salvataggio del pacchetto & UC6.2 \\ \midrule
		
		R[OBB][F]7 & L'utente deve poter creare un nuovo progetto & UC7 \\ \midrule
		R[OBB][F]7.1 & L'utente deve poter inserire il titolo del progetto & UC7.1 \\ \midrule
		R[OBB][F]7.2 & L'utente deve poter scegliere il template di stile per il proprio progetto & UC7.2 \\ \midrule
		
		R[OBB][F]8 & Il proprietario deve poter aprire un progetto da lui precedentemente creato & UC8 \\ \midrule
		R[OBB][F]8.1 & Il proprietario deve poter scegliere il progetto da aprire & UC8.1 \\ \midrule
		
		R[OBB][F]9 & Il proprietario deve poter modificare il contenuto di una slide & UC9, capitolato \\ \midrule
		R[OBB][F]9.1 & Il proprietario deve poter modificare il template di stile della presentazione & UC9.1 \\ \midrule
		R[OBB][F]9.2 & Il proprietario deve poter inserire una nuova slide & UC9.2 \\ \midrule
		R[OBB][F]9.3 & Il proprietario deve poter rimuovere una slide & UC9.3 \\ \midrule
		R[OBB][F]9.4 & Il proprietario deve poter inserire un immagine & UC9.4 \\ \midrule
		R[OBB][F]9.4.1 & Il sistema deve accettare in input immagini di formato: JPG, PNG, GIF & UC9.4 \\ \midrule
		R[OBB][F]9.5 & Il proprietario deve poter inserire elementi di testo & UC9.5 \\ \midrule
		R[OPZ][F]9.6 & Il proprietario deve poter inserire dati real time & UC9.6 \\ \midrule
		R[OPZ][F]9.6.1 & Il proprietario deve poter inserire un file contenente il codice da eseguire  & UC9.6 \\ \midrule
		R[OBB][F]9.7 & Il proprietario deve poter inserire le tabelle & UC9.7 \\ \midrule
		R[OBB][F]9.7.1 & Il proprietario deve poter scegliere il numero di righe della tabella & UC9.7 \\ \midrule
		R[OBB][F]9.7.2 & Il proprietario deve poter scegliere il numero di colonne della tabella & UC9.7 \\ \midrule
		R[OPZ][F]9.8 & Il proprietario deve poter inserire un grafico & UC9.8 \\ \midrule
		R[OPZ][F]9.8.1 & Il proprietario deve poter scegliere il modello di grafico  & UC9.8.1 \\ \midrule
		R[OPZ][F]9.8.2 & Il proprietario deve poter inserire il numero di elementi che avra il grafico & UC9.8.2 \\ \midrule
		R[OPZ][F]9.8.3 & Il proprietario deve poter inserire il nome degli elementi del grafico  & UC9.8.2 \\ \midrule
		R[OPZ][F]9.8.24 & Il proprietario deve poter inserire il valore degli elementi del grafico & UC9.8.2 \\ \midrule
		R[OBB][F]9.9 & Il proprietario deve poter scegliere un effetto di transizione delle slide & UC9.9 \\ \midrule
		
		R[OBB][F]9.10 & Il proprietario deve poter ridimensionare un elemento selezionato & UC9.10 \\ \midrule
		R[OBB][F]9.11 & Il proprietario deve poter spostare un elemento selezionato & UC9.11 \\ \midrule
		R[OBB][F]9.12 & Il proprietario deve poter ruotare un elemento selezionato & UC9.12 \\ \midrule
		R[OBB][F]9.13 & Il proprietario deve poter rimuovere un elemento selezionato & UC9.13 \\ \midrule
		R[OBB][F]9.14 & Il proprietario deve poter caricare un file da inserire nella slide & UC9.14 \\ \midrule
		
		R[OBB][F]9.15 & Il proprietario deve poter modificare la formattazione del testo & UC9.15 \\ \midrule
		R[OBB][F]9.15.1 & Il proprietario deve poter cambiare la grandezza del testo & UC9.15.1 \\ \midrule
		R[OBB][F]9.15.2 & Il proprietario deve poter cambiare il colore del testo & UC9.15.2 \\ \midrule
		R[OBB][F]9.15.3 & Il proprietario deve poter cambiare il font del testo & UC9.15.3 \\ \midrule
		R[OBB][F]9.15.4 & Il proprietario deve poter abilitare/disabilitare il corsivo del testo & UC9.15.4 \\ \midrule
		R[OBB][F]9.15.5 & Il proprietario deve poter abilitare/disabilitare il grassetto del testo & UC9.15.5 \\ \midrule
		R[OBB][F]9.15.6 & Il proprietario deve poter cambiare la posizione del testo all'interno della slide & UC9.15.6 \\ \midrule
		
		R[OBB][F]9.16 & Il proprietario deve poter modificare una tabella & UC9.16 \\ \midrule
		R[OBB][F]9.16.1 & Il proprietario deve poter cambiare il contenuto di una cella della tabella & UC9.16.1 \\ \midrule
		R[OBB][F]9.16.2 & Il proprietario deve poter aggiungere righe alla tabella & UC9.16.2 \\ \midrule
		R[OBB][F]9.16.3 & Il proprietario deve poter cancellare righe alla tabella & UC9.16.2 \\ \midrule
		R[OBB][F]9.16.4 & Il proprietario deve poter aggiungere colonne alla tabella & UC9.16.3 \\ \midrule
		R[OBB][F]9.16.5 & Il proprietario deve poter cancellare colonne alla tabella & UC9.16.3 \\ \midrule
		R[OBB][F]9.16.6 & Il proprietario deve poter modificare la grandezza della tabella & UC9.16.4 \\ \midrule
		
		R[OPZ][F]9.17 & Il proprietario deve poter modificare un grafico & UC9.17 \\ \midrule
		R[OPZ][F]9.17.1 & Il proprietario deve poter aggiungere un elemento al grafico & UC9.17 \\ \midrule
		R[OPZ][F]9.17.2 & Il proprietario deve poter cancellare un elemento dal grafico & UC9.17 \\ \midrule
		R[OPZ][F]9.17.3 & Il proprietario deve poter modificare il nome di un elemento del grafico & UC9.17 \\ \midrule
		R[OPZ][F]9.17.4 & Il proprietario deve poter modificare il valore di un elemento del grafico & UC9.17 \\ \midrule
		R[OPZ][F]9.17.5 & Il proprietario deve poter modificare il modello di grafico & UC9.17.1 \\ \midrule
		R[OPZ][F]9.17.6 & Il proprietario deve poter modificare la grandezza del grafico & UC9.17.2 \\ \midrule
		R[OPZ][F]9.17.7 & Il proprietario deve poter modificare il set di colori del grafico & UC9.17.3 \\ \midrule
		
		R[DES][F]10 & Il proprietario deve poter creare un'infografica & UC10 \\ \midrule
		R[DES][F]10.1 & Il proprietario deve poter scegliere il template dell'infografica & UC10.1 \\ \midrule
		R[DES][F]10.2 & Il proprietario deve poter scegliere le slide da inserire nell'infografica & UC10.2 \\ \midrule
		
		R[OBB][F]11 & Il proprietario deve poter salvare il progetto & UC11 \\ \midrule
		R[OBB][F]11.1 & Il sistema deve essere previsto di una funziona salva & UC11.1 \\ \midrule
		R[OBB][F]11.2 & Il proprietario deve poter modificare il nome inserito alla creazione del progetto & UC11.2 \\ \midrule
		
		R[OBB][F]12 & Il proprietario deve poter eliminare un progetto & UC12 \\ \midrule
		R[OBB][F]12.1 & Il proprietario deve poter selezionare il progetto da eliminare & UC12.1 \\ \midrule
		R[OBB][F]12.2.1 & Il sistema deve permettere la visualizzazione di un messaggio per la conferma eleminazione & UC12.2 \\ \midrule
		R[OBB][F]12.2.2 & Il proprietario deve poter confermare l'eliminazione del progetto  & UC12.2 \\

%		\end{tabular}
		
%	\caption{Tabella dei requisiti funzionali}
%	\end{table}
	\end{longtable}	