\subsection{Caso d'uso UC4: Visualizzazione}
\begin{figure}[h] 
	\centering 
	\includegraphics[scale=0.45] {img/UC4.png} 
	\caption{UC4 - Visualizzazione} 
\end{figure}

\begin{itemize}
	\item \textbf{Attori:} Utente non autenticato, utente autenticato;
	\item \textbf{Scopo e descrizione:} Un utente, una volta aperto un progetto, può scegliere se visualizzare una presentazione o un'infografica;
	\item \textbf{Precondizione:} Il sistema mostra all'utente un progetto;
	\item \textbf{Flusso principale degli eventi:}
	\begin{enumerate}
		\item L'utente sceglie di visualizzare una presentazione [UC4.1];
		\item L'utente sceglie di visualizzare un'infografica [UC4.2].
	\end{enumerate}
	\item \textbf{Postcondizione:} Il sistema mostra all'utente la schermata di visualizzazione di una presentazione o di un'infografica a seconda di cosa è stato scelto.
\end{itemize}

\subsection{Caso d'uso UC4.1: Visualizzazione di una presentazione}
\begin{figure}[h] 
	\centering 
	\includegraphics[scale=0.45] {img/UC4.1.png} 
	\caption{UC4.1 - Visualizzazione di una presentazione} 
\end{figure}

\begin{itemize}
	\item \textbf{Attori:} Utente non autenticato, utente autenticato;
	\item \textbf{Scopo e descrizione:} La visualizzazione di una presentazione permette di scorrere le slide nelle quattro direzioni a seconda di come sono state inserite. In generale, la presentazione segue un flusso principale (da sinistra a destra) nel quale sono presenti i capitoli principali, e un flusso secondario di dettaglio (dall'alto verso il basso) per ogni capitolo che si desidera, nel quale si può esplicitare l'argomento trattato.\\
	Durante il passaggio da un capitolo all'altro avverrà un effetto di "zoom-out, zoom-in" con il quale è possibile avere una panoramica della presentazione e soprattutto delle slide del capitolo che si andrà ad affrontare.\\
	Una volta scelto di visualizzare una presentazione, l'utente può scegliere se visualizzarla come presentatore o come ascoltatore;
	\item \textbf{Precondizione:} L'utente ha scelto di visualizzare una presentazione;
	\item \textbf{Flusso principale degli eventi:}
	\begin{enumerate}
		\item L'utente sceglie di visualizzare la presentazione come presentatore [UC4.1.1];
		\item L'utente sceglie di visualizzare la presentazione come ascoltatore [UC4.1.2].
	\end{enumerate}
	\item \textbf{Postcondizione:} Il sistema mostra all'utente la schermata di visualizzazione di una presentazione con le impostazioni scelte.
\end{itemize}

\subsection{Caso d'uso UC4.1.1: Visualizzare una presentazione come presentatore}
\begin{itemize}
	\item \textbf{Attori:} Utente non autenticato, utente autenticato;
	\item \textbf{Scopo e descrizione:} L'utente ha scelto di visualizzare la presentazione come presentatore e il sistema mostra la presentazione con il supporto visivo dedicato al presentatore;
	\item \textbf{Precondizione:} L'utente ha scelto di visualizzare la presentazione come presentatore;
	\item \textbf{Postcondizione:} Il sistema avvia la presentazione con il supporto visivo dedicato al presentatore.
\end{itemize}

\subsection{Caso d'uso UC4.1.2: Visualizzare una presentazione come ascoltatore}
\begin{itemize}
	\item \textbf{Attori:} Utente non autenticato, utente autenticato;
	\item \textbf{Scopo e descrizione:} L'utente ha scelto di visualizzare la presentazione come ascoltatore e il sistema mostra la presentazione selezionata;
	\item \textbf{Precondizione:} L'utente ha scelto di visualizzare la presentazione come ascoltatore;
	\item \textbf{Postcondizione:} Il sistema avvia la presentazione.
\end{itemize}

\subsection{Caso d'uso UC4.2: Visualizzazione di un'infografica}
\begin{itemize}
	\item \textbf{Attori:} Utente non autenticato, utente autenticato;
	\item \textbf{Scopo e descrizione:} L'utente ha scelto di visualizzare un'infografica e il sistema la mostra;
	\item \textbf{Precondizione:} L'utente ha scelto di visualizzare un'infografica;
	\item \textbf{Postcondizione:} Il sistema mostra all'utente la schermata di visualizzazione dell'infografica.
\end{itemize}