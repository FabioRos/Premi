\subsection{Premi::Model}
	\begin{figure}[h]
		\centering
		\includegraphics[width=0.7\linewidth]{img/premi_model}
		\caption[Premi::Back-End::Model]{Premi::Back-End::Model}
		\label{fig:back_end_premi_model}
	\end{figure}

	
Il package gestisce lo scambio di informazioni tra una sorgente dati e l'interfaccia utente, attraverso i controller. Per ottenere informazioni si comunica con il model. Tutti i model comunicano tra di loro andando a costruire una serie di relazioni che rendono più semplice e veloce il recupero dei dati da parte del controller. \gls{Laravel} utilizza un proprio \gls{ORM}(Object Relational Mapping) chiamato Eloquent. Tutti i model estendono Eloquent che permette l'integrazione del \gls{database} con il tipo di programmazione utilizzata.

\newpage

\newpage
\subsubsection{User}

	\begin{figure}[h]
		\centering
		\includegraphics[width=0.7\linewidth]{img/User}
		\caption[Diagramma della classe User]{Diagramma della classe User}
		\label{fig:User}
	\end{figure}

	\subsubsection*{Descrizione}
	Il model User permette di gestire la collection users del database. Eloquent presume che il nome della classe sia il singolare del nome della collection nel database, quindi collega USer alla collection users.
	\subsubsection*{Utilizzo}
	Il model gestisce la collection users del database.
	\subsubsection*{Attributi}
	\begin{itemize}
		\item \textbf{+ timestamps : boolean = false :}\\
		Di default Eloquent automatizza l'inserimento del timestamp relativo all'inserimento e aggiornamento di un campo. Se alla variabile viene assegnato il valore le informazioni dell'inserimento e del aggiornamento non verranno aggiunto alla collection.
		\item \textbf{\# fillable : array = ['username', 'email', 'firstName', 'lastName', 'password']:}\\
		Quando si crea un model, si deve passare una serie di attributi al costruttore del model stesso. Questi attributi vengono assegnati al model tramite \textbf{mass assignment}. La propietà \textit{fillable} serve a specificare quali attributi devono essere assegnabili tramite il mass-assignment.
		\item \textbf{\# hiddem : array = ['password', 'remember\_token''] : }\\
		La proprietà hidden si aggiunge quando si vuole limitare gli attributi che sono inclusi nel JSON.
	\end{itemize}
	\subsubsection*{Metodi}
	\begin{itemize}
		\item \textbf{+ projects() : Project}\\
		Abbiamo utilizzato la relazione embedsMany per riuscire ad incorporare il model projects all'interno dell'oggetto principale User. Il metodo ritorna Project su cui verrà chiamato il metodo save() nel caso in cui si voglia aggiornare il modello.
	\end{itemize}

\newpage
\subsection{Premi::Http::Controllers}
	\begin{figure}[h]
		\centering
		\includegraphics[width=0.7\linewidth]{img/back_end_premi_http_controllers}
		\caption[Premi::Http::Controllers]{Premi::Http::Controllers}
		\label{fig:back_end_premi_http_controllers}
	\end{figure}

I controller accolgono le richieste dal client sfruttando i dati forniti dal model. Attraverso i controller si organizza il comportamento dell'applicazione. Tutti i controller ereditano la classe \textit{Controller} di base incluso nell'installazione di \gls{Laravel}. I metodi delle classi controller possono ricevere in input la variabile request di tipo Request che rappresenta una richiesta Http proveniente dal front-end, contenente dei valori utilizzati per inizializzare o aggiornare i dati dei vari model.   
\subsubsection{UserController}
\begin{figure}[h]
\centering
\includegraphics[width=0.5\linewidth]{img/back_end_http_controllers_userController}
\caption[Diagramma della classe UserController]{Diagramma della classe UserController}
\label{fig:back_end_http_controllers_userController}
\end{figure}

	\paragraph{Descrizione}
		Questa classe gestisce i dati dell'utente sfruttando i dati forniti dal model.
	\paragraph{Utilizzo}
		La classe è progettata per consentire la creazione, la manipolazione  dei dati utente e l'interrogazione efficiente del database.

	\paragraph{Metodi}
		\begin{itemize}
			\item \textbf{+ show(username = "")}\\
			Il metodo verifica se c'è un utente autenticato. Se la verifica ha avuto successo si procede con il recupero dei dati dell'utente, in particolare il suo profilo e i progetti associati all'utente e restituisce un oggetto JSON contenente le informazioni:\\
			\textbf{Argomenti}
			\begin{itemize}
				\item username : string = "";\\
				Stringa contenente il nome utente con valore di default NULL.
			\end{itemize}
			
			\item \textbf{+ update()}\\
			Il metodo recupera l'utente loggato e aggiorna i dati del profilo. Il metodo ritorno un valore boolean che indica se i dati sono stati aggiornati o meno;
			
			\item \textbf{+ destroy()}
			Il metodo recupera l'utente loggato e lo cancella dal database insieme a tutte le  sue informazioni. Il metodo ritorna un valore boolean che indica se è stato cancellato o meno dal database.
		\end{itemize}
		
\newpage
\subsubsection{ProjectController}
\begin{figure}[h]
\centering
\includegraphics[width=0.5\linewidth]{img/back_end_http_controllers_projectController}
\caption[Diagramma della classe ProjectController]{Diagramma della classe ProjectController}
\label{fig:back_end_http_controllers_projectController}
\end{figure}

	\paragraph{Descrizione}
		Questa classe gestisce i dati di un progetto.
	\paragraph{Utilizzo}
		La classe è progettata per consentire la creazione e la manipolazione dei dati di un progetto.
		
	\paragraph{Metodi}
		\begin{itemize}
			\item \textbf{+ index()}\\
			Il metodo recupera tutti i progetti dell'utente loggato e li restituisce. Il metodo restituisce un JSON con le informazioni richieste;
			\item \textbf{+ store()}\\
			Il metodo crea un nuovo progetto assegnandoli un nome e lo salva nel database. Il metodo restituisce un valore boolean che indica se le informazioni sono state salvate o meno nel database;
			\item \textbf{+ show(project)}\\
			Il metodo interroga il database recuperando il progetto con l'id "project". Il metodo ritorna un JSON con le informazioni di un progetto:\\
			\textbf{Argomenti}
			\begin{itemize}
				\item project : string; \\
				Stringa contenente l'id univoco di un progetto.
			\end{itemize}
			\item \textbf{+ update(project)}\\
			Il metodo recupera il progetto e aggiorna i dati del progetto. Il metodo ritorna un valore boolean che indica se l'aggiornamento delle informazioni è avvenuto o meno:\\
			\textbf{Argomenti}
			\begin{itemize}
				\item project : string; \\
				Stringa contenente l'id univoco di un progetto.
			\end{itemize}
			\item \textbf{+ destroy(project)}\\
			Il metodo recupera il progetto e lo cancella dal database insieme a tutte le  sue informazioni. Il metodo ritorna un valore boolean che indica se la cancellazione del progetto dal database è avvenuta o meno:\\
			\textbf{Argomenti}
			\begin{itemize}
				\item project : string; \\
				Stringa contenente l'id univoco di un progetto.
			\end{itemize}
		\end{itemize}
		
\newpage
\subsubsection{InfographicController}
\begin{figure}[h]
\centering
\includegraphics[width=0.5\linewidth]{img/back_end_http_controllers_infographicController}
\caption[Diagramma della classe InfographicController]{Diagramma della classe InfographicController}
\label{fig:back_end_http_controllers_infographicController}
\end{figure}

	\paragraph{Descrizione}
		Questa classe gestisce i dati di un'infografica.
	\paragraph{Utilizzo}
		La classe è progettata per consentire la creazione e la manipolazione dei dati di un infografica.
		
	\paragraph{Metodi}
		\begin{itemize}
			\item \textbf{+ index(project)}\\
			Il metodo recupera il progetti dell'utente loggato con id = project per poi ricavarsi l'infografica associato a tale progetto e restituisce un oggetto JSON con tutte le infografiche associate al progetto;
			\item \textbf{+ store(project)}\\
			Il metodo crea una nuova infografica assegnandoli un nome e il path per il salvataggio e la salva nel progetto con id = project. Il metodo ritorno un valore boolean che indica se il salvataggio è avvenuto o meno;
			\item \textbf{+ show(project, infographic)}\\
			Il metodo interroga il database recuperando il progetto con l'id "project" e a partire dal progetto recupera l'infografica con id = infographic. Il metodo ritorna un oggetto JSON con le informazioni richieste di una infografica:\\
			\textbf{Argomenti}
			\begin{itemize}
				\item project : string; \\
				Stringa contenente l'id univoco di un progetto;
				\item infographic : string; \\
				Stringa contenente l'id univoco di un'infografica.
			\end{itemize}
			\item \textbf{+ update(project, infographic)}\\
			Il metodo recupera il progetto con id = project e a partire dal progetto recupera l'infografica con id = infografica e aggiorna i dati dell'infografica. Il metodo ritorna un valore boolean che indica se l'aggiornamento è stato effettuato o meno:\\
			\textbf{Argomenti}
			\begin{itemize}
				\item project : string; \\
				Stringa contenente l'id univoco di un progetto;
				\item infographic : string; \\
				Stringa contenente l'id univoco di un'infografica.
			\end{itemize}
			\item \textbf{+ destroy(project, infographic)}\\
			Il metodo recupera il progetto con id = project e a partire dal progetto recupera l'infografica con id = infografica chiamando il metodo delete su tale infografica cancellandola dal database. Il metodo ritorna un valore boolean che indica se la cancellazione è avvenuta o meno:\\
			\textbf{Argomenti}
			\begin{itemize}
				\item project : string; \\
				Stringa contenente l'id univoco di un progetto;
				\item infographic : string; \\
				Stringa contenente l'id univoco di un'infografica.
			\end{itemize}
		\end{itemize}
		
\newpage
\subsubsection{PresentationController}
\begin{figure}[h]
\centering
\includegraphics[width=0.5\linewidth]{img/back_end_premi_http_controllers_presentationController}
\caption[Diagramma della classe PresentationController]{Diagramma della classe PresentationController}
\label{fig:back_end_premi_http_controllers_presentationController}
\end{figure}


	\paragraph{Descrizione}
		Questa classe gestisce i dati della presentazione.
	\paragraph{Utilizzo}
		La classe è stata progettata per consentire la creazione e la manipolazione dei dati di una presentazione.
	
	\paragraph{Metodi}
		\begin{itemize}
			\item \textbf{+ store(project)}\\
			Il metodo crea una nuova presentazione associata al progetto con id = project, assegnandole un titolo e la salva nel database: Il metodo ritorna un valore boolean che indica se il salvataggio nel database è avvenuto o meno:\\
			\textbf{Argomenti}
			\begin{itemize}
				\item project : string; \\
				Stringa contenente l'id univoco di un progetto.
			\end{itemize}
			\item \textbf{+ show(project)}\\
			Il metodo interroga il database recuperando il progetto con l'id "project" e restituisce l'unica presentazione associata a tale progetto. Il metodo ritorna un oggetto JSON contenente le informazioni di una presentazione:\\
			\textbf{Argomenti}
			\begin{itemize}
				\item project : string; \\
				Stringa contenente l'id univoco di un progetto.
			\end{itemize}
			\item \textbf{+ update(project)}\\
			Il metodo recupera il progetto con id = project per poi recuperare l'unica presentazione associata a tale progetto e aggiorna i dati della presentazione. Il metodo ritorna un valore boolean che indica se l'aggiornamento è stato effettuato o meno:\\
			\textbf{Argomenti}
			\begin{itemize}
				\item project : string; \\
				Stringa contenente l'id univoco di un progetto.
			\end{itemize}
			\item \textbf{+ destroy(project)}\\
			Il metodo recupera il progetto con id = project e recupare la presentazione associata a tale progetto e la elimina dal database cancellando tutte le sue informazioni. Il metodo ritorna un valore boolean che indica se la cancellazione nel database è avvenuta o meno:\\
			\textbf{Argomenti}
			\begin{itemize}
				\item project : string; \\
				Stringa contenente l'id univoco di un progetto.
			\end{itemize}
		\end{itemize}
		
\newpage
\subsubsection{SlideController}
\begin{figure}[h]
\centering
\includegraphics[width=0.5\linewidth]{img/back_end_http_controllers_slideController}
\caption[Diagramma della classe SlideController]{Diagramma della classe SlideController}
\label{fig:back_end_http_controllers_slideController}
\end{figure}

	\paragraph{Descrizione}
		Questa classe gestisce i dati di una slide.
	\paragraph{Utilizzo}
		La classe è stata progettata per consentire la creazione e la manipolazione dei dati di una slide.

	\paragraph{Metodi}
		\begin{itemize}
			\item \textbf{+ index(project)}\\
				Il metodo recupera e restituisce tutte le slide associate a una presentazione. Il metodo ritorna un oggetto JSON contenente le slide all'interno della presentazione:\\
				\textbf{Argomenti:}
				\begin{itemize}
					\item project : string; \\
					Stringa contenente l'id univoco di un progetto;
				\end{itemize}
			\item \textbf{+ store(project)}\\
				Il metodo crea una nuova slide.Il metodo ritorna un valore boolean che indica se l'aggiornamento è stato effettuato o meno:\\
				\textbf{Argomenti:}
				\begin{itemize}
					\item project : string; \\
					Stringa contenente l'id univoco di un progetto;
				\end{itemize}
			\item \textbf{+ show(project, slide)}\\
				Il metodo interroga il database recuperando la slide con id = slide. Il metodo ritorna un oggetto JSON contenente tutte le slide associate al progetto:\\
				\textbf{Argomenti:}
					\begin{itemize}
						\item project : string; \\
						Stringa contenente l'id univoco di un progetto;
						\item slide : string;\\
						Stringa contenente l'id univoco di una slide.
					\end{itemize}
			\item \textbf{+ update(project, slide)}\\
				Il metodo recupera il progetto con id = project per poi recuperare l'unica presentazione associata a tale progetto e la slide con id = slide aggiornando i dati della slide. Il metodo ritorna un valore boolean che indica se l'aggiornamento è stato effettuato o meno:\\
					\textbf{Argomenti:}
					\begin{itemize}
						\item project : string; \\
						Stringa contenente l'id univoco di un progetto;
						\item slide : string;\\
						Stringa contenente l'id univoco di una slide.
					\end{itemize}
			\item \textbf{+ destroy(project, slide)}\\
				Il metodo recupera il progetto con id = project e recupare la presentazione associata a tale progetto e la slide con id = slide e la elimina dal database cancellando tutte le sue informazioni. il metodo ritorna un valore boolean che indica se la cancellazione nel database è stata effettuata o meno:\\
				\textbf{Argomenti:}
				\begin{itemize}
					\item project : string; \\
					Stringa contenente l'id univoco di un progetto;
					\item slide : string;\\
					Stringa contenente l'id univoco di una slide.
				\end{itemize}
\end{itemize}

\newpage
\subsubsection{Premi::Http::Controllers::Auth}
Laravel implementa un semplice meccanismo di autenticazione. Infatti, quasi tutto è già configurato "out of the box". Il file di configurazione si trova in config/auth.php, che contiene una serie di opzioni, ben documentate, che useremo per ottimizzare il comportamento del servizio di autenticazione.\\
Ognuno di questi controller usa un trait che include i loro metodi necessari.
	\paragraph{AuthController}
	\begin{figure}[h]
\centering
\includegraphics[width=0.5\linewidth]{img/back_end_http_controllers_authController}
\caption[Diagramma della classe AuthController]{Diagramma della classe AuthController}
\label{fig:back_end_http_controllers_authController}
\end{figure}
		\subparagraph{Descrizione}
			AuthController gestisce la registrazione dei nuovi utenti e i loro accessi.
		\subparagraph{Metodi}
			\begin{itemize}
				\item \textbf{+ \_\_construct()}\\
				Il costruttore della classe AuthController.
				\item \textbf{\# validator(data)}\\
				Il metodo si occupa della validazione di tutte le informazioni che riguardano l'utente al momento della registrazione.\\
					\textbf{Argomenti:}
						\begin{itemize}
							\item data : array;
							Array di valori contenente tutti i dati della registrazione di un utente. 
						\end{itemize}
				\item \textbf{\# create(data)}\\
				Il metodo si occupa della creazione di un nuovo utente.\\
					\textbf{Argomenti:}
						\begin{itemize}
							\item data : array;
							Array di valori contenente tutti i dati della registrazione di un utente.
						\end{itemize}
			\end{itemize}
			
	\paragraph{PasswordController}
	\begin{figure}[h]
\centering
\includegraphics[width=0.5\linewidth]{img/back_end_http_controllers_passwordController}
\caption[Diagramma della classe PasswordController]{Diagramma della classe PasswordController}
\label{fig:back_end_http_controllers_passwordController}
\end{figure}

		\subparagraph{Descrizione}
			PasswordController contiene la logica per aiutare gli utenti per il reset delle loro credenziali di accesso.
		\subparagraph{Metodi}
			\begin{itemize}
				\item \textbf{+ \_\_construct()}\\
				Il costruttore della classe PasswordController.
			\end{itemize}
	

\newpage
\subsection{Premi::Events}
\begin{figure}[h]
	\centering
	\includegraphics[width=0.7\linewidth]{img/premi_back_end_events}
	\caption[Premi::Events]{Premi::Http::Events}
	\label{fig:premi_back_end_events}
\end{figure}
Gli eventi forniscono una semplice implementazione di osservazione che consente all'applicazione di restare in ascolto di eventuali eventi lanciati da qualche funzione, mentre i loro handlers sono salvati all'interno di Premi::Listeners.
\subsubsection{Event}
\begin{figure}[h]
	\centering
	\includegraphics[width=0.5\linewidth]{img/premi_back_end_event}
	\caption[Diagramma della classe Event]{Diagramma della classe Event}
	\label{fig:premi_back_end_event}
\end{figure}


\paragraph{Descrizione}
La classe Events è una classe astratta che viene estesa da tutti i nuovi eventi creati.

\newpage
\subsubsection{ProjectWasCreated}
\begin{figure}[h]
	\centering
	\includegraphics[width=0.5\linewidth]{img/premi_back_end_project_was_created}
	\caption[Diagramma della classe ProjectWasCreated]{Diagramma della classe ProjectWasCreated}
	\label{fig:premi_back_end_project_was_created}
\end{figure}


\paragraph{Descrizione}
La classe ProjectWasCreated è un evento che gestisce la creazione di un nuovo progetto, creando una presentazione con prima slide di default ad esso correlato.

\paragraph{Utilizzo}
Utilizzata quando viene creato un nuovo progetto.

\paragraph{Attributi}
\begin{itemize}
	\item \textbf{+ project : Project}\\
	Variabile di istanza che contiene il progetto che ha generato l'evento.   
\end{itemize}

\paragraph{Metodi:}
\begin{itemize}
	\item \textbf{+ \_\_construct(project: Project) : void}\\
	Crea una nuova istanza dell'evento:\\
	\textbf{Argomenti:}
	\begin{itemize}
		\item project : Project;
		Progetto che ha scatenato l'evento.
	\end{itemize}
\end{itemize}

\newpage
\subsection{Premi::Listeners}
\begin{figure}[h]
	\centering
	\includegraphics[width=0.7\linewidth]{img/back_end_premi_listeners}
	\caption[Premi::Http::Listeners]{Premi::Http::Listeners}
	\label{fig:back_end_premi_listeners}
\end{figure}

I listeners restano in ascolto, in attesa di essere chiamati da qualche evento, e ricevono un istanza di un evento nel loro metodo handle.
\subsubsection{PresentationCreate}
\begin{figure}[h]
	\centering
	\includegraphics[width=0.5\linewidth]{img/premi_back_end_presentation_create}
	\caption[Diagramma della classe PresentationCreate]{Diagramma della classe PresentationCreate}
	\label{fig:premi_back_end_presentation_create}
\end{figure}


\paragraph{Descrizione}
La classe PresentationCreate risponde all'evento ProjectWasCreate e crea una presentazione associata al progetto appena creato.

\paragraph{Utilizzo}
Utilizzata quando viene creato un nuovo progetto e lanciato l'evento ProjectWasCreated.

\paragraph{Metodi:}
\begin{itemize}
	\item \textbf{+ handle(event: ProjectWasCreated) : void}\\
	Crea una nuova presentazione associata al progetto contenuto, e che ha invocato, l'evento ProjectWasCreated:\\
	\textbf{Argomenti:}
	\begin{itemize}
		\item project : Project;
		L'evento che ha chiamato il listener.
	\end{itemize}
\end{itemize}

\newpage
\subsubsection{SlideCreate}
\begin{figure}[h]
	\centering
	\includegraphics[width=0.5\linewidth]{img/premi_back_end_slide_create}
	\caption[Diagramma della classe SlideCreate]{Diagramma della classe SlideCreate}
	\label{fig:premi_back_end_slide_create}
\end{figure}


\paragraph{Descrizione}
La classe SlidCreate risponde all'evento ProjectWasCreate e crea una prima \gls{slide} di default all'interno della presentazione associata al progetto appena creato.

\paragraph{Utilizzo}
Utilizzata quando viene creato un nuovo progetto e lanciato l'evento ProjectWasCreated.

\paragraph{Metodi:}
\begin{itemize}
	\item \textbf{+ handle(event: ProjectWasCreated) : void}\\
	Crea una prima \gls{slide} di default della presentazione associata al progetto contenuto, e che ha invocato, l'evento ProjectWasCreated:\\
	\textbf{Argomenti:}
	\begin{itemize}
		\item project : Project;
		L'evento che ha chiamato il listener.
	\end{itemize}
\end{itemize}

