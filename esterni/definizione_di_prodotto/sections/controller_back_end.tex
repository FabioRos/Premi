\subsubsection{Controller}
La classe base Controller è una classe astratta che usa il \gls{trait} \textit{ValidatesRequest}. \\\textit{ValidatesRequest} permette, tramite un metodo di comodo(validate), di validare le richieste in arrivo con molte tipologie di regole diverse. Il metodo validate() accetta come parametro una richiesta HTTP(la classe Request)e un set di regole di validazione. Se i dati passati nella richiesta "passano" questo test di validazione, il resto del codice viene eseguito normalmente. In caso contrario, invece, l'utente viene possibilmente reindirizzato alla pagina dove si trovava precedentemente, con l'aggiunta di un set di dati (informazioni relative agli errori) da poter mostrare.


\subsubsection{UserController}
\begin{figure}[h]
\centering
\includegraphics[width=0.8\linewidth]{img/back_end_http_controllers_userController}
\caption[Premi::Back-End::Controller::UserController]{Premi::Back-End::Controller::UserController}
\label{fig:back_end_http_controllers_userController}
\end{figure}

	\paragraph{Descrizione}
		Questa classe gestisce i dati dell'utente sfruttando i dati forniti dal model.
	\paragraph{Utilizzo}
		La classe è progettata per consentire l'interrogazione, la manipolazione e l'eliminazione dei dati dell'utente.

	\paragraph{Metodi}
		\begin{itemize}
			\item \textbf{+ show(username: String) : JSON}\\
			Il metodo effettua il recupero dei dati dell'utente tramite la funzione ausiliaria Auth::user() di \gls{Laravel} che ritorna l'utente autenticato nel sistema, a cui corrisponde la variabile \textit{username} univoca e restituisce un oggetto \gls{JSON} contenente le informazioni del profilo:\\
			\textbf{Argomenti}
			\begin{itemize}
				\item username : String;\\
				Stringa contenente il nome utente.
			\end{itemize}
			
			\item \textbf{+ update(request: Request, username: String) : boolean}\\
			Il metodo recupera l'utente autenticato e aggiorna i dati del profilo contenuti nella richiesta HTTP \textit{request}. Il metodo ritorna un valore booleano che indica che i dati sono stati aggiornati:\\
			\textbf{Argomenti}
			\begin{itemize}
				\item request : Request;\\
				Richiesta HTTP contenente i valori con cui eseguire l'aggiornamento del profilo utente.
				\item username : String;\\
				Stringa contenente il nome utente.
			\end{itemize}
			
			\item \textbf{+ destroy(username: String) : boolean}
			Il metodo recupera l'utente autenticato e lo cancella dal \gls{database} insieme a tutte le  sue informazioni. Il metodo ritorna un valore booleano che indica l'avvenuta cancellazione dal \gls{database}:\\
			\textbf{Argomenti}
			\begin{itemize}
				\item username : String;\\
				Stringa contenente il nome utente.
			\end{itemize}
			
			\item \textbf{+ searchByUsername(request: Request) : JSON}\\
			Il metodo ritorna un oggetto \gls{JSON} contenente la lista di tutti gli utenti che hanno \textit{username} uguale a quello inserito nell'apposito form:\\
			\textbf{Argomenti}
			\begin{itemize}
				\item request : Request;\\
				Richiesta HTTP contenente il valore con cui effettuare la ricerca.
			\end{itemize}
		\end{itemize}
		
\newpage
\subsubsection{ProjectController}
\begin{figure}[h]
\centering
\includegraphics[width=0.8\linewidth]{img/back_end_http_controllers_projectController}
\caption[Premi::Back-End::Controller::ProjectController]{Premi::Back-End::Controller::ProjectController}
\label{fig:back_end_http_controllers_projectController}
\end{figure}

	\paragraph{Descrizione}
		Questa classe gestisce i dati di un progetto.
	\paragraph{Utilizzo}
		La classe è progettata per consentire la creazione, la manipolazione dei dati, l'interrogazione e l'eliminazione di un progetto.
		
	\paragraph{Metodi}
		\begin{itemize}
			\item \textbf{+ index(username: String) : JSON}\\
			Il metodo recupera tutti i progetti dell'utente autenticato e li restituisce. Il metodo restituisce un oggetto \gls{JSON} con le informazioni richieste, filtrate dalla funzione statica getParamByProject(Project):\\
			\textbf{Argomenti}
			\begin{itemize}
				\item username : String;\\
				Stringa contenente il nome utente.
			\end{itemize}
			
			\item \textbf{+ store(request: Request, username: String) : boolean}\\
			Il metodo crea un nuovo progetto assegnandoli un nome e lo salva nel \gls{database} all'interno della collection relativa all'utente autenticato. Il metodo restituisce un oggetto \gls{JSON} contenente i parametri filtrati, tramite la funzione getParamByProject(Project), del progetto appena creato. Inoltre emette un segnale di tipo \textit{ProjectWasCreated} per la corretta inizializzazione della presentazione relativa al nuovo progetto:\\
			\textbf{Argomenti}
			\begin{itemize}
				\item request : Request;\\
			 	Richiesta HTTP contenente i valori con cui eseguire la creazione del progetto.
			 	\item username : String;\\
			 	Stringa contenente il nome utente.
			\end{itemize}
			
			\item \textbf{+ update(request: Request, username: String, projectID: String) : boolean}\\
			Il metodo recupera il progetto dell'utente autenticato tramite projectID ed aggiorna i dati del progetto. Il metodo ritorna un valore booleano che indica che l'aggiornamento delle informazioni è avvenuto:\\
			\textbf{Argomenti}
			\begin{itemize}
				\item request : Request;\\
				Richiesta HTTP contenente i valori con cui eseguire l'aggiornamento dei dati del progetto.
				\item username : String;\\
				Stringa contenente il nome utente.
				\item projectID : string; \\
				Stringa contenente l'ID univoco di un progetto.
			\end{itemize}
			
			\item \textbf{+ destroy(username: String, projectID: String) : boolean}\\
			Il metodo recupera il progetto dell'utente autenticato, indicato da projectID, e lo cancella dal \gls{database} insieme a tutte le sue informazioni. Il metodo ritorna un valore booleano che indica che la cancellazione del progetto dal \gls{database} è avvenuta:\\
			\textbf{Argomenti}
			\begin{itemize}
				\item username : String;\\
				Stringa contenente il nome utente.
				\item projectID : string; \\
				Stringa contenente l'ID univoco di un progetto.
			\end{itemize}
			
			\item \textbf{+ searchByProjectName(request: Request) : JSON}\\
			Il metodo ritorna un oggetto \gls{JSON} contenente la lista di tutti i progetti che hanno \textit{name} uguale a quello inserito nell'apposito form:\\
			\textbf{Argomenti}
			\begin{itemize}
				\item request : Request;\\
				Richiesta HTTP contenente il valore con cui effettuare la ricerca.
			\end{itemize}
			
			\item \textbf{+ returnAllFiles(username: String, projectID: String) : JSON}\\
			Il metodo ritorna un oggetto \gls{JSON} contenente tutti i file media dell'utente che ha relativi al progetto con ID uguale a projectID:\\
			\textbf{Argomenti}
			\begin{itemize}
				\item username : String;\\
				Username dell'utente.
				\item projectID : String;\\
				ID del progetto relativo all'utente di cui si vogliono ritornare tutti i file.
			\end{itemize}
			
			\item \textbf{+ returnAllFiles(username: String, projectID: String, filename: String) : void}\\
			Il metodo elimina il file che ha nome uguale a filename relativo al progetto con ID uguale a projectID dell'utente cui corrisponde username:\\
			\textbf{Argomenti}
			\begin{itemize}
				\item username : String;\\
				Username dell'utente.
				\item projectID : String;\\
				ID del progetto relativo all'utente di cui si vuole eliminare un file.
				\item filename : String;\\
				Nome del file che si vuole eliminare.
			\end{itemize}
			
			\item \textbf{+ uploadMedia(username: String, projectID: String) : void}\\
			Il metodo carica nel server un file media nella cartella relativa al progetto con ID uguale a projectID relativo all'utente cui corrisponde username:\\
			\textbf{Argomenti}
			\begin{itemize}
				\item username : String;\\
				Username dell'utente.
				\item projectID : String;\\
				ID del progetto relativo all'utente in cui si vuole caricare un file.
			\end{itemize}
		\end{itemize}
		
\newpage
\subsubsection{InfographicController}
\begin{figure}[h]
\centering
\includegraphics[width=0.8\linewidth]{img/back_end_http_controllers_infographicController}
\caption[Premi::Back-End::Controller::InfographicController]{Premi::Back-End::Controller::InfographicController}
\label{fig:back_end_http_controllers_infographicController}
\end{figure}

	\paragraph{Descrizione}
		Questa classe gestisce i dati di un'\gls{infografica}.
	\paragraph{Utilizzo}
		La classe è progettata per consentire la creazione, la manipolazione dei dati, l'interrogazione e l'eliminazione di un'\gls{infografica}.
		
	\paragraph{Metodi}
		\begin{itemize}
			\item \textbf{+ index(username: String, projectID: String) : JSON}\\
			Il metodo recupera le infografiche relative ad un progetto dell'utente autenticato e restituisce un oggetto \gls{JSON} con tutte le infografiche associate al progetto, filtrandone i parametri con la funzione statica getParamByInfographic(Infographic):\\
			\textbf{Argomenti}
			\begin{itemize}
				\item username : String; \\
				Stringa contenente il nome utente.
				\item projectID : String; \\
				Stringa contenente l'ID univoco di un progetto associato all'utente autenticato.
			\end{itemize}
			
			\item \textbf{+ store(request: Request, username: String, projectID: String) : JSON}\\
			Il metodo crea una nuova \gls{infografica} assegnandoli un nome e il path per il salvataggio e la salva nel progetto con ID = projectID relativo all'utente autenticato. Il metodo ritorna un oggetto \gls{JSON} contenente i dati, filtrati della funzione statica getParamByInfographic(Infographic), dell'\gls{infografica} appena creata:\\
			\textbf{Argomenti}
			\begin{itemize}
				\item request : Request;\\
				Richiesta HTTP contenente i valori con cui creare l'\gls{infografica}.
				\item username : String; \\
				Stringa contenente il nome utente.
				\item projectID : String; \\
				Stringa contenente l'ID univoco di un progetto associato all'utente autenticato.
			\end{itemize}
			
			\item \textbf{+ show(username: String, projectID: String, infographicID: String) : JSON}\\
			Il metodo interroga il \gls{database} recuperando il progetto con l'ID = projectID relativo all'utente autenticato ed a partire dal progetto recupera l'\gls{infografica} con ID = infographicID. Il metodo ritorna un oggetto \gls{JSON} con i parametri, filtrati dalla funzione statica getParamByInfographic(Infographic), dell'\gls{infografica}:\\
			\textbf{Argomenti}
			\begin{itemize}
				\item username : String; \\
				Stringa contenente il nome utente.
				\item projectID : String; \\
				Stringa contenente l'ID univoco di un progetto associato all'utente autenticato.
				\item infographicID : String; \\
				Stringa contenente l'ID univoco di un'\gls{infografica} associato al progetto.
			\end{itemize}
			
			\item \textbf{+ update(request: Request, username: String, projectID: String, infographicID: String) : boolean}\\
			Il metodo recupera il progetto con ID = projectID relativo all'utente autenticato ed a partire dal progetto recupera l'\gls{infografica} con ID = infographicID ed aggiorna i dati \textit{name} e \textit{path} dell'\gls{infografica}. Il metodo ritorna un valore booleano che indica che l'aggiornamento è stato effettuato:\\
			\textbf{Argomenti}
			\begin{itemize}
				\item request : Request;\\
				Richiesta HTTP contenente i valori con cui aggiornare l'\gls{infografica}.
				\item username : String; \\
				Stringa contenente il nome utente.
				\item projectID : String; \\
				Stringa contenente l'ID univoco di un progetto associato all'utente autenticato.
				\item infographicID : String; \\
				Stringa contenente l'ID univoco di un'\gls{infografica} associato al progetto.
			\end{itemize}
			
			\item \textbf{+ destroy(username: String, projectID: String, infographicID: String) : boolean}\\
			Il metodo recupera il progetto con ID = projectID relativo all'utente autenticato ed a partire dal progetto recupera l'\gls{infografica} con ID = infographicID chiamando il metodo \textit{delete} su tale \gls{infografica} cancellandola dal \gls{database}. Il metodo ritorna un valore booleano che indica che la cancellazione è avvenuta:\\
			\textbf{Argomenti}
			\begin{itemize}
				\item username : String; \\
				Stringa contenente il nome utente.
				\item projectID : String; \\
				Stringa contenente l'ID univoco di un progetto associato all'utente autenticato.
				\item infographicID : String; \\
				Stringa contenente l'ID univoco di un'\gls{infografica} associato al progetto.
			\end{itemize}
		\end{itemize}
		
\newpage
\subsubsection{PresentationController}
\begin{figure}[h]
\centering
\includegraphics[width=0.8\linewidth]{img/back_end_http_controllers_presentationController}
\caption[Premi::Back-End::Controller::PresentationController]{Premi::Back-End::Controller::PresentationController}
\end{figure}


	\paragraph{Descrizione}
		Questa classe gestisce i dati della presentazione.
	\paragraph{Utilizzo}
		La classe è stata progettata per consentire l'interrogazione, la manipolazione dei dati e l'eliminazione di una presentazione.
	
	\paragraph{Metodi}
		\begin{itemize}
			\item \textbf{+ update(request: Request, username: String, projectID: String, presentationID: String) : boolean}\\
			Il metodo recupera il progetto con ID = projectID relativo all'utente autenticato ed a partire dal progetto recupera la presentazione con ID = presentationID ed aggiorna i dati \textit{theme} e \textit{transition} della presentazione. Il metodo ritorna un valore booleano che indica che l'aggiornamento è stato effettuato:\\
			\textbf{Argomenti}
			\begin{itemize}
				\item request : Request;\\
				Richiesta HTTP contenente i valori con cui aggiornare la presentazione.
				\item username : String; \\
				Stringa contenente il nome utente.
				\item projectID : String; \\
				Stringa contenente l'ID univoco di un progetto associato all'utente autenticato.
				\item presentationID : String; \\
				Stringa contenente l'ID univoco di una presentazione associata al progetto.
			\end{itemize}
			\newpage
			\item \textbf{+ destroy(username: String, projectID: String, presentationID: String) : boolean}\\
			Il metodo recupera il progetto con ID = projectID relativo all'utente autenticato ed a partire dal progetto recupera la presentazione con ID = presentationID chiamando il metodo \textit{delete} su tale presentazione cancellandola dal \gls{database}. Il metodo ritorna un valore booleano che indica che la cancellazione è avvenuta:\\
			\textbf{Argomenti}
			\begin{itemize}
				\item username : String; \\
				Stringa contenente il nome utente.
				\item projectID : String; \\
				Stringa contenente l'ID univoco di un progetto associato all'utente autenticato.
				\item presentationID : String; \\
				Stringa contenente l'ID univoco di una presentazione associata al progetto.
			\end{itemize}
			\item \textbf{+ updateAxisPosition(request: Request, username: String, projectID: String, presentationID: String) : void}\\
			Il metodo aggiorna la posizione rispetto agli assi X e Y, nella matrice della presentazione, delle \gls{slide} della presentazione che ha ID uguale a presentationID con i valori contenuti all'interno della richiesta Http:\\
			\textbf{Argomenti}
			\begin{itemize}
				\item request : Request;\\
				Richiesta Http contenente i valori con cui aggiornare la posizione rispetto la matrice della presentazione delle \gls{slide};
				\item username : String; \\
				Stringa contenente il nome utente.
				\item projectID : String; \\
				Stringa contenente l'ID univoco di un progetto associato all'utente autenticato.
				\item presentationID : String; \\
				Stringa contenente l'ID univoco della presentazione associata al progetto.
			\end{itemize}
		\end{itemize}
		
\newpage
\subsubsection{SlideController}
\begin{figure}[h]
\centering
\includegraphics[width=0.8\linewidth]{img/back_end_http_controllers_slideController}
\caption[Premi::Back-End::Controller::SlideController]{Premi::Back-End::Controller:: SlideController}
\label{fig:back_end_http_controllers_slideController}
\end{figure}

	\paragraph{Descrizione}
		Questa classe gestisce i dati di una \gls{slide}.
	\paragraph{Utilizzo}
		La classe è stata progettata per consentire la creazione e la manipolazione dei dati di una \gls{slide}.

	\paragraph{Metodi}
		\begin{itemize}
			\item \textbf{+ index(username: String, projectID: String, presentationID: String) : \gls{JSON}}\\
				Il metodo recupera il progetto con ID = projectID relativo all'utente autenticato, per poi recuperare l'unica presentazione associata a tale progetto e restituisce tutte le \gls{slide} associate ad essa. Il metodo ritorna un oggetto \gls{JSON} contenente le \gls{slide} all'interno della presentazione, i cui parametri di ogni slide vengono filtrati dalla funzione statica getSVGBySlides(Slide[]):\\
				\textbf{Argomenti:}
				\begin{itemize}
					\item username : String; \\
					Stringa contenente il nome utente.
					\item projectID : String; \\
					Stringa contenente l'ID univoco di un progetto associato all'utente autenticato.
					\item presentationID : String; \\
					Stringa contenente l'ID univoco di una presentazione associata al progetto.
				\end{itemize}
				
			\item \textbf{+ store(request: Request, username: String, projectID: String, presentationID: String) : \gls{JSON}}\\
				Il metodo crea una nuova \gls{slide} recuperando il progetto con ID = projectID relativo all'utente autenticato, per poi recuperare l'unica presentazione associata a tale progetto e la \gls{slide} con ID = slideID. Invoca il metodo statico della classe Presentation \textit{incrementIndex} che aggiorna in modo corretto tutti gli indici delle \gls{slide} della presentazione. Il metodo ritorna un oggetto \gls{JSON} che contiene la \gls{slide} appena creata:\\
				\textbf{Argomenti:}
				\begin{itemize}
					\item request : Request;\\
					Richiesta HTTP contenente i componenti da inserire nella \gls{slide}.
					\item username : String; \\
					Stringa contenente il nome utente.
					\item projectID : String; \\
					Stringa contenente l'ID univoco di un progetto associato all'utente autenticato.
					\item presentationID : String; \\
					Stringa contenente l'ID univoco di una presentazione associata al progetto.
				\end{itemize}
			
			\newpage
			\item \textbf{+ show(username: String, projectID: String, presentationID: String, slideID: String) : \gls{JSON}}\\
				Il metodo recupera il progetto con ID = projectID relativo all'utente autenticato, per poi recuperare l'unica presentazione associata a tale progetto e la \gls{slide} con ID = slideID. Il metodo ritorna un oggetto \gls{JSON} contenente la \gls{slide} richiesta, i cui parametri sono filtrati dalla funzione statica getComponentsBySlide(Slide):\\
				\textbf{Argomenti:}
					\begin{itemize}
						\item username : String; \\
						Stringa contenente il nome utente.
						\item projectID : String; \\
						Stringa contenente l'ID univoco di un progetto associato all'utente autenticato.
						\item presentationID : String; \\
						Stringa contenente l'ID univoco di una presentazione associata al progetto.
						\item slideID : String; \\
						Stringa contenente l'ID univoco della \gls{slide} associata alla presentazione.
					\end{itemize}
					
			\item \textbf{+ update(request: Request, username: String, projectID: String, presentationID: String, slideID: String) : b}oolean\\
				Il metodo recupera il progetto con ID = projectID relativo all'utente autenticato, per poi recuperare l'unica presentazione associata a tale progetto e la \gls{slide} con ID = slideID. Invoca i metodi statici della classe \gls{Slide} \textit{deleteOldComponents} e \textit{updateNewComponents} per inserire in modo corretto i nuovi componenti passati tramite la richiesta HTTP. Il metodo ritorna un valore booleano che indica che l'aggiornamento è stato effettuato:\\
					\textbf{Argomenti:}
					\begin{itemize}
						\item request : Request;\\
						Richiesta HTTP contenente i componenti con cui aggiornare la \gls{slide}.
						\item username : String; \\
						Stringa contenente il nome utente.
						\item projectID : String; \\
						Stringa contenente l'ID univoco di un progetto associato all'utente autenticato.
						\item presentationID : String; \\
						Stringa contenente l'ID univoco di una presentazione associata al progetto.
						\item slideID : String; \\
						Stringa contenente l'ID univoco della \gls{slide} associata alla presentazione.
					\end{itemize}
					
			\item \textbf{+ destroy(username: String, projectID: String, presentationID: String, slideID: String) : boolean}\\
				Il metodo recupera il progetto con ID = projectID relativo all'utente autenticato, per poi recuperare l'unica presentazione associata a tale progetto e la \gls{slide} con ID = slideID e la elimina dal \gls{database} cancellando tutte le sue informazioni. Invoca il metodo statico della classe Presentation \textit{decrementIndex} che aggiorna in modo corretto tutti gli indici delle \gls{slide} della presentazione. il metodo ritorna un valore booleano che indica che la cancellazione nel \gls{database} è stata effettuata:\\
				\textbf{Argomenti:}
				\begin{itemize}
					\item username : String; \\
					Stringa contenente il nome utente.
					\item projectID : String; \\
					Stringa contenente l'ID univoco di un progetto associato all'utente autenticato.
					\item presentationID : String; \\
					Stringa contenente l'ID univoco di una presentazione associata al progetto.
					\item slideID : String; \\
					Stringa contenente l'ID univoco della \gls{slide} associata alla presentazione.
				\end{itemize}
				
			\item \textbf{+ findByAxis(request: Request, username: String, projectID: String, presentationID: String) : String}\\
			Il metodo recupera il progetto con ID = projectID relativo all'utente autenticato, per poi recuperare l'unica presentazione associata a tale progetto. Il metodo recupera dalla richiesta HTTP gli indici degli assi X ed Y della \gls{slide} che si vuole avere, esegue una query per trovare tale \gls{slide} associata agli indici e ne ritorna l'ID. Ritorna ID = null se non trova nessuna \gls{slide}:\\
			\textbf{Argomenti:}
			\begin{itemize}
				\item request : Request;\\
				Richiesta HTTP contenente i valori degli assi X ed Y della \gls{slide} desiderata.
				\item username : String; \\
				Stringa contenente il nome utente.
				\item projectID : String; \\
				Stringa contenente l'ID univoco di un progetto associato all'utente autenticato.
				\item presentationID : String; \\
				Stringa contenente l'ID univoco di una presentazione associata al progetto.
			\end{itemize}
\end{itemize}

\newpage
\subsubsection{Premi::Http::Controllers::Auth}
\gls{Laravel} utilizza due controller legati all'autenticazione, presenti nel namespace Premi::Http::Controllers::Auth. Il motivo è legato alla comodità di avere due controller che solitamente non subiscono modifiche. 
Implementa un semplice meccanismo di autenticazione. Infatti, quasi tutto è già configurato "out of the box". Il file di configurazione si trova in config/auth.php, che contiene una serie di opzioni, ben documentate, che useremo per ottimizzare il comportamento del servizio di autenticazione.\\
Ognuno di questi controller usa un \gls{trait} che include i loro metodi necessari.
	\paragraph{AuthController}
	\begin{figure}[h]
\centering
\includegraphics[width=0.5\linewidth]{img/back_end_http_controllers_authController}
\caption[Diagramma della classe AuthController]{Diagramma della classe AuthController}
\label{fig:back_end_http_controllers_authController}
\end{figure}
		\subparagraph{Descrizione}
			AuthController gestisce la registrazione dei nuovi utenti e i loro accessi.
		\subparagraph{Metodi}
			\begin{itemize}
				\item \textbf{+ \_\_construct()}\\
				Il costruttore della classe AuthController.
				\item \textbf{\# validator(data: array) : Validator}\\
				Il metodo si occupa della validazione di tutte le informazioni che riguardano l'utente al momento della registrazione.\\
					\textbf{Argomenti:}
						\begin{itemize}
							\item data : array;
							Array di valori contenente tutti i dati della registrazione di un utente. 
						\end{itemize}
				\item \textbf{\# create(data: array) : User}\\
				Il metodo si occupa della creazione di un nuovo utente:\\
					\textbf{Argomenti:}
						\begin{itemize}
							\item data : array;
							Array di valori contenente tutti i dati della registrazione di un utente.
						\end{itemize}
			\end{itemize}
		
\newpage
	\paragraph{PasswordController}
	\begin{figure}[h]
\centering
\includegraphics[width=0.5\linewidth]{img/back_end_http_controllers_passwordController}
\caption[Diagramma della classe PasswordController]{Diagramma della classe PasswordController}
\label{fig:back_end_http_controllers_passwordController}
\end{figure}

		\subparagraph{Descrizione}
			PasswordController contiene la logica per aiutare gli utenti per il reset delle loro credenziali di accesso.
		\subparagraph{Metodi}
			\begin{itemize}
				\item \textbf{+ \_\_construct()}\\
				Il costruttore della classe PasswordController.
			\end{itemize}
	
