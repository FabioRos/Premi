\subsubsection{User}

	\begin{figure}[h]
		\centering
		\includegraphics[width=0.5\linewidth]{img/back_end_user_model}
		\caption[Diagramma della classe User]{Diagramma della classe User}
		\label{fig:back_end_user_model}
	\end{figure}


	\paragraph{Descrizione:}
	Il model User permette di gestire la collection users del database. Eloquent presume che il nome della classe sia il singolare del nome della collection nel database, quindi collega User alla collection users.

	\paragraph{Utilizzo:}
	Il model gestisce la collection users del database.
	
	\paragraph{Attributi:}
	\begin{itemize}
		\item \textbf{+ timestamps : boolean = false :}\\
		Di default Eloquent automatizza l'inserimento del timestamp relativo all'inserimento e aggiornamento di un campo. Se alla variabile viene assegnato il valore \textit{false} le informazioni dell'inserimento e del aggiornamento non verranno aggiunte alla collection.
		\item \textbf{\# fillable : array = ['username', 'email', 'firstName', 'lastName', 'password']:}\\
		Quando si crea un model, si deve passare una serie di attributi al costruttore del model stesso. Questi attributi vengono assegnati al model tramite \textbf{mass assignment}. La propietà \textit{fillable} serve a specificare quali attributi devono essere assegnabili tramite il mass-assignment.
		\begin{itemize}
			\item \textbf{username:} indica il nome utente scelto dall'utente;
			\item \textbf{email:} indica l'indirizzo email inserito dall'utente;
			\item \textbf{firstName:} indica il nome inserito dall'utente;
			\item \textbf{lastName:} indica il cognome inserito dall'utente;
			\item \textbf{password:} indica la password scelta dall'utente.
		\end{itemize}
	\end{itemize}
	
	\paragraph{Metodi:}
	\begin{itemize}
		\item \textbf{+ projects() : Project}\\
		Abbiamo utilizzato la relazione embedsMany per riuscire ad incorporare il model projects all'interno dell'oggetto principale User. Il metodo ritorna Project su cui verrà chiamato il metodo save() nel caso in cui si voglia aggiornare il modello;
	\end{itemize}
\newpage

\subsubsection{Project}

	\begin{figure}[h]
		\centering
		\includegraphics[width=0.5\linewidth]{img/back_end_premi_model_project}
		\caption[Diagramma della classe Project]{Diagramma della classe Project}
		\label{fig:back_end_premi_model_project}
	\end{figure}

	
	\paragraph{Descrizione:}
	Questa classe rappresenta un progetto di un utente, ovvero una presentazione più zero o più infografiche relative a tale presentazione.
	
	\paragraph{Utilizzo:}
	Viene utilizzata alla creazione di un progetto di un utente.
	
	\paragraph{Attributi:}
	\begin{itemize}
		\item \textbf{+ timestamps : boolean = false :}\\
		Di default Eloquent automatizza l'inserimento del timestamp relativo all'inserimento e aggiornamento di un campo. Se alla variabile viene assegnato il valore le informazioni dell'inserimento e del aggiornamento non verranno aggiunto alla collection;
		\item \textbf{\# fillable : array = [’name’]:}\\
		Quando si crea un model, si deve passare una serie di attributi al costruttore del model stesso. Questi attributi vengono assegnati al model tramite \textbf{mass assignment}. La propietà \textit{fillable} serve a specificare quali attributi devono essere assegnabili tramite il mass-assignment.
		\begin{itemize}
			\item \textbf{name:} indica il nome del progetto.
		\end{itemize}
	\end{itemize}
	
	\paragraph{Metodi:}
	\begin{itemize}
		\item \textbf{+ presentations() : Presentation}\\
		Abbiamo utilizzato la relazione embedsOne per riuscire ad incorporare il model presentation all'interno dell'oggetto principale Project. Il metodo ritorna Presentation su cui verrà chiamato il metodo save() nel caso in cui si voglia aggiornare il modello;
		\item \textbf{+ infographics() : Infographic}\\
		Abbiamo utilizzato la relazione embedsMany per riuscire ad incorporare il model Infographic all’interno dell’oggetto principale Project. Il metodo ritorna Infographic su cui verrà chiamato il metodo save() nel caso in cui si voglia aggiornare il modello.
	\end{itemize}

\newpage
\subsubsection{Infographic}

	\begin{figure}[h]
		\centering
		\includegraphics[width=0.5\linewidth]{img/back_end_premi_model_infographic}
		\caption[Diagramma della classe Infographic]{Diagramma della classe Infographic}
		\label{fig:back_end_premi_model_infographic}
	\end{figure}


	\paragraph{Descrizione:}
	Questa classe rappresenta un’infografica di un progetto, ovvero una rappresentazione visuale della presentazione per mostrare in maniera semplice e veloce le informazioni.
	
	\paragraph{Utilizzo:}
	Viene utilizzata alla creazione di un’infografica di una presentazione.

	\paragraph{Attributi:}
		\begin{itemize}
			\item \textbf{+ timestamps : boolean = false :}\\
			Di default Eloquent automatizza l'inserimento del timestamp relativo all'inserimento e aggiornamento di un campo. Se alla variabile viene assegnato il valore le informazioni dell'inserimento e del aggiornamento non verranno aggiunto alla collection;
			\item \textbf{\# fillable : array = [’name’, ’path’]:}\\
			Quando si crea un model, si deve passare una serie di attributi al costruttore del model stesso. Questi attributi vengono assegnati al model tramite \textbf{mass assignment}. La propietà \textit{fillable} serve a specificare quali attributi devono essere assegnabili tramite il mass-assignment.
			\begin{itemize}
				\item \textbf{name:} indica il nome dell'infografica;
				\item \textbf{path:} indica il percorso dove recuperare il file dell'infografica.
			\end{itemize}
		\end{itemize}
\newpage


\subsubsection{Presentation}

	\begin{figure}[h]
		\centering
		\includegraphics[width=0.5\linewidth]{img/back_end_premi_model_presentation}
		\caption[Diagramma della classe Presentation]{Diagramma della classe Presentation}
		\label{fig:back_end_premi_model_presentation}
	\end{figure}


	\paragraph{Descrizione:}
	Questa classe descrive la presentazione di un progetto. Contiene tutte le slide che servono a comporre la presentazione.

	\paragraph{Utilizzo}
	Viene utilizzato alla creazione o caricamento di una presentazione.
	
	\paragraph{Attributi:}
	\begin{itemize}
		\item \textbf{+ timestamps : boolean = false :}\\
		Di default Eloquent automatizza l'inserimento del timestamp relativo all'inserimento e aggiornamento di un campo. Se alla variabile viene assegnato il valore le informazioni dell'inserimento e del aggiornamento non verranno aggiunto alla collection;
		\item \textbf{\# fillable : array = ['title']:}\\
		Quando si crea un model, si deve passare una serie di attributi al costruttore del model stesso. Questi attributi vengono assegnati al model tramite \textbf{mass assignment}. La propietà \textit{fillable} serve a specificare quali attributi devono essere assegnabili tramite il mass-assignment.
		\begin{itemize}
			\item \textbf{title:} indica il titolo della presentazione.
		\end{itemize}
	\end{itemize}

	\paragraph{Metodi:}
	\begin{itemize}
		\item \textbf{+ slides() : Slide}\\
		Abbiamo utilizzato la relazione embedsMany per riuscire ad incorporare il model Slide all’interno dell’oggetto principale Presentation. Il metodo ritorna Slide su cui verrà chiamato il metodo save() nel caso in cui si voglia aggiornare il modello.
	\end{itemize}
\newpage


\subsubsection{Slide}

	\begin{figure}[h]
		\centering
		\includegraphics[width=0.5\linewidth]{img/back_end_premi_model_slide}
		\caption[Diagramma della classe Slide]{Diagramma della classe Slide}
		\label{fig:back_end_premi_model_slide}
	\end{figure}


	\paragraph{Descrizione:}
	Questa classe descrive una singola slide. Contiene tutti i gli oggetti appartenenti alla slide.
	
	\paragraph{Utilizzo:}
	Viene utilizzato alla creazione o caricamento di una slide.

	\paragraph{Attributi:}
	\begin{itemize}
		\item \textbf{+ timestamps : boolean = false :}\\
		Di default Eloquent automatizza l'inserimento del timestamp relativo all'inserimento e aggiornamento di un campo. Se alla variabile viene assegnato il valore le informazioni dell'inserimento e del aggiornamento non verranno aggiunto alla collection;
		\item \textbf{\# fillable : array = [’xIndex’, ’yIndex']:}\\
		Quando si crea un model, si deve passare una serie di attributi al costruttore del model stesso. Questi attributi vengono assegnati al model tramite \textbf{mass assignment}. La propietà \textit{fillable} serve a specificare quali attributi devono essere assegnabili tramite il mass-assignment.
		\begin{itemize}
			\item \textbf{xIndex:} indica la posizione relativa all'asse X delle slide nella matrice di Slide della presentazione;
			\item \textbf{yIndex:} indica la posizione relativa all'asse Y delle slide nella matrice di Slide della presentazione.
		\end{itemize}
	\end{itemize}
	
	\paragraph{Metodi:}
	\begin{itemize}
		\item \textbf{+ components() : Component}\\
		Abbiamo utilizzato la relazione embedsMany per riuscire ad incorporare il model Component all’interno dell’oggetto principale Slide. Il metodo ritorna Component su cui verrà chiamato il metodo save() nel caso in cui si voglia aggiornare il modello.
	\end{itemize}
	\newpage
	

\subsubsection{Component}

	\begin{figure}[h]
		\centering
		\includegraphics[width=0.5\linewidth]{img/back_end_premi_model_component}
		\caption[Diagramma della classe Component]{Diagramma della classe Component}
		\label{fig:back_end_premi_model_component}
	\end{figure}


	\paragraph{Descrizione:}
	Questa classe descrive la struttura genera di un componente.
	
	\paragraph{Utilizzo:}
	Viene utilizzato alla creazione o caricamento di una componente.
	
	\paragraph{Attributi:}
	\begin{itemize}
		\item \textbf{+ timestamps : boolean = false :}\\
		Di default Eloquent automatizza l'inserimento del timestamp relativo all'inserimento e aggiornamento di un campo. Se alla variabile viene assegnato il valore le informazioni dell'inserimento e del aggiornamento non verranno aggiunto alla collection;
		\item \textbf{\# fillable : array = [’type’, ’originX’, ’OriginY’, ’left’, ’top’, ’width’, ’height’, ’fill’, ’stroke’, ’strokeWidth’, ’strokeDashArray’, ’strokeLineCap’, ’strokeLine-Join’, ’strokeMiterLimit’, ’scaleX’, ’scaleY’, ’angle’, ’flipX’, ’flipY’, ’opacity’, ’shadow’, ’visible’, ’clipTo’, ’backgroundColor’, ’fillRule’, ’globalCompositeOperation']:}\\
		Quando si crea un model, si deve passare una serie di attributi al costruttore del model stesso. Questi attributi vengono assegnati al model tramite \textbf{mass assignment}. La propietà \textit{fillable} serve a specificare quali attributi devono essere assegnabili tramite il mass-assignment.
		\begin{itemize}
			\item \textbf{type:} indica il tipo di componente;
			\item \textbf{originX:} indica la posizione relativa all'asse X del vertice d'origine del componente;
			\item \textbf{originY:} indica la posizione relativa all'asse Y del vertice d'origine del componente;
			\item \textbf{left:} indica la posizione relativa all'asse delle X dell'area di disegno in cui verrà disegnato il componente;
			\item \textbf{top:} indica la posizione relativa all'asse delle Y dell'area di disegno in cui verrà disegna il componente;
			\item \textbf{width:} larghezza del riquadro che racchiode il componente;
			\item \textbf{height:} altezza del riquadro che racchiode il componente;
			\item \textbf{fill:} indica il colore di riempimento del
			\item \textbf{stroke:} indica il colore del tratto utilizzato per disegnare il componente;
			\item \textbf{strokeWidth:} indica la larghezza del tratto utilizzato per disegnare il componente;
			\item \textbf{strokeDashArray:} indica il tipo di tratto utilizzato per disegnare il componente;
			\item \textbf{strokeLineCap:} indica lo stile di fine del tratto utilizzato per disegnare il componente;
			\item \textbf{strokeLine-Join:} indica lo stile degli angoli del tratto utilizzato per disegnare il componente;
			\item \textbf{strokeMiterLimit:} indica il limite della distanza di unione tra due linee;
			\item \textbf{scaleX:} indica il volore con cui viene scalato in larghezza il componente;
			\item \textbf{scaleY:} indica il volore con cui viene scalato in altezza il componente;
			\item \textbf{angle:} indica l'angolo di rotazione del componente;
			\item \textbf{flipX:} indica se il componente è capovolto rispetto all'asse X;
			\item \textbf{flipY:} indica se il componente è capovolto rispetto all'asse Y;
			\item \textbf{opacity:} indica il livello di opacità del componente;
			\item \textbf{shadow:} indica il livello dell'ombra del componente;
			\item \textbf{visible:} indica  se il componente è visibile o no;
			\item \textbf{clipTo:} indica se il componente ha l'effetto clip attivo;
			\item \textbf{backgroundColor:} indica il colore di sfondo del componente;
			\item \textbf{fillRule:} indica lo stile di riempimento;
			\item \textbf{globalCompositeOperation:} indica l'ordine in cui vengono disegnati i componenti all'interno di un gruppo.
		\end{itemize}
	\end{itemize}
\newpage

\subsubsection{Chart}




	\paragraph{Descrizione}
	Questa classe rappresenta la struttura di dati necessari per descrivere un grafico all’interno di una slide.
	
	\paragraph{Utilizzo}
	Viene utilizzato alla creazione o caricamento di un grafico.

	\paragraph{Attributi}
	\begin{itemize}
		\item \textbf{+ timestamps : boolean = false :}\\
		Di default Eloquent automatizza l'inserimento del timestamp relativo all'inserimento e aggiornamento di un campo. Se alla variabile viene assegnato il valore le informazioni dell'inserimento e del aggiornamento non verranno aggiunto alla collection.
		\item \textbf{\# fillable : array = [’typeChart’ , ’data’]:}\\
		Quando si crea un model, si deve passare una serie di attributi al costruttore del model stesso. Questi attributi vengono assegnati al model tramite \textbf{mass assignment}. La propietà \textit{fillable} serve a specificare quali attributi devono essere assegnabili tramite il mass-assignment.
		\begin{itemize}
			\item \textbf{typeChart:} indica il tipo di grafico;
			\item \textbf{data:} indica l'array di dati che servono a comporre il grafico.
		\end{itemize}
		
	\end{itemize}
\newpage


\subsubsection{RealTimeData}

	\begin{figure}[h]
		\centering
		\includegraphics[width=0.5\linewidth]{img/back_end_premi_model_realTimeData}
		\caption[Diagramma della classe RealTimeData]{Diagramma della classe RealTimeData}
		\label{fig:back_end_premi_model_realTimeData}
	\end{figure}


	\paragraph{Descrizione}
	Questa classe rappresenta la struttura di dati necessari per descrivere RealTimeData (dato in tempo reale) all’interno di una slide.
	
	\paragraph{Utilizzo}
	Viene utilizzato alla creazione o caricamento di un RealTimeData.
	
	\paragraph{Attributi}
	\begin{itemize}
		\item \textbf{+ timestamps : boolean = false :}\\
		Di default Eloquent automatizza l'inserimento del timestamp relativo all'inserimento e aggiornamento di un campo. Se alla variabile viene assegnato il valore le informazioni dell'inserimento e del aggiornamento non verranno aggiunto alla collection.
		\item \textbf{\# fillable : array = ['pathParser’, ’pathFallback’, ’pathHandlerJs']:}\\
		Quando si crea un model, si deve passare una serie di attributi al costruttore del model stesso. Questi attributi vengono assegnati al model tramite \textbf{mass assignment}. La propietà \textit{fillable} serve a specificare quali attributi devono essere assegnabili tramite il mass-assignment.
		\begin{itemize}
			\item \textbf{pathParser:} Indirizzo del parser php \footnote{file php con il compito di recuperare i dati da un indirizzo remoto (evitando cosi il poroblema della same-origin-policy) fornendo come risultato un oggetto Json che sara' poi processato dall'Handler javcascript};
			\item \textbf{pathFallback:} Indirizzo del file contenente l'oggetto più recente salvato in caso di consultazione offline;
			\item \textbf{pathHandlerJs:} Indirizzo dell'handler javascript con il compitpo di impaginare i risultati provenienti dal parser e riaggiornare la vista dopo n secondi.
		\end{itemize}
	\end{itemize}
\newpage


\subsubsection{Text}

	\begin{figure}[h]
		\centering
		\includegraphics[width=0.5\linewidth]{img/back_end_premi_model_text}
		\caption[Diagramma della classe Text]{Diagramma della classe Text}
		\label{fig:back_end_premi_model_text}
	\end{figure}


	\paragraph{Descrizione}
	Questa classe rappresenta la struttura di dati di un campo di testo di una slide.
	
	\paragraph{Utilizzo}
	Viene utilizzato alla creazione o caricamento di un campo di testo.
	
	\paragraph{Attributi}
	\begin{itemize}
		\item \textbf{+ timestamps : boolean = false :}\\
		Di default Eloquent automatizza l'inserimento del timestamp relativo all'inserimento e aggiornamento di un campo. Se alla variabile viene assegnato il valore le informazioni dell'inserimento e del aggiornamento non verranno aggiunto alla collection.
		\item \textbf{\# fillable : array = [’text’, ’fontSize’, ’fontWeight’, ’fontFamily’, ’fontStyle’, ’lineHeight’, ’textDecoration’, ’textAlign’, ’textBackgroundColor']:}\\
		Quando si crea un model, si deve passare una serie di attributi al costruttore del model stesso. Questi attributi vengono assegnati al model tramite \textbf{mass assignment}. La propietà \textit{fillable} serve a specificare quali attributi devono essere assegnabili tramite il mass-assignment.
		\begin{itemize}
			\item \textbf{text:} indica il contenuto del campo di testo;
			\item \textbf{fontSize:} indica le dimensioni del font del campo di testo;
			\item \textbf{fontWeight:} indica lo spessore del testo
			\item \textbf{fontFamily:} indica la famiglia di font del campo di testo;
			\item \textbf{fontStyle:} indica lo stile del font del campo di testo (corsivo, grasseto, sottolineato);
			\item \textbf{lineHeight:} indica l'altezza della linea del campo di testo;
			\item \textbf{textDecoration:} indica il tipo di decorazione;
			\item \textbf{textAlign:} indica l'allineamento del testo all'interno del campo di testo;
			\item \textbf{textBackgroundColor:} indica il colore di sfondo del campo di testo.
		\end{itemize}
	\end{itemize}
\newpage


\subsubsection{Image}

	\begin{figure}[h]
		\centering
		\includegraphics[width=0.5\linewidth]{img/back_end_premi_model_image}
		\caption[Diagramma della classe Image]{Diagramma della classe Image}
		\label{fig:back_end_premi_model_image}
	\end{figure}


	\paragraph{Descrizione}
	La classe Image rappresenta la struttura dei dati necessari per rapresentare un’immagine all’interno di una slide.
	
	\paragraph{Utilizzo}
	Utilizzata quando viene inserita un’immagine per tenerne traccia.
	
	\paragraph{Attributi}
	\begin{itemize}
		\item \textbf{+ timestamps : boolean = false :}\\
		Di default Eloquent automatizza l'inserimento del timestamp relativo all'inserimento e aggiornamento di un campo. Se alla variabile viene assegnato il valore le informazioni dell'inserimento e del aggiornamento non verranno aggiunto alla collection.
		\item \textbf{\# fillable : array = [’src’, ’filters’, ’crossOrigin’, ’alignX’, ’alignY’,’meetOrSlice’, ’background’]:}\\
		Quando si crea un model, si deve passare una serie di attributi al costruttore del model stesso. Questi attributi vengono assegnati al model tramite \textbf{mass assignment}. La propietà \textit{fillable} serve a specificare quali attributi devono essere assegnabili tramite il mass-assignment.
		\begin{itemize}
			\item \textbf{src:} indica il percorso dove recuperare il file immagine;
			\item \textbf{filters:} indica quali filtri verrano applicati all'immagine;
			\item \textbf{crossOrigin:} utilizzato per passara informazioni assieme all'immagine;
			\item \textbf{alignX:} indica l'allineamento relativo all'asse X dell'immagine se la sua larghezza differisce da quella della sua sulla finestra;
			\item \textbf{alignY:} indica l'allineamento relativo all'asse Y dell'immagine se la sua altezza differisce da quella della sua sulla finestra;
			\item \textbf{meetOrSlice:} indica se l'immagine dev'essere tagliata quando la sua finestra si restringe o essere sempre copletamente visibile;
			\item \textbf{background:} indica lo sfondo dell'immagine.
		\end{itemize}
	\end{itemize}
\newpage


\subsubsection{Table}

	\begin{figure}[h]
		\centering
		\includegraphics[width=0.5\linewidth]{img/back_end_premi_model_table}
		\caption[Diagramma della classe Table]{Diagramma della classe Table}
		\label{fig:back_end_premi_model_table}
	\end{figure}


	\paragraph{Descrizione}
	Questa classe rappresenta la struttura di dati di una tabella di una slide.
	
	\paragraph{Utilizzo}
	Viene utilizzato alla creazione o caricamento di una tabella.
	
	\paragraph{Attributi}
	\begin{itemize}
		\item \textbf{+ timestamps : boolean = false :}\\
		Di default Eloquent automatizza l'inserimento del timestamp relativo all'inserimento e aggiornamento di un campo. Se alla variabile viene assegnato il valore le informazioni dell'inserimento e del aggiornamento non verranno aggiunto alla collection.
		\item \textbf{\# fillable : array = [’row’, ’column’, ’title’, ’cellData’]:}\\
		Quando si crea un model, si deve passare una serie di attributi al costruttore del model stesso. Questi attributi vengono assegnati al model tramite \textbf{mass assignment}. La propietà \textit{fillable} serve a specificare quali attributi devono essere assegnabili tramite il mass-assignment.
		\begin{itemize}
			\item \textbf{row:} indica il numero di righe della tabella;
			\item \textbf{column:} indica il numero dei colonne della tabella;
			\item \textbf{title:} ilndica il titolo della tabella;
			\item \textbf{cellData:} indica l'array di dati necessario a comporre la tabella.
		\end{itemize}
	\end{itemize}