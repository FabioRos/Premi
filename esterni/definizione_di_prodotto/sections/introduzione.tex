\subsection{Scopo del documento}
  Il presente documento serve per definire in dettaglio la struttura e le relazioni tra le componenti del prodotto \PROGETTO{}, riprendendo ed approndendo 
  quanto già delineato nel documento \textit{Specifica Tecnica v3.0.0}. Questo documento servirà da guida per i programmatori, fornendo loro le direttive 
  per l'implementazione del sistema e l'attività di codifica.
\subsection{Scopo del prodotto}
  Lo scopo del progetto è realizzare un software per un sistema di rappresentazione di slide sfruttando la tecnologia HTML5. 
  Lo scopo principale è quello di creare un prodotto che sia di qualità comparabile, in prestazioni, funzionalità ed effetti visivi, 
  ai maggiori concorrenti già presenti sul mercato (Prezi, Powerpoint, Keynote, Impress, ...).
\subsection{Glossario}
Per prevenire ed evitare qualsiasi dubbio e per permettere una maggiore chiarezza e comprensione del testo su termini ambigui, abbreviazioni e acronimi 
utilizzati nei vari documenti, essi sono stati raccolti nel \textit{Glossario v3.0.0} nel quale si possono trovare tutte le informazioni desiderate.
Al fine di rendere subito evidente un termine presente nel Glossario, esso verrà marcato con il pedice \G\footnote{Per le istruzioni si rimanda al documento \textit{Norme di Progetto v2.0.0}.}.
\subsection{Riferimenti}
  \subsubsection{Normativi}
      \begin{itemize}
       \item \textit{Norme di Progetto v3.0.0};
       \item \textit{Analisi dei Requisiti v3.0.0};
       \item \textit{Specifica Tecnica v3.0.0}.
      \end{itemize}
  \subsubsection{Informativi}
      \begin{itemize}
		\item \textbf{Design Patterns, elementi per il riuso di software ad oggetti: } Gamma,  Helm,  Johnson,  Vlissides;
		\item \textbf{SWEBOK v3, Guide to the Software Engineering Body of Knowledge: } IEEE Computer Society;
		\item \textbf{Ingegneria del software, Ian Sommerville};
		\item \textbf{Slide del corso: }
				\begin{itemize}
					\item \textbf{Diagrammi delle classi}: \url{http://www.math.unipd.it/~tullio/IS-1/2014/Dispense/E2a.pdf};
					\item \textbf{Diagrammi dei package}: \url{http://www.math.unipd.it/ ~tullio/IS-1/2014/Dispense/E2b.pdf};
					\item \textbf{Pattern}:
					\begin{itemize}
						\item \textit{Architetturali}
							\begin{itemize}
								\item \url{http://www.math.unipd.it/~tullio/IS-1/2014/Dispense/E9.pdf}; 
								\item \url{http://www.math.unipd.it/~rcardin/pdf/Design\%20Pattern\%20Architetturali\%20-\%20Model\%20View\%20Controller\_4x4.pdf};
							\end{itemize}
						\item \textit{Strutturali}:
						\begin{itemize}
						\item \url{http://www.math.unipd.it/~tullio/IS-1/2014/Dispense/E6.pdf};
						\end{itemize}
						\item \textit{Creazionali}:
						\begin{itemize}
						\item \url{http://www.math.unipd.it/~tullio/IS-1/2014/Dispense/E7.pdf};
						\end{itemize}
						\item \textit{Comportamentali}:
						\begin{itemize}
						\item \url{http://www.math.unipd.it/~tullio/IS-1/2014/Dispense/E8.pdf};
						\end{itemize}
					\end{itemize}
					\item \textbf{Documentazione di Chart.js}: \url{http://chartjs.org/docs};
					\item \textbf{Documentazione di Angular.js}: \url{https://docs.angularjs.org/guide};
					\item \textbf{Documentazione di Reveal.js}: \url{http://github.com/hakimel/reveal.js};
					\item \textbf{Manuale di MongoDb}: \url{https://docs.mongodb.org/manual};
					\item \textbf{Documentazione di Foundation}: \url{http://foundation.zurb.com/docs};
 					\item \textbf{Documentazione di Php}: \url{http://php.net/docs.php};
 					\item \textbf{Documentazione di Laravel}: \url{http://laravel.com/docs/5.1}.
				\end{itemize}

	\end{itemize}