\subsection{Standard di progettazione architetturale}
Gli standard di progettazione architetturale seguiti sono definiti e descritti nel documento \textit{Specifica Tecnica v3.0.0}. Si faccia riferimento a tale
documento per approfondimenti.

\subsection{Standard di documentazione del codice}
Gli standard di documentazione del codice seguiti sono definiti e descritti nel documento \textit{Norme di Progetto v3.0.0}. Si faccia riferimento a tale
documento per approfondimenti.

\subsection{Standard di denominazione di entità e relazioni}
La nomenclatura di tutti gli elementi definiti in questo documento, siano essi package, classi, metodi o attributi, deve essere chiara e concisa. 
La chiarezza del nome sarà anteposta alla sua lunghezza, che potrà essere appositamente abbreviata. \\
\noindent Sono ammesse abbreviazioni quando esse risultino:
\begin{itemize}
	 \item Immediatamente comprensibili;
	 \item Non ambigue;
	 \item Sufficientemente contestualizzate.
\end{itemize}
Per tutte le regole tipografiche adottate si faccia riferimento al documento \textit{Norme di Progetto v3.0.0}.

\subsection{Standard di programmazione}
Gli standard di programmazione seguiti sono definiti e descritti nel documento \textit{Norme di Progetto v3.0.0}. Si faccia riferimento a tale
documento per approfondimenti.

\subsection{Strumenti di lavoro}
Tutti gli strumenti di lavoro e le procedure da seguire per la corretta realizzazione del prodotto sono definiti nel documento \textit{Norme di Progetto v3.0.0}.
Si faccia riferimento a tale documento per approfondimenti.

\subsection{Notazioni derivate dai framework}
	\subsubsection{AngularJS}
		\paragraph{Directive}
		Nelle descrizioni degli \textit{scope} delle directive, le modalità con cui vengono passati i parametri utilizzano questa notazione:
		\begin{itemize}
			\item @: indica che il parametro passato è una stringa, quindi non è possibile vedere i cambiamenti fatti su di essa all'esterno;
			\item =: indica che il parametro passato allo scope è un riferimento ad un oggetto;
			\item \&: indica che il parametro passato è una funzione, con i relativi parametri.
		\end{itemize}
		
		\paragraph{Controller}
		Le funzioni che un controller definisce all'interno dell'oggetto \SCOPE sono state modellate come metodi pubblici del controller stesso.