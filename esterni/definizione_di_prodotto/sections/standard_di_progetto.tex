\subsection{Standard di progettazione architetturale}
Gli standard di progettazione architetturale seguiti sono definiti e descritti nel documento \textit{Specifica Tecnica v2.0.0}. Si faccia riferimento a tale
documento per approfondimenti.

\subsection{Standard di documentazione del codice}
Gli standard di documentazione del codice seguiti sono definiti e descritti nel documento \textit{Norme di Progetto v3.0.0}. Si faccia riferimento a tale
documento per approfondimenti.

\subsection{Standard di denominazione di entità e relazioni}
La nomenclatura di tutti gli elementi definiti in questo documento, siano essi package, classi, metodi o attributi, deve essere chiara e concisa. 
La chiarezza del nome sarà anteposta alla sua lunghezza, che potrà essere appositamente abbreviata. \\
\noindent Sono ammesse abbreviazioni quando esse risultino:
\begin{itemize}
	 \item Immediatamente comprensibili;
	 \item Non ambigue;
	 \item Sufficientemente contestualizzate.
\end{itemize}
Per tutte le regole tipografiche adottate si faccia riferimento al documento \textit{Norme di Progetto v3.0.0}.

\subsection{Standard di programmazione}
Gli standard di programmazione seguiti sono definiti e descritti nel documento \textit{Norme di Progetto v3.0.0}. Si faccia riferimento a tale
documento per approfondimenti.

\subsection{Strumenti di lavoro}
Tutti gli strumenti di lavoro e le procedure da seguire per la corretta realizzazione del prodotto sono definiti nel documento \textit{Norme di Progetto v3.0.0}.
Si faccia riferimento a tale documento per approfondimenti.

\subsection{Note derivate dai framework}
	\subsubsection{AngularJS}
		\paragraph{Controller}
		Le funzioni che un controller definisce all'interno dell'oggetto \$scope sono state modellate come metodi pubblici del controller stesso.
