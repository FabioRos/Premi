Al fine di migliorare l'andamento del progetto si è effettuata un'attenta analisi dei rischi. Essa è suddivisa in quattro fasi:
\begin{itemize}
	\item \textbf{Identificazione:} individuare i rischi che possono interessare il progetto, indicandone le cause e cercando di prevederne
	le conseguenze. I rischi possono riguardare:
	\begin{itemize}
		\item \textbf{Progetto:} relativi a pianificazione, risorse e strumenti;
		\item \textbf{Prodotto:} relativi a conformità alle aspettative del committente;
		\item \textbf{Business:} relativi a costi e concorrenza sul mercato.
	\end{itemize}
	\item \textbf{Analisi:} prevedere le probabilità di occorrenza del rischio e si cercano le possibili conseguenze;
	\item \textbf{Pianificazione:} ideare una strategia per prevenire i possibili rischi;
	\item \textbf{Monitoraggio:} elaborare metodi di verifica e controllo per riconoscere per tempo i rischi, per il loro trattamento e la loro mitigazione.
\end{itemize}
Ogni rischio identificato è descritto con: nome, probabilità di occorrenza del rischio, livello di pericolosità, descrizione e contromisure. Ciascun rischio verrà monitorato nel tempo e ne verrà descritto l'effettivo riscontro con l'avanzare del progetto.

\subsection{Livello tecnologico}
\subsubsection{Tecnologie adottate}
\begin{description}
	\item[Livello di rischio:] medio.
	\item[Probabilità:] bassa.
	\item[Descrizione:] le tecnologie richieste dal capitolato sono HTML5, Javascript e CSS. Tali linguaggi sono noti a tutto il gruppo in quanto progetti didattici svolti in anni passati ne richiedevano la conoscenza. Non è da escludere che si possano incontrare difficoltà per alcuni particolari requisiti da soddisfare e per l'utilizzo di framework totalmente sconosciuti come \textit{Reveal.js}.
	\item[Contromisure:] per evitare di incorrere in gravi ritardi, ogni membro del gruppo ha il compito di studiare ed approfondire la propria conoscenza delle tecnologie utilizzate attraverso l'uso di documentazione appropriata.
\end{description}
\subsubsection{Rotture hardware}
\begin{description}
	\item[Livello di rischio:] basso.
	\item[Probabilità:] bassa.
	\item[Descrizione:] i computer portatili utilizzati dai componenti del gruppo sono di tipo commerciale e non professionale, quindi è da tenere presente che si potrebbe incappare in alcune rotture hardware che ne potrebbero compromettere il corretto funzionamento.
	\item[Contromisure:] ogni membro del gruppo si impegna ad avere cura dei propri strumenti di lavoro. Inoltre è stata messa a disposizione una cartella condivisa in Google Drive nella quale poter inserire un backup di tutti i file ai quali sta lavorando, in modo che essi possano essere reperibili anche in caso di guasti al proprio portatile.
\end{description}

\subsection{Livello del personale}
\subsubsection{Problemi dei componenti del gruppo}
\begin{description}
	\item[Livello di rischio:] alto.
	\item[Probabilità:] media.
	\item[Descrizione:] alcuni componenti del gruppo hanno impegni lavorativi esterni al progetto, inoltre si deve tenere in considerazione che tutti i membri possono avere altre attività personali da svolgere, tutto ciò può portare a subire forti rallentamenti nello sviluppo del progetto. Bisogna inoltre tener conto anche di eventuali assenze o indisposizioni dovute a motivi di salute.
	\item[Contromisure:] è di vitale importanza organizzare per tempo tutte le attività da svolgere ed in caso di assenza di un membro del gruppo deve essere segnalata per tempo al \textit{Responsabile di Progetto} in modo che possa suddividere il carico di lavoro tra i restanti membri attraverso l'interscambiabilità dei ruoli per non incorrere in ritardi difficilmente recuperabili. L'utilizzo del calendario offerto dallo strumento Teamwork, il cui uso è regolato dalle \textit{Norme di Progetto v1.0.0}, permette di avere una visione complessiva delle indisponibilità di ogni componente del gruppo. 
\end{description}
\subsubsection{Problemi tra componenti del gruppo}
\begin{description}
	\item[Livello di rischio:] alto.
	\item[Probabilità:] basso.
	\item[Descrizione:] tutti i membri collaborano tra loro in un gruppo numeroso per la prima volta, ognuno ha idee e linee di pensiero differenti.
	Tali premesse potrebbero portare a scontri interni e a conseguenti problemi di collaborazione, che andrebbero a riflettersi sul normale	avanzamento del progetto, creando un clima di lavoro pesante e difficilmente proficuo.
	\item[Contromisure:] nel caso in cui si verifichino forti contrasti tra due membri del gruppo, sarà compito del \textit{Responsabile di Progetto} cercare una mediazione tra i due. Se ciò non bastasse, il \textit{Responsabile di Progetto} allocherà le risorse in modo da minimizzare il contatto tra i due.
\end{description}
\subsubsection{Inesperienza del gruppo}
\begin{description}
	\item[Livello di rischio:] alto.
	\item[Probabilità:] alta.
	\item[Descrizione:] l'approccio al metodo di lavoro utilizzato per sviluppare il progetto risulta totalmente nuovo. Sono richieste capacità di analisi	e pianificazione che il gruppo non ha ancora acquisito a causa dell'inesperienza. Inoltre viene richiesto l'utilizzo di alcune tecnologie che non tutti i membri hanno potuto utilizzare e che richiedono tempo per essere apprese.
	\item[Contromisure:] ogni membro si impegna a riempire le lacune di stampo tecnologico che si trova a fronteggiare, sarà poi compito del \textit{Responsabile di Progetto} fornire per tempo, e nei periodi dove il carico di lavoro è minore, il materiale necessario allo studio. 
\end{description}

\subsection{Livello organizzativo}
\begin{description}
	\item[Livello di rischio:] alto.
	\item[Probabilità:] media.
	\item[Descrizione:] durante la fase di pianificazione è possibile che i tempi per lo svolgimento di alcune attività vengano calcolati in modo errato. Una sottostima dei tempi, in particolare, può portare ad un aumento dei costi e conseguente ritardo di consegna del materiale previsto. 
	\item[Contromisure:] si è deciso di prevedere un periodo di slack per ogni attività con alta criticità, in modo che un eventuale ritardo non vada ad influenzare la durata totale del progetto. Inoltre, la caratteristica dinamica di questo rischio impone che si debba controllare lo stato dei ticket periodicamente in modo da verificare eventuali ritardi nello sviluppo delle attività. 
\end{description}

\subsection{Livello dei requisiti}
\begin{description}
	\item[Livello di rischio:] medio.
	\item[Probabilità:] media.
	\item[Descrizione:] durante lo studio del capitolato c'è la possibilità che alcuni requisiti espressi nel non vengano capiti o peggio, male interpretati e quindi implementati in modo non corretto. Questo può portare a delle divergenze tra le aspettative del Proponente e la visione del gruppo sul prodotto. 
	\item[Contromisure:] è necessario che il gruppo fissi degli incontri con il Proponente per chiarire ogni dubbio e fissare dei requisiti certi sui quali non verranno effettuate future modifiche. Avere un contatto diretto con il proponente è poi uno dei fattori che contribuiscono a migliorare lo sviluppo del prodotto. Sarà inoltre indispensabile correggere eventuali errori o imprecisioni indicati dal Committente all'esito di ogni revisione.  
\end{description}