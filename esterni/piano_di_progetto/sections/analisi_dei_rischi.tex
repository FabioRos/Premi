Al fine di migliorare l'andamento del progetto si è effettuata un'attenta analisi dei rischi. Essa è suddivisa in quattro fasi:
\begin{itemize}
	\item \textbf{Identificazione:} individuare i rischi che possono interessare il progetto, indicandone le cause e cercando di prevederne
	le conseguenze. I rischi possono riguardare:
	\begin{itemize}
		\item \textbf{Progetto:} relativi a pianificazione, risorse e strumenti;
		\item \textbf{Prodotto:} relativi a conformità alle aspettative del committente;
		\item \textbf{\gls{Business}:} relativi a costi e concorrenza sul mercato.
	\end{itemize}
	\item \textbf{Analisi:} prevedere le probabilità di occorrenza del rischio e determinare le possibili conseguenze;
	\item \textbf{Pianificazione:} ideare una strategia per prevenire i possibili rischi;
	\item \textbf{Monitoraggio:} elaborare metodi di verifica e controllo per riconoscere anticipatamente i rischi, per il loro trattamento e la loro mitigazione.
\end{itemize}
Ogni rischio identificato è descritto con: nome, probabilità di occorrenza del rischio, livello di pericolosità, descrizione strategie per la rilevazione e contromisure. Ciascun rischio verrà monitorato nel tempo e ne verrà descritto l'effettivo riscontro con l'avanzare del progetto.

\subsection{Livello tecnologico}
\subsubsection{Tecnologie adottate}
\begin{description}
	\item[Livello di rischio:] alto.
	\item[Probabilità:] medio.
	\item[Descrizione:] le tecnologie richieste dal capitolato sono \gls{HTML5}, \gls{Javascript} e \gls{CSS}. Tali linguaggi sono noti a tutto il gruppo in quanto progetti didattici svolti in anni passati ne richiedevano la conoscenza. Non è da escludere che si possano incontrare difficoltà per alcuni particolari requisiti da soddisfare e per l'utilizzo di \gls{framework} totalmente sconosciuti come \textit{\gls{Reveal.js}}, \textit{\gls{Angular.js}} e \textit{\gls{Laravel}}.
	\item[Identificazione:] sarà compito del \textit{Responsabile di Progetto} monitorare costantemente il grado di conoscenza dei membri del gruppo riguardo le tecnologie che si andranno ad utilizzare.
	\item[Gestione:] per evitare di incorrere in gravi ritardi, ogni membro del gruppo ha il compito di studiare ed approfondire la propria conoscenza delle tecnologie utilizzate attraverso l'uso di documentazione appropriata. Il piano di lavoro dovrà essere organizzato tenendo conto anche dell'inesperienza dei componenti del gruppo con le tecnologie utilizzate. Se il tempo investito nello studio non sarà sufficiente le attività saranno riprogrammate dando il tempo necessario ai membri di acquisire le conoscenze richieste, posticipando le date di scadenza ma senza andare ad incidere sul preventivo economico.
	\item[Riscontro:] il gruppo ha utilizzato il tempo a disposizione, soprattutto durante il periodo di \textit{Analisi di Dettaglio}, per approfondire e colmare le carenze riscontrate. Nonostante ciò, è stato necessario riservare del tempo per lo studio ottimale del \gls{framework} \textit{\gls{Angular.js}} anche all'inizio della fase di \textit{Progettazione Architetturale}, causando un leggero ritardo nel concludere alcune attività. Anche la scelta di utilizzare, per la parte di \gls{back-end}, il \gls{framework} \textit{\gls{Laravel}}, effettuata all'inizio della fase di \textit{Progettazione di Dettaglio e Codifica}, per quanto abbia reso più semplice la programmazione lato server del software, ha causato un ritardo dovuto alla comprensione di tale \gls{framework} ed è stata una delle cause che ha portato alla modifica della pianificazione delle scadenze.
\end{description}
\subsubsection{Rotture hardware}
\begin{description}
	\item[Livello di rischio:] basso.
	\item[Probabilità:] bassa.
	\item[Descrizione:] i computer portatili utilizzati dai componenti del gruppo sono di tipo commerciale e non professionale, quindi è da tenere presente la possibilità di incappare in alcune rotture hardware che ne potrebbero compromettere il corretto funzionamento e che potrebbero comportare la perdita dei dati.
	\item[Identificazione:] ogni membro del gruppo dovrà avere una particolare cura della propria strumentazione, verificandone periodicamente il corretto funzionamento.
	\item[Gestione:] è stata messa a disposizione una cartella condivisa in \gls{Google Drive} nella quale poter inserire un backup di tutti i file ai quali si sta lavorando, in modo che essi possano essere reperibili anche in caso di guasti al proprio portatile. Non è possibile impedire del tutto la probabilità del verificarsi di guasti hardware.
	\item[Riscontro:] tale problema è stato riscontrato durante il periodo di \textit{Progettazione di Dettaglio e Codifica} in quanto un membro del gruppo ha subito la quasi contemporanea rottura del proprio PC personale e furto dello smartphone, utilizzato per la comunicazione col gruppo via WhatsApp. Si è riusciti a reperire in pochi giorni un nuovo PC riuscendo quindi in modo quasi immediato, facendo si che la pianificazione delle attività ne risentisse in minima parte. È stata utilizzata una chat di gruppo, messa a disposizione dal social network \gls{Facebook}, per sopperire alla mancanza dello smartphone utilizzato per la comunicazione tra membri del gruppo.  
\end{description}

\subsection{Livello del personale}
\subsubsection{Problemi dei componenti del gruppo}
\begin{description}
	\item[Livello di rischio:] alto.
	\item[Probabilità:] alta.
	\item[Descrizione:] alcuni componenti del gruppo hanno impegni lavorativi esterni al progetto, inoltre si deve tenere in considerazione che tutti i membri possono avere altre attività personali da svolgere, tutto ciò può portare a subire forti rallentamenti nello sviluppo del progetto. Bisogna inoltre tener conto anche di eventuali assenze o indisposizioni dovute a motivi di salute. Infine tutti i componenti del gruppo hanno altri esami universitari da preparare durante il periodo di svolgimento del progetto.
	\item[Identificazione:] ogni membro del gruppo si dovrà impegnare a comunicare tempestivamente al \textit{Responsabile di Progetto} eventuali impegni personali che sopraggiungano durante i periodi di svolgimento del progetto. 
	\item[Gestione:] è di vitale importanza organizzare per tempo tutte le attività da svolgere. In caso di assenza di un membro del gruppo il \textit{Responsabile di Progetto} dovrà riorganizzare le attività in modo che si possa suddividere il carico di lavoro tra i restanti membri attraverso l'interscambiabilità dei ruoli per non incorrere in ritardi difficilmente recuperabili. L'utilizzo di \gls{Google Calendar}, il cui uso è regolato dalle \textit{Norme di Progetto v3.0.0}, permette di avere una visione complessiva delle indisponibilità di ogni componente del gruppo. 
	\item[Riscontro:] i membri del gruppo hanno sempre segnalato per tempo eventuali indisponibilità e impegni lavorativi, anche grazie all'utilizzo di \gls{Google Calendar}. Durante il periodo di \textit{Progettazione di Dettaglio e Codifica}, coincidente con l'inizio del periodo di esami del secondo semestre, ci si è accorti che la programmazione precedentemente fatta non aveva tenuto abbastanza conto del carico di studio che i membri del gruppo dovevano affrontare in questa fase, portando ad una riprogrammazione delle scadenze descritta nella sezione \ref{scadenze}. Inoltre è stato riscontrato che un membro del gruppo, non seguendo le \textit{Norme di Progetto v3.0.0} per quanto riguarda comunicazioni di difficoltà ed impegni personali, ha causato alcuni ritardi nello svolgimento delle attività programmate.
\end{description}
\subsubsection{Problemi tra componenti del gruppo}
\label{rischi}
\begin{description}
	\item[Livello di rischio:] alto.
	\item[Probabilità:] media.
	\item[Descrizione:] tutti i membri collaborano tra loro in un gruppo numeroso per la prima volta, ognuno ha idee e linee di pensiero differenti.
	Tali premesse potrebbero portare a scontri interni e a conseguenti problemi di collaborazione, che andrebbero a riflettersi sul normale	avanzamento del progetto, creando un clima di lavoro pesante e difficilmente proficuo.
	\item[Identificazione:] è compito del \textit{Responsabile di Progetto} monitorare lo stato dei rapporti tra i vari componenti del gruppo. È invece compito di tutti i membri riferire al \textit{Responsabile di Progetto} eventuali problemi di cui lui non è a conoscenza.
	\item[Gestione:] nel caso in cui si verifichino forti contrasti tra due membri del gruppo, sarà compito del \textit{Responsabile di Progetto} cercare una mediazione tra i due. Se ciò non bastasse, il \textit{Responsabile di Progetto} allocherà le risorse in modo da minimizzare il contatto tra i due.
	\item[Riscontro:] all'inizio del periodo di \textit{Progettazione di Dettaglio e Codifica} ci sono stati degli scontri tra qualche membro dovuti a ripetute assenze e scarsa comunicazione con un componente del gruppo. Il tutto, dopo lo spostamento della data di consegna dei documenti inerenti alla \textit{Revisione di Qualifica}, è culminato con una lettera di richiamo inviata dal \textit{Responsabile di Progetto} a quei membri che hanno causato ritardi nel completamento delle attività programmate. Tale intervento del \textit{Responsabile di Progetto}, che ha riorganizzato le attività in modo da minimizzare il contatto tra i membri fra cui la discussione è stata più accesa, ha riportato la serenità nel gruppo. Per questo motivo il livello di rischio è stato aumentato. 
\end{description}
\subsubsection{Inesperienza del gruppo}
\begin{description}
	\item[Livello di rischio:] alto.
	\item[Probabilità:] media.
	\item[Descrizione:] l'approccio al metodo di lavoro utilizzato per sviluppare il progetto risulta totalmente nuovo. Sono richieste capacità di analisi	e pianificazione che il gruppo non ha ancora acquisito a causa dell'inesperienza. Inoltre viene richiesto l'utilizzo di alcune tecnologie che non tutti i membri hanno potuto utilizzare e che richiedono tempo per essere apprese.
	\item[Identificazione:] è compito del \textit{Responsabile di Progetto} assicurarsi che tutti i membri del gruppo riescano a lavorare seguendo la pianificazione individuata ed utilizzino correttamente gli strumenti messi a disposizione dall'\textit{Amministratore di Progetto}.
	\item[Gestione:] ogni membro si impegna a riempire le lacune di stampo tecnologico che si trova a fronteggiare, sarà poi compito dell'\textit{Amministratore di Progetto} fornire per tempo, e nei periodi dove il carico di lavoro è minore, il materiale necessario allo studio.
	\item[Riscontro:] durante le fasi iniziali è successo spesso di dover correggere molte parti di documenti in quanto non erano state seguite le norme che ne regolavano la stesura. Col passare del tempo tale problema è diminuito sensibilmente. Anche durante la scrittura del codice le norme delineate dall'\textit{Amministratore di Progetto} sono state seguite fedelmente.
\end{description}

\subsection{Livello organizzativo}
\begin{description}
	\item[Livello di rischio:] medio.
	\item[Probabilità:] alto.
	\item[Descrizione:] durante la fase di pianificazione è possibile che i tempi per lo svolgimento di alcune attività vengano calcolati in modo errato. Una sottostima dei tempi, in particolare, può portare ad un aumento dei costi e conseguente ritardo di consegna del materiale previsto. 
	\item[Identificazione:] il \textit{Responsabile di Progetto} tramite la consultazione dei diagrammi di \gls{Gantt} potrà consultare giornalmente se lo stato di avanzamento dei lavori sta seguendo la pianificazione effettuata.
	\item[Gestione:] si è deciso di prevedere un periodo di \gls{slack} per ogni attività con alta criticità, in modo che un eventuale ritardo non vada ad influenzare la durata totale del progetto. Inoltre, la caratteristica dinamica di questo rischio impone che si debba controllare lo stato dei \gls{ticket} periodicamente in modo da verificare eventuali ritardi nello sviluppo delle attività. 
	\item[Riscontro:] nei primi giorni di svolgimento di progetto sono state riscontrati alcuni ritardi nella consegna del materiale. Pianificando in modo diverso alcune attività non si sono verificati aumenti di costi né slittamenti dei tempi di consegna. All'inizio del periodo di \textit{Progettazione di Dettaglio e Codifica} ci si è accorti di avere sottostimato il tempo necessario per lo studio di preparazione agli esami di altri corsi che i membri del gruppo devono sostenere durante questo periodo. È stato necessaria una ripianificazione dei periodi di \textit{Progettazione di Dettaglio e Codifica} e \textit{Verifica e Validazione} in modo da permettere sia il completamento del progetto sia lo svolgimento degli esami con profitto da parte dei componenti del gruppo. 
\end{description}

\subsection{Livello dei requisiti}
\begin{description}
	\item[Livello di rischio:] medio.
	\item[Probabilità:] bassa.
	\item[Descrizione:] durante lo studio del capitolato c'è la possibilità che alcuni requisiti espressi non vengano capiti o peggio, male interpretati e quindi implementati in modo non corretto. Questo può portare a delle divergenze tra le aspettative del Proponente e la visione del gruppo sul prodotto. 
	\item[Identificazione:] durante i periodi di analisi il gruppo fisserà degli incontri con il Proponente per chiarire ogni dubbio e fissare dei requisiti certi sui quali non verranno effettuate future modifiche.
	\item[Gestione:] avere un contatto diretto con il Proponente è uno dei fattori che contribuiscono a migliorare lo sviluppo del prodotto. Gli incontri saranno sfruttati al massimo per chiarire ogni dubbio prima di cominciare la progettazione. Sarà inoltre indispensabile correggere eventuali errori o imprecisioni indicati dal Committente all'esito di ogni revisione.
	\item[Riscontro:] Al momento è stata ricevuta una conferma da parte del Proponente sia per quanto riguarda i requisiti individuati dal gruppo sia per le tecnologie da utilizzare, mentre sono stati richiesti alcuni dettagli nel documento di \textit{Analisi dei Requisiti} da parte del Committente. Nonostante ciò, come ribadito in sede di correzione della \textit{Revisione di Progettazione}, il grado di dettaglio del documento è migliorato per cui il livello di rischio è stato abbassato
\end{description}

