\subsection{Redazione}

\begin{table}[h]
	\centering
	\begin{tabular}{|l|c|c|c|c|c|c|c|}
		\toprule
		\textbf{Cognome e Nome} & \textbf{Data} & \textbf{Firma} \\
		
		\midrule
		Burlin Valerio & & \\
		
		\bottomrule
	\end{tabular}
	\caption{Redazione}
\end{table}

\subsection{Accettazione componenti}

\begin{table}[h]
	\centering
	\begin{tabular}{|l|c|c|c|c|c|c|c|}
		\toprule
		\textbf{Cognome e Nome} & \textbf{Data} & \textbf{Firma} \\
		
		\midrule
		Agostinetto Matteo & & \\
		Burlin Valerio & & \\ 
		Carraro Nicola & & \\
		Crespan Emanuele & & \\
		Ros Fabio & & \\
		Suierica Bogdan & & \\
		
		\bottomrule
	\end{tabular}
	\caption{Accettazione componenti}
\end{table}

\subsection{Componenti}

\begin{table}[h]
	\centering
	\begin{tabular}{|l|c|c|c|c|c|c|c|}
		\toprule
		\textbf{Cognome e Nome} & \textbf{Matricola} & \textbf{email} \\
	
		\midrule
		Agostinetto Matteo & 611267 & \\
		Burlin Valerio & 1029442 & valerio.burlin@gmail.com \\ 
		Carraro Nicola & 1002050 & \\
		Crespan Emanuele & 1004994 & \\
		Ros Fabio & 609724 & fabio.ros90@gmail.com \\
		Suierica Bogdan & 1008089 & \\
		
		\bottomrule
	\end{tabular}
	\caption{Componenti}
\end{table}

\subsection{Definizione dei ruoli}
Durante lo sviluppo di \PROGETTO{} i membri del gruppo dovranno ricoprire diversi ruoli. Tali ruoli corrispondono a figure aziendali specializzate, le cui mansioni e responsabilità sono riportate nel documento \textit{Norme di Progetto v1.0.0}. Ogni componente del gruppo dovrà ricoprire almeno una volta ogni ruolo\footnote{Regola derivante dai vincoli di organigramma 2.2}. Si deve inoltre prestare attenzione che non ci siano periodi in cui un membro del gruppo non sia verificatore di se stesso.

\noindent I ruoli da ricoprire, ognuno con un costo orario\footnote{Tali costi sono riportati nei vincoli di organigramma 2.2} espresso in euro, sono: 
\begin{itemize}
	\item \textbf{\textit{Responsabile di Progetto}};
	\item \textbf{\textit{Amministratore di Progetto}};
	\item \textbf{\textit{Analista}};
	\item \textbf{\textit{Progettista}};
	\item \textbf{\textit{Programmatore}};
	\item \textbf{\textit{Verificatore}}.	
\end{itemize}


