Si riportano i report redatti dal \textit{Responsabile di Progetto} che mette in relazione lo stato del gruppo con quelli identificati nel modello \gls{SEMAT}.

\subsection{Report Progettazione Architetturale}
\begin{description}
	\item[Opportunity:] è stata individuata una soluzione che rispetta tutti i vincoli di progetto, alcuni rischi individuati si sono palesati e sono stati gestiti correttamente secondo quanto descritto nell'analisi dei rischi in sezione \ref{sezione 2}.
	\item[Stakeholders:] tutti i dubbi nati durante il periodo di analisi sono stati approfonditamente discussi con gli stakeholder arrivando ad una visione condivisa di funzionalità da soddisfare e tecnologie da utilizzare nello sviluppo del progetto.
	\item[Requirements:] i requisiti individuati rappresentano una soluzione accettabile per gli stakeholder e la loro frequenza di modifica è diminuita sensibilmente. Non è esclusa la possibilità che debbano essere rivisti in seguito ad ulteriori valutazioni degli stakeholder. 
	\item[Software System:] l'architettura ad alto livello del sistema software è stata individuata, così come le tecnologie e gli strumenti utili per lo sviluppo.
	\item[Team:] non si sono mai palesate difficoltà di relazione tra i componenti del gruppo. Il team lavora in modo coordinato e collaborativo per raggiungere gli obiettivi fissati, tornando poche volte su quanto pensato e progettato e producendo documenti di buona qualità.
	\item[Work:] l'avanzamento dei lavori è monitorato dagli strumenti individuati dal \textit{Responsabile di Progetto}. Il gruppo lavora coerentemente con quanto pianificato nella suddivisione del lavoro in sezione \ref{sezione 4}.
	\item[Way of Work:] i membri del team producono i progressi pianificati con tempistiche adeguate applicando in modo assiduo le norme stabilite. 
\end{description}
