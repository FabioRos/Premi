Si riportano i report redatti dal \textit{Responsabile di Progetto} che mette in relazione lo stato del gruppo con quelli identificati nel modello \gls{SEMAT}.

\subsection{Report Progettazione Architetturale}
\begin{description}
	\item[Opportunity:] è stata individuata una soluzione che rispetta tutti i vincoli di progetto, alcuni rischi individuati si sono palesati e sono stati gestiti correttamente secondo quanto descritto nell'analisi dei rischi in sezione \ref{sezione 2}.
	\item[Stakeholders:] tutti i dubbi nati durante il periodo di analisi sono stati approfonditamente discussi con gli \gls{stakeholders} arrivando ad una visione condivisa di funzionalità da soddisfare e tecnologie da utilizzare nello sviluppo del progetto.
	\item[Requirements:] i requisiti individuati rappresentano una soluzione accettabile per gli \gls{stakeholders} e la loro frequenza di modifica è diminuita sensibilmente. Non è esclusa la possibilità che debbano essere rivisti in seguito ad ulteriori valutazioni degli \gls{stakeholders}. 
	\item[Software System:] l'architettura ad alto livello del sistema software è stata individuata, così come le tecnologie e gli strumenti utili per lo sviluppo.
	\item[Team:] non si sono mai palesate difficoltà di relazione tra i componenti del gruppo. Il team lavora in modo coordinato e collaborativo per raggiungere gli obiettivi fissati, tornando poche volte su quanto pensato e progettato e producendo documenti di buona qualità.
	\item[Work:] l'avanzamento dei lavori è monitorato dagli strumenti individuati dal \textit{Responsabile di Progetto}. Il gruppo lavora coerentemente con quanto pianificato nella suddivisione del lavoro in sezione \ref{sezione 4}.
	\item[Way of Work:] i membri del team producono i progressi pianificati con tempistiche adeguate applicando in modo assiduo le norme stabilite. 
\end{description}

\subsection{Report Progettazione di Dettaglio e Codifica}
\begin{description}
	\item[Opportunity:] confermiamo la pianificazione al livello \textit{Addressed} in quanto stimiamo di giungere alla \gls{milestone} con un sistema globalmente funzionante che possa soddisfare gli \gls{stakeholders}.
	\item[Stakeholders:] date le difficoltà avute, sopratutto all'inizio di questo periodo, e l'adozione di un nuovo \gls{framework} abbiamo ripianificato di giungere alla \gls{milestone} allo stato \textit{In Agreement}.
	\item[Requirements:] i requisiti implementati hanno portato ad un prodotto software che soddisfa pienamente i requisiti obbligatori individuati, rappresentando una soluzione accettabile per gli \gls{stakeholders}. 
	\item[Software System:] il prodotto software realizzato sarà usabile e conforme alle caratteristiche di qualità descritte nel \textit{Piano di Qualifica v3.0.0}. Parte dei test verranno svolti, riducendo il numero di difetti ad un livello accettabile.
	\item[Team:] viste le difficoltà incontrate in questo periodo ripianifichiamo lo stato di avanzamento in \textit{Collaborating}.
	\item[Work:] confermiamo lo stato \textit{Under Controll} poichè le attività vengono svolte e gestite dal gruppo in maniera soddisfacente.
	\item[Way of Work:] viene confermato lo stato \textit{Working Well} in quanto le attività vengono svolte con le tempistiche adeguate. 
\end{description}

\subsection{Report Verifica e Validazione}
\begin{description}
	\item[Opportunity:] riprogrammiamo la pianificazione al livello \textit{Addressed} in quanto, a causa dei problemi incontrati sopratutto nel periodo di \textit{Verifica e Validazione}, non siamo riusciti ad implementare molto di più oltre ai requisiti obbligatori, soddisfando solo in parte le richieste degli \textit{Stakeholders}.
	\item[Stakeholders:] confermiamo il livello \textit{Satisfied in Use} in quanto il sistema soddisfa tutti i requisiti minimi richiesti.
	\item[Requirements:] riprogrammiamo la pianificazione al livello \textit{Addressed} in quanto con i requisiti implementati il software permette di eseguire tutte le attività richieste, ma non sono ancora stati sviluppati tutti i requisiti definiti nell'\textit{Analisi dei Requisiti v4.0.0}.  
	\item[Software System:] il prodotto software realizzato è usabile e conforme alle caratteristiche di qualità descritte nel \textit{Piano di Qualifica v4.0.0}. I test di verifica e validazione del sistema sono stati eseguiti ed il software realizzato incontra le richieste degli \textit{Stakeholders}.
	\item[Team:] viste le difficoltà incontrate anche in questo periodo ripianifichiamo lo stato di avanzamento in \textit{Performing} in quanto, a parte le difficoltà incontrate nel rapportarsi con uno dei membri del gruppo, i risultati delle attività svolte in questa fase sono stati ottimali richiedendo una minor attività di correzione a seguito delle verifiche.
	\item[Work:] confermiamo lo stato \textit{Conclused} poichè tutte le attività pianificate sono state concluse in maniera soddisfacente.
	\item[Way of Work:] viene confermato lo stato \textit{Retired} in quanto le attività sono terminate e i membri del gruppo hanno acquisito buone capacità di pianificazione ed organizzazione del lavoro da applicare in futuro. 
\end{description}