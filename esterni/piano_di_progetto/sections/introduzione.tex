\subsection{Scopo del documento}
Con il seguente documento si intende specificare il piano di lavoro seguito dal gruppo \GRUPPO{} nello svolgimento del progetto \PROGETTO{}.
Gli obbiettivi del documento sono:
\begin{itemize}
	\item Presentare la pianificazione delle attività da svolgere;
	\item Pianificare i tempi per le attività da svolgere;
	\item Fornire un preventivo e mettere a consuntivo le risorse durante lo svolgersi dei lavori;
	\item Analizzare e individuare eventuali fattori di rischio ed attuare delle opportune contromisure.
\end{itemize}

\subsection{Riferimenti}
\subsubsection{Normativi}
\begin{itemize}
	\item \textbf{Capitolato d'appalto C4:} Premi: Software di presentazione "better than Prezi". Reperibile all'indirizzo: \url{http://www.math.unipd.it/~tullio/IS-1/2014/Progetto/C4.pdf};
	\item \textbf{Vincoli di organigramma:} \url{http://www.math.unipd.it/~tullio/IS-1/2014/Progetto/PD01b.html}; 
	\item \textbf{Norme di Progetto:} \textit{Norme di Progetto v1.0.0}.
\end{itemize}
\subsubsection{Informativi}
\begin{itemize}
	\item \textbf{Software Engineering - Ian Sommerville - 9th Edition (2010):} 
	\begin{itemize}
		\item Part 4: Software Management.
	\end{itemize}
	\item \textbf{\gls{Slide} del corso di Ingegneria Del Software modulo A (2014):} 
	\begin{itemize}
		\item Il ciclo di vita del software; 
		\item Gestione di progetto.
	\end{itemize}
	\url{http://www.math.unipd.it/~tullio/IS-1/2014/}
\end{itemize}

\subsection{Glossario}
Per prevenire ed evitare qualsiasi dubbio e per permettere una maggiore chiarezza e comprensione del testo su termini ambigui, abbreviazioni e acronimi utilizzati nei vari documenti, essi sono stati raccolti nel \textit{Glossario v1.0.0} nel quale si possono trovare tutte le informazioni desiderate.
Al fine di rendere subito evidente un termine presente nel \textit{Glossario}, esso verrà marcato con il pedice \G\footnote{Per le istruzioni si rimanda al documento \textit{Norme di Progetto v1.0.0} .}.

\subsection{Note sulle tabelle}
Come riportato nelle \textit{Norme di Progetto v1.0.0} si è scelto di lasciare uno spazio vuoto in tutte le celle delle tabelle che riportano valori uguali a 0 (zero) per migliorarne comprensione e leggibilità della stessa.

\subsection{Ciclo di vita}
Come modello di ciclo di vita da applicare ai processi si è scelto il \textbf{modello incrementale}, che porta allo sviluppo del progetto in varie fasi. La fine di ogni fase è segnata da una \gls{milestone}.
La scelta di tale modello si deve alle seguenti proprietà:
\begin{itemize}
	\item È previsto che il sistema sia diviso in sottosistemi. Utilizzando questa tecnica si hanno alcuni particolari aspetti positivi:
	\begin{itemize}
		\item In un breve lasso di tempo le risorse vengono utilizzate su un numero limitato di attività. Questa suddivisione permette una gestione più semplice e maggiormente controllabile di risorse e tempi;
		\item Si effettuano test più dettagliati e di conseguenza più approfonditi.
	\end{itemize}
	\item I requisiti utente sono trattati in base alla loro importanza, quelli con maggiore criticità hanno la priorità;
	\item Ad ogni incremento si ha una consolidazione della sezione coinvolta, il che riduce il rischio di fallimento;
	\item I cicli di incremento vengono pianificati ottenendo un maggior controllo di tempi e costi;
	\item Prevede rilasci multipli e successivi con i seguenti benefici:
	\begin{itemize}
		\item Permette di incrementare le funzionalità del prodotto e di migliorare le funzionalità implementate in precedenza;
		\item I primi rilasci sono relativi ai requisiti più importanti. Di conseguenza tali requisiti passeranno attraverso più stadi di verifica, risultando alla fine più raffinati e migliorati;
		\item È prevista la possibilità di rilasciare prototipi che permettano di isolare requisiti per i successivi incrementi.
	\end{itemize}
\end{itemize}
Il modello incrementale prevede:
\begin{itemize}
	\item \textbf{Analisi e progettazione architetturale:} Queste macro-fasi non sono ripetute. I requisiti principali e l'architettura del sistema sono identificati e fissati definitivamente in modo da pianificare correttamente i cicli di incremento; 
	\item \textbf{Progettazione di dettaglio, codifica e verifiche:} tali macro-fasi sono realizzate in modo incrementale. Sono previste due fasi di codifica distinte per permettere il rilascio del software con le funzionalità obbligatorie\footnote{Si rimanda al documento \textit{Norme di Progetto v1.0.0} per chiarimenti sulla tipologia di requisiti.} il prima possibile e successivamente un nuovo rilascio\footnote{Tale fase potrebbe non avere luogo nel caso in cui la codifica dei requisiti obbligatori richieda più tempo del previsto.} con le funzionalità opzionali ed eventuali altri miglioramenti.
\end{itemize}
Utilizzando tale modello si può quindi rilasciare al committente un \gls{prototipo}, contenente funzionalità di primaria importanza, nel minor tempo possibile. Così facendo il committente può accertarsi dell'andamento dei lavori e valutare in corso d'opera lo sviluppo del prodotto. Seguendo questo modello quindi si avrà il vantaggio di spendere le iniziali risorse per produrre una base solida e funzionante che presenti il modello nei suoi aspetti di maggiore importanza, e successivamente utilizzarla per sviluppare ed ampliare il prodotto con funzionalità desiderabili e funzionali.
 
\subsection{Scadenze}
Di seguito le scadenze che il gruppo \GRUPPO{} ha deciso di rispettare e sulle quali si baserà la pianificazione del progetto:
\begin{itemize}
	\item \textbf{Revisione dei requisiti (RR):} 2015.04.27;
	\item \textbf{Revisione di progetto (RP):} 2015.05.29;
	\item \textbf{Revisione di qualifica (RQ):} 2015.06.18;
	\item \textbf{Revisione di accettazione (RA):} 2015.07.06 .
\end{itemize}





