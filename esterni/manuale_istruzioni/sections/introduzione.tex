Il prodotto \PROGETTO{} è un'applicazione web, quindi se caricato in un server è possibile utilizzarlo senza ulteriori installazioni nel proprio sistema ma semplicemente collegandosi al rispettivo indirizzo.

\section{Utilizzo del prodotto in locale}
Per utilizzare \PROGETTO{} in una macchina locale si dovranno installare le seguenti componenti:
\begin{itemize}
	\item \textbf{Php versione 5 o superiore :} si consiglia di utilizzare il sito ufficiale di Php \url{http://php.net/downloads.php} ;
	\item \textbf{MongoDB: } si consiglia di utilizzare il sito ufficiale di MongoDB \url{https://www.mongodb.org/downloads} ;
	\item \textbf{Drive Php di MongoDB: } si consiglia di utilizzare il sito ufficiale di MongoDB \url{http://docs.mongodb.org/ecosystem/drivers/php/} ;  
	\item \textbf{Composer: } si consiglia di utilizzare il sito ufficiale di Composer \url{https://getcomposer.org/download/} ;
\end{itemize}
Dopo aver installato le componenti sopra citate è necessario avviare un server in locale e il servizio MongoDB. Si consiglia di utilizzare il server Php presente all'interno del sistema, che è possibile avviare attraverso il commando: "php artisan serve"(Il server utilizza di default la porta 8000).