% per essere usato è necessario il pacchetto \usepackage{glossaries}
% da includere nel file in cui si deve usare tramite il comando \loadglsentries{glossary/glossario.tex}
\makeglossaries

% GLOSSARIO DEI TERMINI
\newglossaryentry{infografica}{name={infografica\G}, description={L'infografica (anche nota con termini inglesi "information design", information graphic o infographic) è l'informazione proiettata in forma più grafica e visuale che testuale.}}

\newglossaryentry{HTML5}{name={HTML5\G},description={Linguaggio di \gls{markup} per la strutturazione delle pagine web, e da ottobre 2014 pubblicato come \acrfull{W3C} Recommendation\footnote{linee guida per l'accessibilità dei contenuti web}.}}

\newglossaryentry{markup}{name={markup\G},description={In generale un linguaggio di markup è un insieme di regole che descrivono i meccanismi di rappresentazione (strutturali, semantici o presentazionali) di un testo che, utilizzando convenzioni standardizzate, sono utilizzabili su più supporti.}}

\newglossaryentry{UML}{name={UML\G},description={In ingegneria del software, UML (Unified Modeling Language, "linguaggio di modellazione unificato") è un linguaggio di modellazione e specifica basato sul paradigma \gls{object-oriented}. Un modello UML è costituito da una collezione organizzata di diagrammi correlati, costruiti componendo elementi grafici (con significato formalmente definito), elementi testuali formali, ed elementi di testo libero.}}

\newglossaryentry{object-oriented}{name={object-oriented\G},description={In ingegneria del software, l'espressione \gls{object-oriented} si riferisce a un insieme di concetti introdotti dai linguaggi di programmazione orientati agli oggetti e in seguito estesi a numerosi altri contesti della \gls{information technology}.}}

\newglossaryentry{information technology}{name={information technology\G},description={Le tecnologie dell'informazione e della comunicazione (in inglese Information and Communication Technology), sono l'insieme dei metodi e delle tecnologie che realizzano i sistemi di trasmissione, ricezione ed elaborazione di informazioni (tecnologie digitali comprese).}}

\newglossaryentry{Windows}{name={Windows\G}, description={È una famiglia di ambienti operativi e sistemi operativi commerciali della Microsoft Corporation dedicati ai personal computer, alle workstation e ai server.}}

\newglossaryentry{Linux}{name={Linux\G}, description={Linux è una famiglia di sistemi operativi di tipo Unix-like, rilasciati sotto varie possibili distribuzioni, aventi la caratteristica comune di utilizzare come nucleo il kernel Linux.}}

\newglossaryentry{Mac OS X}{name={Mac OS X\G}, description={Mac OS X è il sistema operativo sviluppato da Apple per i computer Macintosh, nato nel 2001 per combinare la note caratteristiche dell'interfaccia utente dell'originiario Mac OS con l'architettura di un sistema operativo di derivazione UNIX.}}

\newglossaryentry{browser}{name={browser\G}, description={Un web browser o navigatore di rete, in informatica, è un programma che consente di usufruire dei servizi di connettività in Internet, o di una rete di computer, e di navigare sul World Wide Web.}}

\newglossaryentry{GUI}{name={GUI\G}, description={La GUI (dall'inglese Graphical User Interface), comunemente abbreviata in interfaccia grafica, è un tipo di interfaccia utente che consente all'utente di interagire con la macchina controllando oggetti grafici convenzionali.}}

\newglossaryentry{real time}{name={real time\G}, description={Real-time (in italiano "tempo reale") è un termine utilizzato in ambito informatico per indicare quei programmi per i quali la correttezza del risultato dipende dal tempo di risposta. Ciò comporta che tali programmi devono rispondere ad eventi esterni entro tempi prestabiliti.}}

\newglossaryentry{font}{name={font\G}, description={In tipografia e in informatica il tipo di carattere o font è un insieme di caratteri tipografici caratterizzati e accomunati da un certo stile grafico o intesi per svolgere una data funzione.}}

\newglossaryentry{layout}{name={layout\G}, description={In informatica, il layout è l'impaginazione e la struttura grafica di un sito web, di un programma o di un documento (come quelli generati da un programma di videoscrittura).}}

\newglossaryentry{Google Chrome}{name={Google Chrome\G}, description={Google Chrome (detto anche semplicemente Chrome) è un \gls{browser} basato su WebKit sviluppato da Google.}}

\newglossaryentry{Mozilla Firefox}{name={Mozilla Firefox\G}, description={Mozilla Firefox è un \gls{browser} open source multipiattaforma prodotto da Mozilla Foundation.}}

\newglossaryentry{Internet Explorer}{name={Internet Explorer\G}, description={Internet Explorer (IE), oggi noto anche con il nome Windows Internet Explorer (WIE), è un \gls{browser} proprietario sviluppato da Microsoft e incluso in \gls{Windows} a partire dal 1995.}}

\newglossaryentry{Opera}{name={Opera\G}, description={Opera è un \gls{browser} web freeware e multipiattaforma prodotto da Opera Software}}

\newglossaryentry{Safari}{name={Safari\G}, description={Safari è un \gls{browser} web sviluppato da Apple Inc. per il sistema operativo \gls{Mac OS X}.}}

\newglossaryentry{caso d'uso}{name={caso d'uso\G}, description={È una tecnica usata nei processi di ingegneria del software per effettuare in maniera esaustiva e non ambigua, la raccolta dei requisiti al fine di produrre software di qualità. Consiste nel valutare ogni requisito, focalizzandosi sugli attori che interagiscono col sistema e valutandone le varie interazioni. Il documento dei casi d'uso, individua e descrive gli scenari elementari di utilizzo del sistema, da parte degli attori che si interfacciano con esso.}}

\newglossaryentry{filesystem}{name={filesystem\G}, description={Un filesystem, in informatica, indica informalmente un meccanismo con il quale i file sono posizionati e organizzati o su un dispositivo di archiviazione o una memoria di massa, come un disco rigido o un CD-ROM e, in casi eccezionali, anche sulla RAM.}}

\newglossaryentry{JavaScript}{name={JavaScript\G}, description={In informatica JavaScript è un linguaggio di programmazione orientato agli oggetti e agli eventi, comunemente utilizzato nella programmazione Web lato client.}}



% GLOSSARIO DEGLI ACRONIMI
\newacronym{SWE}{SWE\G}{Software Engineering}
\newacronym{W3C}{W3C}{World Wide Web Consortium}
