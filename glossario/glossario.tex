% per essere usato è necessario il pacchetto \usepackage{glossaries}
% da includere nel file in cui si deve usare tramite il comando \loadglsentries{glossary/glossario.tex}
\makeglossaries

% GLOSSARIO DEI TERMINI
\newglossaryentry{milestone}{name={milestone\G}, description={Punto nel tempo che determina importanti traguardi intermedi nello svolgimento del progetto. Indica a che distanza si è dalla fine del progetto.}}

\newglossaryentry{repository}{name={repository\G},description={Database in grado di contenere svariate tipologie di dati, corredate da relative informazioni (metadati). Offre inoltre un sistema di versionamento in grado di tener traccia delle modifiche effettuate al suo interno. Generalmente condiviso da più utenti, ognuno in grado di accedervi autonomamente per apportare modifiche. È implicitamente un servizio di condivisione dati.}}

\newglossaryentry{HTML5}{name={HTML5\G},description={Linguaggio di \gls{markup} per la strutturazione delle pagine web, e da ottobre 2014 pubblicato come \acrfull{W3C} Recommendation\footnote{linee guida per l'accessibilità dei contenuti web}.}}

\newglossaryentry{markup}{name={markup\G},description={In generale un linguaggio di markup è un insieme di regole che descrivono i meccanismi di rappresentazione (strutturali, semantici o presentazionali) di un testo che, utilizzando convenzioni standardizzate, sono utilizzabili su più supporti.}}

\newglossaryentry{mailing list}{name={mailing list\G},description={Rappresenta un metodo di comunicazione in cui un messaggio e-mail inviato ad un sistema server viene inoltrato automaticamente alla lista di destinatari interessati: solitamente infatti gli utenti condividono un interesse o uno scopo, e quando ci sono novità, il gestore invia mail a tutta la lista per far nascere discussioni, commenti o condividere informazioni utili.}}

\newglossaryentry{verbale}{name={verbale\G},description={Documento descrittivo di atti o fatti compiuti alla presenza di un soggetto verbalizzante appositamente incaricato}}

\newglossaryentry{PDF}{name={PDF\G},description={sta per Portable Document Format, è un formato di file basato su un linguaggio di descrizione di pagina sviluppato da Adobe System, per rappresentare documenti in modo indipendente dall’hardware e dal software utilizzato per generarli o visualizzarli.}}

\newglossaryentry{GitHub}{name={GitHub\G},description={servizio web di hosting per lo sviluppo di progetti software, che usa il sistema di controllo di versione \gls{Git}. GitHub offre la possibilità di gestire \gls{repository} privati a pagamento o pubblici, molto utilizzati per lo sviluppo di progetti open source.}}

\newglossaryentry{Git}{name={Git\G},description={sistema software di controllo di versione distribuito, creato da Linus Torvalds nel 2005.}}

\newglossaryentry{branch}{name={branch\G},description={Il branch o ramo si crea quando un utente decide di personalizzare un progetto già esistente e crea una propria \gls{repository} dove ospitare le modifiche a quello stesso progetto.}}

% GLOSSARIO DEGLI ACRONIMI
\newacronym{SWE}{SWE\G}{Software Engineering}
\newacronym{W3C}{W3C}{World Wide Web Consortium}
