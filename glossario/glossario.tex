% per essere usato è necessario il pacchetto \usepackage{glossaries}
% da includere nel file in cui si deve usare tramite il comando \loadglsentries{glossario/glossary.tex}
\makeglossaries

% GLOSSARIO DEI TERMINI
\newglossaryentry{Astah}{name={Astah\G}, description={Software proprietario usato per la modellazione di diagrammi UML.}}

\newglossaryentry{branch}{name={Branch\G}, description={Ramo di un sistema di gestione per gestire i file separatamente.}}

\newglossaryentry{browser}{name={Browser\G}, description={Particolare programma per navigare in Internet che inoltra la richiesta di un documento alla rete e ne consente la visualizzazione una volta arrivato.}}

\newglossaryentry{business}{name={Business\G}, description={Affare, commercio.}}

\newglossaryentry{casi d'uso}{name={Casi d'uso\G}, description={Tecnica usata nei processi di ingegneria del software per effettuare in maniera esaustiva e non ambigua, la raccolta dei requisiti.}}

\newglossaryentry{conciso}{name={Conciso\G}, description={Di scritto o discorso breve, ma completo ed efficace.}}

\newglossaryentry{Design pattern}{name={Design pattern\G}, description={Si tratta di una descrizione o modello logico da applicare per la risoluzione di un problema che può presentarsi in diverse situazioni durante le fasi di progettazione e sviluppo del software.}}

\newglossaryentry{Facebook}{name={Facebook\G}, description={Servizio di rete sociale lanciato nel febbraio del 2004, basato su una piattaforma software scritta in vari linguaggi di programmazione.}}

\newglossaryentry{feedback}{name={Feedback\G}, description={Processo per cui il risultato dall'azione di un sistema si riflette sul sistema stesso per correggerne o modificarne il comportamento.}}

\newglossaryentry{font}{name={Font\G}, description={Insieme di caratteri tipografici caratterizzati e accomunati da un certo stile grafico o intesi per svolgere una data funzione.}}

\newglossaryentry{Gantt}{name={Gantt\G}, description={Il diagramma di Gantt è uno strumento di supporto alla gestione dei progetti.}}

\newglossaryentry{Git}{name={Git\G}, description={È un software di controllo di versione distribuito, creato da Linus Torvalds.}}

\newglossaryentry{GitHub}{name={GitHub\G}, description={Servizio web di hosting per lo sviluppo di progetti software che usa il sistema di controllo di versione Git.}}

\newglossaryentry{Google Calendar}{name={Google Calendar\G}, description={Google Calendar è un sistema di calendari concepito da Google.}}

\newglossaryentry{Google Docs}{name={Google Docs\G}, description={Fa parte di una serie di applicazioni Web e di Office automation offerte da Google. Esso permette di salvare documenti di testo e fogli di calcolo nei formati .doc, .odt e .pdf, creare delle presentazioni, fogli di calcolo e moduli HTML.}}

\newglossaryentry{Google Drive}{name={Google Drive\G}, description={Servizio, in ambiente cloud computing, di memorizzazione e sincronizzazione online introdotto da Google che permette la condivisione e la modifica concorrente di documenti online.}}

\newglossaryentry{hosting}{name={hosting\G}, description={servizio di rete che consiste nell'allocare su un server web le pagine web di un sito web, rendendolo così accessibile dalla rete Internet e ai suoi utenti.}}

\newglossaryentry{HTML5}{name={HTML5\G}, description={Linguaggio di markup per la strutturazione delle pagine web, da ottobre 2014 pubblicato come W3C Recommendation.}}

\newglossaryentry{infografica}{name={Infografica\G}, description={Tecnica che consiste nel proiettare l'informazione in una forma più grafica e visuale che testuale.}}

\newglossaryentry{Inspection}{name={Inspection\G}, description={Tecnica di esaminazione che involvono misurazioni e test applicate su alcune caratteristiche nei confronti di un oggetto o di attività.}}

\newglossaryentry{Java}{name={Java\G}, description={Linguaggio di programmazione orientato agli oggetti che permette di sviluppare programmi eseguibili su diversi tipi di computer e compatibili con qualsiasi sistema operativo.}}

\newglossaryentry{Javascript}{name={Javascript\G}, description={Un linguaggio di scripting spesso utilizzato in pagine web per la gestione di effetti speciali e per l'interattività.}}

\newglossaryentry{layout}{name={Layout\G}, description={Rappresentazione grafica di un sistema di elaborazione o di un archivio di dati.}}

\newglossaryentry{Linguaggio di markup}{name={linguaggio di markup\G}, description={Un linguaggio di markup è un insieme di regole che descrivono i meccanismi di rappresentazione (strutturali, semantici o presentazionali) di un testo che, utilizzando convenzioni standardizzate, sono utilizzabili su più supporti.}}

\newglossaryentry{Linguaggio di programmazione}{name={linguaggio di programmazione\G}, description={Insieme di regole che descrivono i meccanismi di rappresentazione di un testo che, utilizzando convenzioni standardizzate, sono utilizzabili su più supporti; la tecnica di composizione di un testo con l’uso di marcatori (o espressioni codificate), richiede quindi una serie di convenzioni, ovvero appunto di un linguaggio a marcatori di documenti.}}

\newglossaryentry{Linux}{name={Linux\G}, description={Famiglia di sistemi operativi di tipo Unix-like, rilasciati sotto varie possibili distribuzioni, aventi la caratteristica comune di utilizzare come nucleo il kernel Linux.}}

\newglossaryentry{Mac OsX}{name={Mac Os\G}, description={Sistema operativo di Apple dedicato ai computer Macintosh; il nome è l’acronimo di Macintosh Operating System}}

\newglossaryentry{Mercurial}{name={Mercurial\G}, description={Mercurial è un software multipiattaforma di controllo di versione distribuito creato da Matt Mackall e rilasciato sotto GNU General Public License 2.0.}}

\newglossaryentry{milestone}{name={Milestone\G}, description={Termine inglese che letteralmente significa pietra miliare. Viene tipicamente utilizzato nella pianificazione e gestione di progetti complessi per indicare il raggiungimento di obiettivi stabiliti in fase di definizione del progetto stesso. Le milestone indicano importanti traguardi intermedi nello svolgimento del progetto.}}

\newglossaryentry{parser}{name={Parser\G}, description={Un parser è un programma che analizza un flusso continuo di dati in ingresso in modo da determinare la sua struttura grazie ad una data grammatica formale.}}

\newglossaryentry{plugin}{name={Plugin\G}, description={Programma non autonomo che interagisce con un altro programma per ampliarne o estenderne le funzionalità originarie.}}

\newglossaryentry{prototipo}{name={Prototipo\G}, description={Il prototipo è il modello originale o il primo esemplare di un manufatto, rispetto a una sequenza di eguali o similari realizzazioni successive.}}

\newglossaryentry{real time}{name={Real time\G}, description={Real-time (in italiano "tempo reale") è un termine utilizzato in ambito informatico per indicare quei programmi per i quali la correttezza del risultato dipende dal tempo di risposta.}}

\newglossaryentry{repository}{name={Repository\G}, description={Database in grado di contenere svariate tipologie di dati, corredate da relative informazioni (metadati). Offre inoltre un sistema di versionamento in grado di tener traccia delle modifiche effettuate al suo interno.}}

\newglossaryentry{reveal.js}{name={Reveal.js\G}, description={Framework per presentazioni di slide.}}

\newglossaryentry{Servizio cloud}{name={Servizio cloud\G}, description={Servizio che fornisce la possibilità di erogazione di risorse informatiche, come l'archiviazione, l'elaborazione o la trasmissione di dati}}

\newglossaryentry{sourceforge}{name={Sourceforge\G}, description={\'{E} una piattaforma e un sito web che fornisce gli strumenti per portare avanti un progetto di sviluppo software in modo collaborativo tra gli sviluppatori.}}

\newglossaryentry{Task list}{name={Task list\G}, description={Una lista di task.}}

\newglossaryentry{Task}{name={Task\G}, description={\'{E} un'attività che deve essere realizzata entro un determinato periodo di tempo.}}

\newglossaryentry{template}{name={Template\G}, description={Struttura generica o standard;}}
\newglossaryentry{Template}{name={Template\G}, description={Struttura generica o standard;}}

\newglossaryentry{Ticketing}{name={Ticketing\G}, description={Sistema con cui si richiede a qualcuno di eseguire un compito.}}

\newglossaryentry{Ubuntu}{name={Ubuntu\G}, description={Distribuzione GNU/Linux, basata su Debian.}}

\newglossaryentry{Verbale}{name={Verbale\G}, description={Documento che contiene una verbalizzazione, atto giuridico consistente nella narrazione per iscritto, in maniera sintetica ma fedele, fatta dalla persona incaricata, di dichiarazioni, operazioni o altri fatti giuridici avvenuti in sua presenza, allo scopo di ricordarli e costituirne prova}}

\newglossaryentry{Versionamento}{name={Versionamento\G}, description={Gestione di versioni multiple di un insieme di informazioni.}}

\newglossaryentry{Walkthrough}{name={Walkthrough\G}, description={Tecnica di esaminazione nel quale si ispeziona meticolosamente un documento per ricercarne eventuali errori.}}

\newglossaryentry{Windows}{name={Windows\G}, description={Famiglia di ambienti operativi e sistemi operativi dedicati ai personal computer, alle workstation, ai server e agli smartphone.}}


\newglossaryentry{WordPress}{name={WordPress\G}, description={\'{E} una piattaforma software di "personal publishing" e content management system (CMS), sviluppata in PHP con appoggio al gestore di database MySQL e che consente la creazione e distribuzione di un sito Internet formato da contenuti testuali o multimediali.}}

%\newglossaryentry{}{name={\G}, description={}}


% GLOSSARIO DEGLI ACRONIMI

\newacronym{CMS}{CMS\G}{Content Management System}
\newacronym{CSS}{CSS\G}{Cascading Style Sheets}
\newacronym{GUI}{GUI\G}{Graphical User Interface}
\newacronym{JSON}{JSON\G}{JavaScript Object Notation}
\newacronym{PERT}{PERT\G}{Project Evaluation and Review Technique}
\newacronym{PNG}{PNG\G}{Portable Network Graphics}
\newacronym{SVN}{SVN\G}{Subversion}
\newacronym{SWE}{SWE\G}{Software Engineering}
\newacronym{UI}{UI\G}{User Interface}
\newacronym{UML}{UML\G}{Unified Modeling Language}
\newacronym{WBS}{WBS\G}{Work Breakdown Structure}

%\newacronym{}{\G}{}