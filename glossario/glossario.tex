% per essere usato è necessario il pacchetto \usepackage{glossaries}
% da includere nel file in cui si deve usare tramite il comando \loadglsentries{glossary/glossario.tex}
\makeglossaries

% GLOSSARIO DEI TERMINI

\newglossaryentry{prototipo}{name={prototipo\G}, description={è il modello originale o il primo esemplare di un manufatto.}}

\newglossaryentry{front-end}{name={front-end\G}, description={front.}}

\newglossaryentry{back-end}{name={back-end\G}, description={front.}}

\newglossaryentry{Laravel}{name={Laravel\G}, description={front.}}

\newglossaryentry{PHP}{name={PHP\G}, description={front.}}

\newglossaryentry{infografiche}{name={infografiche\G}, description={front.}}

\newglossaryentry{Mustache.js}{name={Mustache.js\G}, description={front.}}

\newglossaryentry{framework}{name={framework\G}, description={oooo.}}

\newglossaryentry{Angular.js}{name={Angular.js\G}, description={javascript.}}

\newglossaryentry{MongoDB}{name={MongoDB\G}, description={database.}}

\newglossaryentry{DBMS}{name={DBMS\G}, description={database.}}

\newglossaryentry{milestone}{name={milestone\G}, description={Punto nel tempo che determina importanti traguardi intermedi nello svolgimento del progetto. Indica a che distanza si è dalla fine del progetto.}}

\newglossaryentry{infografica}{name={infografica\G}, description={Tecnica che consiste nel proiettare l'informazione in una forma più grafica e visuale che testuale.}}

% GLOSSARIO DEGLI ACRONIMI
\newacronym{SWE}{SWE\G}{Software Engineering}
