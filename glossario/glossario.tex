% per essere usato è necessario il pacchetto \usepackage{glossaries}
% da includere nel file in cui si deve usare tramite il comando \loadglsentries{glossary/glossario.tex}
\makeglossaries

% GLOSSARIO DEI TERMINI

\newglossaryentry{prototipo}{name={prototipo\G}, description={è il modello originale o il primo esemplare di un manufatto.}}

\newglossaryentry{front-end}{name={front-end\G}, description={front.}}

\newglossaryentry{AJAX}{name={AJAX\G}, description={front.}}

\newglossaryentry{back-end}{name={back-end\G}, description={front.}}

\newglossaryentry{Laravel}{name={Laravel\G}, description={front.}}

\newglossaryentry{PHP}{name={PHP\G}, description={front.}}

\newglossaryentry{infografiche}{name={infografiche\G}, description={front.}}

\newglossaryentry{Mustache.js}{name={Mustache.js\G}, description={front.}}

\newglossaryentry{framework}{name={framework\G}, description={oooo.}}

\newglossaryentry{Angular.js}{name={Angular.js\G}, description={javascript.}}

\newglossaryentry{MongoDB}{name={MongoDB\G}, description={database.}}

\newglossaryentry{DBMS}{name={DBMS\G}, description={database.}}

\newglossaryentry{milestone}{name={milestone\G}, description={Punto nel tempo che determina importanti traguardi intermedi nello svolgimento del progetto. Indica a che distanza si è dalla fine del progetto.}}

\newglossaryentry{infografica}{name={infografica\G}, description={Tecnica che consiste nel proiettare l'informazione in una forma più grafica e visuale che testuale.}}

\newglossaryentry{Back-End}{name={Back-End\G}, description={front.}}

\newglossaryentry{Front-End}{name={Front-End\G}, description={front.}}

\newglossaryentry{slide}{name={slide\G}, description={front.}}

\newglossaryentry{Slide}{name={Slide\G}, description={front.}}

\newglossaryentry{template}{name={template\G}, description={front.}}

\newglossaryentry{database}{name={database\G}, description={front.}}

\newglossaryentry{business}{name={business\G}, description={front.}}

\newglossaryentry{Design Pattern}{name={Design Pattern\G}, description={front.}}

\newglossaryentry{design pattern}{name={design pattern\G}, description={front.}}

\newglossaryentry{Real Time}{name={Real Time\G}, description={front.}}

\newglossaryentry{same-origin-policy}{name={same-origin-policy\G}, description={front.}}

\newglossaryentry{real time}{name={real time\G}, description={front.}}

\newglossaryentry{real-time}{name={real-time\G}, description={front.}}

\newglossaryentry{Reveal.js}{name={Reveal.js\G}, description={front.}}

\newglossaryentry{HTML5}{name={HTML5\G}, description={front.}}

\newglossaryentry{linguaggio di markup}{name={linguaggio di markup\G}, description={front.}}

\newglossaryentry{JavaScript}{name={JavaScript\G}, description={front.}}

\newglossaryentry{Javascript}{name={Javascript\G}, description={front.}}

\newglossaryentry{linguaggio di programmazione}{name={linguaggio di programmazione\G}, description={front.}}

\newglossaryentry{web-scraping}{name={web-scraping\G}, description={front.}}

\newglossaryentry{Angular}{name={Angular\G}, description={front.}}

\newglossaryentry{HTML}{name={HTML\G}, description={front.}}

\newglossaryentry{JQuery}{name={JQuery\G}, description={front.}}

\newglossaryentry{top-down}{name={top-down\G}, description={front.}}

\newglossaryentry{REST-like}{name={REST-like\G}, description={front.}}

\newglossaryentry{REST}{name={REST\G}, description={front.}}

\newglossaryentry{Rest}{name={Rest\G}, description={front.}}


% GLOSSARIO DEGLI ACRONIMI
\newacronym{SWE}{SWE\G}{Software Engineering}
\newacronym{CMS}{CMS\G}{Content Management System}
\newacronym{CRUD}{CRUD\G}{Create, Read, Update, Delete}
\newacronym{CSS}{CSS\G}{Cascading Style Sheets}
\newacronym{GUI}{GUI\G}{Graphical User Interface}
\newacronym{JSON}{JSON\G}{JavaScript Object Notation}
\newacronym{PERT}{PERT\G}{Project Evaluation and Review Technique}
\newacronym{PNG}{PNG\G}{Portable Network Graphics}
\newacronym{SDK}{SDK\G}{Software Development Kit}
\newacronym{SEMAT}{SEMAT\G}{Software Engineering Method and Theory}
\newacronym{SVN}{SVN\G}{Subversion}
\newacronym{UI}{UI\G}{User Interface}
\newacronym{UML}{UML\G}{Unified Modeling Language}
\newacronym{URI}{URI\G}{Uniform Resource Identifier}
\newacronym{WBS}{WBS\G}{Work Breakdown Structure}
\newacronym{W3C}{W3C\G}{World Wide Web Consortium}
\newacronym{MVC}{MVC\G}{Model View Controller}
